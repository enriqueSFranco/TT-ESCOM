\section{Trabajo Realizado en Trabajo Terminal I}
El trabajo realizado lo largo de \textbf{Trabajo Terminal I} se realizarón actividades de levantamiento de requerimientos y análisis de casos de uso atravez de juntas 
o sesiones en Meet con el área de extensión y apoyos educativos de la mano del licenciado NOMBREDELICENCIADO, de manera breve se redactan los resultados obtenidos al finalizar Trabajo Terminal I.\\

Como resultado de las juntas con el área de  extensión y apoyos educativos de la ESCOM se identificaron 4 actores indispensables del sistema:
    \begin{itemize}
        \item Encargado del sistema.
        \item Colaborador del sistema.
        \item Reclutador de una empresa.
        \item Candidato.
    \end{itemize}

Se levantaron al rededor de 40 requerimientos funcionales los cuales fueron desglosados en 80 casos de uso y se separaron en 4 módulos diferentes con base en el negocio que
manejan.
    \begin{description}
        \item[Módulo general:] en este módulo se abordan los requerimientos de iniciar sesión, crear cuenta, recuperar contraseña, modificar contraseña y que cualquier usuario registrado o no puede acceder.
        \item[Módulo de usuarios:] aquí se abarcan todos los requerimientos de configuración de perfiles de cuenta para todos los candidatos.
        \item[Módulo de vacantes:] en este módulo habitan toda la gestión vacantes para los diferentes tipos de usuario
        \item[Módulo de postulaciones:]  en este módulo habitan toda la gestión de postulaciones para los diferentes tipos de usuario
        \item[Módulo de administración:] aquí habitan los requerimientos de manejo de cuentas, control de permisos, y generación de reportes
    \end{description} 

Para tener el control de todos los casos de uso, pantallas y diagramas se utilizaron las siguientes herramientas:
    \begin{itemize} 
        \item Balsamiq Cloud para el diseño de interfaces de usuario.
        \item Visual Paradigm en su versión Online para todos los diagramas.
    \end{itemize} 

Para dichas herramientas se adquirieron sus correspondientes licencias y configuraron plataformas que se utilizarón durante el desarrollo del proyecto.
Se creo un repositorio en la plataforma GitHub usando el Sistema de Control de Versiones Git, el cual está conformado por la siguiente estructura de carpetas:
    \begin{itemize}
        \item Backend: Esta carpeta contiene todo lo relacionado con el servidor web que se estará programando en Python y usando el framework Django.
        \item Frontend: Esta carpeta contiene todo lo relacionado a las interfaces gráficas, componentes, imágenes, servicios que nos ayudarán a comunicar el Front End con el Backend, archivos para el manejo de rutas públicas y privadas del sistema y scripts que nos ayudarán a darle funcionalidad al sistema.
        \item Análisis:  Esta carpeta contiene todo lo relacionado con el documento de nuestro trabajo terminal.
    \end{itemize}

    Se hizo la instalación de los diferentes entornos y librerías para que las tecnologías que se seleccionaron puedan funcionar sin problemas, a continuación, se menciona cuales fueron estos entornos y librerías:
    \begin{itemize}
        \item Editor de código:  Visual Studio Code, versión 1.67
        \item ReactJS:
        \begin{itemize}
            \item node 16.6.0
            \item create-react-app
            \item npm
            \item material ui
            \item axios
        \end{itemize}
        \item Django
        \begin{itemize}
            \item Django REST framework 3.13.1
            \item Django 4.0.2
            \item Pillow 9.0.1
            \item Psycopg2-binary 2.9.3
            \item django-environ 0.8.1
            \item djangorestframework-simplejwt 5.1.0
        \end{itemize}
    \end{itemize}

    Los casos de uso que se describieron y desarrollaron durante los meses de \textbf{Trabajo Terminal I} se listan a continuación:
    \begin{itemize}
        \item \refElem{GRL-CU01}
        \item \refElem{GRL-CU02}
        \item \refElem{GRL-CU03}
        \item \refElem{PST-CU01}
        \item \refElem{PST-CU01-1}
        \item \refElem{PST-CU01-2}
        \item \refElem{VCT-CU01}
        \item \refElem{VCT-CU02} 
        \item \refElem{VCT-CU02-1} 
        \item \refElem{VCT-CU03}
        \item \refElem{VCT-CU04}
    \end{itemize} 

    Al finalizar los meses de \textbf{Trabajo Terminal I}  se obtuvo la base de datos que  y el análisis tentativo con el algoritmo .....