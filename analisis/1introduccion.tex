%!TEX root = ../main-ejemplo.tex
\chapter{Introducción}
\label{ch:introduccion}

    \instrucciones{Añade una descripción acerca del documento, por ejemplo:}
    \instrucciones{Para quitar estos comentarios quita la opcion ''instrucciones'' del packete cdtControl\\}
    El presente documento contiene la especificación del entregable \varCveEntregable:\varEntregable\ del Proyecto \varProyecto, correspondiente a ...

    \section{Intención del documento}

        \instrucciones{Indica el objetivo del documento y a quien va dirigido, por ejemplo:\\}
        En este documento refleja el resultado de los avances realizados en el sistema \varCveSistema-\varSistema y tiene como objetivo ...\\

        Va dirigido a ...

    \section{Estructura del documento}

        \instrucciones{Indica como se encuentra organizado el documento, y en que consiste cada parte \\}
        El presente documento se encuentra organizado en ... partes, cada una dividida a su vez en capítulos, como se muestra a continuación.

        \begin{itemize}
	        \item El capítulo \ref{ch:nomenclatura} presenta ...
	        \item Parte 1: Modelo de Negocio
		        \begin{itemize}
			        \item El capítulo \ref{ch:glosario} presenta ...
		        \end{itemize}
        \end{itemize}
	
\chapter{Nomenclatura}
\label{ch:nomenclatura}
    
    \instrucciones{Indica en que consiste este capitulo y cual es la finalidad del mismo, por ejemplo:\\}
   	La información del presente documento se encuentra estructurada mediante ...

    \section{Modelado del negocio}
    
        \instrucciones{Indica la forma en que se modela el negocio, así como los modelados que se usan (glosario de términos, estructura de información, reglas de negocio, máquinas de estados, etc); por ejemplo:\\}

        Para modelar el negocio, se presenta el glosario de términos, el modelado de la estructura de la información que tendrá el sistema, las reglas de negocio que rigen las operaciones y máquinas de estados para modelar los ciclos de gestión en el proyecto \varProyecto.
        

        \subsection{Modelado de estructura de datos}
        \label{sec:ModEstructuraDatos}
        
        \subsection{Reglas de Negocio}
        \label{sec:ModReglasNegocio}
        
        \subsection{Maquinas de Estados}
        \label{sec:ModMaquinas}
    
    \clearpage
    \section{Modelado de casos de uso}
    \section{Modelado de interfaces}
    \section{Modelado de mensajes}
    
    
