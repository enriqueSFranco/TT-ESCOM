%!TEX root = ../example.tex

\chapter{Modelado Dinámico}
\label{ch:modDinamico}

\begin{UseCase}[Autor/DanielO,Version/1.0,Estado/Ok]{HR-PR-CU1}{Llenar solicitud de unidades de aprendizaje }{
		El Profesor\footnote{Ver \refElem{tProfesor}} podrá llenar la solicitud de las Unidades de Aprendizaje\footnote{Ver \refElem{tUnidadDeAprendizaje}} que le gustaría impartir en el periodo escolar\footnote{Ver \refElem{tPeriodoEscolar}} inmediato siguiente, sin que esta solicitud genere una obligación de asignarle esas unidades de aprendizaje en su carga académica\footnote{Ver \refElem{tCargaAcadémica}}. El Profesor podrá definir su disponibilidad, así como guardar y cerrar su solicitud. Cuando el Profesor guarde su solicitud, podrá editarla posteriormente siempre y cuando esté dentro del periodo de recepción de las mismas y en el caso de que cierre la solicitud, el Profesor ya no podrá hacer cambios.}

	\UCitem[control]{Superviso}{  Panchito Lopez Cordero (Panchito LoCo)}
	\UCitem[control]{Operación}{  Registro de propuesta de Unidades de Aprendizaje}	
	\UCitem[control]{Revisor}{    David Ortega Pacheco / Sandra Ivette Bautista Rosales}
	\UCitem[control]{Último cambio}{04 de Septiembre del 2017}
	
	\UCsection[control]{Revisión Version 0.2}
	    \UCitem[control]{Fecha}{05 de Septiembre del 2017}
	    \UCitem[control]{Resultado}{Con correcciones}
	    \UCitem[control]{Observaciones}{
		    \begin{Titemize}
			    \Titem \TODO En el resumen, la palabra profesor no debería tener superíndice, debe aparecer el término como link porque está en singular como la definición.
			    \Titem \TODO Link roto de carga académica (9), falta agregar y definir ese término "carga académica"
			    \Titem \TODO Verificar si los nuevos cambios al mockup tienen RN, por ejemplo: 1) Si el profesor solicita materias en más de una UA, el cierre de solicitud implica que cerrará en todas las unidades académicas en las que puede solicitar?
			    \Titem \TODO Revisar entradas y salidas de acuerdo a los cambios que resulten en el mockup de solicitud de unidades de aprendizaje
		    \end{Titemize}}
	
	
	\UCsection{Atributos}
	    \UCitem{Actor}{\refElem{aGerenteVentas}}
        \UCitem{Propósito}{
	        \begin{Titemize}
		        \Titem Permitir al profesor hacer la solicitud de las unidades de aprendizaje que le gustaría impartir de acuerdo a la disponibilidad que él tiene.
	        \end{Titemize}
        }
	    \UCitem{Entradas}{ \imprimeEntrada }
	    \UCitem{Origen}{El actor selecciona los datos con el mouse y del teclado.}
	    \UCitem{Salida}{
		    \begin{Titemize}
			    \Titem Datos Laborales del Profesor.
			    \Titem Descarga por Nombramiento.
			    \Titem Unidad Académica.
			    \Titem Academia
			    \Titem Unidades de Aprendizaje
			    \Titem Se muestra el mensaje \refIdElem{MSG1} para indicar que la solicitud fue guardada exitosamente.
			    \Titem Se muestra el mensaje \refIdElem{MSG12} en pantalla para indicar que la solicitud será cerrada y no podrá hacer cambios futuros.
		    \end{Titemize}
	    }   
	    \UCitem{Destino}{
		    \begin{Titemize}
			    \Titem \refIdElem{MSG1} en pantalla.
			    \Titem \refIdElem{MSG12} en pantalla.
			    \Titem \refIdElem{MSG17} en pantalla.
		    \end{Titemize}
	    }
	    \UCitem{Precondición}{
		    \begin{Titemize}
			    \Titem\textbf{Sistematizada}: Que exista el registro de los datos del profesor.
			    \Titem\textbf{Sistematizada}: Que el periodo para recepción de solicitudes esté habilitado.
			    \Titem\textbf{Sistematizada}: Que el profesor esté asociado por lo menos a una unidad académica.
		    \end{Titemize}
	    }
	
	    \UCitem{Postcondición}{
		    \begin{Titemize}
			    \Titem \textbf{Sistematizada}: Si el profesor cerró la solicitud, entonces ésta será enviada. 
			    \Titem \textbf{Sistematizada}: Si el profesor cerró la solicitud, cambia el estado de la solicitud de unidades de aprendizaje a " Enviado "
			    \Titem \textbf{Sistematizada}: Si el profesor guardó la solicitud, ésta debe quedar en estado de editable hasta que el profesor la cierre o se termine el periodo de envío.
			    \Titem \textbf{Sistematizada}: El profesor podrá consultar la solicitud que envió.
		    \end{Titemize}
	    }
	    \UCitem{Reglas de Negocio}{
		    \begin{Titemize}
			    \Titem \refIdElem{BR-S001}		
			    \Titem \refIdElem{BR-N027}
		    \end{Titemize}
	    }	
	    \UCitem{Errores}{
		    \begin{Titemize}
			    \Titem \UCerr{Uno}{Cuando no hay información en el catálogo \textbf{Periodo escolar},}{se muestra el mensaje \refIdElem{MSG15} en la pantalla \refIdElem{HR-PR-UI1} y termina el caso de uso.}
			    
			    \Titem \UCerr{Dos}{Cuando el tiempo para realizar la solicitud se acabo},{se muestra el mensaje \refIdElem{MSG58} en la pantalla \refIdElem{HR-PR-UI1} y termina el caso de uso.}
					    
		    \end{Titemize}
	    }
	    \UCitem{Referencia Documental}{}
	    \UCitem{Datos sensibles}{}
	
	\UCsection[design]{Datos de Diseño}
	    \UCitem[design]{Disparador}{El Profesor desea realizar su solicitud de Unidades de Aprendizaje} %progr definir un periodo para la selección de unidades de aprendizaje.
	    \UCitem[design]{Condiciones de Término}{Se tendrá registrado la solicitud.}
	    \UCitem[design]{Efectos Colaterales}{Los profesores podrán hacer su solicitud de unidades de aprendizaje para un periodo escolar siguiente durante el periodo definido.}
	    \UCitem[design]{Suposiciones}{Se considera que el usuario debio haber sido capacitado por la empresa}
	    \UCitem[design]{Casos de Prueba}{Caso de Prueba CP13: Casos de Pruebas}
	    
	\UCsection[admin]{Datos de Administración de Requerimiento}
	    \UCitem[admin]{Viene de}{\refIdElem{HR-PR-CU2} }
	    \UCitem[admin]{Requerimientos Funcionales}{}%
	    \UCitem[admin]{Requerimientos del Sistema}{}%
        \UCitem[admin]{Madurez}{Alta}               %
	    \UCitem[admin]{Volatilidad}{Media}          %
	    \UCitem[admin]{Complejidad}{Baja}           %
        \UCitem[admin]{Tamaño}{Grande}              %
	    \UCitem[admin]{Prioridad}{Media}            %
        \UCitem[admin]{Auditable}{}                 
        \UCitem[admin]{Issues}{}                    %
	    \UCitem[admin]{Dificultades}{               %
		    \begin{Titemize}
			    \Titem Por definir RN: ¿Qué pasa si un profesor guarda su solicitud para terminarla después y antes de que la cierre se termina el tiempo de envío?
			    \Titem Por definir: Siempre se tiene que guardar antes de cerrar? o cerrar solicitud implica que el sistema va a guardar?
			    \Titem En condiciones de termino, no se si es con que termina el caso de uso o que pasa cuando termina
			    \Titem De las pantallas falta revisar: 1) Si tiene sentido dejar el label "Descarga por nombramiento" ya que no tenemos un catálogo de todos los puestos de la UA y también ya hay un responsable de SD que se encarga de validar los cargos y verificar si aplica algunas horas de descarga
			    \Titem 2) Pasar a otra pantalla lo de "Definir Disponibilidad"
			    \Titem 3) Quitar el ícono del lápiz para editar en la tabla de las materias seleccionadas
		    \end{Titemize}
	    }
	
\end{UseCase}

%Trayectoria Principal
\begin{UCtrayectoria}
	\UCpaso [\UCactor] Selecciona Solicitud de Carga desde el Menú de Profesores
	\UCpaso [\UCsist] Muestra la pantalla \refIdElem{HR-PR-UI1}. 
	\UCpaso [\UCactor] Selecciona la \entrada{Descarga por Nombramiento}, la \entrada{Unidad Académica} y la \entrada{Academia}.
	\UCpaso [\UCactor] Presiona el icono \IUBuscar. \label{HR-PR-IU1:Materias}
	\UCpaso [\UCsist] Muestra el catalogo de Unidades de Aprendizaje. \refErr{Uno}
	\UCpaso [\UCactor] Selecciona la \entrada{Unidad de Aprendizaje} desde el catalogo.
	\UCpaso [\UCactor] Presiona el botón \IUbutton{Seleccionar}. \refTray{A}
	\UCpaso [\UCactor] Selecciona la cantidad de \entrada{Grupos}.
	\UCpaso [\UCactor] Presiona el botón \IUbutton{Agregar Unidad de Aprendizaje}.
	\UCpaso [\UCsist] Muestra la unidad de aprendizaje en la tabla de la unidad académica seleccionada
	\UCpaso [\UCsist] Habilita los iconos \IURegistrar y \IUEliminar.
	\UCpaso [\UCactor] Selecciona si quiere Definir la \entrada{Disponibilidad} presionando el Icono "No". \refTray{B} \refTray{C} \label{HR-PR-CU1:EdiEli}
	\UCpaso [\UCactor] Presiona el botón \IUbutton{Cerrar Solicitud}. \refTray{D} \refTray{E} \label{HR-PR-IU1:Dispo}
	\UCpaso [\UCsist] Muestra el mensajes \refIdElem{MSG12}.
	\UCpaso [\UCactor] Presiona el botón \IUbutton{Definir Disponibilidad}. \refTray{F}
	\UCpaso [\UCsist] Verifica que este en el tiempo adecuado para entregar la solicitud. \refErr{Dos}
\end{UCtrayectoria}

%Trayectorias Alternativas
\begin{UCtrayectoriaA}[Fin del caso de uso]{A}{El actor desea cancelar la operación.}
	\UCpaso [\UCactor] Presiona el botón \IUbutton{Cancelar} de la pantalla \refIdElem{HR-UA-UI1}.
	\UCpaso regresa al paso \ref{HR-PR-IU1:Materias} de la trayectoria principal.
\end{UCtrayectoriaA} 

\begin{UCtrayectoriaA}[Fin del caso de uso]{B}{El actor desea Definir la Disponibilidad.}
	\UCpaso [\UCactor] Presiona el icono "Si" de la pantalla \refIdElem{HR-PR-UI1}.
	\UCpaso [\UCsist] Habilita los días para poder modificar el horario
	\UCpaso [\UCactor] Llena los datos que requiere cumpliendo la \refIdElem{BR-N004X} 
	\UCpaso regresa al paso \ref{HR-PR-IU1:Dispo} de la trayectoria principal.
\end{UCtrayectoriaA} 

\begin{UCtrayectoriaA}[Fin de la trayectoria]{C}{El actor desea editar o eliminar.}
	\UCpaso [\UCactor] Presiona el icono \IURegistrar. \refTray{C1}
	\UCpaso [\UCsist] Habilita el campo unidad de aprendizaje
	\UCpaso [\UCactor] Presiona el campo unidad de aprendizaje 
	\UCpaso [\UCsist] Muestra el catalogo de Unidades de Aprendizaje. \refErr{Uno}
	\UCpaso [\UCactor] Selecciona la Unidad de Aprendizaje desde el catalogo.
	\UCpaso [\UCactor] Presiona el botón \IUbutton{Seleccionar}. \refTray{A}
	\UCpaso [\UCsist] regresa al paso \ref{HR-PR-CU1:EdiEli} de la trayectoria principal
\end{UCtrayectoriaA} 

\begin{UCtrayectoriaA}[Fin de la trayectoria]{C1}{El actor desea editar o eliminar.}
	\UCpaso [\UCactor] Presiona el icono \IUEliminar.
	\UCpaso [\UCsist] Muestra el mensaje \refIdElem{MSG17}.
	\UCpaso [\UCactor] Presiona el boton \IUbutton{Aceptar}
	\UCpaso [\UCsist] Elimina la Unidad de Aprendizaje
	\UCpaso [\UCsist] regresa al paso \ref{HR-PR-CU1:EdiEli} de la trayectoria principal
\end{UCtrayectoriaA} 


\begin{UCtrayectoriaA}[Fin del caso de uso]{D}{El actor desea cancelar la operación.}
	\UCpaso [\UCactor] Presiona el botón \IUbutton{Cancelar} de la pantalla \refIdElem{HR-PR-UI1}.
	\UCpaso regresa al paso \ref{HR-PR-IU1:Dispo} de la trayectoria principal.
\end{UCtrayectoriaA}

\begin{UCtrayectoriaA}[Fin del caso de uso]{E}{El actor desea guardar la operación.}
	\UCpaso [\UCactor] Presiona el botón \IUbutton{Guardar} de la pantalla \refIdElem{HR-PR-UI1}.
\end{UCtrayectoriaA} 

\begin{UCtrayectoriaA}[Fin del caso de uso]{F}{El actor desea cancelar la operación.}
	\UCpaso [\UCactor] Presiona el botón \IUbutton{Cancelar} de la pantalla \refIdElem{HR-PR-UI1}.
	\UCpaso regresa al paso \ref{HR-PR-IU1:Dispo} de la trayectoria principal.
\end{UCtrayectoriaA}

