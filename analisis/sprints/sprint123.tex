\section{Sprint 1: Pruebas de concepto}
    \begin{description}
        \item \textbf{Duración}: 15 días.
        \item \textbf{Inicio del sprint }: 31 de enero de 2022.
        \item \textbf{Cierre del sprint }: 14 de febrero de 2022.
    \end{description}

    Esta iteración tuvo como propósito capacitarnos en las tecnologías que se van ha estar utilizando 
    a lo largo del desarrollo del sistema. La actividades que se realizaron son las siguientes:
    \begin{enumerate}
        \item Capacitación en React Js el cual se va utilizar en el sistema y fue para todo el equipo.
        \item Capacitación en Django el cual se va utilizar en el sistema y fue para todo el equipo.
    \end{enumerate}

%=========================================================================================================%

\section{Sprint 2: Análisis general del sistema}
    \begin{description}
        \item \textbf{Duración}: 15 días.
        \item \textbf{Inicio del sprint }: 15 de febrero de 2022.
        \item \textbf{Cierre del sprint }: 1 de marzo de 2022.
    \end{description}

    Esta iteración tuvo como propósito identificar todos los requerimientos fucnionales y no funcionales del sistema. 
    Las actividades que se realizaron de manera general fueron las siguientes:
    \begin{enumerate}
        \item Se hizo el levantamiento de requerimientos a travez de juntas con el personal de la bolsa de trabajo de ESCOM.
        \item Se creo el product backlog con base en los requerimientos identificados.
        \item Se elaboró el diagrama de procesos con base en las juntas con el personal de la bolsa de trabajo.
        \item Se hizo la estimación de casos de uso de acuerdo a los requerimientos identificados.
    \end{enumerate} 

%=========================================================================================================%

\section{Sprint 3: Preparación del entorno}
    \begin{description}
        \item \textbf{Duración}: 9 días. 
        \item \textbf{Inicio del sprint }: 2 de marzo de 2022.
        \item \textbf{Cierre del sprint }: 10 de marzo de 2022.
    \end{description}
    
    Esta iteración tuvo como propósito adquirir licencias y configurar plataformas que se van utilizar
    durante el desarrollo del proyecto.Las actividades que se realizaron de manera general fueron las siguientes:
    \begin{enumerate}
        \item Se adquirieron las licencias de Balsamiq Cloud y Visual Paradigm  para todo el análisis de
        requerimientos y de diseño.
        \item Configurar las diferentes plataformas para el analisis y diseño del sistema.
        \item Se creo un repositorio usando el Sistema de Control de Versiones Git para almacenar el código y la documentación del proyecto
        utilizando la prataforma de GitHub.
        \item Se configuró el proyecto en front end utilizando React Js con la versión 17.0.2.
    \end{enumerate}
