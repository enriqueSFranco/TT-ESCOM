\subsection{Preparación del entorno}
    Esta iteración tuvo como propósito adquirir licencias y configurar plataformas que se van utilizar
    durante el desarrollo del proyecto.Las actividades que se realizaron de manera general fueron las siguientes:
    \begin{enumerate}
        \item Se adquirieron las licencias de Balsamiq Cloud y Visual Paradigm  para todo el análisis de
        requerimientos y de diseño.
        \item Se creo un repositorio en la plataforma GitHub usando el Sistema de Control de Versiones Git, el cual está conformado por la siguiente estructura de carpetas:
        \begin{itemize}
            \item Backend: Esta carpeta contiene todo lo relacionado con el servidor web que se estará programando en Python y usando el framework Django.
            \item Frontend: Esta carpeta contiene todo lo relacionado a las interfaces gráficas, componentes, imágenes, servicios que nos ayudarán a comunicar el Front End con el Backend, archivos para el manejo de rutas públicas y privadas del sistema y scripts que nos ayudarán a darle funcionalidad al sistema.
            \item Análisis:  Esta carpeta contiene todo lo relacionado con el documento de nuestro trabajo terminal.
        \end{itemize}
    \end{enumerate}

    Se hizo la instalación de los diferentes entornos y librerías para que las tecnologías que se seleccionaron puedan funcionar sin problemas, a continuación, se menciona cuales fueron estos entornos y librerías:
\begin{itemize}
    \item Editor de código:  Visual Studio Code, versión 1.67
\item ReactJS:
\begin{itemize}
    \item node 16.6.0
    \item create-react-app
    \item npm
    \item material ui
    \item axios
\end{itemize}
\item Django
\begin{itemize}
    \item Django REST framework 3.13.1
    \item Django 4.0.2
    \item Pillow 9.0.1
    \item Psycopg2-binary 2.9.3
    \item django-environ 0.8.1
    \item djangorestframework-simplejwt 5.1.0
\end{itemize}
\end{itemize}
