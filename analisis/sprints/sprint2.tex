\clearpage
\section{Análisis general del sistema}
    \begin{description}
        \item \textbf{Duración}: 15 días.
        \item \textbf{Inicio del sprint }: 15 de febrero de 2022.
        \item \textbf{Cierre del sprint }: 1 de marzo de 2022.
    \end{description}

    Esta iteración tuvo como propósito identificar todos los requerimientos funcionales y no funcionales del sistema. 
    Las actividades que se realizaron de manera general fueron las siguientes:
    \begin{enumerate}
        \item Se hizo el levantamiento de requerimientos a travez de juntas con el personal de la bolsa de trabajo de ESCOM.
        \item Se elaboró el diagrama de procesos con base en las juntas con el personal de la bolsa de trabajo.
        \item Se hizo la estimación de casos de uso de acuerdo a los requerimientos identificados.
    \end{enumerate} 

    Durante este sprint  se hicieron juntas de levantamiento de requerimientos con el jefe de departamento de atención de apoyo y recursos educativos de la ESCOM  a través de Google Meet , en donde se hacían preguntas sobre el o los pasos y condiciones que tienen que hacer para publicar una vacante de una empresa.
    Como resultado se identificaron 4  factores:
    \begin{enumerate}
        \item Encargado de la bolsa de trabajo
        \item El colaborador de la bolsa de trabajo
        \item El reclutador de una empresa 
        \item El candidato que en su mayoría son estudiantes.
    \end{enumerate} 

    Los requerimientos que se obtuvieron en dichas sesiones fueron los siguientes:
    \begin{enumerate}
        \item El sistema debe proporcionar un mecanismo que permita al actor acceder al sistema mediante un correo y una contraseña.
        \item El sistema debe proporcionar un mecanismo que permita al actor registrarse como nuevo usuario en el sistema.
        \item El sistema debe proporcionar un mecanismo que permita al actor recuperar su contraseña en caso de haberla olvidado.
        \item El sistema debe permitir al actor acceder a su perfil para consultar la información registrada.
        \item El sistema debe permitir al actor modificar la información registrada en su perfil, como nombre, apellidos, información de contacto y si el usuario es un candidato, debe de permitir modificar su información académica y datos de habilidades y conocimientos del mismo.
        \item El sistema debe permitir al actor modificar su contraseña actual por una nueva en el momento que él lo requiera.
        \item El sistema debe proporcionar un mecanismo que permita eliminar la cuenta de un actor, ya sea que el propio actor la elimine o si en dado caso es el encargado o colaborador la pueda eliminar.
        \item El sistema debe proporcionar  un mecanismo que le permita al actor recuperar su cuenta que anteriormente haya eliminado.
        \item El sistema debe permitir a cualquier persona consultar las vacantes que se tengan registradas siempre y cuando aun estén abiertas.
        \item El sistema debe permitir  a los usuarios registrados publicar vacantes.
        \item El sistema debe permitir a los usuarios editar las vacantes que ellos hayan publicado.
        \item El sistema debe permitir a los usuarios reportar vacantes publicadas si es que el usuario lo crea necesario.
        \item El sistema debe permitir a los usuarios eliminar las vacantes que ellos hayan publicado.
        \item El sistema debe permitir a los usuarios registrados enviar postularse a  las vacantes que esté abiertas.
        \item El sistema debe permitir al actor consultar el estado de su o sus postulaciones que haya hecho.
        \item El sistema debe permitir al actor indicar  el estado de las postulaciones que tiene según sus vacantes publicadas.
        \item El sistema debe permitir a los actores registrados consultar las postulaciones que se hayan hecho de acuerdo a cada vacante.
        \item El sistema debe permitir a los actores registrados consultar los candidatos registrados hayan o no postulado a una vacante.
        \item El sistema debe permitir publicar comunicados a los actores que tengan cuenta. 
        \item El sistema debe permitir consultar todos los comunicados que se tengan registrados y que aún están vigentes
        \item El sistema debe permitir editar todos los comunicados que se tengan registrados y que aún están vigentes
        \item El sistema debe permitir eliminar todos los comunicados que se tengan registrados.
        \item El sistema debe proporcionar un mecanismo para editar la información general de una empres: misión, visión, objetivos, nombre, RFC  y razón social
        \item El sistema debe proporcionar un mecanismo para que el representante de una empresa pueda enviar una solicitud o pre-registro para poder acceder y publicar vacantes.
        \item El sistema debe proporcionar un mecanismo para consultar la información general de una empres: misión, visión, objetivos, nombre, RFC  y razón social
        \item El sistema debe proporcionar un mecanismo para validar la información general de una empresa y así corroborar que sean empresas constituidas
        \item El sistema debe proporcionar un mecanismo para validar la información de un reclutador y así corroborar que trabajen para la empresa que representan.
        \item El sistema debe proporcionar un mecanismo para generar reportes sobre vacantes, cuentas, empresas, candidatos, reclutadores, postulaciones y uso de la plataforma.
        \item El sistema debe proporcionar un mecanismo para dar solución a los casos de vacantes reportadas por los propios usuarios
        \item El sistema debe proporcionar un mecanismo para que los colaboradores y el encargado puedan gestionar las cuentas de otros usuarios, es decir, eliminarlas, crearlas y recuperarlas si así lo requiere.
        \item El sistema debe proporcionar un mecanismo para validar la información de un reclutador y así corroborar que trabajen para la empresa que representan.
        \item El sistema debe proporcionar un mecanismo para recomendar y filtrar candidatos a los reclutadores, con base en las características y requerimientos de las vacantes publicadas por el propio reclutador  
        \item El sistema debe proporcionar un mecanismo para recomendar y filtrar vacantes a los candidatos, con base en las características, habilidades, conocimientos y preferencias del candidato.
        \item El sistema debe proporcionar un mecanismo para generar un curriculum en pdf con la información registrada del perfil de un candidato
        \item El sistema debe proporcionar un mecanismo para enviar notificaciones a los usuarios sobre los estados de sus vacantes, publicaciones, reportes y actividad en general de su cuenta.
        \item El sistema debe proporcionar un mecanismo para confirmar por medio de correos el envío de credenciales para las cuentas de reclutadores, recuperar contraseñas y recuperar cuentas
        \item El sistema debe proporcionar un mecanismo para guardar todos los registros de las cuentas creadas, eliminadas y recuperadas de los últimos 5 años
        \item El sistema debe proporcionar un mecanismo para guardar todos los registros de las vacantes, empresas, postulaciones y reportes de los últimos 5 años
        \item El sistema debe proporcionar un mecanismo para descargar y subir archivos según lo requieran los actores
        \item El sistema debe proporcionar un mecanismo para crear usuarios y contraseñas de nuevas cuentas

    \end{enumerate} 

    Una vez identificados los requerimientos, se hizo una estimación de casos de uso, donde se identificaron 80 casos de uso, y se separaron en 4 módulos, de los cuales para esta entrega de TT1 solo se van a comenzar a desarrollar 3.
    \begin{description}
        \item[Módulo general] en este módulo se abordan los requerimientos de iniciar sesión, crear cuenta, recuperar contraseña, modificar contraseña y que cualquier usuario registrado o no puede acceder.
        \item[Módulo de usuarios] aquí se abarcan todos los requerimientos de configuración de perfiles de cuenta para todos los candidatos.
        \item[Módulo de vacantes] en este módulo habitan toda la gestión vacantes para los diferentes tipos de usuario
        \item[Módulo de postulaciones]  en este módulo habitan toda la gestión de postulaciones para los diferentes tipos de usuario

        \item[Módulo de administración] aquí habitan los requerimientos de manejo de cuentas, control de permisos, y generación de reportes
\end{description} 


Para tener el control de todos los casos de uso, pantallas y diagramas se identificaron las siguientes herramientas:
\begin{itemize} 
    \item Balsamiq Cloud para el diseño de interfaces de usuario 
    \item Visual Paradigm en su versión Online para todos los diagramas
    \item Insta Gantt para el control de tiempo de los sprints
\end{itemize} 
