%!TEX root = ../example.tex
\clearpage
\section{Conclusiones}
Una vez que se finalizó con el desarrollo y todas las etapas especificadas en la metodología SCRUM, se consiguió un software capaz de gestionar la bolsa de trabajo de la ESCOM y de la capacidad de recomendar vacantes que mejor se ajusten acorde a los conocimientos que tenga un candidato, cumpliendo así, de manera satisfactoria con el objetivo general propuesto.\\
\newline
Con el desarrollo del sistema pudimos entender como es el desarrollo de un sistema en la vida real, teniendo que generar por nosotros mismos la planeación y la identificación de las tecnologías que se necesitaron para llevar a cabo la implementación del proyecto. Con el continuo esfuerzo y respectivas correcciones señaladas por los profesores directores del proyecto y los encargados de los boletines de la bolsa de trabajo se ha generado una plataforma en la cuál se pueda hacer uso de las funciones principales del sistema, mejorando los módulos proyectas e implementado nuevas funciones que no se habían proyectado inicialmente lo que muestra el avance que hemos tenido como equipo a lo largo del año que ha durado el desarrollo del trabajo y nos muestra lo complejo que es proponer y crear una nueva solución.\\
\newline
Sin embargo, a pesar que se cumplió con el objetivo general, se llego a la conclusión de que este software puede tener un mayor alcance, por lo que el sistema se fue construyendo a base de componentes reutilizables.




\section{Trabajo a futuro} 
A continuación vamos a listar los diferentes elementos que consideramos relevantes que se pueden tomar en cuenta para terminar de implementar el sistema.
\begin{itemize}
	\item Implementar la gestión de comunicados para que los reclutadores, colaboradores y el encargado del sistema puedan publicar anuncios dentro del sistema.
	\item Desarrollar el módulo de reportes y estadísticas de uso del sistema, cuyo objetivo es generar reportes de uso para que el encargado pueda presentar estadísticas  a sus superiores. El formato de descarga para dichos reportes debe ser .pdf, .csv. Recomendamos que se puedan seleccionar los datos que deben de tener cada reporte y de diferente módulo en el sistema.
	\item Implementar un gestor de versiones para las validaciones de las vacantes. Recomendamos que las observaciones sean de forma específica, es decir, que el usuario pueda seleccionar el párrafo y/o la sección que requiera una observación.
	\item Implementar el módulo de notificación y envío de correos electrónicos para indicar al usuario los nuevos cambios en el sistema y aquellas partes del negocio que requieran su atención inmediata. 
\end{itemize}

