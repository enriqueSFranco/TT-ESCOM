%!TEX root = ../example.tex
\chapter{Glosario}
\label{ch:glosario}

    \instrucciones{Indica en que consiste este capitulo y cual es la finalidad del mismo, por ejemplo:\\}

	El glosario mostrado a continuación presenta los términos utilizados a lo largo del documento ... 
	
    \noindent La lista de términos se encuentra agrupada por áreas de conocimiento:

	\begin{Citemize}
		\item Términos técnicos: Agrupa los términos que tienen que ver con el sistema.
		\item Términos del negocio: Agrupa los términos que tienen significado dentro del IPN.
	\end{Citemize}

\section{Términos técnicos}
\label{gls:terminosTecnicos}

\instrucciones{Describe cada uno de los elementos de este apartado:\\}

\begin{bGlosario}

	\bTerm{tAlfanumerico}{Alfanumérico} Es un \refElem{tTipoDato} definido por el conjunto de caracteres numéricos y alfabéticos.

\end{bGlosario}
