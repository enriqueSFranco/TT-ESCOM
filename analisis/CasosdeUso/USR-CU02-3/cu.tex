\begin{UseCase}[]{USR-CU02.3}{Editar Habilidades}{
	Permite al candidato editar su información personal y su dirección. así como permitir que suba su CV para guardarlo en el sistema.
}
	%----------------------------------------------------------------
	% Datos generales del CU:
	\UCsection{Atributos}
	\UCitem{Actor(es)}{
		Candidato. 
	}
	\UCitem[admin]{Prioridad}{ 
		Alta
	}
	\UCitem[admin]{Complejidad}{
		Media
	}
	\UCitem{Precondiciones}{
		\Titem El candidato debe de tener una cuenta en el sistema.
	}
	\UCitem{Destino}{
		\Titem \refElem{USR-IU2.3}
	}
	\UCitem{Viene de}{
		Caso de uso \refIdElem{GRL-CU03}.
	}	
\end{UseCase}

%Trayectoria Principal
\begin{UCtrayectoria}
	\UCpaso [\UCactor] Presiona el ícono \IUEditar{} en la sección ``Habilidades'' en la interfaz \refIdElem{USR-IU2}. 
	\UCpaso Obtiene el nombre de las habilidades registradas en el sistema y asociadas al candidato.
	\UCpaso Obtiene el catalogo completo de  las habilidades registradas en el sistema.
	\UCpaso [\UCsist] Muestra la información obtenida en la interfaz \refIdElem{USR-IU2.3}.
	\UCpaso \label{USRCU02-1-1} Elimina y/o agrega nuevas habilidades desde el catalogo del sistema.
	\UCpaso [\UCsist] Presiona el botón \IUbutton{Aceptar}.\refTray{A}
	\UCpaso Actualiza la información ingresada en el sistema.
	\UCpaso Muestra la información actualizada en la interfaz \refIdElem{USR-IU2}. 
\end{UCtrayectoria}

\begin{UCtrayectoriaA}[Fin del caso de uso]{A}{El actor no desea editar las habilidades}
	\UCpaso [\UCsist] Presiona el botón \IUbutton{Cancelar}.
	\extendUC{USR-CU02}.
\end{UCtrayectoriaA} 


