\begin{UseCase}{ADMN-CU02.3}{Editar información de un Colaborador}{
	Permite al actor editar la infromación de un colaborador que está en el sistema.
}
	%----------------------------------------------------------------
	% Datos generales del CU:
	\UCsection{Atributos}
	\UCitem{Actor(es)}{
		Encargado del sistema.
	}
	\UCitem{Prioridad}{
		Alta.
	}
	\UCitem{Complejidad}{
		Media.
	}
	\UCitem{Precondiciones}{
		Ninguna.
		%El correo electrónico del nuevo usuario no debe estar registrado en el sistema.
	}
	\UCitem{Postcondiciones}{
		Ninguna.
	}
	  
	\UCitem{Destino}{
		\refIdElem{ADMN-IU02.1}
		
	}

	\UCitem{Viene de}{
		Caso de uso \refIdElem{ADMN-CU02}.
	}	
\end{UseCase}

%Trayectoria Principal
\begin{UCtrayectoria}
	\UCpaso [\UCactor] Da clic en el ícono \IUEditar{} desde la interfaz \refElem{ADMN-IU02} en el recuadro del colaborador que desea eliminar.
	\UCpaso [\UCsist] Obtiene el nombre, primer apellido, segundo apellido, cargo que desempeña.
	\UCpaso [\UCsist] Muestra la información obtenida en la interfaz \refElem{ADMN-IU02.3}.
	\UCpaso [\UCactor] \label{ADMNCU02-3-1} Actualiza el nombre, primer apellido, segundo apellido y/o cargo que desempeña en ESCOM.
	\UCpaso [\UCactor] Presiona el botón \IUbutton{Aceptar}.\refTray{A}
	\UCpaso Verifica que los campos marcados como obligatorios hayan sido ingresados con blase ne la regla de negocio \refIdElem{RN-N001}.\refTray{B}
	\UCpaso Verifica que el formato de los campos ingresados sea correcto con blase ne la regla de negocio \refIdElem{RN-N003}.\refTray{C}
    \UCpaso [\UCsist] Muestra la información actualizada en interfaz \refElem{ADMN-IU02}.
	
\end{UCtrayectoria}

\begin{UCtrayectoriaA}[Fin del caso de uso]{A}{El actor no desea registrar un nuevo colaborador}
	\UCpaso [\UCsist] Presiona el botón \IUbutton{Cancelar}.
	\extendUC{ADMN-CU02}.
\end{UCtrayectoriaA} 

\begin{UCtrayectoriaA}[Fin de la trayectoria]{B}{El actor no ingresó uno o más datos obligatorios.}
	\UCpaso [\UCsist] Muestra el mensaje \refIdElem{MSG4} en los campos que no fueron ingresados.
	\UCpaso [\UCsist] Continúa en el paso \ref{ADMNCU02-3-1} de la trayectoria principal.
\end{UCtrayectoriaA} 

\begin{UCtrayectoriaA}[Fin de la trayectoria]{C}{El actor ingresó el dato con formato incorrecto.}
	\UCpaso [\UCsist] Muestra el mensaje \refIdElem{MSG2}.
	\UCpaso [\UCsist] Continúa en el paso \ref{ADMNCU02-3-1} de la trayectoria principal.
\end{UCtrayectoriaA}



