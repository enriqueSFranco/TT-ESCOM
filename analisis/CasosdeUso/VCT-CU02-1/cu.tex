\begin{UseCase}[]{VCT-CU02-1}{Consultar vacante}{
	Permite al actor consultar la infomación completa de una vacante publicada en el sistema.
}
	%----------------------------------------------------------------
	% Datos generales del CU:
	\UCsection{Atributos}
	\UCitem{Actor(es)}{
		Cualquier persona que pueda acceder al sistema.

	}
	\UCitem[admin]{Prioridad}{
		Media
	}
	\UCitem[admin]{Complejidad}{
		Alta
	}
	\UCitem{Precondiciones}{
		Ninguna
	}
	\UCitems{Salidas}{
		\imprimeUC{salida}
	}
	\UCitem{Destino}{
		\Titem \refElem{VCT-IU02}
	}
	\UCitem{Reglas de Negocio}{
		Ninguna
		
	}
	\UCitem{Viene de}{
		%Caso de uso \refIdElem{GRL-CU03}.
	}	
\end{UseCase}

%Trayectoria Principal
\begin{UCtrayectoria}
	\UCpaso [\UCactor] Seleciona una vacante desde la interfaz \refElem{VCT-IU02}.
	\UCpaso [\UCsist] Optiene el \salida[puesto]{vacante.puesto}, 
	\salida[empresa]{empresa.nombre}, 
	\salida[salario minímo]{vacante.salariomin},
	\salida[salario maximo]{vacante.saraliomax},
	la \salida[ubicación]{vacante.ubicacion},
	los \salida[días]{vacante.diaslab} de trabajo,
	el \salida[horario de entrada]{vacante.horaentra} y \salida[horario de salida]{vacante.horasal},
	la \salida[descripción]{vacante.descripcion}  de la vacante.
	\UCpaso [\UCsist] Muestra la información optenida en el componente \refElem{VCT-IU02-3} en el lado izquierdo de la interfaz \refElem{VCT-IU02}
\end{UCtrayectoria}

