\clearpage
\subsection{VCT-IU02-2 Consultar vacante}

\subsubsection{Objetivo}
En la figura \refElem{VCT-IU02} se muestra la interfaz correspondiente con la funcionalidad descrita en las
trayectorias del caso de uso \refElem{VCT-CU2} , la cual permite al actor registrado o no dentro del sistema consultar 
las vacantes publicadas.

La interfaz esta dividida en dos ``secciones'':
\begin{itemize}
   \item \textbf{Listado de vacantes}: esta sección tiene como objetivo listar cada una de las vacantesque hay en el sistema.
   \item \textbf{Consultar de vacante}: esta sección tiene como objetivo mostrar la información de una sola vacante, la cual se explica en 
   la interfaz \refIdElem{VCT-IU02-3}.
\end{itemize}

\IUfig{.9}{analisisydiseno/CasosdeUso/VCT-CU2/imagenes/VCT-IU02.png}{VCT-IU02}{Listar vacantes} 

\subsubsection{Consulta de vacante}
Cada vacante al selecionarla para su consulta se muestra por un componente llamando ``card'' el cual se puede ver en la figura \refElem{GRL-IU02-2}, la informacion que 
muestra es la sigueinte: 
\begin{itemize}
   \item Titulo de la vacante (1).
   \item Nombre de la empresa (2).
   \item Salario de la vacante (3).
   \item Ubicación de la vacante (4).
   \item Icono para reportar la vacante (5).
   \item Horario laboral de la vacante (6).
   \item Botón para postularse a la vacante (7).
   \item Descripción de la vacante (8).
\end{itemize}

\IUfig{.9}{analisisydiseno/CasosdeUso/VCT-CU2/imagenes/GRL-IU02-3.png}{VCT-IU02-3}{Consulta de una vacante} 



\clearpage
