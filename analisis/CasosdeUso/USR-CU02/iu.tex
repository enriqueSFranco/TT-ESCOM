\clearpage
\subsection{USR-IU02 Consultar perfil}

\subsubsection{Objetivo}
En la figura \refElem{USR-IU02} se muestra la interfaz correspondiente con la funcionalidad descrita en las
trayectorias del caso de uso \refElem{USR-CU02} , la cual permite al actor la gestión su perfil y la consulta del mismo.

La interfaz esta compuesta ``secciones'' y cada sección corresponde a un formulario diferente, las secciones
son las siguientes:
\begin{itemize}
   \item \textbf{Datos personales}: esta sección tiene como objetivo que el actor actualice su información
   personal,sus datos de contacto y/o habilidades que posee (ver la figura \refElem{USR-IU02a}).
   \item \textbf{Objetivos y metas personales}: esta sección tiene como objetivo que el actor actualice sus objetivos, metas personales y laborales 
   (ver la figura \refElem{USR-IU02b}).
   \item \textbf{Historial académico}: esta sección tiene como objetivo que el actor actualice su información
   académica referente a todos los grados de estudios que tiene hasta la fecha (ver la figura \refElem{USR-IU02c}).
   \item \textbf{Idiomas}: esta sección tiene como objetivo preguntarle que el actor actualice la información de idiomas o en su caso, elimine
   o agregue nuevos idiomas a su perfil  (ver la figura \refElem{USR-IU02d}).
   \item \textbf{Experiencia laboral}: esta sección tiene como objetivo que el actor actualice su información de su experiencia
   laborar que tiene hasta la fecha (ver la figura \refElem{USR-IU02e}).
   \item \textbf{Cursos/Certificaciones}:  esta sección tiene como objetivo que el actor actualice su información de sus Certificaciones
   o cursos que ha tenido durante toda su trayectoria académica y laboral (ver la figura \refElem{USR-IU02f}).
\end{itemize}

\subsubsection{Comandos}
Los siguientes comandos aparecen durante toda la interfaz es decir, cada sección los tiene.

%\Titem \IUPass : Al da clic en el ícono, se muestra la contraseña de lo contrario aparecerá \IUOculto \thinspace sustituyendo cada caracter de la contraseña. \\

\Titem \IUEditar{} : Cuando presiona el ícono, habilita la sección para hacer los datos editables acorde a la sección o elemento seleccionado.
\Titem \IUEliminar{} : Cuando presiona el ícono, habilita la sección para eliminar el elemento seleccionado.
\Titem \IUAgregar{} : Cuando presiona el ícono, habilita la sección para agregar un nuevo el elemento.

\IUfig{.9}{CasosdeUso/USR-CU02/imagenes/USR-IU02.png}{USR-IU02}{Consultar perfil}  
\IUfig{.5}{CasosdeUso/USR-CU02/imagenes/USR-IU02a.png}{USR-IU02a}{Consultar perfil: Datos personales}
\IUfig{.9}{CasosdeUso/USR-CU02/imagenes/USR-IU02b.png}{USR-IU02b}{Consultar perfil: Objetivos y metas personales}  
\IUfig{.9}{CasosdeUso/USR-CU02/imagenes/USR-IU02c.png}{USR-IU02c}{Consultar perfil: Historial académico}  
\IUfig{.9}{CasosdeUso/USR-CU02/imagenes/USR-IU02d.png}{USR-IU02d}{Consultar perfil: Idiomas}
\IUfig{.9}{CasosdeUso/USR-CU02/imagenes/USR-IU02e.png}{USR-IU02e}{Consultar perfil: Experiencia laboral}  
\IUfig{.9}{CasosdeUso/USR-CU02/imagenes/USR-IU02f.png}{USR-IU02f}{Consultar perfil: Cursos/Certificaciones}  


\clearpage
