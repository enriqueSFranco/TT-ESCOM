\begin{UseCase}[]{VCT-CU02}{Listar vacantes}{
	Permite al actor consultar la información general de todas las vacantes registradas en el sistema.
	}
	%----------------------------------------------------------------
	% Datos generales del CU:
	\UCsection{Atributos}
	\UCitem{Actor(es)}{
		Cualquier persona que pueda acceder al sistema.

	}
	\UCitem[admin]{Prioridad}{
		Media.
	}
	\UCitem[admin]{Complejidad}{
		Alta.
	}
	\UCitem{Precondiciones}{
		La vacante no debe de estar en estado de \textbf{cerrada}.
	}
	\UCitem{Destino}{
		\refElem{VCT-IU01}
	}
	\UCitem{Viene de}{
		Caso de uso \refIdElem{GRL-CU03}.
	}	
\end{UseCase}

%Trayectoria Principal
\begin{UCtrayectoria}
	\UCpaso [\UCactor] Presiona el botón \IUbutton{Vacantes} desde la interfaz \refIdElem{GRL-IU03}.
    \UCpaso [\UCsist] Verifica que exista al menos un registro de una vacante que no se encuentre es estado \textbf{cerrada} en la base de datos.\refTray{A}
	\UCpaso [\UCsist] Obtiene el titulo, empresa, salario mínimo, salario máximo,ubicación y la descripción  de las vacantes.
	\UCpaso [\UCsist] Lista la información obtenida en la parte derecha de la interfaz \refElem{VCT-IU01}
\end{UCtrayectoria}

%Trayectorias Alternativas
\begin{UCtrayectoriaA}[Fin del caso de uso]{A}{No hay vacantes registradas en el sistema.}
	\UCpaso [\UCsist] Muestra la interfaz \refElem{VCT-IU02-2}.
\end{UCtrayectoriaA}

