\clearpage
\subsection{PST-IU01 Consultar postulaciones}

\subsubsection{Objetivo}
En la figura \refElem{PST-IU01} se muestra la interfaz correspondiente con la funcionalidad descrita en las
trayectorias del caso de uso \refElem{PST-CU01} , la cual permite registrado como reclutador pueda consultar las postulaciones de sus vacantes que el haya publicado.
\IUfig{.8}{CasosdeUso/PST-CU01/imagenes/PST-IU01.jpeg}{PST-IU01}{Consultar postulacioness} 

La interfaz \refElem{VCT-IU03} esta dividida en dos ``secciones'':
\begin{itemize}
    \item \textbf{Resumen de la vacante}: esta sección tiene como objetivo dar un pequeño resumen de la información más sobresaliente de la vacante vacante. (Ver figura \refElem{PST-IU01a})
   \item \textbf{Listado de postulaciones}: esta sección tiene como objetivo listar cada una de las vacantes que hay registradas en el sistema. (Ver figura \refElem{PST-IU01b})
\end{itemize}

\IUfig{.5}{CasosdeUso/PST-CU01/imagenes/PST-IU01a.jpeg}{PST-IU01a}{Resumen de la vacante} 
\IUfig{.8}{CasosdeUso/PST-CU01/imagenes/PST-IU01b.jpeg}{PST-IU01b}{Listado de postulaciones} 

\subsubsection{Acciones}
En la columna ``Acciones'' del componente \refElem{PST-IU01b} se pueden ver dos iconos, los cuales hacen referencia a dar seguimiento y/o descartar una postulación