\clearpage
\begin{UseCase}[]{PST-CU01-2}{Descartar postulación}{
	Permite al reclutador de una empresa dar seguimiento a una postulación recibida de un candidato.
	}
	%----------------------------------------------------------------
	% Datos generales del CU:
	\UCsection{Atributos}
	\UCitem{Actor(es)}{
		Reclutadores.
	}
	\UCitem[admin]{Prioridad}{
		Media
	}
	\UCitem[admin]{Complejidad}{
		Alta
	}
	\UCitem{Precondiciones}{
		\Titem El reclutador debe de estar registrado en el sistema.
		\Titem La vacante debe de tener al menos una postulación asociada.
	}
	\UCitem{Destino}{
		\Titem \refElem{PST-IU01}
	}
	\UCitem{Reglas de Negocio}{
		Ninguna.
		
	}
	\UCitem{Viene de}{
		\refElem{PST-CU01}.
	}	
\end{UseCase}

%Trayectoria Principal
\begin{UCtrayectoria}
	\UCpaso [\UCactor] Da clic en el \IUbutton{Descartar} de la postulación que desee, en la interfaz \refElem{VCT-IU03}.
	\UCpaso [\UCsist] Cambia el estado de la postulación a ``Descartada'', con base en la \refElem{mq:post}.
	\UCpaso [\UCsist] Notifica al candidato que su postulación ha sido descartada.
	\UCpaso [\UCsist] Habilitaa los botones de acciones en el componenete \refElem{VCT-IU03b} de la postulaciones descartada, con base en la \refElem{mq:post}.
	\UCpaso [\UCsist] Muestra la pantalla \refElem{PST-IU01}. \refTray{A}\label{PST-CU01:1}

\end{UCtrayectoria}

%Trayectorias Alternativas
%\begin{UCtrayectoriaA}[Fin del caso de uso]{A}{El actor no tiene vacantes registradas en el sistema.}
%	\UCpaso [\UCsist] Muestra la interfaz \refElem{VCT-IU03-1}.
%\end{UCtrayectoriaA} 
