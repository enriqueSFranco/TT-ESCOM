\begin{UseCase}[]{USR-CU05}{Consultar Postulaciones}{
	Permite al alumno consultar la información y el estatus de todas las postulaciones que ha realizado.
}
	%----------------------------------------------------------------
	% Datos generales del CU:
	\UCsection{Atributos}
	\UCitem{Actor(es)}{
		\Titem Candidato. 
	}
	\UCitem[admin]{Prioridad}{ 
		\Titem Alta
	}
	\UCitem[admin]{Complejidad}{
		\Titem Media
	}
	\UCitem{Precondiciones}{
		\Titem El alumno debe de tener una cuenta en el sistema.
	}
	\UCitem{Destino}{
		\Titem \refElem{USR-IU05}
	}
	\UCitem{Viene de}{
		\Titem Caso de uso \refIdElem{USR-CU01}.
	}	
\end{UseCase}

%Trayectoria Principal
\begin{UCtrayectoria}
	\UCpaso [\UCactor] Presiona el botón \IUbutton{Postulaciones} desde la interfaz \refElem{USR-IU02}.
    \UCpaso [\UCsist] Obtiene el nombre de la vacante, nombre de la empresa, el estatus  y la fecha en la que se realizó cada una de las postulaciones asociadas al usuario.
	\UCpaso [\UCsist] Muestra la interfaz \refElem{USR-IU05}.\refTray{A}
\end{UCtrayectoria}

\begin{UCtrayectoriaA}[Fin del caso de uso]{A}{El actor desea consultar vacantes}
	\UCpaso [\UCsist] Presiona el botón \IUbutton{Vacantes}.
	\extendUC{VCT-CU01}.
\end{UCtrayectoriaA} 