\clearpage
\begin{UseCase}[]{PST-CU01}{Consultar postulaciones}{
	Permite al reclutador de una empresa  consultar las postulaciones que tenga de una vacante que haya publicado ya se ha que hayan sido recientemente publicadas 
	o ya las haya cerrado.
	}
	%----------------------------------------------------------------
	% Datos generales del CU:
	\UCsection{Atributos}
	\UCitem{Actor(es)}{
		Reclutadores.
	}
	\UCitem[admin]{Prioridad}{
		Media
	}
	\UCitem[admin]{Complejidad}{
		Alta
	}
	\UCitem{Precondiciones}{
		\Titem El reclutador debe de estar registrado en el sistema.
	}
	\UCitem{Destino}{
		\Titem \refElem{PST-IU01}
	}
	\UCitem{Reglas de Negocio}{
		Ninguna.
		
	}
	\UCitem{Viene de}{
		\refElem{VCT-CU03} trayectoria A.
	}	
\end{UseCase}

%Trayectoria Principal
\begin{UCtrayectoria}
	\UCpaso [\UCactor] Da clic en el componente \refElem{VCT-IU03a} de la vacante que quiera consultar, en la interfaz \refElem{VCT-IU03}.
	\UCpaso [\UCsist] Obtiene el número de postulaciones recibidas de esa vacante, el número de postulaciones en seguimiento, el número de postulaciones
	descartadas y el número de postulaciones que aun no ha revisado, con base en la \refElem{mq:post}.
	\UCpaso [\UCactor] Da clic en el botón \IUbutton{Ver postulaciones}
	\UCpaso [\UCsist] Obtiene el nombre, primer y segundo apellido, carrera, habilidades técnicas y habilidades blandas y la
	información de contacto de cada candidato que se haya postulado a la vacante seleccionada.
	\UCpaso [\UCsist] Muestra la pantalla \refElem{PST-IU01} con la información obtenida. \refTray{A} \refTray{B} \refTray{C}\label{PST-CU01:1}

\end{UCtrayectoria}

%Trayectorias Alternativas
%\begin{UCtrayectoriaA}[Fin del caso de uso]{A}{El actor no tiene vacantes registradas en el sistema.}
%	\UCpaso [\UCsist] Muestra la interfaz \refElem{VCT-IU03-1}.
%\end{UCtrayectoriaA} 

%Trayectorias Alternativas
\begin{UCtrayectoriaA}[Fin de la trayectoria]{A}{El actor quiere consultar la información de un candidato.}
	\UCpaso [\UCactor] Da clic en el nombre del candidato que quiera consultar, en la interfaz \refElem{PST-IU01} 
	\UCpaso [\UCsist] Muestra la interfaz \refElem{PST-IU01a} con la información correspondiente al candidato.
\end{UCtrayectoriaA} 

\begin{UCtrayectoriaA}[Fin del caso de uso]{B}{El actor quiere dar seguimiento a una postulación.}
	\extendUC{PST-CU01-1}.
\end{UCtrayectoriaA} 

\begin{UCtrayectoriaA}[Fin del caso de uso]{C}{El actor quiere descartar una postulación.}
	\extendUC{PST-CU01-2}.
\end{UCtrayectoriaA} 

\subsubsection{Puntos de extensión}

\UCExtensionPoint{Aceptar postulación}{El actor requiere quiere dar seguimiento a una postulación}{En el paso \ref{PST-CU01:1} de la trayectoria principal}{\refIdElem{GRL-CU01}}
\UCExtensionPoint{Rechazar postulación}{El actor quiere descartar una postulación}{En el paso \ref{PST-CU01:1} de la trayectoria principal}{\refIdElem{GRL-CU02}}
