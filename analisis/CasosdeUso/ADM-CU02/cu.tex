\begin{UseCase}{ADMN-CU02}{Gestionar colaboradores de la plataforma}{
	Permite al actor Consultrar, registrar y/o eliminar un nuevo colaborador en el sistema.
}
	%----------------------------------------------------------------
	% Datos generales del CU:
	\UCsection{Atributos}
	\UCitem{Actor(es)}{
		Encargado del sistema.
	}
	\UCitem{Prioridad}{
		Alta.
	}
	\UCitem{Complejidad}{
		Media.
	}
	\UCitem{Precondiciones}{
		Ninguna.
		%El correo electrónico del nuevo usuario no debe estar registrado en el sistema.
	}
	\UCitem{Postcondiciones}{
		Ninguna.
	}
	  
	\UCitem{Destino}{
		\refIdElem{ADMN-IU02}
		
	}

	\UCitem{Viene de}{
		Caso de uso \refIdElem{GRL-CU03}.
	}	
\end{UseCase}

%Trayectoria Principal
\begin{UCtrayectoria}
	\UCpaso [\UCactor] Presiona el botón \IUbutton{Colaboradores} desde la barra superior del menú.
	\UCpaso [\UCsist] Obtiene nombre, primer apellido, segundo apellido y el cargo que desempeña den la ESCOM de todos los colaboradores registrados en el sistema. 
    \UCpaso [\UCsist] Muestra la información obtenida en interfaz \refElem{ADMN-IU02}.
	
\end{UCtrayectoria}

%Trayectorias Alternativas
\begin{UCtrayectoriaA}[Fin de la trayectoria]{A}{El actor desea registrar un nuevo colaborador.}
	\UCpaso [\UCactor] Da clic en el ícono \IUAgregar{}.
	\extendUC{ADMN-CU02.1}.
\end{UCtrayectoriaA} 

%Trayectorias Alternativas
\begin{UCtrayectoriaA}[Fin de la trayectoria]{B}{El actor desea eliminar un coloborador.}
	\UCpaso [\UCactor] Da clic en el ícono \IUEliminar{}.
	\extendUC{ADMN-CU02.2}.
\end{UCtrayectoriaA} 
