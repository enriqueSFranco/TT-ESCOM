\begin{UseCase}[]{USR-CU04.2}{Editar historial académico}{
	Permite al candidato editar alguna carrera/escuela  que haya cursado y que esté previamiente registrado en el usuario perfil.
}
	%----------------------------------------------------------------
	% Datos generales del CU:
	\UCsection{Atributos}
	\UCitem{Actor(es)}{
		Candidato. 
	}
	\UCitem[admin]{Prioridad}{ 
		Media
	}
	\UCitem[admin]{Complejidad}{
		Media
	}
	\UCitem{Precondiciones}{
		\Titem El candidato debe de tener una cuenta en el sistema.
	}
	\UCitem{Destino}{
		\Titem \refElem{USR-IU02}
	}
	\UCitem{Viene de}{
		Caso de uso \refIdElem{GRL-CU04}.
	}	
\end{UseCase}

%Trayectoria Principal
\begin{UCtrayectoria}
	\UCpaso [\UCactor] Presiona el ícono \IUEditar{} en la sección en la interfaz \refIdElem{USR-IU4}.
	\UCpaso Obtiene el nombre de la carrera, la institución, nivel académico, estado académico, fecha de inicio y la fecha en la que se planea terminar o terminó la carrera del registro que se seleccionó.
	\UCpaso Muestra información obtenida en la interfaz \refIdElem{USR-IU4.2}.
	\UCpaso [\UCsist] \label{USRCU04-2} Edita el nombre de la carrera, la institución, nivel académico, estado académico, fecha de inicio y la fecha en la que se planea terminar o terminó la carrera.
	\UCpaso [\UCsist] Presiona el botón \IUbutton{Aceptar}.\refTray{A}
	\UCpaso Verifica que los campos marcados como obligatorios hayan sido ingresados con blase ne la regla de negocio \refIdElem{RN-N001}.\refTray{B}
	\UCpaso Verifica que el formato de los campos ingresados sea correcto con blase ne la regla de negocio \refIdElem{RN-N003}.\refTray{C}
	\UCpaso Actualiza la información ingresada en el sistema.
	\UCpaso Muestra la información actualizada en la interfaz \refIdElem{USR-IU2}. 
\end{UCtrayectoria}

\begin{UCtrayectoriaA}[Fin del caso de uso]{A}{El actor no desea agregar una historial academico}
	\UCpaso [\UCsist] Presiona el botón \IUbutton{Cancelar}.
	\extendUC{USR-CU02}.
\end{UCtrayectoriaA} 

\begin{UCtrayectoriaA}[Fin de la trayectoria]{B}{El actor no ingresó uno o más datos obligatorios.}
	\UCpaso [\UCsist] Muestra el mensaje \refIdElem{MSG4} en los campos que no fueron ingresados.
	\UCpaso [\UCsist] Continúa en el paso \ref{USRCU04-2} de la trayectoria principal.
\end{UCtrayectoriaA} 

\begin{UCtrayectoriaA}[Fin de la trayectoria]{C}{El actor ingresó el dato con formato incorrecto.}
	\UCpaso [\UCsist] Muestra el mensaje \refIdElem{MSG2}.
	\UCpaso [\UCsist] Continúa en el paso \ref{USRCU04-2} de la trayectoria principal.
\end{UCtrayectoriaA}



