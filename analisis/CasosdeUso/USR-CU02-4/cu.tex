\begin{UseCase}[]{USR-CU02.4}{Editar Objetivos profesionales}{
	Permite al candidato editar su información de presentación, salario deseado y el trabajo que desea en su perfil.
}
	%----------------------------------------------------------------
	% Datos generales del CU:
	\UCsection{Atributos}
	\UCitem{Actor(es)}{
		Candidato. 
	}
	\UCitem[admin]{Prioridad}{ 
		Alta
	}
	\UCitem[admin]{Complejidad}{
		Media
	}
	\UCitem{Precondiciones}{
		\Titem El candidato debe de tener una cuenta en el sistema.
	}
	\UCitem{Destino}{
		\Titem \refElem{USR-IU2.4}
	}
	\UCitem{Viene de}{
		Caso de uso \refIdElem{GRL-CU03}.
	}	
\end{UseCase}

%Trayectoria Principal
\begin{UCtrayectoria}
	\UCpaso [\UCactor] Presiona el ícono \IUEditar{} en la sección ``Objetivos profesionales'' en la interfaz \refIdElem{USR-IU2}. 
	\UCpaso Obtiene el objetivo, el salario deseado y el trabajo deseado que se tiene registrados actualmente en el sistema.
	\UCpaso [\UCsist] Muestra la información obtenida en la interfaz \refIdElem{USR-IU2.4}.
	\UCpaso \label{USRCU01-4-1} Modifica el objetivo, el salario deseado y/o el trabajo deseado.
	\UCpaso [\UCsist] Presiona el botón \IUbutton{Aceptar}.\refTray{A}
	\UCpaso Verifica que los campos marcados como obligatorios hayan sido ingresados con blase ne la regla de negocio \refIdElem{RN-N001}.\refTray{B}
	\UCpaso Verifica que el formato de los campos ingresados sea correcto con blase ne la regla de negocio \refIdElem{RN-N003}.\refTray{C}
	\UCpaso Actualiza la información ingresada en el sistema.
	\UCpaso Muestra la información actualizada en la interfaz \refIdElem{USR-IU2}. 
\end{UCtrayectoria}

\begin{UCtrayectoriaA}[Fin del caso de uso]{A}{El actor no desea editar los objetivos profesionales}
	\UCpaso [\UCsist] Presiona el botón \IUbutton{Cancelar}.
	\extendUC{USR-CU02}.
\end{UCtrayectoriaA} 

\begin{UCtrayectoriaA}[Fin de la trayectoria]{B}{El actor no ingresó uno o más datos obligatorios.}
	\UCpaso [\UCsist] Muestra el mensaje \refIdElem{MSG4} en los campos que no fueron ingresados.
	\UCpaso [\UCsist] Continúa en el paso \ref{USRCU01-4-1} de la trayectoria principal.
\end{UCtrayectoriaA} 

\begin{UCtrayectoriaA}[Fin de la trayectoria]{C}{El actor ingresó el dato con formato incorrecto.}
	\UCpaso [\UCsist] Muestra el mensaje \refIdElem{MSG2}.
	\UCpaso [\UCsist] Continúa en el paso \ref{USRCU01-4-1} de la trayectoria principal.
\end{UCtrayectoriaA}



