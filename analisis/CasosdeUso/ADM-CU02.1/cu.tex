\begin{UseCase}{ADMN-CU02.1}{Eliminar  Colaborador}{
	Permite al actor eliminado un colaborador que está en el sistema.
}
	%----------------------------------------------------------------
	% Datos generales del CU:
	\UCsection{Atributos}
	\UCitem{Actor(es)}{
		Encargado del sistema.
	}
	\UCitem{Prioridad}{
		Alta.
	}
	\UCitem{Complejidad}{
		Media.
	}
	\UCitem{Precondiciones}{
		Ninguna.
		%El correo electrónico del nuevo usuario no debe estar registrado en el sistema.
	}
	\UCitem{Postcondiciones}{
		Ninguna.
	}
	  
	\UCitem{Destino}{
		\refIdElem{ADMN-IU02.1}
		
	}

	\UCitem{Viene de}{
		Caso de uso \refIdElem{ADMN-CU02}.
	}	
\end{UseCase}

%Trayectoria Principal
\begin{UCtrayectoria}
	\UCpaso [\UCactor] Da clic en el ícono \IUEliminar{} desde la interfaz \refElem{ADMN-IU02} en el recuadro del colaborador que desea eliminar.
	\UCpaso [\UCsist] Muestra la interfaz \refElem{ADMN-IU02.2}.
	\UCpaso [\UCactor] Presiona el botón \IUbutton{Aceptar}.\refTray{A}
	\UCpaso Elimina el registro del sistema junto con su accesos.
	\UCpaso Envía un mensaje por correo electrónico indicando que su cuenta ha sido eliminada.
    \UCpaso [\UCsist] Muestra la información actualizada en interfaz \refElem{ADMN-IU02}.
	
\end{UCtrayectoria}

\begin{UCtrayectoriaA}[Fin del caso de uso]{A}{El actor no desea registrar un nuevo colaborador}
	\UCpaso [\UCsist] Presiona el botón \IUbutton{Cancelar}.
	\extendUC{ADMN-CU02}.
\end{UCtrayectoriaA} 

