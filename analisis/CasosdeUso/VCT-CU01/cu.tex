\begin{UseCase}[]{VCT-CU01}{Buscar vacantes}{
	Permite al actor buscar las vacantes registradas en el sistema por medio de los filtros que tiene la interfaz o
	ingresar palabras clave en los campos de busqueda.
	}
	%----------------------------------------------------------------
	% Datos generales del CU:
	\UCsection{Atributos}
	\UCitem{Actor(es)}{
		Cualquier persona que pueda acceder al sistema.

	}
	\UCitem[admin]{Prioridad}{
		Media
	}
	\UCitem[admin]{Complejidad}{
		Alta
	}
	\UCitem{Precondiciones}{
		La vacante el estado de la vacante de \textbf{abierta} con base en la \refElem{mq:vacante}.
	}
	\UCitem{Destino}{
		\Titem \refElem{VCT-IU01}
	}
	\UCitem{Reglas de Negocio}{
		Ninguna.
		
	}
	\UCitem{Viene de}{
		Primario.
	}	
\end{UseCase}

%Trayectoria Principal
\begin{UCtrayectoria}
	\UCpaso [\UCactor] Ingresa la palabra clave de la vacane que desea buscar en la interfaz \refElem{VCT-IU01}.
	\UCpaso [\UCactor] Da clic en el botón \IUbutton{Buscar} en la interfaz \refElem{VCT-IU01}.
    \UCpaso [\UCsist] Verifica que exista almenos un registro de una vacante que no se encuente es estado \textbf{cerrada}.\refTray{A}
	\UCpaso [\UCsist] Optiene el titulo, empresa, salario minímo
	, salario maximo,ubicación y la descripción  de las vacantes que coincidan con 
	la palabra ingresada en el paso 1 de la trayectoria princial. \refTray{B}
	\extendUC{VCT-CU02}.
\end{UCtrayectoria}

%Trayectorias Alternativas
\begin{UCtrayectoriaA}[Fin del caso de uso]{A}{No hay vacantes en estado abierta.}
	\UCpaso [\UCsist] Muestra la interfaz \refElem{GRL-IU02-1}.
\end{UCtrayectoriaA}

%Trayectorias Alternativas
\begin{UCtrayectoriaA}[Fin del caso de uso]{A}{No hay vacantes registradas en el sistema que coincidan con la palabra clave.}
	\UCpaso [\UCsist] Muestra la interfaz \refElem{GRL-IU02-1}.
\end{UCtrayectoriaA}
