\clearpage
\begin{UseCase}[]{VCT-CU03}{Consultar vacantes}{
	Permite al reclutador de una empresa  consultar las vacantes que haya publicado ya seha que hayan sido recientemente publicadas 
	o ya las haya cerrado.
	}
	%----------------------------------------------------------------
	% Datos generales del CU:
	\UCsection{Atributos}
	\UCitem{Actor(es)}{
		Reclutadores.

	}
	\UCitem[admin]{Prioridad}{
		Media.
	}
	\UCitem[admin]{Complejidad}{
		Alta.
	}
	\UCitem{Precondiciones}{
		El reclutador debe de estar registrado en el sistema.
	}
	\UCitem{Destino}{
		\refElem{VCT-IU03}
	}
	\UCitem{Viene de}{
		\refElem{VCT-CU03}
	}	
\end{UseCase}

%Trayectoria Principal
\begin{UCtrayectoria}
	\UCpaso [\UCactor] Da clic en el botón \IUbutton{Vacantes} en la interfaz de inicio del reclutador
	\UCpaso [\UCsist] Verifica que el usuario tenga al menos una vacante registrada en el sistema.
	\UCpaso [\UCsist] Obtiene el título y el número de plazas de la vacante, ingresa el código postal, estado, municipio y colonia donde 
	se va labora, de cada vacante registrada por el reclutador.
	\UCpaso [\UCsist] Obtiene el perfil, la experiencia y el tipo de contratación a la que la vacante va dirigida,de cada vacante registrada por el reclutador.
	\UCpaso [\UCsist] Obtiene el horario laboral y el rango salarial indicando si es neto o no el salario de cada vacante registrada por el reclutador.
	\UCpaso [\UCsist] Muestra la información obtenida en la interfaz \refElem{VCT-IU03}. \refTray{A}
\end{UCtrayectoria}

%Trayectorias Alternativas
\begin{UCtrayectoriaA}[Fin de la trayectoria]{A}{El actor quiere consultar la información de una vacante.}
	\UCpaso [\UCactor] Da clic en el componente \refElem{VCT-IU03a} de la vacante que quiera consultar, en la interfaz \refElem{VCT-IU03} 
	\UCpaso [\UCsist] Muestra el componente \refElem{VCT-IU03b} en la interfaz \refElem{VCT-IU03}.
\end{UCtrayectoriaA} 