\clearpage
\subsection{VCT-IU03 Consultar vacantes}

\subsubsection{Objetivo}
En la figura \refElem{VCT-IU03} se muestra la interfaz correspondiente con la funcionalidad descrita en las
trayectorias del caso de uso \refElem{VCT-CU03} , la cual permite registrado como reclutador pueda consultar las vacantes que el haya publicado.


La interfaz \refElem{VCT-IU03} esta dividida en tres ``secciones'':
\begin{itemize}
    \item \textbf{Resumes de postulaciones}: esta sección tiene como objetivo dar un pequeño resumen de las postulaciones que tenga esta vacante.
   \item \textbf{Listado de vacantes}: esta sección tiene como objetivo listar cada una de las vacantes que hay registradas en el sistema.
   \item \textbf{Consulta de vacante}: esta sección tiene como objetivo mostrar la información de una sola vacante.
\end{itemize}
\IUfig{.9}{CasosdeUso/VCT-CU03/imagenes/VCT-IU3.jpeg}{VCT-IU03}{Consultar vacantes} 

\subsubsection{Listado de vacantes}
Cada vacante listada se muestra por un componente llamando ``contenedor'' el cual se puede ver en la figura \refElem{VCT-IU03a}, la información que 
muestra es la siguiente: 
\begin{itemize}
   \item Titulo de la vacante (1).
   \item Logo y nombre de la empresa (2).
   \item Salario de la vacante (3).
   \item Ubicación de la vacante (4).
   \item Idioma obligatorio (5): este dato es opcional siempre y cuando el reclutador no haya indicado que es requisito obligatorio 
   para la vacante.
   \item Estado de la publicación (si tiene menos de 3 día publicada) (6).
   \item Número de plazas que tiene la vacante (7).
   \item Dias que lleva publicada la vacante (8).
\end{itemize}
\IUfig{.9}{CasosdeUso/VCT-CU02/imagenes/VCT-IU02-2.png}{VCT-IU03a}{Listar vacantes} 

\subsubsection{Consultar de vacante}
Cada vacante al ser seleccionada para su consulta se muestra por un componente llamando ``card'' el cual se puede ver en la figura \refElem{VCT-IU03b}, la información que 
muestra es la siguiente: 
\begin{itemize}
   \item Titulo de la vacante (1).
   \item Nombre de la empresa (2).
   \item Salario de la vacante (3).
   \item Ubicación de la vacante (4).
   \item Icono para reportar la vacante (5).
   \item Horario laboral de la vacante (6).
   \item Botón para postularse a la vacante (7).
   \item Descripción de la vacante (8).
\end{itemize}

\IUfig{.9}{CasosdeUso/VCT-CU02/imagenes/VCT-IU02-3.png}{VCT-IU03b}{Consulta de una vacante}




