% Actualizar para indicar cual es el archivo raíz a compilar.
%!TEX root = ../../main-ejemplo.tex


\begin{UseCase}[%
	Autor/Ulises Vélez Saldaña,%
	Version/0.1,% 			
	Estado/\ucstEnEdicion%	% \ucstEnEdicion, \ucstEnRevision, \ucstEnAprobacion, \ucstAprobado.
]{%
	RE-CU17%
}{
	Inscribirse a un Seminario de titulación
}{
	Un alumno que está por terminar la carrera podrá inscribirse a un seminario de tesis como una de las opciones de titulación. En el caso de uso el sistema verificará si el alumno cumple con los requisitos y le mostrará los seminarios disponibles y compatibles con su carreta. El alumno podrá seleccionar el que mas le guste y el sistema le apartará su lugar en el seminario por 5 días en lo que realiza y registra el pago correspondiente.
}
	% Datos generales del CU:
	\UCitem{Actor}{
		\refElem{Alumno}
	}
	\UCitems{Propósito}{
		\Titem Facilitar la inscripción a los seminarios de titulación.
		\Titem Dar cumplimiento a la opción de titulación por seminario de tesis.
		\Titem Que el alumno logre su titulación mediante un seminario de tesis.
	}
	\UCitem[control]{Operación}{% Alta, Baja, Cambio, Consulta, Reporte, Negocio. 
		Negocio
	}
	\UCitem[control]{Revisor}{
		Luis Enrique Hernandez Olvera
	}
	\UCitem[control]{Último cambio}{
		jueves 11 de Julio del 2019
	}
	
%% BEGIN-BLOQUE REVISION ------------------------------------->
%	\UCsection[control]{
%		Version
%	}
%	\UCitem[control]{Revisó}{
%		\TODO Especificar
%	}
%	\UCitem[control]{Fecha}{
%		\TODO Especificar
%	}
%	\UCitem[control]{Resultado}{% Pendiente o Aprobado.
%		\TODO Especificar
%	}
%	\UCitems[control]{Observaciones}{% usar \TODO %\TOCHK \DONE.
%		\Titem \TODO Agregar observaciones 
%	}
%% <------------------------------------------ END-BLOQUE PARA AGREGAR UNA REVISION
	
	%----------------------------------------------
	\UCsection{Atributos}
	\UCitems{Entradas}{
		\imprimeEntrada
	}
	\UCitems{Origen}{
		\Titem Escribir mediante el teclado.
		\Titem Selección con el mouse.
	}
	\UCitems{Salida}{
		\imprimeSalida
	}   
	\UCitems{Destino}{
		\Titem Impresora: los datos correspondientes a la \refElem{OrdenDePago}.
		\Titem Pantalla: todas las demás salidas.
	}
	
	\UCitems{Precondiciones}{
		\Titem\textbf{Restrictiva}: El estudiante debe estar activo en la universidad.
		\Titem\textbf{Restrictiva}: El estudiante debe estar en proceso de titulación.
		\Titem\textbf{Restrictiva}: Debe haber al menos un Seminario Disponible compatible con la carrera del estudiante.
		\Titem\textbf{Restrictiva}: El seminario elegido debe tener cupo para al menos un estudiante.
	}
	\UCitems{Postcondiciones}{%Directa, Colateral, Condición de término
		\Titem \textbf{Directa}: El estudiante estará registrado en el Seminario seleccionado.
		\Titem \textbf{Directa}: El lugar del estudiante se respetará por 5 días mientras no concluya su pago.
		\Titem \textbf{Directa}: El estudiante no podrá registrarse en otro seminario de titulación a menos que cancele el registro actual o lo concluya con calificación reprobatoria.
		\Titem \textbf{Colateral}: En caso de que no haya mas lugares en el seminario ya no podrán inscribirse mas estudiantes, a menos que algún estudiante cancele su registro. 
		\Titem \textbf{Condición de término}: Se imprimirá la Orden de pago.
		
	}
	\UCitems{Reglas de Negocio}{
		\Titem \refIdElem{BR-100}% Alumno activo
		\Titem \refIdElem{BR-129}% Alumno en proceso de titulación.
		\Titem \refIdElem{BR-130}% Seminarios elegibles por carrera
		\Titem \refIdElem{BR-131}% Seminarios disponibles
		\Titem \refIdElem{BR-143}% Cálculo del costo de seminario por tipo de estudiante.
		\Titem \refIdElem{BR-152}% Aplicación de impuestos a servicios generales.
	}
	\UCitems{Errores}{
		\Titem \UCerr{NUMERO}{CAUSA DEL ERROR,}{REACCION DEL SISTEMA}
	}
    \UCitem[admin]{Auditable}{
    	\Titem No.
    }                 
	\UCitems{Datos sensibles}{
    	\Titem Ninguno
	}
	\UCitem{Viene de}{
		\refElem{CU-1}
	}	
	%----------------------------------------------
	\UCsection[design]{Datos de Diseño}
	\UCitems[design]{Disparador}{% Evento + Frecuencia + Cantidad de usuarios
		\Titem \textbf{Evento}: El alumno ha concluido todas sus materias y opta por la titulación por seminario.
		\Titem \textbf{Frecuencia}: Al final de cada ciclo escolar
		\Titem \textbf{Usuarios}: Aproximadamente 45 en cada ciclo escolar.
	}
	\UCitems[design]{Casos de Prueba}{% Especificar todos los casos de prueba que pueda identificar.
	                                  % Considere:
	                                  % - Casos correctos.
	                                  % - Validación de datos.
	                                  % - Valores a la frontera.
	                                  % - Validar precondiciones.
	                                  % - Validar postcondiciones.
	                                  % - Validar Reglas de negocio.
	                                  % - Verificar efectos colaterales.
		% EJEMPLOS:
		% - CPS-1: Proporcionando todos los datos correctos para un cliente Persona moral
		% - CPS-2: Proporcionando todos los datos correctos para un cliente Persona física
		% - CPS-3: Dejando vacío uno por uno cada uno de los datos obligatorios.
		% - CPS-4: Especificar una fecha de nacimiento posterior a la del día de hoy.
		% - CPS-5: Registrar la venta especificando un precio con valor negativo.
		% - CPS-6: Registrar de una venta sin productos.
		% - CPS-7: Realizar una venta de un producto sin existencias.
		\Titem Ejecute el CU proporcionando todos los datos correctos para un estudiante elegible y con seminarios disponibles de carreras afines.
		\Titem Ejecute el CU proporcionando todos los datos correctos para un estudiante elegible y con seminarios disponibles de carreras afines y no afines.
		\Titem Ejecute el CU proporcionando todos los datos correctos para un estudiante elegible y con seminarios disponibles y no disponibles de carreras afines y no afines.
		\Titem Ejecute el CU proporcionando todos los datos correctos para un estudiante no elegible.
		\Titem Ejecute el CU proporcionando todos los datos correctos para un estudiante no elegible.
		\Titem Ejecute el CU proporcionando datos incorrectos (de tipos no válidos para las entradas).
		\Titem Ejecute el CU omitiendo datos que son marcados como requeridos.
		\Titem Ejecute el CU proporcionando todos los datos correctos para un estudiante ya inscrito en un seminario pero que no haya pagado.
		\Titem Ejecute el CU proporcionando todos los datos correctos para un estudiante ya inscrito en un seminario ya pagado.
	}
	%----------------------------------------------    
	\UCsection[admin]{Datos de Administración de Requerimiento}
	\UCitem[admin]{Prioridad}{
		Alta
	}
	\UCitems[admin]{Referencia Documental}{
		\Titem Toma de requerimientos minuta TR-23
		\Titem Reglamento interno del Instituto.
		\Titem Manual de procesos 34- Proceso de inscripción a seminario de tesis.
	}
    \UCitems[admin]{Impedimentos}{
    	\Titem Ninguno.
	}
	\UCitems[admin]{Suposiciones}{
    	\Titem Ninguna.
	}
	\UCitems[admin]{Observaciones}{
    	\Titem Ninguna.
	}
\end{UseCase}

%Trayectoria Principal
\begin{UCtrayectoria}
	\UCpaso[\UCactor] Solicita su inscripción a un seminario de titulación presionando el botón \IUbutton{Inscribir Seminario} del menú {\bf Opciones de titulación} de la pantalla \refIdElem{IU23}\label{RE-CU-17-Solicita-inscripcion}.
		\UCpaso Verifica que el Estudiante sea elegible para inscribirse al Seminario con base en las reglas \refIdElem{BR-100} y \refIdElem{BR-129}. \refErr{1}, \refErr{2}.
		\UCpaso Busca los seminarios compatibles con la carrera del estudiante y que tengan lugares disponibles con base en las reglas \refIdElem{BR-130} y \refIdElem{BR-131} \refErr{3} y \refErr{4}.
		\UCpaso Solicita los datos del seminario mediante la \refElem{IU32} con la lista de Seminarios Disponibles.
		\UCpaso[\UCactor] Selecciona el {\sc\entrada{Seminario}} en el que desea inscribirse \refTray{B}\label{RE-CU-17-SeleccionarSeminario}.
		\UCpaso Calcula el costo del Seminario basado en el costo publicado en el catálogo de cursos, los costos aplicables al alumno y los impuestos aplicables, con base en las reglas \refIdElem{BR-143} y \refIdElem{BR-152}.
		\UCpaso Despliega los datos de la {\sc\salida{OrdenDePago}} en la pantalla \refIdElem{IU33}.
		\UCpaso Pide al Estudiante que confirme la inscripción alSeminario.
		\UCpaso[\UCactor] Confirma la inscripción al Seminario.
		\UCpaso Registra al Estudiante en el seminario seleccionado por cinco días hasta que se registre el pago correspondiente.
		\UCpaso Informa que la inscripción se realizó exitosamente vía la \refElem{UI88}
		\UCpaso Imprime el recibo de pago con base en la regla \refIdElem{BR-100}.
		\UCpaso Pregunta al estudiante si desea imprimir un comprobante de la inscripción.
\end{UCtrayectoria}

%Trayectorias Alternativas
\begin{UCtrayectoriaA}[Fin del caso de uso]{A}{El actor desea eliminar elemento.}
	\UCpaso [\UCactor] Presiona el icono "quitar" de la pantalla de agregar.
	\UCpaso [\UCsist] Quita elemento de la compra.
	\UCpaso regresa al paso \ref{elemento} de la trayectoria principal.
\end{UCtrayectoriaA} 
\begin{UCtrayectoriaA}[Fin del caso de uso]{B}{El actor no igresa todos los campos}
	\UCpaso [\UCsist] Manda mensaje: "Faltan campos por llenar".
	\UCpaso regresa al paso \ref{direccion} de la trayectoria principal.
\end{UCtrayectoriaA} 

\begin{UCtrayectoriaA}[Fin del caso de uso]{C}{El actor desea cancelar la operación.}
	\UCpaso [\UCactor] Presiona el botón \IUbutton{Cancelar}
	\UCpaso [\UCsist] Manda mensaje: "Pedido cancelado no se cobrara ningún tipo de monto"
\end{UCtrayectoriaA}

\begin{UCtrayectoriaA}[Fin del caso de uso]{D}{El actor introduce dirección no válida.}
	\UCpaso [\UCsist] Manda mensaje ''Dirección no válida'' 
	\UCpaso regresa al paso \ref{nuevDir} de la trayectoria principal.
\end{UCtrayectoriaA}

\begin{UCtrayectoriaA}[Fin del caso de uso]{E}{El actor presiona botón de pagar en efectivo.} 
	\UCpaso regresa al paso \ref{confDatos} de la trayectoria principal.
\end{UCtrayectoriaA}


