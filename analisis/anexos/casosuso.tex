\chapter{Casos de uso}

%Los casos de uso se identificarán de acuerdo a la siguiente nomenclatura:

%- - - - - - - - - - - - - - - - - - - - - - - - - - - - -
\section{Especificación de casos de uso}

Para poder entender un caso de uso más allá de un diagrama, se lleva a cabo la tarea de especificar cada uno de los casos de uso identificados en el sistema con el fin de describir las secuencias de acciones que realiza el sistema y que lleva a un resultado de valor a un actor específico. Los casos de uso tienen atributos los cuales se describen a continuación:

\begin{description}
	\item[Id] Identificador del caso de uso, el cual debe ser único.
	\item[Nombre] Nombre del caso de uso el cual es descriptivo basándose en la transacción que se realiza.
	\item[Resumen] Es una descripción resumida en la que se especifica la transacción realizada por el caso de uso.
	\item[Actores] Lista de los actores que interactúan con el caso de uso.
	\item[Entradas] Lista los datos de entrada que el caso de uso recibe, los cuales harán referencia al modelo de información.
	\item[Salidas] Lista los datos de salida que el caso de uso genera.
	\item[Destino] Indica a dónde se dirigen los datos de salida, por ejemplo: interfaz, impresora, repositorio, hacia un servidor o un archivo.
	\item[Precondiciones] En lista las cosas que deben haber sucedido para que el caso  de uso se lleve a cabo.
	\item[Postcondiciones] En lista las cosas que suceden en el sistema o negocio de forma inmediata o a corto plazo una vez que se ejecute el caso de uso.
	\item[Reglas de Negocio] Lista las reglas de negocio que se van a ejecutar en el caso de uso.
	\item[Viene de] Indica cuando el caso de uso se extiende de otro o se incluye en otro.
	%\item[Casos de prueba] Indican los datos o condiciones con las que se debe probar este caso de uso.
	\item[Trayectoria principal] Secuencia de pasos que llevan al caso de uso al éxito.
	\item[Trayectoria(s) alternativa(s)] Secuencias de pasos que llevan al caso de uso al éxito o al fracaso.
	\item[Puntos de extensión] Cuando existen casos de uso que pueden ejecutarse a partir del caso de uso en proceso.
\end{description}

\section{Módulo General}
	En la figura \ref{adcu:grl} se muestra el diagrama de casos de uso del módulo general del sistema.

	\begin{figure}[hbtp!]
		\begin{center}
			\includegraphics[width=.8\textwidth]{sprints/imagenes/CUGRL.png}
		\end{center}
		
		\caption{Diagrama de casos de uso del \textit{Módulo General}.}
		\label{adcu:grl}
	\end{figure}

	\begin{itemize}
        \item Los casos de uso \IUazul{} , son aquellos que se pertenecen a esta primera entrega del proyecto.
        \item Los casos de uso \IUblanco{}, se tienen planeados para la segunda entrega del proyecto.
    \end{itemize} 

	\clearpage
\begin{UseCase}[]{VCT-CU04}{Registrar vacante}{
	Permite al reclutador de una empresa  publicar una vacante en el sistema durante cierto periodo de tiempo y así poder gestionas las postulaciones 
	que los usarios(canditos) hagan a dicha vacante.
	}
	%----------------------------------------------------------------
	% Datos generales del CU:
	\UCsection{Atributos}
	\UCitem{Actor(es)}{
		Reclutadores.

	}
	\UCitem[admin]{Prioridad}{
		Media
	}
	\UCitem[admin]{Complejidad}{
		Alta
	}
	\UCitem{Precondiciones}{
		El reclutador debe de estar registrado en el sistema.
	}
	\UCitem{Destino}{
		\Titem \refElem{VCT-IU03}
	}
	\UCitem{Reglas de Negocio}{
		\Titem \refIdElem{RN-N001}
		
	}
	\UCitem{Viene de}{
		\refElem{VCT-CU03}
	}	
\end{UseCase}

%Trayectoria Principal
\begin{UCtrayectoria}
	\UCpaso [\UCactor] Da clic en el icono \IUAgregar{} (Publicar vacante) en la interfaz \refElem{VCT-IU03}.
	\UCpaso [\UCsist] Muestra la interfaz \refElem{VCT-IU04a} en la interfaz \refElem{VCT-IU03}.
	\UCpaso [\UCactor] \label{VCT-CU04:vadata} Ingresa el título y el número de plazas de la vacante, ingresa el código postal, estado, municipio y colonia donde se va laboral.\refTray{A}
	\UCpaso [\UCactor] Ingresa el perfil, la experiencia y el tipo de contratación a la que la vacante va dirigida.
	\UCpaso [\UCactor] Ingresa el horario laboral y el rango salarial indicando si es neto o no el salario.
	\UCpaso [\UCsist] Valida que todos los campos marcados como obligatorios hayan sido ingresados de acuerdo a la regla de negocio \refIdElem{RN-N001}. \refTray{B}
	\UCpaso [\UCactor] Ingresa la descripción de la vacante.
	\UCpaso [\UCactor] \label{VCT-CU04:hab}Selecciona las habilidades y su correspondiente experiencia deseadas para la vacante.\refTray{D}
	\UCpaso [\UCactor] Selecciona la fecha de cierre de la vacante.
	\UCpaso [\UCactor] Da clic en el botón \IUbutton{Publicar} en la interfaz \refElem{VCT-IU04b}.\refTray{A}\refTray{C} \refTray{F}
	\UCpaso [\UCsist] \label{VCT-CU04:vadata2}Valida que todos los campos marcados como obligatorios hayan sido ingresados de acuerdo a la regla de negocio \refIdElem{RN-N001}.\refTray{E}
	\UCpaso [\UCsist] Muestra la interfaz \refElem{VCT-IU03} mostrando la nueva vacante registrada.
	\UCpaso [\UCsist] Notifica a los encargados y colaboradores de que hay una nueva vacante por revisar.
\end{UCtrayectoria}

%Trayectorias Alternativas
\begin{UCtrayectoriaA}[Fin de la trayectoria]{A}{El actor decide cancelar el registro.}
	\UCpaso [\UCactor] Da clic en el botón \IUbutton{Cancelar} en la interfaz \refElem{VCT-IU04a}.
	\UCpaso [\UCsist] Muestra el mensaje \refIdElem{MSG6} en la interfaz \refElem{VCT-IU04a}.
	\UCpaso [\UCactor] Da clic en el botón \IUbutton{Sí} en la interfaz \refElem{VCT-IU04a}.\refTray{G}
	\UCpaso [\UCsist] Muestra la interfaz \refElem{VCT-IU03}.
\end{UCtrayectoriaA} 

%Trayectorias Alternativas
\begin{UCtrayectoriaA}[Fin de la trayectoria]{B}{El actor no registro al menos un campo obligatorio.}
	\UCpaso [\UCsist] Muestra el mensaje \refIdElem{MSG4} en la interfaz \refElem{VCT-IU03a} en los campos que no
	fueron ingresados.
	\UCpaso [\UCsist] Continúa en el paso \ref{VCT-CU04:vadata} de la trayectoria principal.
\end{UCtrayectoriaA} 

%Trayectorias Alternativas
\begin{UCtrayectoriaA}[Fin de la trayectoria]{C}{El actor decide regresa a la pantalla anterior.}
	\UCpaso [\UCactor] Da clic en el botón \IUbutton{Regresar} en la interfaz \refElem{VCT-IU04b}.
	\UCpaso [\UCsist] Muestra la interfaz \refElem{VCT-IU04a}.
	\UCpaso [\UCsist] Continúa en el paso \ref{VCT-CU04:vadata} de la trayectoria principal.
\end{UCtrayectoriaA} 

%Trayectorias Alternativas
\begin{UCtrayectoriaA}[Fin de la trayectoria]{D}{El actor decide eliminar una habilidad.}
	\UCpaso [\UCactor] Da clic en el botón \IUbutton{x} de la habilidad seleccionada en la interfaz \refElem{VCT-IU04b}.
	\UCpaso [\UCsist] Muestra la interfaz \refElem{VCT-IU04b}.
	\UCpaso [\UCsist] Continúa en el paso \ref{VCT-CU04:hab} de la trayectoria principal.
\end{UCtrayectoriaA} 

%Trayectorias Alternativas
\begin{UCtrayectoriaA}[Fin de la trayectoria]{E}{El actor no registro al menos un campo obligatorio.}
	\UCpaso [\UCsist] Muestra el mensaje \refIdElem{MSG4} en la interfaz \refElem{VCT-IU03b} en los campos que no
	fueron ingresados.
	\UCpaso [\UCsist] Continúa en el paso \ref{VCT-CU04:vadata2} de la trayectoria principal.
\end{UCtrayectoriaA} 

%Trayectorias Alternativas
\begin{UCtrayectoriaA}[Fin de la trayectoria]{F}{El actor decide cancelar el registro.}
	\UCpaso [\UCactor] Da clic en el botón \IUbutton{Cancelar} en la interfaz \refElem{VCT-IU04b}.
	\UCpaso [\UCsist] Muestra el mensaje \refIdElem{MSG6} en la interfaz \refElem{VCT-IU04b}.
	\UCpaso [\UCactor] Da clic en el botón \IUbutton{Sí} en la interfaz \refElem{VCT-IU04b}.\refTray{H}
	\UCpaso [\UCsist] Muestra la interfaz \refElem{VCT-IU03}.
\end{UCtrayectoriaA} 

%Trayectorias Alternativas
\begin{UCtrayectoriaA}[Fin de la trayectoria]{G}{El actor decide cancelar la acción.}
	\UCpaso [\UCactor] Da clic en el botón \IUbutton{No} en la interfaz \refElem{VCT-IU04a}.
	\UCpaso [\UCsist] Muestra la interfaz \refElem{VCT-IU04a}.
	\UCpaso [\UCsist] Continúa en el paso \ref{VCT-CU04:vadata} de la trayectoria principal.
\end{UCtrayectoriaA} 

%Trayectorias Alternativas
\begin{UCtrayectoriaA}[Fin de la trayectoria]{H}{El actor decide cancelar la acción.}
	\UCpaso [\UCactor] Da clic en el botón \IUbutton{No} en la interfaz \refElem{VCT-IU04b}.
	\UCpaso [\UCsist] Muestra la interfaz \refElem{VCT-IU04b}.
	\UCpaso [\UCsist] Continúa en el paso \ref{VCT-CU04:vadata2} de la trayectoria principal.
\end{UCtrayectoriaA} 
	\clearpage
\subsection{USR-IU02 Consultar perfil}

\subsubsection{Objetivo}
En la figura \refElem{USR-IU02} se muestra la interfaz correspondiente con la funcionalidad descrita en las
trayectorias del caso de uso \refElem{USR-CU02} , la cual permite al actor la gestión su perfil y la consulta del mismo.

La interfaz esta compuesta ``secciones'' y cada sección corresponde a un formulario diferente, las secciones
son las siguientes:
\begin{itemize}
   \item \textbf{Datos personales}: esta sección tiene como objetivo que el actor actualice su información
   personal,sus datos de contacto y/o habilidades que posee (ver la figura \refElem{USR-IU02a}).
   \item \textbf{Objetivos y metas personales}: esta sección tiene como objetivo que el actor actualice sus objetivos, metas personales y laborales 
   (ver la figura \refElem{USR-IU02b}).
   \item \textbf{Historial académico}: esta sección tiene como objetivo que el actor actualice su información
   académica referente a todos los grados de estudios que tiene hasta la fecha (ver la figura \refElem{USR-IU02c}).
   \item \textbf{Idiomas}: esta sección tiene como objetivo preguntarle que el actor actualice la información de idiomas o en su caso, elimine
   o agregue nuevos idiomas a su perfil  (ver la figura \refElem{USR-IU02d}).
   \item \textbf{Experiencia laboral}: esta sección tiene como objetivo que el actor actualice su información de su experiencia
   laborar que tiene hasta la fecha (ver la figura \refElem{USR-IU02e}).
   \item \textbf{Cursos/Certificaciones}:  esta sección tiene como objetivo que el actor actualice su información de sus Certificaciones
   o cursos que ha tenido durante toda su trayectoria académica y laboral (ver la figura \refElem{USR-IU02f}).
\end{itemize}

\subsubsection{Comandos}
Los siguientes comandos aparecen durante toda la interfaz es decir, cada sección los tiene.

%\Titem \IUPass : Al da clic en el ícono, se muestra la contraseña de lo contrario aparecerá \IUOculto \thinspace sustituyendo cada caracter de la contraseña. \\

\Titem \IUEditar{} : Cuando presiona el ícono, habilita la sección para hacer los datos editables acorde a la sección o elemento seleccionado.
\Titem \IUEliminar{} : Cuando presiona el ícono, habilita la sección para eliminar el elemento seleccionado.
\Titem \IUAgregar{} : Cuando presiona el ícono, habilita la sección para agregar un nuevo el elemento.

\IUfig{.9}{CasosdeUso/USR-CU02/imagenes/USR-IU02.png}{USR-IU02}{Consultar perfil}  
\IUfig{.5}{CasosdeUso/USR-CU02/imagenes/USR-IU02a.png}{USR-IU02a}{Consultar perfil: Datos personales}
\IUfig{.9}{CasosdeUso/USR-CU02/imagenes/USR-IU02b.png}{USR-IU02b}{Consultar perfil: Objetivos y metas personales}  
\IUfig{.9}{CasosdeUso/USR-CU02/imagenes/USR-IU02c.png}{USR-IU02c}{Consultar perfil: Historial académico}  
\IUfig{.9}{CasosdeUso/USR-CU02/imagenes/USR-IU02d.png}{USR-IU02d}{Consultar perfil: Idiomas}
\IUfig{.9}{CasosdeUso/USR-CU02/imagenes/USR-IU02e.png}{USR-IU02e}{Consultar perfil: Experiencia laboral}  
\IUfig{.9}{CasosdeUso/USR-CU02/imagenes/USR-IU02f.png}{USR-IU02f}{Consultar perfil: Cursos/Certificaciones}  


\clearpage


	\clearpage
\begin{UseCase}[]{VCT-CU04}{Registrar vacante}{
	Permite al reclutador de una empresa  publicar una vacante en el sistema durante cierto periodo de tiempo y así poder gestionas las postulaciones 
	que los usarios(canditos) hagan a dicha vacante.
	}
	%----------------------------------------------------------------
	% Datos generales del CU:
	\UCsection{Atributos}
	\UCitem{Actor(es)}{
		Reclutadores.

	}
	\UCitem[admin]{Prioridad}{
		Media
	}
	\UCitem[admin]{Complejidad}{
		Alta
	}
	\UCitem{Precondiciones}{
		El reclutador debe de estar registrado en el sistema.
	}
	\UCitem{Destino}{
		\Titem \refElem{VCT-IU03}
	}
	\UCitem{Reglas de Negocio}{
		\Titem \refIdElem{RN-N001}
		
	}
	\UCitem{Viene de}{
		\refElem{VCT-CU03}
	}	
\end{UseCase}

%Trayectoria Principal
\begin{UCtrayectoria}
	\UCpaso [\UCactor] Da clic en el icono \IUAgregar{} (Publicar vacante) en la interfaz \refElem{VCT-IU03}.
	\UCpaso [\UCsist] Muestra la interfaz \refElem{VCT-IU04a} en la interfaz \refElem{VCT-IU03}.
	\UCpaso [\UCactor] \label{VCT-CU04:vadata} Ingresa el título y el número de plazas de la vacante, ingresa el código postal, estado, municipio y colonia donde se va laboral.\refTray{A}
	\UCpaso [\UCactor] Ingresa el perfil, la experiencia y el tipo de contratación a la que la vacante va dirigida.
	\UCpaso [\UCactor] Ingresa el horario laboral y el rango salarial indicando si es neto o no el salario.
	\UCpaso [\UCsist] Valida que todos los campos marcados como obligatorios hayan sido ingresados de acuerdo a la regla de negocio \refIdElem{RN-N001}. \refTray{B}
	\UCpaso [\UCactor] Ingresa la descripción de la vacante.
	\UCpaso [\UCactor] \label{VCT-CU04:hab}Selecciona las habilidades y su correspondiente experiencia deseadas para la vacante.\refTray{D}
	\UCpaso [\UCactor] Selecciona la fecha de cierre de la vacante.
	\UCpaso [\UCactor] Da clic en el botón \IUbutton{Publicar} en la interfaz \refElem{VCT-IU04b}.\refTray{A}\refTray{C} \refTray{F}
	\UCpaso [\UCsist] \label{VCT-CU04:vadata2}Valida que todos los campos marcados como obligatorios hayan sido ingresados de acuerdo a la regla de negocio \refIdElem{RN-N001}.\refTray{E}
	\UCpaso [\UCsist] Muestra la interfaz \refElem{VCT-IU03} mostrando la nueva vacante registrada.
	\UCpaso [\UCsist] Notifica a los encargados y colaboradores de que hay una nueva vacante por revisar.
\end{UCtrayectoria}

%Trayectorias Alternativas
\begin{UCtrayectoriaA}[Fin de la trayectoria]{A}{El actor decide cancelar el registro.}
	\UCpaso [\UCactor] Da clic en el botón \IUbutton{Cancelar} en la interfaz \refElem{VCT-IU04a}.
	\UCpaso [\UCsist] Muestra el mensaje \refIdElem{MSG6} en la interfaz \refElem{VCT-IU04a}.
	\UCpaso [\UCactor] Da clic en el botón \IUbutton{Sí} en la interfaz \refElem{VCT-IU04a}.\refTray{G}
	\UCpaso [\UCsist] Muestra la interfaz \refElem{VCT-IU03}.
\end{UCtrayectoriaA} 

%Trayectorias Alternativas
\begin{UCtrayectoriaA}[Fin de la trayectoria]{B}{El actor no registro al menos un campo obligatorio.}
	\UCpaso [\UCsist] Muestra el mensaje \refIdElem{MSG4} en la interfaz \refElem{VCT-IU03a} en los campos que no
	fueron ingresados.
	\UCpaso [\UCsist] Continúa en el paso \ref{VCT-CU04:vadata} de la trayectoria principal.
\end{UCtrayectoriaA} 

%Trayectorias Alternativas
\begin{UCtrayectoriaA}[Fin de la trayectoria]{C}{El actor decide regresa a la pantalla anterior.}
	\UCpaso [\UCactor] Da clic en el botón \IUbutton{Regresar} en la interfaz \refElem{VCT-IU04b}.
	\UCpaso [\UCsist] Muestra la interfaz \refElem{VCT-IU04a}.
	\UCpaso [\UCsist] Continúa en el paso \ref{VCT-CU04:vadata} de la trayectoria principal.
\end{UCtrayectoriaA} 

%Trayectorias Alternativas
\begin{UCtrayectoriaA}[Fin de la trayectoria]{D}{El actor decide eliminar una habilidad.}
	\UCpaso [\UCactor] Da clic en el botón \IUbutton{x} de la habilidad seleccionada en la interfaz \refElem{VCT-IU04b}.
	\UCpaso [\UCsist] Muestra la interfaz \refElem{VCT-IU04b}.
	\UCpaso [\UCsist] Continúa en el paso \ref{VCT-CU04:hab} de la trayectoria principal.
\end{UCtrayectoriaA} 

%Trayectorias Alternativas
\begin{UCtrayectoriaA}[Fin de la trayectoria]{E}{El actor no registro al menos un campo obligatorio.}
	\UCpaso [\UCsist] Muestra el mensaje \refIdElem{MSG4} en la interfaz \refElem{VCT-IU03b} en los campos que no
	fueron ingresados.
	\UCpaso [\UCsist] Continúa en el paso \ref{VCT-CU04:vadata2} de la trayectoria principal.
\end{UCtrayectoriaA} 

%Trayectorias Alternativas
\begin{UCtrayectoriaA}[Fin de la trayectoria]{F}{El actor decide cancelar el registro.}
	\UCpaso [\UCactor] Da clic en el botón \IUbutton{Cancelar} en la interfaz \refElem{VCT-IU04b}.
	\UCpaso [\UCsist] Muestra el mensaje \refIdElem{MSG6} en la interfaz \refElem{VCT-IU04b}.
	\UCpaso [\UCactor] Da clic en el botón \IUbutton{Sí} en la interfaz \refElem{VCT-IU04b}.\refTray{H}
	\UCpaso [\UCsist] Muestra la interfaz \refElem{VCT-IU03}.
\end{UCtrayectoriaA} 

%Trayectorias Alternativas
\begin{UCtrayectoriaA}[Fin de la trayectoria]{G}{El actor decide cancelar la acción.}
	\UCpaso [\UCactor] Da clic en el botón \IUbutton{No} en la interfaz \refElem{VCT-IU04a}.
	\UCpaso [\UCsist] Muestra la interfaz \refElem{VCT-IU04a}.
	\UCpaso [\UCsist] Continúa en el paso \ref{VCT-CU04:vadata} de la trayectoria principal.
\end{UCtrayectoriaA} 

%Trayectorias Alternativas
\begin{UCtrayectoriaA}[Fin de la trayectoria]{H}{El actor decide cancelar la acción.}
	\UCpaso [\UCactor] Da clic en el botón \IUbutton{No} en la interfaz \refElem{VCT-IU04b}.
	\UCpaso [\UCsist] Muestra la interfaz \refElem{VCT-IU04b}.
	\UCpaso [\UCsist] Continúa en el paso \ref{VCT-CU04:vadata2} de la trayectoria principal.
\end{UCtrayectoriaA} 
	\clearpage
\subsection{USR-IU02 Consultar perfil}

\subsubsection{Objetivo}
En la figura \refElem{USR-IU02} se muestra la interfaz correspondiente con la funcionalidad descrita en las
trayectorias del caso de uso \refElem{USR-CU02} , la cual permite al actor la gestión su perfil y la consulta del mismo.

La interfaz esta compuesta ``secciones'' y cada sección corresponde a un formulario diferente, las secciones
son las siguientes:
\begin{itemize}
   \item \textbf{Datos personales}: esta sección tiene como objetivo que el actor actualice su información
   personal,sus datos de contacto y/o habilidades que posee (ver la figura \refElem{USR-IU02a}).
   \item \textbf{Objetivos y metas personales}: esta sección tiene como objetivo que el actor actualice sus objetivos, metas personales y laborales 
   (ver la figura \refElem{USR-IU02b}).
   \item \textbf{Historial académico}: esta sección tiene como objetivo que el actor actualice su información
   académica referente a todos los grados de estudios que tiene hasta la fecha (ver la figura \refElem{USR-IU02c}).
   \item \textbf{Idiomas}: esta sección tiene como objetivo preguntarle que el actor actualice la información de idiomas o en su caso, elimine
   o agregue nuevos idiomas a su perfil  (ver la figura \refElem{USR-IU02d}).
   \item \textbf{Experiencia laboral}: esta sección tiene como objetivo que el actor actualice su información de su experiencia
   laborar que tiene hasta la fecha (ver la figura \refElem{USR-IU02e}).
   \item \textbf{Cursos/Certificaciones}:  esta sección tiene como objetivo que el actor actualice su información de sus Certificaciones
   o cursos que ha tenido durante toda su trayectoria académica y laboral (ver la figura \refElem{USR-IU02f}).
\end{itemize}

\subsubsection{Comandos}
Los siguientes comandos aparecen durante toda la interfaz es decir, cada sección los tiene.

%\Titem \IUPass : Al da clic en el ícono, se muestra la contraseña de lo contrario aparecerá \IUOculto \thinspace sustituyendo cada caracter de la contraseña. \\

\Titem \IUEditar{} : Cuando presiona el ícono, habilita la sección para hacer los datos editables acorde a la sección o elemento seleccionado.
\Titem \IUEliminar{} : Cuando presiona el ícono, habilita la sección para eliminar el elemento seleccionado.
\Titem \IUAgregar{} : Cuando presiona el ícono, habilita la sección para agregar un nuevo el elemento.

\IUfig{.9}{CasosdeUso/USR-CU02/imagenes/USR-IU02.png}{USR-IU02}{Consultar perfil}  
\IUfig{.5}{CasosdeUso/USR-CU02/imagenes/USR-IU02a.png}{USR-IU02a}{Consultar perfil: Datos personales}
\IUfig{.9}{CasosdeUso/USR-CU02/imagenes/USR-IU02b.png}{USR-IU02b}{Consultar perfil: Objetivos y metas personales}  
\IUfig{.9}{CasosdeUso/USR-CU02/imagenes/USR-IU02c.png}{USR-IU02c}{Consultar perfil: Historial académico}  
\IUfig{.9}{CasosdeUso/USR-CU02/imagenes/USR-IU02d.png}{USR-IU02d}{Consultar perfil: Idiomas}
\IUfig{.9}{CasosdeUso/USR-CU02/imagenes/USR-IU02e.png}{USR-IU02e}{Consultar perfil: Experiencia laboral}  
\IUfig{.9}{CasosdeUso/USR-CU02/imagenes/USR-IU02f.png}{USR-IU02f}{Consultar perfil: Cursos/Certificaciones}  


\clearpage


	\clearpage
\begin{UseCase}[]{VCT-CU04}{Registrar vacante}{
	Permite al reclutador de una empresa  publicar una vacante en el sistema durante cierto periodo de tiempo y así poder gestionas las postulaciones 
	que los usarios(canditos) hagan a dicha vacante.
	}
	%----------------------------------------------------------------
	% Datos generales del CU:
	\UCsection{Atributos}
	\UCitem{Actor(es)}{
		Reclutadores.

	}
	\UCitem[admin]{Prioridad}{
		Media
	}
	\UCitem[admin]{Complejidad}{
		Alta
	}
	\UCitem{Precondiciones}{
		El reclutador debe de estar registrado en el sistema.
	}
	\UCitem{Destino}{
		\Titem \refElem{VCT-IU03}
	}
	\UCitem{Reglas de Negocio}{
		\Titem \refIdElem{RN-N001}
		
	}
	\UCitem{Viene de}{
		\refElem{VCT-CU03}
	}	
\end{UseCase}

%Trayectoria Principal
\begin{UCtrayectoria}
	\UCpaso [\UCactor] Da clic en el icono \IUAgregar{} (Publicar vacante) en la interfaz \refElem{VCT-IU03}.
	\UCpaso [\UCsist] Muestra la interfaz \refElem{VCT-IU04a} en la interfaz \refElem{VCT-IU03}.
	\UCpaso [\UCactor] \label{VCT-CU04:vadata} Ingresa el título y el número de plazas de la vacante, ingresa el código postal, estado, municipio y colonia donde se va laboral.\refTray{A}
	\UCpaso [\UCactor] Ingresa el perfil, la experiencia y el tipo de contratación a la que la vacante va dirigida.
	\UCpaso [\UCactor] Ingresa el horario laboral y el rango salarial indicando si es neto o no el salario.
	\UCpaso [\UCsist] Valida que todos los campos marcados como obligatorios hayan sido ingresados de acuerdo a la regla de negocio \refIdElem{RN-N001}. \refTray{B}
	\UCpaso [\UCactor] Ingresa la descripción de la vacante.
	\UCpaso [\UCactor] \label{VCT-CU04:hab}Selecciona las habilidades y su correspondiente experiencia deseadas para la vacante.\refTray{D}
	\UCpaso [\UCactor] Selecciona la fecha de cierre de la vacante.
	\UCpaso [\UCactor] Da clic en el botón \IUbutton{Publicar} en la interfaz \refElem{VCT-IU04b}.\refTray{A}\refTray{C} \refTray{F}
	\UCpaso [\UCsist] \label{VCT-CU04:vadata2}Valida que todos los campos marcados como obligatorios hayan sido ingresados de acuerdo a la regla de negocio \refIdElem{RN-N001}.\refTray{E}
	\UCpaso [\UCsist] Muestra la interfaz \refElem{VCT-IU03} mostrando la nueva vacante registrada.
	\UCpaso [\UCsist] Notifica a los encargados y colaboradores de que hay una nueva vacante por revisar.
\end{UCtrayectoria}

%Trayectorias Alternativas
\begin{UCtrayectoriaA}[Fin de la trayectoria]{A}{El actor decide cancelar el registro.}
	\UCpaso [\UCactor] Da clic en el botón \IUbutton{Cancelar} en la interfaz \refElem{VCT-IU04a}.
	\UCpaso [\UCsist] Muestra el mensaje \refIdElem{MSG6} en la interfaz \refElem{VCT-IU04a}.
	\UCpaso [\UCactor] Da clic en el botón \IUbutton{Sí} en la interfaz \refElem{VCT-IU04a}.\refTray{G}
	\UCpaso [\UCsist] Muestra la interfaz \refElem{VCT-IU03}.
\end{UCtrayectoriaA} 

%Trayectorias Alternativas
\begin{UCtrayectoriaA}[Fin de la trayectoria]{B}{El actor no registro al menos un campo obligatorio.}
	\UCpaso [\UCsist] Muestra el mensaje \refIdElem{MSG4} en la interfaz \refElem{VCT-IU03a} en los campos que no
	fueron ingresados.
	\UCpaso [\UCsist] Continúa en el paso \ref{VCT-CU04:vadata} de la trayectoria principal.
\end{UCtrayectoriaA} 

%Trayectorias Alternativas
\begin{UCtrayectoriaA}[Fin de la trayectoria]{C}{El actor decide regresa a la pantalla anterior.}
	\UCpaso [\UCactor] Da clic en el botón \IUbutton{Regresar} en la interfaz \refElem{VCT-IU04b}.
	\UCpaso [\UCsist] Muestra la interfaz \refElem{VCT-IU04a}.
	\UCpaso [\UCsist] Continúa en el paso \ref{VCT-CU04:vadata} de la trayectoria principal.
\end{UCtrayectoriaA} 

%Trayectorias Alternativas
\begin{UCtrayectoriaA}[Fin de la trayectoria]{D}{El actor decide eliminar una habilidad.}
	\UCpaso [\UCactor] Da clic en el botón \IUbutton{x} de la habilidad seleccionada en la interfaz \refElem{VCT-IU04b}.
	\UCpaso [\UCsist] Muestra la interfaz \refElem{VCT-IU04b}.
	\UCpaso [\UCsist] Continúa en el paso \ref{VCT-CU04:hab} de la trayectoria principal.
\end{UCtrayectoriaA} 

%Trayectorias Alternativas
\begin{UCtrayectoriaA}[Fin de la trayectoria]{E}{El actor no registro al menos un campo obligatorio.}
	\UCpaso [\UCsist] Muestra el mensaje \refIdElem{MSG4} en la interfaz \refElem{VCT-IU03b} en los campos que no
	fueron ingresados.
	\UCpaso [\UCsist] Continúa en el paso \ref{VCT-CU04:vadata2} de la trayectoria principal.
\end{UCtrayectoriaA} 

%Trayectorias Alternativas
\begin{UCtrayectoriaA}[Fin de la trayectoria]{F}{El actor decide cancelar el registro.}
	\UCpaso [\UCactor] Da clic en el botón \IUbutton{Cancelar} en la interfaz \refElem{VCT-IU04b}.
	\UCpaso [\UCsist] Muestra el mensaje \refIdElem{MSG6} en la interfaz \refElem{VCT-IU04b}.
	\UCpaso [\UCactor] Da clic en el botón \IUbutton{Sí} en la interfaz \refElem{VCT-IU04b}.\refTray{H}
	\UCpaso [\UCsist] Muestra la interfaz \refElem{VCT-IU03}.
\end{UCtrayectoriaA} 

%Trayectorias Alternativas
\begin{UCtrayectoriaA}[Fin de la trayectoria]{G}{El actor decide cancelar la acción.}
	\UCpaso [\UCactor] Da clic en el botón \IUbutton{No} en la interfaz \refElem{VCT-IU04a}.
	\UCpaso [\UCsist] Muestra la interfaz \refElem{VCT-IU04a}.
	\UCpaso [\UCsist] Continúa en el paso \ref{VCT-CU04:vadata} de la trayectoria principal.
\end{UCtrayectoriaA} 

%Trayectorias Alternativas
\begin{UCtrayectoriaA}[Fin de la trayectoria]{H}{El actor decide cancelar la acción.}
	\UCpaso [\UCactor] Da clic en el botón \IUbutton{No} en la interfaz \refElem{VCT-IU04b}.
	\UCpaso [\UCsist] Muestra la interfaz \refElem{VCT-IU04b}.
	\UCpaso [\UCsist] Continúa en el paso \ref{VCT-CU04:vadata2} de la trayectoria principal.
\end{UCtrayectoriaA} 
	\clearpage
\subsection{USR-IU02 Consultar perfil}

\subsubsection{Objetivo}
En la figura \refElem{USR-IU02} se muestra la interfaz correspondiente con la funcionalidad descrita en las
trayectorias del caso de uso \refElem{USR-CU02} , la cual permite al actor la gestión su perfil y la consulta del mismo.

La interfaz esta compuesta ``secciones'' y cada sección corresponde a un formulario diferente, las secciones
son las siguientes:
\begin{itemize}
   \item \textbf{Datos personales}: esta sección tiene como objetivo que el actor actualice su información
   personal,sus datos de contacto y/o habilidades que posee (ver la figura \refElem{USR-IU02a}).
   \item \textbf{Objetivos y metas personales}: esta sección tiene como objetivo que el actor actualice sus objetivos, metas personales y laborales 
   (ver la figura \refElem{USR-IU02b}).
   \item \textbf{Historial académico}: esta sección tiene como objetivo que el actor actualice su información
   académica referente a todos los grados de estudios que tiene hasta la fecha (ver la figura \refElem{USR-IU02c}).
   \item \textbf{Idiomas}: esta sección tiene como objetivo preguntarle que el actor actualice la información de idiomas o en su caso, elimine
   o agregue nuevos idiomas a su perfil  (ver la figura \refElem{USR-IU02d}).
   \item \textbf{Experiencia laboral}: esta sección tiene como objetivo que el actor actualice su información de su experiencia
   laborar que tiene hasta la fecha (ver la figura \refElem{USR-IU02e}).
   \item \textbf{Cursos/Certificaciones}:  esta sección tiene como objetivo que el actor actualice su información de sus Certificaciones
   o cursos que ha tenido durante toda su trayectoria académica y laboral (ver la figura \refElem{USR-IU02f}).
\end{itemize}

\subsubsection{Comandos}
Los siguientes comandos aparecen durante toda la interfaz es decir, cada sección los tiene.

%\Titem \IUPass : Al da clic en el ícono, se muestra la contraseña de lo contrario aparecerá \IUOculto \thinspace sustituyendo cada caracter de la contraseña. \\

\Titem \IUEditar{} : Cuando presiona el ícono, habilita la sección para hacer los datos editables acorde a la sección o elemento seleccionado.
\Titem \IUEliminar{} : Cuando presiona el ícono, habilita la sección para eliminar el elemento seleccionado.
\Titem \IUAgregar{} : Cuando presiona el ícono, habilita la sección para agregar un nuevo el elemento.

\IUfig{.9}{CasosdeUso/USR-CU02/imagenes/USR-IU02.png}{USR-IU02}{Consultar perfil}  
\IUfig{.5}{CasosdeUso/USR-CU02/imagenes/USR-IU02a.png}{USR-IU02a}{Consultar perfil: Datos personales}
\IUfig{.9}{CasosdeUso/USR-CU02/imagenes/USR-IU02b.png}{USR-IU02b}{Consultar perfil: Objetivos y metas personales}  
\IUfig{.9}{CasosdeUso/USR-CU02/imagenes/USR-IU02c.png}{USR-IU02c}{Consultar perfil: Historial académico}  
\IUfig{.9}{CasosdeUso/USR-CU02/imagenes/USR-IU02d.png}{USR-IU02d}{Consultar perfil: Idiomas}
\IUfig{.9}{CasosdeUso/USR-CU02/imagenes/USR-IU02e.png}{USR-IU02e}{Consultar perfil: Experiencia laboral}  
\IUfig{.9}{CasosdeUso/USR-CU02/imagenes/USR-IU02f.png}{USR-IU02f}{Consultar perfil: Cursos/Certificaciones}  


\clearpage


	\clearpage
\begin{UseCase}[]{VCT-CU04}{Registrar vacante}{
	Permite al reclutador de una empresa  publicar una vacante en el sistema durante cierto periodo de tiempo y así poder gestionas las postulaciones 
	que los usarios(canditos) hagan a dicha vacante.
	}
	%----------------------------------------------------------------
	% Datos generales del CU:
	\UCsection{Atributos}
	\UCitem{Actor(es)}{
		Reclutadores.

	}
	\UCitem[admin]{Prioridad}{
		Media
	}
	\UCitem[admin]{Complejidad}{
		Alta
	}
	\UCitem{Precondiciones}{
		El reclutador debe de estar registrado en el sistema.
	}
	\UCitem{Destino}{
		\Titem \refElem{VCT-IU03}
	}
	\UCitem{Reglas de Negocio}{
		\Titem \refIdElem{RN-N001}
		
	}
	\UCitem{Viene de}{
		\refElem{VCT-CU03}
	}	
\end{UseCase}

%Trayectoria Principal
\begin{UCtrayectoria}
	\UCpaso [\UCactor] Da clic en el icono \IUAgregar{} (Publicar vacante) en la interfaz \refElem{VCT-IU03}.
	\UCpaso [\UCsist] Muestra la interfaz \refElem{VCT-IU04a} en la interfaz \refElem{VCT-IU03}.
	\UCpaso [\UCactor] \label{VCT-CU04:vadata} Ingresa el título y el número de plazas de la vacante, ingresa el código postal, estado, municipio y colonia donde se va laboral.\refTray{A}
	\UCpaso [\UCactor] Ingresa el perfil, la experiencia y el tipo de contratación a la que la vacante va dirigida.
	\UCpaso [\UCactor] Ingresa el horario laboral y el rango salarial indicando si es neto o no el salario.
	\UCpaso [\UCsist] Valida que todos los campos marcados como obligatorios hayan sido ingresados de acuerdo a la regla de negocio \refIdElem{RN-N001}. \refTray{B}
	\UCpaso [\UCactor] Ingresa la descripción de la vacante.
	\UCpaso [\UCactor] \label{VCT-CU04:hab}Selecciona las habilidades y su correspondiente experiencia deseadas para la vacante.\refTray{D}
	\UCpaso [\UCactor] Selecciona la fecha de cierre de la vacante.
	\UCpaso [\UCactor] Da clic en el botón \IUbutton{Publicar} en la interfaz \refElem{VCT-IU04b}.\refTray{A}\refTray{C} \refTray{F}
	\UCpaso [\UCsist] \label{VCT-CU04:vadata2}Valida que todos los campos marcados como obligatorios hayan sido ingresados de acuerdo a la regla de negocio \refIdElem{RN-N001}.\refTray{E}
	\UCpaso [\UCsist] Muestra la interfaz \refElem{VCT-IU03} mostrando la nueva vacante registrada.
	\UCpaso [\UCsist] Notifica a los encargados y colaboradores de que hay una nueva vacante por revisar.
\end{UCtrayectoria}

%Trayectorias Alternativas
\begin{UCtrayectoriaA}[Fin de la trayectoria]{A}{El actor decide cancelar el registro.}
	\UCpaso [\UCactor] Da clic en el botón \IUbutton{Cancelar} en la interfaz \refElem{VCT-IU04a}.
	\UCpaso [\UCsist] Muestra el mensaje \refIdElem{MSG6} en la interfaz \refElem{VCT-IU04a}.
	\UCpaso [\UCactor] Da clic en el botón \IUbutton{Sí} en la interfaz \refElem{VCT-IU04a}.\refTray{G}
	\UCpaso [\UCsist] Muestra la interfaz \refElem{VCT-IU03}.
\end{UCtrayectoriaA} 

%Trayectorias Alternativas
\begin{UCtrayectoriaA}[Fin de la trayectoria]{B}{El actor no registro al menos un campo obligatorio.}
	\UCpaso [\UCsist] Muestra el mensaje \refIdElem{MSG4} en la interfaz \refElem{VCT-IU03a} en los campos que no
	fueron ingresados.
	\UCpaso [\UCsist] Continúa en el paso \ref{VCT-CU04:vadata} de la trayectoria principal.
\end{UCtrayectoriaA} 

%Trayectorias Alternativas
\begin{UCtrayectoriaA}[Fin de la trayectoria]{C}{El actor decide regresa a la pantalla anterior.}
	\UCpaso [\UCactor] Da clic en el botón \IUbutton{Regresar} en la interfaz \refElem{VCT-IU04b}.
	\UCpaso [\UCsist] Muestra la interfaz \refElem{VCT-IU04a}.
	\UCpaso [\UCsist] Continúa en el paso \ref{VCT-CU04:vadata} de la trayectoria principal.
\end{UCtrayectoriaA} 

%Trayectorias Alternativas
\begin{UCtrayectoriaA}[Fin de la trayectoria]{D}{El actor decide eliminar una habilidad.}
	\UCpaso [\UCactor] Da clic en el botón \IUbutton{x} de la habilidad seleccionada en la interfaz \refElem{VCT-IU04b}.
	\UCpaso [\UCsist] Muestra la interfaz \refElem{VCT-IU04b}.
	\UCpaso [\UCsist] Continúa en el paso \ref{VCT-CU04:hab} de la trayectoria principal.
\end{UCtrayectoriaA} 

%Trayectorias Alternativas
\begin{UCtrayectoriaA}[Fin de la trayectoria]{E}{El actor no registro al menos un campo obligatorio.}
	\UCpaso [\UCsist] Muestra el mensaje \refIdElem{MSG4} en la interfaz \refElem{VCT-IU03b} en los campos que no
	fueron ingresados.
	\UCpaso [\UCsist] Continúa en el paso \ref{VCT-CU04:vadata2} de la trayectoria principal.
\end{UCtrayectoriaA} 

%Trayectorias Alternativas
\begin{UCtrayectoriaA}[Fin de la trayectoria]{F}{El actor decide cancelar el registro.}
	\UCpaso [\UCactor] Da clic en el botón \IUbutton{Cancelar} en la interfaz \refElem{VCT-IU04b}.
	\UCpaso [\UCsist] Muestra el mensaje \refIdElem{MSG6} en la interfaz \refElem{VCT-IU04b}.
	\UCpaso [\UCactor] Da clic en el botón \IUbutton{Sí} en la interfaz \refElem{VCT-IU04b}.\refTray{H}
	\UCpaso [\UCsist] Muestra la interfaz \refElem{VCT-IU03}.
\end{UCtrayectoriaA} 

%Trayectorias Alternativas
\begin{UCtrayectoriaA}[Fin de la trayectoria]{G}{El actor decide cancelar la acción.}
	\UCpaso [\UCactor] Da clic en el botón \IUbutton{No} en la interfaz \refElem{VCT-IU04a}.
	\UCpaso [\UCsist] Muestra la interfaz \refElem{VCT-IU04a}.
	\UCpaso [\UCsist] Continúa en el paso \ref{VCT-CU04:vadata} de la trayectoria principal.
\end{UCtrayectoriaA} 

%Trayectorias Alternativas
\begin{UCtrayectoriaA}[Fin de la trayectoria]{H}{El actor decide cancelar la acción.}
	\UCpaso [\UCactor] Da clic en el botón \IUbutton{No} en la interfaz \refElem{VCT-IU04b}.
	\UCpaso [\UCsist] Muestra la interfaz \refElem{VCT-IU04b}.
	\UCpaso [\UCsist] Continúa en el paso \ref{VCT-CU04:vadata2} de la trayectoria principal.
\end{UCtrayectoriaA} 
	\clearpage
\subsection{USR-IU02 Consultar perfil}

\subsubsection{Objetivo}
En la figura \refElem{USR-IU02} se muestra la interfaz correspondiente con la funcionalidad descrita en las
trayectorias del caso de uso \refElem{USR-CU02} , la cual permite al actor la gestión su perfil y la consulta del mismo.

La interfaz esta compuesta ``secciones'' y cada sección corresponde a un formulario diferente, las secciones
son las siguientes:
\begin{itemize}
   \item \textbf{Datos personales}: esta sección tiene como objetivo que el actor actualice su información
   personal,sus datos de contacto y/o habilidades que posee (ver la figura \refElem{USR-IU02a}).
   \item \textbf{Objetivos y metas personales}: esta sección tiene como objetivo que el actor actualice sus objetivos, metas personales y laborales 
   (ver la figura \refElem{USR-IU02b}).
   \item \textbf{Historial académico}: esta sección tiene como objetivo que el actor actualice su información
   académica referente a todos los grados de estudios que tiene hasta la fecha (ver la figura \refElem{USR-IU02c}).
   \item \textbf{Idiomas}: esta sección tiene como objetivo preguntarle que el actor actualice la información de idiomas o en su caso, elimine
   o agregue nuevos idiomas a su perfil  (ver la figura \refElem{USR-IU02d}).
   \item \textbf{Experiencia laboral}: esta sección tiene como objetivo que el actor actualice su información de su experiencia
   laborar que tiene hasta la fecha (ver la figura \refElem{USR-IU02e}).
   \item \textbf{Cursos/Certificaciones}:  esta sección tiene como objetivo que el actor actualice su información de sus Certificaciones
   o cursos que ha tenido durante toda su trayectoria académica y laboral (ver la figura \refElem{USR-IU02f}).
\end{itemize}

\subsubsection{Comandos}
Los siguientes comandos aparecen durante toda la interfaz es decir, cada sección los tiene.

%\Titem \IUPass : Al da clic en el ícono, se muestra la contraseña de lo contrario aparecerá \IUOculto \thinspace sustituyendo cada caracter de la contraseña. \\

\Titem \IUEditar{} : Cuando presiona el ícono, habilita la sección para hacer los datos editables acorde a la sección o elemento seleccionado.
\Titem \IUEliminar{} : Cuando presiona el ícono, habilita la sección para eliminar el elemento seleccionado.
\Titem \IUAgregar{} : Cuando presiona el ícono, habilita la sección para agregar un nuevo el elemento.

\IUfig{.9}{CasosdeUso/USR-CU02/imagenes/USR-IU02.png}{USR-IU02}{Consultar perfil}  
\IUfig{.5}{CasosdeUso/USR-CU02/imagenes/USR-IU02a.png}{USR-IU02a}{Consultar perfil: Datos personales}
\IUfig{.9}{CasosdeUso/USR-CU02/imagenes/USR-IU02b.png}{USR-IU02b}{Consultar perfil: Objetivos y metas personales}  
\IUfig{.9}{CasosdeUso/USR-CU02/imagenes/USR-IU02c.png}{USR-IU02c}{Consultar perfil: Historial académico}  
\IUfig{.9}{CasosdeUso/USR-CU02/imagenes/USR-IU02d.png}{USR-IU02d}{Consultar perfil: Idiomas}
\IUfig{.9}{CasosdeUso/USR-CU02/imagenes/USR-IU02e.png}{USR-IU02e}{Consultar perfil: Experiencia laboral}  
\IUfig{.9}{CasosdeUso/USR-CU02/imagenes/USR-IU02f.png}{USR-IU02f}{Consultar perfil: Cursos/Certificaciones}  


\clearpage


	\clearpage
\begin{UseCase}[]{VCT-CU04}{Registrar vacante}{
	Permite al reclutador de una empresa  publicar una vacante en el sistema durante cierto periodo de tiempo y así poder gestionas las postulaciones 
	que los usarios(canditos) hagan a dicha vacante.
	}
	%----------------------------------------------------------------
	% Datos generales del CU:
	\UCsection{Atributos}
	\UCitem{Actor(es)}{
		Reclutadores.

	}
	\UCitem[admin]{Prioridad}{
		Media
	}
	\UCitem[admin]{Complejidad}{
		Alta
	}
	\UCitem{Precondiciones}{
		El reclutador debe de estar registrado en el sistema.
	}
	\UCitem{Destino}{
		\Titem \refElem{VCT-IU03}
	}
	\UCitem{Reglas de Negocio}{
		\Titem \refIdElem{RN-N001}
		
	}
	\UCitem{Viene de}{
		\refElem{VCT-CU03}
	}	
\end{UseCase}

%Trayectoria Principal
\begin{UCtrayectoria}
	\UCpaso [\UCactor] Da clic en el icono \IUAgregar{} (Publicar vacante) en la interfaz \refElem{VCT-IU03}.
	\UCpaso [\UCsist] Muestra la interfaz \refElem{VCT-IU04a} en la interfaz \refElem{VCT-IU03}.
	\UCpaso [\UCactor] \label{VCT-CU04:vadata} Ingresa el título y el número de plazas de la vacante, ingresa el código postal, estado, municipio y colonia donde se va laboral.\refTray{A}
	\UCpaso [\UCactor] Ingresa el perfil, la experiencia y el tipo de contratación a la que la vacante va dirigida.
	\UCpaso [\UCactor] Ingresa el horario laboral y el rango salarial indicando si es neto o no el salario.
	\UCpaso [\UCsist] Valida que todos los campos marcados como obligatorios hayan sido ingresados de acuerdo a la regla de negocio \refIdElem{RN-N001}. \refTray{B}
	\UCpaso [\UCactor] Ingresa la descripción de la vacante.
	\UCpaso [\UCactor] \label{VCT-CU04:hab}Selecciona las habilidades y su correspondiente experiencia deseadas para la vacante.\refTray{D}
	\UCpaso [\UCactor] Selecciona la fecha de cierre de la vacante.
	\UCpaso [\UCactor] Da clic en el botón \IUbutton{Publicar} en la interfaz \refElem{VCT-IU04b}.\refTray{A}\refTray{C} \refTray{F}
	\UCpaso [\UCsist] \label{VCT-CU04:vadata2}Valida que todos los campos marcados como obligatorios hayan sido ingresados de acuerdo a la regla de negocio \refIdElem{RN-N001}.\refTray{E}
	\UCpaso [\UCsist] Muestra la interfaz \refElem{VCT-IU03} mostrando la nueva vacante registrada.
	\UCpaso [\UCsist] Notifica a los encargados y colaboradores de que hay una nueva vacante por revisar.
\end{UCtrayectoria}

%Trayectorias Alternativas
\begin{UCtrayectoriaA}[Fin de la trayectoria]{A}{El actor decide cancelar el registro.}
	\UCpaso [\UCactor] Da clic en el botón \IUbutton{Cancelar} en la interfaz \refElem{VCT-IU04a}.
	\UCpaso [\UCsist] Muestra el mensaje \refIdElem{MSG6} en la interfaz \refElem{VCT-IU04a}.
	\UCpaso [\UCactor] Da clic en el botón \IUbutton{Sí} en la interfaz \refElem{VCT-IU04a}.\refTray{G}
	\UCpaso [\UCsist] Muestra la interfaz \refElem{VCT-IU03}.
\end{UCtrayectoriaA} 

%Trayectorias Alternativas
\begin{UCtrayectoriaA}[Fin de la trayectoria]{B}{El actor no registro al menos un campo obligatorio.}
	\UCpaso [\UCsist] Muestra el mensaje \refIdElem{MSG4} en la interfaz \refElem{VCT-IU03a} en los campos que no
	fueron ingresados.
	\UCpaso [\UCsist] Continúa en el paso \ref{VCT-CU04:vadata} de la trayectoria principal.
\end{UCtrayectoriaA} 

%Trayectorias Alternativas
\begin{UCtrayectoriaA}[Fin de la trayectoria]{C}{El actor decide regresa a la pantalla anterior.}
	\UCpaso [\UCactor] Da clic en el botón \IUbutton{Regresar} en la interfaz \refElem{VCT-IU04b}.
	\UCpaso [\UCsist] Muestra la interfaz \refElem{VCT-IU04a}.
	\UCpaso [\UCsist] Continúa en el paso \ref{VCT-CU04:vadata} de la trayectoria principal.
\end{UCtrayectoriaA} 

%Trayectorias Alternativas
\begin{UCtrayectoriaA}[Fin de la trayectoria]{D}{El actor decide eliminar una habilidad.}
	\UCpaso [\UCactor] Da clic en el botón \IUbutton{x} de la habilidad seleccionada en la interfaz \refElem{VCT-IU04b}.
	\UCpaso [\UCsist] Muestra la interfaz \refElem{VCT-IU04b}.
	\UCpaso [\UCsist] Continúa en el paso \ref{VCT-CU04:hab} de la trayectoria principal.
\end{UCtrayectoriaA} 

%Trayectorias Alternativas
\begin{UCtrayectoriaA}[Fin de la trayectoria]{E}{El actor no registro al menos un campo obligatorio.}
	\UCpaso [\UCsist] Muestra el mensaje \refIdElem{MSG4} en la interfaz \refElem{VCT-IU03b} en los campos que no
	fueron ingresados.
	\UCpaso [\UCsist] Continúa en el paso \ref{VCT-CU04:vadata2} de la trayectoria principal.
\end{UCtrayectoriaA} 

%Trayectorias Alternativas
\begin{UCtrayectoriaA}[Fin de la trayectoria]{F}{El actor decide cancelar el registro.}
	\UCpaso [\UCactor] Da clic en el botón \IUbutton{Cancelar} en la interfaz \refElem{VCT-IU04b}.
	\UCpaso [\UCsist] Muestra el mensaje \refIdElem{MSG6} en la interfaz \refElem{VCT-IU04b}.
	\UCpaso [\UCactor] Da clic en el botón \IUbutton{Sí} en la interfaz \refElem{VCT-IU04b}.\refTray{H}
	\UCpaso [\UCsist] Muestra la interfaz \refElem{VCT-IU03}.
\end{UCtrayectoriaA} 

%Trayectorias Alternativas
\begin{UCtrayectoriaA}[Fin de la trayectoria]{G}{El actor decide cancelar la acción.}
	\UCpaso [\UCactor] Da clic en el botón \IUbutton{No} en la interfaz \refElem{VCT-IU04a}.
	\UCpaso [\UCsist] Muestra la interfaz \refElem{VCT-IU04a}.
	\UCpaso [\UCsist] Continúa en el paso \ref{VCT-CU04:vadata} de la trayectoria principal.
\end{UCtrayectoriaA} 

%Trayectorias Alternativas
\begin{UCtrayectoriaA}[Fin de la trayectoria]{H}{El actor decide cancelar la acción.}
	\UCpaso [\UCactor] Da clic en el botón \IUbutton{No} en la interfaz \refElem{VCT-IU04b}.
	\UCpaso [\UCsist] Muestra la interfaz \refElem{VCT-IU04b}.
	\UCpaso [\UCsist] Continúa en el paso \ref{VCT-CU04:vadata2} de la trayectoria principal.
\end{UCtrayectoriaA} 
	\clearpage
\subsection{USR-IU02 Consultar perfil}

\subsubsection{Objetivo}
En la figura \refElem{USR-IU02} se muestra la interfaz correspondiente con la funcionalidad descrita en las
trayectorias del caso de uso \refElem{USR-CU02} , la cual permite al actor la gestión su perfil y la consulta del mismo.

La interfaz esta compuesta ``secciones'' y cada sección corresponde a un formulario diferente, las secciones
son las siguientes:
\begin{itemize}
   \item \textbf{Datos personales}: esta sección tiene como objetivo que el actor actualice su información
   personal,sus datos de contacto y/o habilidades que posee (ver la figura \refElem{USR-IU02a}).
   \item \textbf{Objetivos y metas personales}: esta sección tiene como objetivo que el actor actualice sus objetivos, metas personales y laborales 
   (ver la figura \refElem{USR-IU02b}).
   \item \textbf{Historial académico}: esta sección tiene como objetivo que el actor actualice su información
   académica referente a todos los grados de estudios que tiene hasta la fecha (ver la figura \refElem{USR-IU02c}).
   \item \textbf{Idiomas}: esta sección tiene como objetivo preguntarle que el actor actualice la información de idiomas o en su caso, elimine
   o agregue nuevos idiomas a su perfil  (ver la figura \refElem{USR-IU02d}).
   \item \textbf{Experiencia laboral}: esta sección tiene como objetivo que el actor actualice su información de su experiencia
   laborar que tiene hasta la fecha (ver la figura \refElem{USR-IU02e}).
   \item \textbf{Cursos/Certificaciones}:  esta sección tiene como objetivo que el actor actualice su información de sus Certificaciones
   o cursos que ha tenido durante toda su trayectoria académica y laboral (ver la figura \refElem{USR-IU02f}).
\end{itemize}

\subsubsection{Comandos}
Los siguientes comandos aparecen durante toda la interfaz es decir, cada sección los tiene.

%\Titem \IUPass : Al da clic en el ícono, se muestra la contraseña de lo contrario aparecerá \IUOculto \thinspace sustituyendo cada caracter de la contraseña. \\

\Titem \IUEditar{} : Cuando presiona el ícono, habilita la sección para hacer los datos editables acorde a la sección o elemento seleccionado.
\Titem \IUEliminar{} : Cuando presiona el ícono, habilita la sección para eliminar el elemento seleccionado.
\Titem \IUAgregar{} : Cuando presiona el ícono, habilita la sección para agregar un nuevo el elemento.

\IUfig{.9}{CasosdeUso/USR-CU02/imagenes/USR-IU02.png}{USR-IU02}{Consultar perfil}  
\IUfig{.5}{CasosdeUso/USR-CU02/imagenes/USR-IU02a.png}{USR-IU02a}{Consultar perfil: Datos personales}
\IUfig{.9}{CasosdeUso/USR-CU02/imagenes/USR-IU02b.png}{USR-IU02b}{Consultar perfil: Objetivos y metas personales}  
\IUfig{.9}{CasosdeUso/USR-CU02/imagenes/USR-IU02c.png}{USR-IU02c}{Consultar perfil: Historial académico}  
\IUfig{.9}{CasosdeUso/USR-CU02/imagenes/USR-IU02d.png}{USR-IU02d}{Consultar perfil: Idiomas}
\IUfig{.9}{CasosdeUso/USR-CU02/imagenes/USR-IU02e.png}{USR-IU02e}{Consultar perfil: Experiencia laboral}  
\IUfig{.9}{CasosdeUso/USR-CU02/imagenes/USR-IU02f.png}{USR-IU02f}{Consultar perfil: Cursos/Certificaciones}  


\clearpage

	
	\clearpage
\begin{UseCase}[]{VCT-CU04}{Registrar vacante}{
	Permite al reclutador de una empresa  publicar una vacante en el sistema durante cierto periodo de tiempo y así poder gestionas las postulaciones 
	que los usarios(canditos) hagan a dicha vacante.
	}
	%----------------------------------------------------------------
	% Datos generales del CU:
	\UCsection{Atributos}
	\UCitem{Actor(es)}{
		Reclutadores.

	}
	\UCitem[admin]{Prioridad}{
		Media
	}
	\UCitem[admin]{Complejidad}{
		Alta
	}
	\UCitem{Precondiciones}{
		El reclutador debe de estar registrado en el sistema.
	}
	\UCitem{Destino}{
		\Titem \refElem{VCT-IU03}
	}
	\UCitem{Reglas de Negocio}{
		\Titem \refIdElem{RN-N001}
		
	}
	\UCitem{Viene de}{
		\refElem{VCT-CU03}
	}	
\end{UseCase}

%Trayectoria Principal
\begin{UCtrayectoria}
	\UCpaso [\UCactor] Da clic en el icono \IUAgregar{} (Publicar vacante) en la interfaz \refElem{VCT-IU03}.
	\UCpaso [\UCsist] Muestra la interfaz \refElem{VCT-IU04a} en la interfaz \refElem{VCT-IU03}.
	\UCpaso [\UCactor] \label{VCT-CU04:vadata} Ingresa el título y el número de plazas de la vacante, ingresa el código postal, estado, municipio y colonia donde se va laboral.\refTray{A}
	\UCpaso [\UCactor] Ingresa el perfil, la experiencia y el tipo de contratación a la que la vacante va dirigida.
	\UCpaso [\UCactor] Ingresa el horario laboral y el rango salarial indicando si es neto o no el salario.
	\UCpaso [\UCsist] Valida que todos los campos marcados como obligatorios hayan sido ingresados de acuerdo a la regla de negocio \refIdElem{RN-N001}. \refTray{B}
	\UCpaso [\UCactor] Ingresa la descripción de la vacante.
	\UCpaso [\UCactor] \label{VCT-CU04:hab}Selecciona las habilidades y su correspondiente experiencia deseadas para la vacante.\refTray{D}
	\UCpaso [\UCactor] Selecciona la fecha de cierre de la vacante.
	\UCpaso [\UCactor] Da clic en el botón \IUbutton{Publicar} en la interfaz \refElem{VCT-IU04b}.\refTray{A}\refTray{C} \refTray{F}
	\UCpaso [\UCsist] \label{VCT-CU04:vadata2}Valida que todos los campos marcados como obligatorios hayan sido ingresados de acuerdo a la regla de negocio \refIdElem{RN-N001}.\refTray{E}
	\UCpaso [\UCsist] Muestra la interfaz \refElem{VCT-IU03} mostrando la nueva vacante registrada.
	\UCpaso [\UCsist] Notifica a los encargados y colaboradores de que hay una nueva vacante por revisar.
\end{UCtrayectoria}

%Trayectorias Alternativas
\begin{UCtrayectoriaA}[Fin de la trayectoria]{A}{El actor decide cancelar el registro.}
	\UCpaso [\UCactor] Da clic en el botón \IUbutton{Cancelar} en la interfaz \refElem{VCT-IU04a}.
	\UCpaso [\UCsist] Muestra el mensaje \refIdElem{MSG6} en la interfaz \refElem{VCT-IU04a}.
	\UCpaso [\UCactor] Da clic en el botón \IUbutton{Sí} en la interfaz \refElem{VCT-IU04a}.\refTray{G}
	\UCpaso [\UCsist] Muestra la interfaz \refElem{VCT-IU03}.
\end{UCtrayectoriaA} 

%Trayectorias Alternativas
\begin{UCtrayectoriaA}[Fin de la trayectoria]{B}{El actor no registro al menos un campo obligatorio.}
	\UCpaso [\UCsist] Muestra el mensaje \refIdElem{MSG4} en la interfaz \refElem{VCT-IU03a} en los campos que no
	fueron ingresados.
	\UCpaso [\UCsist] Continúa en el paso \ref{VCT-CU04:vadata} de la trayectoria principal.
\end{UCtrayectoriaA} 

%Trayectorias Alternativas
\begin{UCtrayectoriaA}[Fin de la trayectoria]{C}{El actor decide regresa a la pantalla anterior.}
	\UCpaso [\UCactor] Da clic en el botón \IUbutton{Regresar} en la interfaz \refElem{VCT-IU04b}.
	\UCpaso [\UCsist] Muestra la interfaz \refElem{VCT-IU04a}.
	\UCpaso [\UCsist] Continúa en el paso \ref{VCT-CU04:vadata} de la trayectoria principal.
\end{UCtrayectoriaA} 

%Trayectorias Alternativas
\begin{UCtrayectoriaA}[Fin de la trayectoria]{D}{El actor decide eliminar una habilidad.}
	\UCpaso [\UCactor] Da clic en el botón \IUbutton{x} de la habilidad seleccionada en la interfaz \refElem{VCT-IU04b}.
	\UCpaso [\UCsist] Muestra la interfaz \refElem{VCT-IU04b}.
	\UCpaso [\UCsist] Continúa en el paso \ref{VCT-CU04:hab} de la trayectoria principal.
\end{UCtrayectoriaA} 

%Trayectorias Alternativas
\begin{UCtrayectoriaA}[Fin de la trayectoria]{E}{El actor no registro al menos un campo obligatorio.}
	\UCpaso [\UCsist] Muestra el mensaje \refIdElem{MSG4} en la interfaz \refElem{VCT-IU03b} en los campos que no
	fueron ingresados.
	\UCpaso [\UCsist] Continúa en el paso \ref{VCT-CU04:vadata2} de la trayectoria principal.
\end{UCtrayectoriaA} 

%Trayectorias Alternativas
\begin{UCtrayectoriaA}[Fin de la trayectoria]{F}{El actor decide cancelar el registro.}
	\UCpaso [\UCactor] Da clic en el botón \IUbutton{Cancelar} en la interfaz \refElem{VCT-IU04b}.
	\UCpaso [\UCsist] Muestra el mensaje \refIdElem{MSG6} en la interfaz \refElem{VCT-IU04b}.
	\UCpaso [\UCactor] Da clic en el botón \IUbutton{Sí} en la interfaz \refElem{VCT-IU04b}.\refTray{H}
	\UCpaso [\UCsist] Muestra la interfaz \refElem{VCT-IU03}.
\end{UCtrayectoriaA} 

%Trayectorias Alternativas
\begin{UCtrayectoriaA}[Fin de la trayectoria]{G}{El actor decide cancelar la acción.}
	\UCpaso [\UCactor] Da clic en el botón \IUbutton{No} en la interfaz \refElem{VCT-IU04a}.
	\UCpaso [\UCsist] Muestra la interfaz \refElem{VCT-IU04a}.
	\UCpaso [\UCsist] Continúa en el paso \ref{VCT-CU04:vadata} de la trayectoria principal.
\end{UCtrayectoriaA} 

%Trayectorias Alternativas
\begin{UCtrayectoriaA}[Fin de la trayectoria]{H}{El actor decide cancelar la acción.}
	\UCpaso [\UCactor] Da clic en el botón \IUbutton{No} en la interfaz \refElem{VCT-IU04b}.
	\UCpaso [\UCsist] Muestra la interfaz \refElem{VCT-IU04b}.
	\UCpaso [\UCsist] Continúa en el paso \ref{VCT-CU04:vadata2} de la trayectoria principal.
\end{UCtrayectoriaA} 
	\clearpage
\subsection{USR-IU02 Consultar perfil}

\subsubsection{Objetivo}
En la figura \refElem{USR-IU02} se muestra la interfaz correspondiente con la funcionalidad descrita en las
trayectorias del caso de uso \refElem{USR-CU02} , la cual permite al actor la gestión su perfil y la consulta del mismo.

La interfaz esta compuesta ``secciones'' y cada sección corresponde a un formulario diferente, las secciones
son las siguientes:
\begin{itemize}
   \item \textbf{Datos personales}: esta sección tiene como objetivo que el actor actualice su información
   personal,sus datos de contacto y/o habilidades que posee (ver la figura \refElem{USR-IU02a}).
   \item \textbf{Objetivos y metas personales}: esta sección tiene como objetivo que el actor actualice sus objetivos, metas personales y laborales 
   (ver la figura \refElem{USR-IU02b}).
   \item \textbf{Historial académico}: esta sección tiene como objetivo que el actor actualice su información
   académica referente a todos los grados de estudios que tiene hasta la fecha (ver la figura \refElem{USR-IU02c}).
   \item \textbf{Idiomas}: esta sección tiene como objetivo preguntarle que el actor actualice la información de idiomas o en su caso, elimine
   o agregue nuevos idiomas a su perfil  (ver la figura \refElem{USR-IU02d}).
   \item \textbf{Experiencia laboral}: esta sección tiene como objetivo que el actor actualice su información de su experiencia
   laborar que tiene hasta la fecha (ver la figura \refElem{USR-IU02e}).
   \item \textbf{Cursos/Certificaciones}:  esta sección tiene como objetivo que el actor actualice su información de sus Certificaciones
   o cursos que ha tenido durante toda su trayectoria académica y laboral (ver la figura \refElem{USR-IU02f}).
\end{itemize}

\subsubsection{Comandos}
Los siguientes comandos aparecen durante toda la interfaz es decir, cada sección los tiene.

%\Titem \IUPass : Al da clic en el ícono, se muestra la contraseña de lo contrario aparecerá \IUOculto \thinspace sustituyendo cada caracter de la contraseña. \\

\Titem \IUEditar{} : Cuando presiona el ícono, habilita la sección para hacer los datos editables acorde a la sección o elemento seleccionado.
\Titem \IUEliminar{} : Cuando presiona el ícono, habilita la sección para eliminar el elemento seleccionado.
\Titem \IUAgregar{} : Cuando presiona el ícono, habilita la sección para agregar un nuevo el elemento.

\IUfig{.9}{CasosdeUso/USR-CU02/imagenes/USR-IU02.png}{USR-IU02}{Consultar perfil}  
\IUfig{.5}{CasosdeUso/USR-CU02/imagenes/USR-IU02a.png}{USR-IU02a}{Consultar perfil: Datos personales}
\IUfig{.9}{CasosdeUso/USR-CU02/imagenes/USR-IU02b.png}{USR-IU02b}{Consultar perfil: Objetivos y metas personales}  
\IUfig{.9}{CasosdeUso/USR-CU02/imagenes/USR-IU02c.png}{USR-IU02c}{Consultar perfil: Historial académico}  
\IUfig{.9}{CasosdeUso/USR-CU02/imagenes/USR-IU02d.png}{USR-IU02d}{Consultar perfil: Idiomas}
\IUfig{.9}{CasosdeUso/USR-CU02/imagenes/USR-IU02e.png}{USR-IU02e}{Consultar perfil: Experiencia laboral}  
\IUfig{.9}{CasosdeUso/USR-CU02/imagenes/USR-IU02f.png}{USR-IU02f}{Consultar perfil: Cursos/Certificaciones}  


\clearpage


	\clearpage
\begin{UseCase}[]{VCT-CU04}{Registrar vacante}{
	Permite al reclutador de una empresa  publicar una vacante en el sistema durante cierto periodo de tiempo y así poder gestionas las postulaciones 
	que los usarios(canditos) hagan a dicha vacante.
	}
	%----------------------------------------------------------------
	% Datos generales del CU:
	\UCsection{Atributos}
	\UCitem{Actor(es)}{
		Reclutadores.

	}
	\UCitem[admin]{Prioridad}{
		Media
	}
	\UCitem[admin]{Complejidad}{
		Alta
	}
	\UCitem{Precondiciones}{
		El reclutador debe de estar registrado en el sistema.
	}
	\UCitem{Destino}{
		\Titem \refElem{VCT-IU03}
	}
	\UCitem{Reglas de Negocio}{
		\Titem \refIdElem{RN-N001}
		
	}
	\UCitem{Viene de}{
		\refElem{VCT-CU03}
	}	
\end{UseCase}

%Trayectoria Principal
\begin{UCtrayectoria}
	\UCpaso [\UCactor] Da clic en el icono \IUAgregar{} (Publicar vacante) en la interfaz \refElem{VCT-IU03}.
	\UCpaso [\UCsist] Muestra la interfaz \refElem{VCT-IU04a} en la interfaz \refElem{VCT-IU03}.
	\UCpaso [\UCactor] \label{VCT-CU04:vadata} Ingresa el título y el número de plazas de la vacante, ingresa el código postal, estado, municipio y colonia donde se va laboral.\refTray{A}
	\UCpaso [\UCactor] Ingresa el perfil, la experiencia y el tipo de contratación a la que la vacante va dirigida.
	\UCpaso [\UCactor] Ingresa el horario laboral y el rango salarial indicando si es neto o no el salario.
	\UCpaso [\UCsist] Valida que todos los campos marcados como obligatorios hayan sido ingresados de acuerdo a la regla de negocio \refIdElem{RN-N001}. \refTray{B}
	\UCpaso [\UCactor] Ingresa la descripción de la vacante.
	\UCpaso [\UCactor] \label{VCT-CU04:hab}Selecciona las habilidades y su correspondiente experiencia deseadas para la vacante.\refTray{D}
	\UCpaso [\UCactor] Selecciona la fecha de cierre de la vacante.
	\UCpaso [\UCactor] Da clic en el botón \IUbutton{Publicar} en la interfaz \refElem{VCT-IU04b}.\refTray{A}\refTray{C} \refTray{F}
	\UCpaso [\UCsist] \label{VCT-CU04:vadata2}Valida que todos los campos marcados como obligatorios hayan sido ingresados de acuerdo a la regla de negocio \refIdElem{RN-N001}.\refTray{E}
	\UCpaso [\UCsist] Muestra la interfaz \refElem{VCT-IU03} mostrando la nueva vacante registrada.
	\UCpaso [\UCsist] Notifica a los encargados y colaboradores de que hay una nueva vacante por revisar.
\end{UCtrayectoria}

%Trayectorias Alternativas
\begin{UCtrayectoriaA}[Fin de la trayectoria]{A}{El actor decide cancelar el registro.}
	\UCpaso [\UCactor] Da clic en el botón \IUbutton{Cancelar} en la interfaz \refElem{VCT-IU04a}.
	\UCpaso [\UCsist] Muestra el mensaje \refIdElem{MSG6} en la interfaz \refElem{VCT-IU04a}.
	\UCpaso [\UCactor] Da clic en el botón \IUbutton{Sí} en la interfaz \refElem{VCT-IU04a}.\refTray{G}
	\UCpaso [\UCsist] Muestra la interfaz \refElem{VCT-IU03}.
\end{UCtrayectoriaA} 

%Trayectorias Alternativas
\begin{UCtrayectoriaA}[Fin de la trayectoria]{B}{El actor no registro al menos un campo obligatorio.}
	\UCpaso [\UCsist] Muestra el mensaje \refIdElem{MSG4} en la interfaz \refElem{VCT-IU03a} en los campos que no
	fueron ingresados.
	\UCpaso [\UCsist] Continúa en el paso \ref{VCT-CU04:vadata} de la trayectoria principal.
\end{UCtrayectoriaA} 

%Trayectorias Alternativas
\begin{UCtrayectoriaA}[Fin de la trayectoria]{C}{El actor decide regresa a la pantalla anterior.}
	\UCpaso [\UCactor] Da clic en el botón \IUbutton{Regresar} en la interfaz \refElem{VCT-IU04b}.
	\UCpaso [\UCsist] Muestra la interfaz \refElem{VCT-IU04a}.
	\UCpaso [\UCsist] Continúa en el paso \ref{VCT-CU04:vadata} de la trayectoria principal.
\end{UCtrayectoriaA} 

%Trayectorias Alternativas
\begin{UCtrayectoriaA}[Fin de la trayectoria]{D}{El actor decide eliminar una habilidad.}
	\UCpaso [\UCactor] Da clic en el botón \IUbutton{x} de la habilidad seleccionada en la interfaz \refElem{VCT-IU04b}.
	\UCpaso [\UCsist] Muestra la interfaz \refElem{VCT-IU04b}.
	\UCpaso [\UCsist] Continúa en el paso \ref{VCT-CU04:hab} de la trayectoria principal.
\end{UCtrayectoriaA} 

%Trayectorias Alternativas
\begin{UCtrayectoriaA}[Fin de la trayectoria]{E}{El actor no registro al menos un campo obligatorio.}
	\UCpaso [\UCsist] Muestra el mensaje \refIdElem{MSG4} en la interfaz \refElem{VCT-IU03b} en los campos que no
	fueron ingresados.
	\UCpaso [\UCsist] Continúa en el paso \ref{VCT-CU04:vadata2} de la trayectoria principal.
\end{UCtrayectoriaA} 

%Trayectorias Alternativas
\begin{UCtrayectoriaA}[Fin de la trayectoria]{F}{El actor decide cancelar el registro.}
	\UCpaso [\UCactor] Da clic en el botón \IUbutton{Cancelar} en la interfaz \refElem{VCT-IU04b}.
	\UCpaso [\UCsist] Muestra el mensaje \refIdElem{MSG6} en la interfaz \refElem{VCT-IU04b}.
	\UCpaso [\UCactor] Da clic en el botón \IUbutton{Sí} en la interfaz \refElem{VCT-IU04b}.\refTray{H}
	\UCpaso [\UCsist] Muestra la interfaz \refElem{VCT-IU03}.
\end{UCtrayectoriaA} 

%Trayectorias Alternativas
\begin{UCtrayectoriaA}[Fin de la trayectoria]{G}{El actor decide cancelar la acción.}
	\UCpaso [\UCactor] Da clic en el botón \IUbutton{No} en la interfaz \refElem{VCT-IU04a}.
	\UCpaso [\UCsist] Muestra la interfaz \refElem{VCT-IU04a}.
	\UCpaso [\UCsist] Continúa en el paso \ref{VCT-CU04:vadata} de la trayectoria principal.
\end{UCtrayectoriaA} 

%Trayectorias Alternativas
\begin{UCtrayectoriaA}[Fin de la trayectoria]{H}{El actor decide cancelar la acción.}
	\UCpaso [\UCactor] Da clic en el botón \IUbutton{No} en la interfaz \refElem{VCT-IU04b}.
	\UCpaso [\UCsist] Muestra la interfaz \refElem{VCT-IU04b}.
	\UCpaso [\UCsist] Continúa en el paso \ref{VCT-CU04:vadata2} de la trayectoria principal.
\end{UCtrayectoriaA} 
	\clearpage
\subsection{USR-IU02 Consultar perfil}

\subsubsection{Objetivo}
En la figura \refElem{USR-IU02} se muestra la interfaz correspondiente con la funcionalidad descrita en las
trayectorias del caso de uso \refElem{USR-CU02} , la cual permite al actor la gestión su perfil y la consulta del mismo.

La interfaz esta compuesta ``secciones'' y cada sección corresponde a un formulario diferente, las secciones
son las siguientes:
\begin{itemize}
   \item \textbf{Datos personales}: esta sección tiene como objetivo que el actor actualice su información
   personal,sus datos de contacto y/o habilidades que posee (ver la figura \refElem{USR-IU02a}).
   \item \textbf{Objetivos y metas personales}: esta sección tiene como objetivo que el actor actualice sus objetivos, metas personales y laborales 
   (ver la figura \refElem{USR-IU02b}).
   \item \textbf{Historial académico}: esta sección tiene como objetivo que el actor actualice su información
   académica referente a todos los grados de estudios que tiene hasta la fecha (ver la figura \refElem{USR-IU02c}).
   \item \textbf{Idiomas}: esta sección tiene como objetivo preguntarle que el actor actualice la información de idiomas o en su caso, elimine
   o agregue nuevos idiomas a su perfil  (ver la figura \refElem{USR-IU02d}).
   \item \textbf{Experiencia laboral}: esta sección tiene como objetivo que el actor actualice su información de su experiencia
   laborar que tiene hasta la fecha (ver la figura \refElem{USR-IU02e}).
   \item \textbf{Cursos/Certificaciones}:  esta sección tiene como objetivo que el actor actualice su información de sus Certificaciones
   o cursos que ha tenido durante toda su trayectoria académica y laboral (ver la figura \refElem{USR-IU02f}).
\end{itemize}

\subsubsection{Comandos}
Los siguientes comandos aparecen durante toda la interfaz es decir, cada sección los tiene.

%\Titem \IUPass : Al da clic en el ícono, se muestra la contraseña de lo contrario aparecerá \IUOculto \thinspace sustituyendo cada caracter de la contraseña. \\

\Titem \IUEditar{} : Cuando presiona el ícono, habilita la sección para hacer los datos editables acorde a la sección o elemento seleccionado.
\Titem \IUEliminar{} : Cuando presiona el ícono, habilita la sección para eliminar el elemento seleccionado.
\Titem \IUAgregar{} : Cuando presiona el ícono, habilita la sección para agregar un nuevo el elemento.

\IUfig{.9}{CasosdeUso/USR-CU02/imagenes/USR-IU02.png}{USR-IU02}{Consultar perfil}  
\IUfig{.5}{CasosdeUso/USR-CU02/imagenes/USR-IU02a.png}{USR-IU02a}{Consultar perfil: Datos personales}
\IUfig{.9}{CasosdeUso/USR-CU02/imagenes/USR-IU02b.png}{USR-IU02b}{Consultar perfil: Objetivos y metas personales}  
\IUfig{.9}{CasosdeUso/USR-CU02/imagenes/USR-IU02c.png}{USR-IU02c}{Consultar perfil: Historial académico}  
\IUfig{.9}{CasosdeUso/USR-CU02/imagenes/USR-IU02d.png}{USR-IU02d}{Consultar perfil: Idiomas}
\IUfig{.9}{CasosdeUso/USR-CU02/imagenes/USR-IU02e.png}{USR-IU02e}{Consultar perfil: Experiencia laboral}  
\IUfig{.9}{CasosdeUso/USR-CU02/imagenes/USR-IU02f.png}{USR-IU02f}{Consultar perfil: Cursos/Certificaciones}  


\clearpage


	%\clearpage
\begin{UseCase}[]{VCT-CU04}{Registrar vacante}{
	Permite al reclutador de una empresa  publicar una vacante en el sistema durante cierto periodo de tiempo y así poder gestionas las postulaciones 
	que los usarios(canditos) hagan a dicha vacante.
	}
	%----------------------------------------------------------------
	% Datos generales del CU:
	\UCsection{Atributos}
	\UCitem{Actor(es)}{
		Reclutadores.

	}
	\UCitem[admin]{Prioridad}{
		Media
	}
	\UCitem[admin]{Complejidad}{
		Alta
	}
	\UCitem{Precondiciones}{
		El reclutador debe de estar registrado en el sistema.
	}
	\UCitem{Destino}{
		\Titem \refElem{VCT-IU03}
	}
	\UCitem{Reglas de Negocio}{
		\Titem \refIdElem{RN-N001}
		
	}
	\UCitem{Viene de}{
		\refElem{VCT-CU03}
	}	
\end{UseCase}

%Trayectoria Principal
\begin{UCtrayectoria}
	\UCpaso [\UCactor] Da clic en el icono \IUAgregar{} (Publicar vacante) en la interfaz \refElem{VCT-IU03}.
	\UCpaso [\UCsist] Muestra la interfaz \refElem{VCT-IU04a} en la interfaz \refElem{VCT-IU03}.
	\UCpaso [\UCactor] \label{VCT-CU04:vadata} Ingresa el título y el número de plazas de la vacante, ingresa el código postal, estado, municipio y colonia donde se va laboral.\refTray{A}
	\UCpaso [\UCactor] Ingresa el perfil, la experiencia y el tipo de contratación a la que la vacante va dirigida.
	\UCpaso [\UCactor] Ingresa el horario laboral y el rango salarial indicando si es neto o no el salario.
	\UCpaso [\UCsist] Valida que todos los campos marcados como obligatorios hayan sido ingresados de acuerdo a la regla de negocio \refIdElem{RN-N001}. \refTray{B}
	\UCpaso [\UCactor] Ingresa la descripción de la vacante.
	\UCpaso [\UCactor] \label{VCT-CU04:hab}Selecciona las habilidades y su correspondiente experiencia deseadas para la vacante.\refTray{D}
	\UCpaso [\UCactor] Selecciona la fecha de cierre de la vacante.
	\UCpaso [\UCactor] Da clic en el botón \IUbutton{Publicar} en la interfaz \refElem{VCT-IU04b}.\refTray{A}\refTray{C} \refTray{F}
	\UCpaso [\UCsist] \label{VCT-CU04:vadata2}Valida que todos los campos marcados como obligatorios hayan sido ingresados de acuerdo a la regla de negocio \refIdElem{RN-N001}.\refTray{E}
	\UCpaso [\UCsist] Muestra la interfaz \refElem{VCT-IU03} mostrando la nueva vacante registrada.
	\UCpaso [\UCsist] Notifica a los encargados y colaboradores de que hay una nueva vacante por revisar.
\end{UCtrayectoria}

%Trayectorias Alternativas
\begin{UCtrayectoriaA}[Fin de la trayectoria]{A}{El actor decide cancelar el registro.}
	\UCpaso [\UCactor] Da clic en el botón \IUbutton{Cancelar} en la interfaz \refElem{VCT-IU04a}.
	\UCpaso [\UCsist] Muestra el mensaje \refIdElem{MSG6} en la interfaz \refElem{VCT-IU04a}.
	\UCpaso [\UCactor] Da clic en el botón \IUbutton{Sí} en la interfaz \refElem{VCT-IU04a}.\refTray{G}
	\UCpaso [\UCsist] Muestra la interfaz \refElem{VCT-IU03}.
\end{UCtrayectoriaA} 

%Trayectorias Alternativas
\begin{UCtrayectoriaA}[Fin de la trayectoria]{B}{El actor no registro al menos un campo obligatorio.}
	\UCpaso [\UCsist] Muestra el mensaje \refIdElem{MSG4} en la interfaz \refElem{VCT-IU03a} en los campos que no
	fueron ingresados.
	\UCpaso [\UCsist] Continúa en el paso \ref{VCT-CU04:vadata} de la trayectoria principal.
\end{UCtrayectoriaA} 

%Trayectorias Alternativas
\begin{UCtrayectoriaA}[Fin de la trayectoria]{C}{El actor decide regresa a la pantalla anterior.}
	\UCpaso [\UCactor] Da clic en el botón \IUbutton{Regresar} en la interfaz \refElem{VCT-IU04b}.
	\UCpaso [\UCsist] Muestra la interfaz \refElem{VCT-IU04a}.
	\UCpaso [\UCsist] Continúa en el paso \ref{VCT-CU04:vadata} de la trayectoria principal.
\end{UCtrayectoriaA} 

%Trayectorias Alternativas
\begin{UCtrayectoriaA}[Fin de la trayectoria]{D}{El actor decide eliminar una habilidad.}
	\UCpaso [\UCactor] Da clic en el botón \IUbutton{x} de la habilidad seleccionada en la interfaz \refElem{VCT-IU04b}.
	\UCpaso [\UCsist] Muestra la interfaz \refElem{VCT-IU04b}.
	\UCpaso [\UCsist] Continúa en el paso \ref{VCT-CU04:hab} de la trayectoria principal.
\end{UCtrayectoriaA} 

%Trayectorias Alternativas
\begin{UCtrayectoriaA}[Fin de la trayectoria]{E}{El actor no registro al menos un campo obligatorio.}
	\UCpaso [\UCsist] Muestra el mensaje \refIdElem{MSG4} en la interfaz \refElem{VCT-IU03b} en los campos que no
	fueron ingresados.
	\UCpaso [\UCsist] Continúa en el paso \ref{VCT-CU04:vadata2} de la trayectoria principal.
\end{UCtrayectoriaA} 

%Trayectorias Alternativas
\begin{UCtrayectoriaA}[Fin de la trayectoria]{F}{El actor decide cancelar el registro.}
	\UCpaso [\UCactor] Da clic en el botón \IUbutton{Cancelar} en la interfaz \refElem{VCT-IU04b}.
	\UCpaso [\UCsist] Muestra el mensaje \refIdElem{MSG6} en la interfaz \refElem{VCT-IU04b}.
	\UCpaso [\UCactor] Da clic en el botón \IUbutton{Sí} en la interfaz \refElem{VCT-IU04b}.\refTray{H}
	\UCpaso [\UCsist] Muestra la interfaz \refElem{VCT-IU03}.
\end{UCtrayectoriaA} 

%Trayectorias Alternativas
\begin{UCtrayectoriaA}[Fin de la trayectoria]{G}{El actor decide cancelar la acción.}
	\UCpaso [\UCactor] Da clic en el botón \IUbutton{No} en la interfaz \refElem{VCT-IU04a}.
	\UCpaso [\UCsist] Muestra la interfaz \refElem{VCT-IU04a}.
	\UCpaso [\UCsist] Continúa en el paso \ref{VCT-CU04:vadata} de la trayectoria principal.
\end{UCtrayectoriaA} 

%Trayectorias Alternativas
\begin{UCtrayectoriaA}[Fin de la trayectoria]{H}{El actor decide cancelar la acción.}
	\UCpaso [\UCactor] Da clic en el botón \IUbutton{No} en la interfaz \refElem{VCT-IU04b}.
	\UCpaso [\UCsist] Muestra la interfaz \refElem{VCT-IU04b}.
	\UCpaso [\UCsist] Continúa en el paso \ref{VCT-CU04:vadata2} de la trayectoria principal.
\end{UCtrayectoriaA} 
	%\clearpage
\begin{UseCase}[]{VCT-CU04}{Registrar vacante}{
	Permite al reclutador de una empresa  publicar una vacante en el sistema durante cierto periodo de tiempo y así poder gestionas las postulaciones 
	que los usarios(canditos) hagan a dicha vacante.
	}
	%----------------------------------------------------------------
	% Datos generales del CU:
	\UCsection{Atributos}
	\UCitem{Actor(es)}{
		Reclutadores.

	}
	\UCitem[admin]{Prioridad}{
		Media
	}
	\UCitem[admin]{Complejidad}{
		Alta
	}
	\UCitem{Precondiciones}{
		El reclutador debe de estar registrado en el sistema.
	}
	\UCitem{Destino}{
		\Titem \refElem{VCT-IU03}
	}
	\UCitem{Reglas de Negocio}{
		\Titem \refIdElem{RN-N001}
		
	}
	\UCitem{Viene de}{
		\refElem{VCT-CU03}
	}	
\end{UseCase}

%Trayectoria Principal
\begin{UCtrayectoria}
	\UCpaso [\UCactor] Da clic en el icono \IUAgregar{} (Publicar vacante) en la interfaz \refElem{VCT-IU03}.
	\UCpaso [\UCsist] Muestra la interfaz \refElem{VCT-IU04a} en la interfaz \refElem{VCT-IU03}.
	\UCpaso [\UCactor] \label{VCT-CU04:vadata} Ingresa el título y el número de plazas de la vacante, ingresa el código postal, estado, municipio y colonia donde se va laboral.\refTray{A}
	\UCpaso [\UCactor] Ingresa el perfil, la experiencia y el tipo de contratación a la que la vacante va dirigida.
	\UCpaso [\UCactor] Ingresa el horario laboral y el rango salarial indicando si es neto o no el salario.
	\UCpaso [\UCsist] Valida que todos los campos marcados como obligatorios hayan sido ingresados de acuerdo a la regla de negocio \refIdElem{RN-N001}. \refTray{B}
	\UCpaso [\UCactor] Ingresa la descripción de la vacante.
	\UCpaso [\UCactor] \label{VCT-CU04:hab}Selecciona las habilidades y su correspondiente experiencia deseadas para la vacante.\refTray{D}
	\UCpaso [\UCactor] Selecciona la fecha de cierre de la vacante.
	\UCpaso [\UCactor] Da clic en el botón \IUbutton{Publicar} en la interfaz \refElem{VCT-IU04b}.\refTray{A}\refTray{C} \refTray{F}
	\UCpaso [\UCsist] \label{VCT-CU04:vadata2}Valida que todos los campos marcados como obligatorios hayan sido ingresados de acuerdo a la regla de negocio \refIdElem{RN-N001}.\refTray{E}
	\UCpaso [\UCsist] Muestra la interfaz \refElem{VCT-IU03} mostrando la nueva vacante registrada.
	\UCpaso [\UCsist] Notifica a los encargados y colaboradores de que hay una nueva vacante por revisar.
\end{UCtrayectoria}

%Trayectorias Alternativas
\begin{UCtrayectoriaA}[Fin de la trayectoria]{A}{El actor decide cancelar el registro.}
	\UCpaso [\UCactor] Da clic en el botón \IUbutton{Cancelar} en la interfaz \refElem{VCT-IU04a}.
	\UCpaso [\UCsist] Muestra el mensaje \refIdElem{MSG6} en la interfaz \refElem{VCT-IU04a}.
	\UCpaso [\UCactor] Da clic en el botón \IUbutton{Sí} en la interfaz \refElem{VCT-IU04a}.\refTray{G}
	\UCpaso [\UCsist] Muestra la interfaz \refElem{VCT-IU03}.
\end{UCtrayectoriaA} 

%Trayectorias Alternativas
\begin{UCtrayectoriaA}[Fin de la trayectoria]{B}{El actor no registro al menos un campo obligatorio.}
	\UCpaso [\UCsist] Muestra el mensaje \refIdElem{MSG4} en la interfaz \refElem{VCT-IU03a} en los campos que no
	fueron ingresados.
	\UCpaso [\UCsist] Continúa en el paso \ref{VCT-CU04:vadata} de la trayectoria principal.
\end{UCtrayectoriaA} 

%Trayectorias Alternativas
\begin{UCtrayectoriaA}[Fin de la trayectoria]{C}{El actor decide regresa a la pantalla anterior.}
	\UCpaso [\UCactor] Da clic en el botón \IUbutton{Regresar} en la interfaz \refElem{VCT-IU04b}.
	\UCpaso [\UCsist] Muestra la interfaz \refElem{VCT-IU04a}.
	\UCpaso [\UCsist] Continúa en el paso \ref{VCT-CU04:vadata} de la trayectoria principal.
\end{UCtrayectoriaA} 

%Trayectorias Alternativas
\begin{UCtrayectoriaA}[Fin de la trayectoria]{D}{El actor decide eliminar una habilidad.}
	\UCpaso [\UCactor] Da clic en el botón \IUbutton{x} de la habilidad seleccionada en la interfaz \refElem{VCT-IU04b}.
	\UCpaso [\UCsist] Muestra la interfaz \refElem{VCT-IU04b}.
	\UCpaso [\UCsist] Continúa en el paso \ref{VCT-CU04:hab} de la trayectoria principal.
\end{UCtrayectoriaA} 

%Trayectorias Alternativas
\begin{UCtrayectoriaA}[Fin de la trayectoria]{E}{El actor no registro al menos un campo obligatorio.}
	\UCpaso [\UCsist] Muestra el mensaje \refIdElem{MSG4} en la interfaz \refElem{VCT-IU03b} en los campos que no
	fueron ingresados.
	\UCpaso [\UCsist] Continúa en el paso \ref{VCT-CU04:vadata2} de la trayectoria principal.
\end{UCtrayectoriaA} 

%Trayectorias Alternativas
\begin{UCtrayectoriaA}[Fin de la trayectoria]{F}{El actor decide cancelar el registro.}
	\UCpaso [\UCactor] Da clic en el botón \IUbutton{Cancelar} en la interfaz \refElem{VCT-IU04b}.
	\UCpaso [\UCsist] Muestra el mensaje \refIdElem{MSG6} en la interfaz \refElem{VCT-IU04b}.
	\UCpaso [\UCactor] Da clic en el botón \IUbutton{Sí} en la interfaz \refElem{VCT-IU04b}.\refTray{H}
	\UCpaso [\UCsist] Muestra la interfaz \refElem{VCT-IU03}.
\end{UCtrayectoriaA} 

%Trayectorias Alternativas
\begin{UCtrayectoriaA}[Fin de la trayectoria]{G}{El actor decide cancelar la acción.}
	\UCpaso [\UCactor] Da clic en el botón \IUbutton{No} en la interfaz \refElem{VCT-IU04a}.
	\UCpaso [\UCsist] Muestra la interfaz \refElem{VCT-IU04a}.
	\UCpaso [\UCsist] Continúa en el paso \ref{VCT-CU04:vadata} de la trayectoria principal.
\end{UCtrayectoriaA} 

%Trayectorias Alternativas
\begin{UCtrayectoriaA}[Fin de la trayectoria]{H}{El actor decide cancelar la acción.}
	\UCpaso [\UCactor] Da clic en el botón \IUbutton{No} en la interfaz \refElem{VCT-IU04b}.
	\UCpaso [\UCsist] Muestra la interfaz \refElem{VCT-IU04b}.
	\UCpaso [\UCsist] Continúa en el paso \ref{VCT-CU04:vadata2} de la trayectoria principal.
\end{UCtrayectoriaA} 

\section{Módulo de Usuarios}
	En la figura \ref{adcu:usr} se muestra el diagrama de casos de uso del módulo de usuarios del sistema.

	\begin{figure}[hbtp!]
		\begin{center}
			\includegraphics[width=1 \textwidth]{anexos/imagenes/CUUSR.png}
		\end{center}
		
		\caption{Diagrama de casos de uso del \textit{Módulo de Usuarios}.}
		\label{adcu:usr}
	\end{figure}

	\begin{itemize}
        \item Los casos de uso \IUazul{} , son aquellos que se pertenecen a esta primera entrega del proyecto.
        \item Los casos de uso \IUblanco{}, se tienen planeados para la segunda entrega del proyecto.
    \end{itemize} 

	\clearpage
\begin{UseCase}[]{VCT-CU04}{Registrar vacante}{
	Permite al reclutador de una empresa  publicar una vacante en el sistema durante cierto periodo de tiempo y así poder gestionas las postulaciones 
	que los usarios(canditos) hagan a dicha vacante.
	}
	%----------------------------------------------------------------
	% Datos generales del CU:
	\UCsection{Atributos}
	\UCitem{Actor(es)}{
		Reclutadores.

	}
	\UCitem[admin]{Prioridad}{
		Media
	}
	\UCitem[admin]{Complejidad}{
		Alta
	}
	\UCitem{Precondiciones}{
		El reclutador debe de estar registrado en el sistema.
	}
	\UCitem{Destino}{
		\Titem \refElem{VCT-IU03}
	}
	\UCitem{Reglas de Negocio}{
		\Titem \refIdElem{RN-N001}
		
	}
	\UCitem{Viene de}{
		\refElem{VCT-CU03}
	}	
\end{UseCase}

%Trayectoria Principal
\begin{UCtrayectoria}
	\UCpaso [\UCactor] Da clic en el icono \IUAgregar{} (Publicar vacante) en la interfaz \refElem{VCT-IU03}.
	\UCpaso [\UCsist] Muestra la interfaz \refElem{VCT-IU04a} en la interfaz \refElem{VCT-IU03}.
	\UCpaso [\UCactor] \label{VCT-CU04:vadata} Ingresa el título y el número de plazas de la vacante, ingresa el código postal, estado, municipio y colonia donde se va laboral.\refTray{A}
	\UCpaso [\UCactor] Ingresa el perfil, la experiencia y el tipo de contratación a la que la vacante va dirigida.
	\UCpaso [\UCactor] Ingresa el horario laboral y el rango salarial indicando si es neto o no el salario.
	\UCpaso [\UCsist] Valida que todos los campos marcados como obligatorios hayan sido ingresados de acuerdo a la regla de negocio \refIdElem{RN-N001}. \refTray{B}
	\UCpaso [\UCactor] Ingresa la descripción de la vacante.
	\UCpaso [\UCactor] \label{VCT-CU04:hab}Selecciona las habilidades y su correspondiente experiencia deseadas para la vacante.\refTray{D}
	\UCpaso [\UCactor] Selecciona la fecha de cierre de la vacante.
	\UCpaso [\UCactor] Da clic en el botón \IUbutton{Publicar} en la interfaz \refElem{VCT-IU04b}.\refTray{A}\refTray{C} \refTray{F}
	\UCpaso [\UCsist] \label{VCT-CU04:vadata2}Valida que todos los campos marcados como obligatorios hayan sido ingresados de acuerdo a la regla de negocio \refIdElem{RN-N001}.\refTray{E}
	\UCpaso [\UCsist] Muestra la interfaz \refElem{VCT-IU03} mostrando la nueva vacante registrada.
	\UCpaso [\UCsist] Notifica a los encargados y colaboradores de que hay una nueva vacante por revisar.
\end{UCtrayectoria}

%Trayectorias Alternativas
\begin{UCtrayectoriaA}[Fin de la trayectoria]{A}{El actor decide cancelar el registro.}
	\UCpaso [\UCactor] Da clic en el botón \IUbutton{Cancelar} en la interfaz \refElem{VCT-IU04a}.
	\UCpaso [\UCsist] Muestra el mensaje \refIdElem{MSG6} en la interfaz \refElem{VCT-IU04a}.
	\UCpaso [\UCactor] Da clic en el botón \IUbutton{Sí} en la interfaz \refElem{VCT-IU04a}.\refTray{G}
	\UCpaso [\UCsist] Muestra la interfaz \refElem{VCT-IU03}.
\end{UCtrayectoriaA} 

%Trayectorias Alternativas
\begin{UCtrayectoriaA}[Fin de la trayectoria]{B}{El actor no registro al menos un campo obligatorio.}
	\UCpaso [\UCsist] Muestra el mensaje \refIdElem{MSG4} en la interfaz \refElem{VCT-IU03a} en los campos que no
	fueron ingresados.
	\UCpaso [\UCsist] Continúa en el paso \ref{VCT-CU04:vadata} de la trayectoria principal.
\end{UCtrayectoriaA} 

%Trayectorias Alternativas
\begin{UCtrayectoriaA}[Fin de la trayectoria]{C}{El actor decide regresa a la pantalla anterior.}
	\UCpaso [\UCactor] Da clic en el botón \IUbutton{Regresar} en la interfaz \refElem{VCT-IU04b}.
	\UCpaso [\UCsist] Muestra la interfaz \refElem{VCT-IU04a}.
	\UCpaso [\UCsist] Continúa en el paso \ref{VCT-CU04:vadata} de la trayectoria principal.
\end{UCtrayectoriaA} 

%Trayectorias Alternativas
\begin{UCtrayectoriaA}[Fin de la trayectoria]{D}{El actor decide eliminar una habilidad.}
	\UCpaso [\UCactor] Da clic en el botón \IUbutton{x} de la habilidad seleccionada en la interfaz \refElem{VCT-IU04b}.
	\UCpaso [\UCsist] Muestra la interfaz \refElem{VCT-IU04b}.
	\UCpaso [\UCsist] Continúa en el paso \ref{VCT-CU04:hab} de la trayectoria principal.
\end{UCtrayectoriaA} 

%Trayectorias Alternativas
\begin{UCtrayectoriaA}[Fin de la trayectoria]{E}{El actor no registro al menos un campo obligatorio.}
	\UCpaso [\UCsist] Muestra el mensaje \refIdElem{MSG4} en la interfaz \refElem{VCT-IU03b} en los campos que no
	fueron ingresados.
	\UCpaso [\UCsist] Continúa en el paso \ref{VCT-CU04:vadata2} de la trayectoria principal.
\end{UCtrayectoriaA} 

%Trayectorias Alternativas
\begin{UCtrayectoriaA}[Fin de la trayectoria]{F}{El actor decide cancelar el registro.}
	\UCpaso [\UCactor] Da clic en el botón \IUbutton{Cancelar} en la interfaz \refElem{VCT-IU04b}.
	\UCpaso [\UCsist] Muestra el mensaje \refIdElem{MSG6} en la interfaz \refElem{VCT-IU04b}.
	\UCpaso [\UCactor] Da clic en el botón \IUbutton{Sí} en la interfaz \refElem{VCT-IU04b}.\refTray{H}
	\UCpaso [\UCsist] Muestra la interfaz \refElem{VCT-IU03}.
\end{UCtrayectoriaA} 

%Trayectorias Alternativas
\begin{UCtrayectoriaA}[Fin de la trayectoria]{G}{El actor decide cancelar la acción.}
	\UCpaso [\UCactor] Da clic en el botón \IUbutton{No} en la interfaz \refElem{VCT-IU04a}.
	\UCpaso [\UCsist] Muestra la interfaz \refElem{VCT-IU04a}.
	\UCpaso [\UCsist] Continúa en el paso \ref{VCT-CU04:vadata} de la trayectoria principal.
\end{UCtrayectoriaA} 

%Trayectorias Alternativas
\begin{UCtrayectoriaA}[Fin de la trayectoria]{H}{El actor decide cancelar la acción.}
	\UCpaso [\UCactor] Da clic en el botón \IUbutton{No} en la interfaz \refElem{VCT-IU04b}.
	\UCpaso [\UCsist] Muestra la interfaz \refElem{VCT-IU04b}.
	\UCpaso [\UCsist] Continúa en el paso \ref{VCT-CU04:vadata2} de la trayectoria principal.
\end{UCtrayectoriaA} 
	\clearpage
\subsection{USR-IU02 Consultar perfil}

\subsubsection{Objetivo}
En la figura \refElem{USR-IU02} se muestra la interfaz correspondiente con la funcionalidad descrita en las
trayectorias del caso de uso \refElem{USR-CU02} , la cual permite al actor la gestión su perfil y la consulta del mismo.

La interfaz esta compuesta ``secciones'' y cada sección corresponde a un formulario diferente, las secciones
son las siguientes:
\begin{itemize}
   \item \textbf{Datos personales}: esta sección tiene como objetivo que el actor actualice su información
   personal,sus datos de contacto y/o habilidades que posee (ver la figura \refElem{USR-IU02a}).
   \item \textbf{Objetivos y metas personales}: esta sección tiene como objetivo que el actor actualice sus objetivos, metas personales y laborales 
   (ver la figura \refElem{USR-IU02b}).
   \item \textbf{Historial académico}: esta sección tiene como objetivo que el actor actualice su información
   académica referente a todos los grados de estudios que tiene hasta la fecha (ver la figura \refElem{USR-IU02c}).
   \item \textbf{Idiomas}: esta sección tiene como objetivo preguntarle que el actor actualice la información de idiomas o en su caso, elimine
   o agregue nuevos idiomas a su perfil  (ver la figura \refElem{USR-IU02d}).
   \item \textbf{Experiencia laboral}: esta sección tiene como objetivo que el actor actualice su información de su experiencia
   laborar que tiene hasta la fecha (ver la figura \refElem{USR-IU02e}).
   \item \textbf{Cursos/Certificaciones}:  esta sección tiene como objetivo que el actor actualice su información de sus Certificaciones
   o cursos que ha tenido durante toda su trayectoria académica y laboral (ver la figura \refElem{USR-IU02f}).
\end{itemize}

\subsubsection{Comandos}
Los siguientes comandos aparecen durante toda la interfaz es decir, cada sección los tiene.

%\Titem \IUPass : Al da clic en el ícono, se muestra la contraseña de lo contrario aparecerá \IUOculto \thinspace sustituyendo cada caracter de la contraseña. \\

\Titem \IUEditar{} : Cuando presiona el ícono, habilita la sección para hacer los datos editables acorde a la sección o elemento seleccionado.
\Titem \IUEliminar{} : Cuando presiona el ícono, habilita la sección para eliminar el elemento seleccionado.
\Titem \IUAgregar{} : Cuando presiona el ícono, habilita la sección para agregar un nuevo el elemento.

\IUfig{.9}{CasosdeUso/USR-CU02/imagenes/USR-IU02.png}{USR-IU02}{Consultar perfil}  
\IUfig{.5}{CasosdeUso/USR-CU02/imagenes/USR-IU02a.png}{USR-IU02a}{Consultar perfil: Datos personales}
\IUfig{.9}{CasosdeUso/USR-CU02/imagenes/USR-IU02b.png}{USR-IU02b}{Consultar perfil: Objetivos y metas personales}  
\IUfig{.9}{CasosdeUso/USR-CU02/imagenes/USR-IU02c.png}{USR-IU02c}{Consultar perfil: Historial académico}  
\IUfig{.9}{CasosdeUso/USR-CU02/imagenes/USR-IU02d.png}{USR-IU02d}{Consultar perfil: Idiomas}
\IUfig{.9}{CasosdeUso/USR-CU02/imagenes/USR-IU02e.png}{USR-IU02e}{Consultar perfil: Experiencia laboral}  
\IUfig{.9}{CasosdeUso/USR-CU02/imagenes/USR-IU02f.png}{USR-IU02f}{Consultar perfil: Cursos/Certificaciones}  


\clearpage


	%\clearpage
\begin{UseCase}[]{VCT-CU04}{Registrar vacante}{
	Permite al reclutador de una empresa  publicar una vacante en el sistema durante cierto periodo de tiempo y así poder gestionas las postulaciones 
	que los usarios(canditos) hagan a dicha vacante.
	}
	%----------------------------------------------------------------
	% Datos generales del CU:
	\UCsection{Atributos}
	\UCitem{Actor(es)}{
		Reclutadores.

	}
	\UCitem[admin]{Prioridad}{
		Media
	}
	\UCitem[admin]{Complejidad}{
		Alta
	}
	\UCitem{Precondiciones}{
		El reclutador debe de estar registrado en el sistema.
	}
	\UCitem{Destino}{
		\Titem \refElem{VCT-IU03}
	}
	\UCitem{Reglas de Negocio}{
		\Titem \refIdElem{RN-N001}
		
	}
	\UCitem{Viene de}{
		\refElem{VCT-CU03}
	}	
\end{UseCase}

%Trayectoria Principal
\begin{UCtrayectoria}
	\UCpaso [\UCactor] Da clic en el icono \IUAgregar{} (Publicar vacante) en la interfaz \refElem{VCT-IU03}.
	\UCpaso [\UCsist] Muestra la interfaz \refElem{VCT-IU04a} en la interfaz \refElem{VCT-IU03}.
	\UCpaso [\UCactor] \label{VCT-CU04:vadata} Ingresa el título y el número de plazas de la vacante, ingresa el código postal, estado, municipio y colonia donde se va laboral.\refTray{A}
	\UCpaso [\UCactor] Ingresa el perfil, la experiencia y el tipo de contratación a la que la vacante va dirigida.
	\UCpaso [\UCactor] Ingresa el horario laboral y el rango salarial indicando si es neto o no el salario.
	\UCpaso [\UCsist] Valida que todos los campos marcados como obligatorios hayan sido ingresados de acuerdo a la regla de negocio \refIdElem{RN-N001}. \refTray{B}
	\UCpaso [\UCactor] Ingresa la descripción de la vacante.
	\UCpaso [\UCactor] \label{VCT-CU04:hab}Selecciona las habilidades y su correspondiente experiencia deseadas para la vacante.\refTray{D}
	\UCpaso [\UCactor] Selecciona la fecha de cierre de la vacante.
	\UCpaso [\UCactor] Da clic en el botón \IUbutton{Publicar} en la interfaz \refElem{VCT-IU04b}.\refTray{A}\refTray{C} \refTray{F}
	\UCpaso [\UCsist] \label{VCT-CU04:vadata2}Valida que todos los campos marcados como obligatorios hayan sido ingresados de acuerdo a la regla de negocio \refIdElem{RN-N001}.\refTray{E}
	\UCpaso [\UCsist] Muestra la interfaz \refElem{VCT-IU03} mostrando la nueva vacante registrada.
	\UCpaso [\UCsist] Notifica a los encargados y colaboradores de que hay una nueva vacante por revisar.
\end{UCtrayectoria}

%Trayectorias Alternativas
\begin{UCtrayectoriaA}[Fin de la trayectoria]{A}{El actor decide cancelar el registro.}
	\UCpaso [\UCactor] Da clic en el botón \IUbutton{Cancelar} en la interfaz \refElem{VCT-IU04a}.
	\UCpaso [\UCsist] Muestra el mensaje \refIdElem{MSG6} en la interfaz \refElem{VCT-IU04a}.
	\UCpaso [\UCactor] Da clic en el botón \IUbutton{Sí} en la interfaz \refElem{VCT-IU04a}.\refTray{G}
	\UCpaso [\UCsist] Muestra la interfaz \refElem{VCT-IU03}.
\end{UCtrayectoriaA} 

%Trayectorias Alternativas
\begin{UCtrayectoriaA}[Fin de la trayectoria]{B}{El actor no registro al menos un campo obligatorio.}
	\UCpaso [\UCsist] Muestra el mensaje \refIdElem{MSG4} en la interfaz \refElem{VCT-IU03a} en los campos que no
	fueron ingresados.
	\UCpaso [\UCsist] Continúa en el paso \ref{VCT-CU04:vadata} de la trayectoria principal.
\end{UCtrayectoriaA} 

%Trayectorias Alternativas
\begin{UCtrayectoriaA}[Fin de la trayectoria]{C}{El actor decide regresa a la pantalla anterior.}
	\UCpaso [\UCactor] Da clic en el botón \IUbutton{Regresar} en la interfaz \refElem{VCT-IU04b}.
	\UCpaso [\UCsist] Muestra la interfaz \refElem{VCT-IU04a}.
	\UCpaso [\UCsist] Continúa en el paso \ref{VCT-CU04:vadata} de la trayectoria principal.
\end{UCtrayectoriaA} 

%Trayectorias Alternativas
\begin{UCtrayectoriaA}[Fin de la trayectoria]{D}{El actor decide eliminar una habilidad.}
	\UCpaso [\UCactor] Da clic en el botón \IUbutton{x} de la habilidad seleccionada en la interfaz \refElem{VCT-IU04b}.
	\UCpaso [\UCsist] Muestra la interfaz \refElem{VCT-IU04b}.
	\UCpaso [\UCsist] Continúa en el paso \ref{VCT-CU04:hab} de la trayectoria principal.
\end{UCtrayectoriaA} 

%Trayectorias Alternativas
\begin{UCtrayectoriaA}[Fin de la trayectoria]{E}{El actor no registro al menos un campo obligatorio.}
	\UCpaso [\UCsist] Muestra el mensaje \refIdElem{MSG4} en la interfaz \refElem{VCT-IU03b} en los campos que no
	fueron ingresados.
	\UCpaso [\UCsist] Continúa en el paso \ref{VCT-CU04:vadata2} de la trayectoria principal.
\end{UCtrayectoriaA} 

%Trayectorias Alternativas
\begin{UCtrayectoriaA}[Fin de la trayectoria]{F}{El actor decide cancelar el registro.}
	\UCpaso [\UCactor] Da clic en el botón \IUbutton{Cancelar} en la interfaz \refElem{VCT-IU04b}.
	\UCpaso [\UCsist] Muestra el mensaje \refIdElem{MSG6} en la interfaz \refElem{VCT-IU04b}.
	\UCpaso [\UCactor] Da clic en el botón \IUbutton{Sí} en la interfaz \refElem{VCT-IU04b}.\refTray{H}
	\UCpaso [\UCsist] Muestra la interfaz \refElem{VCT-IU03}.
\end{UCtrayectoriaA} 

%Trayectorias Alternativas
\begin{UCtrayectoriaA}[Fin de la trayectoria]{G}{El actor decide cancelar la acción.}
	\UCpaso [\UCactor] Da clic en el botón \IUbutton{No} en la interfaz \refElem{VCT-IU04a}.
	\UCpaso [\UCsist] Muestra la interfaz \refElem{VCT-IU04a}.
	\UCpaso [\UCsist] Continúa en el paso \ref{VCT-CU04:vadata} de la trayectoria principal.
\end{UCtrayectoriaA} 

%Trayectorias Alternativas
\begin{UCtrayectoriaA}[Fin de la trayectoria]{H}{El actor decide cancelar la acción.}
	\UCpaso [\UCactor] Da clic en el botón \IUbutton{No} en la interfaz \refElem{VCT-IU04b}.
	\UCpaso [\UCsist] Muestra la interfaz \refElem{VCT-IU04b}.
	\UCpaso [\UCsist] Continúa en el paso \ref{VCT-CU04:vadata2} de la trayectoria principal.
\end{UCtrayectoriaA} 
	%\clearpage
\begin{UseCase}[]{VCT-CU04}{Registrar vacante}{
	Permite al reclutador de una empresa  publicar una vacante en el sistema durante cierto periodo de tiempo y así poder gestionas las postulaciones 
	que los usarios(canditos) hagan a dicha vacante.
	}
	%----------------------------------------------------------------
	% Datos generales del CU:
	\UCsection{Atributos}
	\UCitem{Actor(es)}{
		Reclutadores.

	}
	\UCitem[admin]{Prioridad}{
		Media
	}
	\UCitem[admin]{Complejidad}{
		Alta
	}
	\UCitem{Precondiciones}{
		El reclutador debe de estar registrado en el sistema.
	}
	\UCitem{Destino}{
		\Titem \refElem{VCT-IU03}
	}
	\UCitem{Reglas de Negocio}{
		\Titem \refIdElem{RN-N001}
		
	}
	\UCitem{Viene de}{
		\refElem{VCT-CU03}
	}	
\end{UseCase}

%Trayectoria Principal
\begin{UCtrayectoria}
	\UCpaso [\UCactor] Da clic en el icono \IUAgregar{} (Publicar vacante) en la interfaz \refElem{VCT-IU03}.
	\UCpaso [\UCsist] Muestra la interfaz \refElem{VCT-IU04a} en la interfaz \refElem{VCT-IU03}.
	\UCpaso [\UCactor] \label{VCT-CU04:vadata} Ingresa el título y el número de plazas de la vacante, ingresa el código postal, estado, municipio y colonia donde se va laboral.\refTray{A}
	\UCpaso [\UCactor] Ingresa el perfil, la experiencia y el tipo de contratación a la que la vacante va dirigida.
	\UCpaso [\UCactor] Ingresa el horario laboral y el rango salarial indicando si es neto o no el salario.
	\UCpaso [\UCsist] Valida que todos los campos marcados como obligatorios hayan sido ingresados de acuerdo a la regla de negocio \refIdElem{RN-N001}. \refTray{B}
	\UCpaso [\UCactor] Ingresa la descripción de la vacante.
	\UCpaso [\UCactor] \label{VCT-CU04:hab}Selecciona las habilidades y su correspondiente experiencia deseadas para la vacante.\refTray{D}
	\UCpaso [\UCactor] Selecciona la fecha de cierre de la vacante.
	\UCpaso [\UCactor] Da clic en el botón \IUbutton{Publicar} en la interfaz \refElem{VCT-IU04b}.\refTray{A}\refTray{C} \refTray{F}
	\UCpaso [\UCsist] \label{VCT-CU04:vadata2}Valida que todos los campos marcados como obligatorios hayan sido ingresados de acuerdo a la regla de negocio \refIdElem{RN-N001}.\refTray{E}
	\UCpaso [\UCsist] Muestra la interfaz \refElem{VCT-IU03} mostrando la nueva vacante registrada.
	\UCpaso [\UCsist] Notifica a los encargados y colaboradores de que hay una nueva vacante por revisar.
\end{UCtrayectoria}

%Trayectorias Alternativas
\begin{UCtrayectoriaA}[Fin de la trayectoria]{A}{El actor decide cancelar el registro.}
	\UCpaso [\UCactor] Da clic en el botón \IUbutton{Cancelar} en la interfaz \refElem{VCT-IU04a}.
	\UCpaso [\UCsist] Muestra el mensaje \refIdElem{MSG6} en la interfaz \refElem{VCT-IU04a}.
	\UCpaso [\UCactor] Da clic en el botón \IUbutton{Sí} en la interfaz \refElem{VCT-IU04a}.\refTray{G}
	\UCpaso [\UCsist] Muestra la interfaz \refElem{VCT-IU03}.
\end{UCtrayectoriaA} 

%Trayectorias Alternativas
\begin{UCtrayectoriaA}[Fin de la trayectoria]{B}{El actor no registro al menos un campo obligatorio.}
	\UCpaso [\UCsist] Muestra el mensaje \refIdElem{MSG4} en la interfaz \refElem{VCT-IU03a} en los campos que no
	fueron ingresados.
	\UCpaso [\UCsist] Continúa en el paso \ref{VCT-CU04:vadata} de la trayectoria principal.
\end{UCtrayectoriaA} 

%Trayectorias Alternativas
\begin{UCtrayectoriaA}[Fin de la trayectoria]{C}{El actor decide regresa a la pantalla anterior.}
	\UCpaso [\UCactor] Da clic en el botón \IUbutton{Regresar} en la interfaz \refElem{VCT-IU04b}.
	\UCpaso [\UCsist] Muestra la interfaz \refElem{VCT-IU04a}.
	\UCpaso [\UCsist] Continúa en el paso \ref{VCT-CU04:vadata} de la trayectoria principal.
\end{UCtrayectoriaA} 

%Trayectorias Alternativas
\begin{UCtrayectoriaA}[Fin de la trayectoria]{D}{El actor decide eliminar una habilidad.}
	\UCpaso [\UCactor] Da clic en el botón \IUbutton{x} de la habilidad seleccionada en la interfaz \refElem{VCT-IU04b}.
	\UCpaso [\UCsist] Muestra la interfaz \refElem{VCT-IU04b}.
	\UCpaso [\UCsist] Continúa en el paso \ref{VCT-CU04:hab} de la trayectoria principal.
\end{UCtrayectoriaA} 

%Trayectorias Alternativas
\begin{UCtrayectoriaA}[Fin de la trayectoria]{E}{El actor no registro al menos un campo obligatorio.}
	\UCpaso [\UCsist] Muestra el mensaje \refIdElem{MSG4} en la interfaz \refElem{VCT-IU03b} en los campos que no
	fueron ingresados.
	\UCpaso [\UCsist] Continúa en el paso \ref{VCT-CU04:vadata2} de la trayectoria principal.
\end{UCtrayectoriaA} 

%Trayectorias Alternativas
\begin{UCtrayectoriaA}[Fin de la trayectoria]{F}{El actor decide cancelar el registro.}
	\UCpaso [\UCactor] Da clic en el botón \IUbutton{Cancelar} en la interfaz \refElem{VCT-IU04b}.
	\UCpaso [\UCsist] Muestra el mensaje \refIdElem{MSG6} en la interfaz \refElem{VCT-IU04b}.
	\UCpaso [\UCactor] Da clic en el botón \IUbutton{Sí} en la interfaz \refElem{VCT-IU04b}.\refTray{H}
	\UCpaso [\UCsist] Muestra la interfaz \refElem{VCT-IU03}.
\end{UCtrayectoriaA} 

%Trayectorias Alternativas
\begin{UCtrayectoriaA}[Fin de la trayectoria]{G}{El actor decide cancelar la acción.}
	\UCpaso [\UCactor] Da clic en el botón \IUbutton{No} en la interfaz \refElem{VCT-IU04a}.
	\UCpaso [\UCsist] Muestra la interfaz \refElem{VCT-IU04a}.
	\UCpaso [\UCsist] Continúa en el paso \ref{VCT-CU04:vadata} de la trayectoria principal.
\end{UCtrayectoriaA} 

%Trayectorias Alternativas
\begin{UCtrayectoriaA}[Fin de la trayectoria]{H}{El actor decide cancelar la acción.}
	\UCpaso [\UCactor] Da clic en el botón \IUbutton{No} en la interfaz \refElem{VCT-IU04b}.
	\UCpaso [\UCsist] Muestra la interfaz \refElem{VCT-IU04b}.
	\UCpaso [\UCsist] Continúa en el paso \ref{VCT-CU04:vadata2} de la trayectoria principal.
\end{UCtrayectoriaA} 
	%\clearpage
\begin{UseCase}[]{VCT-CU04}{Registrar vacante}{
	Permite al reclutador de una empresa  publicar una vacante en el sistema durante cierto periodo de tiempo y así poder gestionas las postulaciones 
	que los usarios(canditos) hagan a dicha vacante.
	}
	%----------------------------------------------------------------
	% Datos generales del CU:
	\UCsection{Atributos}
	\UCitem{Actor(es)}{
		Reclutadores.

	}
	\UCitem[admin]{Prioridad}{
		Media
	}
	\UCitem[admin]{Complejidad}{
		Alta
	}
	\UCitem{Precondiciones}{
		El reclutador debe de estar registrado en el sistema.
	}
	\UCitem{Destino}{
		\Titem \refElem{VCT-IU03}
	}
	\UCitem{Reglas de Negocio}{
		\Titem \refIdElem{RN-N001}
		
	}
	\UCitem{Viene de}{
		\refElem{VCT-CU03}
	}	
\end{UseCase}

%Trayectoria Principal
\begin{UCtrayectoria}
	\UCpaso [\UCactor] Da clic en el icono \IUAgregar{} (Publicar vacante) en la interfaz \refElem{VCT-IU03}.
	\UCpaso [\UCsist] Muestra la interfaz \refElem{VCT-IU04a} en la interfaz \refElem{VCT-IU03}.
	\UCpaso [\UCactor] \label{VCT-CU04:vadata} Ingresa el título y el número de plazas de la vacante, ingresa el código postal, estado, municipio y colonia donde se va laboral.\refTray{A}
	\UCpaso [\UCactor] Ingresa el perfil, la experiencia y el tipo de contratación a la que la vacante va dirigida.
	\UCpaso [\UCactor] Ingresa el horario laboral y el rango salarial indicando si es neto o no el salario.
	\UCpaso [\UCsist] Valida que todos los campos marcados como obligatorios hayan sido ingresados de acuerdo a la regla de negocio \refIdElem{RN-N001}. \refTray{B}
	\UCpaso [\UCactor] Ingresa la descripción de la vacante.
	\UCpaso [\UCactor] \label{VCT-CU04:hab}Selecciona las habilidades y su correspondiente experiencia deseadas para la vacante.\refTray{D}
	\UCpaso [\UCactor] Selecciona la fecha de cierre de la vacante.
	\UCpaso [\UCactor] Da clic en el botón \IUbutton{Publicar} en la interfaz \refElem{VCT-IU04b}.\refTray{A}\refTray{C} \refTray{F}
	\UCpaso [\UCsist] \label{VCT-CU04:vadata2}Valida que todos los campos marcados como obligatorios hayan sido ingresados de acuerdo a la regla de negocio \refIdElem{RN-N001}.\refTray{E}
	\UCpaso [\UCsist] Muestra la interfaz \refElem{VCT-IU03} mostrando la nueva vacante registrada.
	\UCpaso [\UCsist] Notifica a los encargados y colaboradores de que hay una nueva vacante por revisar.
\end{UCtrayectoria}

%Trayectorias Alternativas
\begin{UCtrayectoriaA}[Fin de la trayectoria]{A}{El actor decide cancelar el registro.}
	\UCpaso [\UCactor] Da clic en el botón \IUbutton{Cancelar} en la interfaz \refElem{VCT-IU04a}.
	\UCpaso [\UCsist] Muestra el mensaje \refIdElem{MSG6} en la interfaz \refElem{VCT-IU04a}.
	\UCpaso [\UCactor] Da clic en el botón \IUbutton{Sí} en la interfaz \refElem{VCT-IU04a}.\refTray{G}
	\UCpaso [\UCsist] Muestra la interfaz \refElem{VCT-IU03}.
\end{UCtrayectoriaA} 

%Trayectorias Alternativas
\begin{UCtrayectoriaA}[Fin de la trayectoria]{B}{El actor no registro al menos un campo obligatorio.}
	\UCpaso [\UCsist] Muestra el mensaje \refIdElem{MSG4} en la interfaz \refElem{VCT-IU03a} en los campos que no
	fueron ingresados.
	\UCpaso [\UCsist] Continúa en el paso \ref{VCT-CU04:vadata} de la trayectoria principal.
\end{UCtrayectoriaA} 

%Trayectorias Alternativas
\begin{UCtrayectoriaA}[Fin de la trayectoria]{C}{El actor decide regresa a la pantalla anterior.}
	\UCpaso [\UCactor] Da clic en el botón \IUbutton{Regresar} en la interfaz \refElem{VCT-IU04b}.
	\UCpaso [\UCsist] Muestra la interfaz \refElem{VCT-IU04a}.
	\UCpaso [\UCsist] Continúa en el paso \ref{VCT-CU04:vadata} de la trayectoria principal.
\end{UCtrayectoriaA} 

%Trayectorias Alternativas
\begin{UCtrayectoriaA}[Fin de la trayectoria]{D}{El actor decide eliminar una habilidad.}
	\UCpaso [\UCactor] Da clic en el botón \IUbutton{x} de la habilidad seleccionada en la interfaz \refElem{VCT-IU04b}.
	\UCpaso [\UCsist] Muestra la interfaz \refElem{VCT-IU04b}.
	\UCpaso [\UCsist] Continúa en el paso \ref{VCT-CU04:hab} de la trayectoria principal.
\end{UCtrayectoriaA} 

%Trayectorias Alternativas
\begin{UCtrayectoriaA}[Fin de la trayectoria]{E}{El actor no registro al menos un campo obligatorio.}
	\UCpaso [\UCsist] Muestra el mensaje \refIdElem{MSG4} en la interfaz \refElem{VCT-IU03b} en los campos que no
	fueron ingresados.
	\UCpaso [\UCsist] Continúa en el paso \ref{VCT-CU04:vadata2} de la trayectoria principal.
\end{UCtrayectoriaA} 

%Trayectorias Alternativas
\begin{UCtrayectoriaA}[Fin de la trayectoria]{F}{El actor decide cancelar el registro.}
	\UCpaso [\UCactor] Da clic en el botón \IUbutton{Cancelar} en la interfaz \refElem{VCT-IU04b}.
	\UCpaso [\UCsist] Muestra el mensaje \refIdElem{MSG6} en la interfaz \refElem{VCT-IU04b}.
	\UCpaso [\UCactor] Da clic en el botón \IUbutton{Sí} en la interfaz \refElem{VCT-IU04b}.\refTray{H}
	\UCpaso [\UCsist] Muestra la interfaz \refElem{VCT-IU03}.
\end{UCtrayectoriaA} 

%Trayectorias Alternativas
\begin{UCtrayectoriaA}[Fin de la trayectoria]{G}{El actor decide cancelar la acción.}
	\UCpaso [\UCactor] Da clic en el botón \IUbutton{No} en la interfaz \refElem{VCT-IU04a}.
	\UCpaso [\UCsist] Muestra la interfaz \refElem{VCT-IU04a}.
	\UCpaso [\UCsist] Continúa en el paso \ref{VCT-CU04:vadata} de la trayectoria principal.
\end{UCtrayectoriaA} 

%Trayectorias Alternativas
\begin{UCtrayectoriaA}[Fin de la trayectoria]{H}{El actor decide cancelar la acción.}
	\UCpaso [\UCactor] Da clic en el botón \IUbutton{No} en la interfaz \refElem{VCT-IU04b}.
	\UCpaso [\UCsist] Muestra la interfaz \refElem{VCT-IU04b}.
	\UCpaso [\UCsist] Continúa en el paso \ref{VCT-CU04:vadata2} de la trayectoria principal.
\end{UCtrayectoriaA} 
	%\clearpage
\begin{UseCase}[]{VCT-CU04}{Registrar vacante}{
	Permite al reclutador de una empresa  publicar una vacante en el sistema durante cierto periodo de tiempo y así poder gestionas las postulaciones 
	que los usarios(canditos) hagan a dicha vacante.
	}
	%----------------------------------------------------------------
	% Datos generales del CU:
	\UCsection{Atributos}
	\UCitem{Actor(es)}{
		Reclutadores.

	}
	\UCitem[admin]{Prioridad}{
		Media
	}
	\UCitem[admin]{Complejidad}{
		Alta
	}
	\UCitem{Precondiciones}{
		El reclutador debe de estar registrado en el sistema.
	}
	\UCitem{Destino}{
		\Titem \refElem{VCT-IU03}
	}
	\UCitem{Reglas de Negocio}{
		\Titem \refIdElem{RN-N001}
		
	}
	\UCitem{Viene de}{
		\refElem{VCT-CU03}
	}	
\end{UseCase}

%Trayectoria Principal
\begin{UCtrayectoria}
	\UCpaso [\UCactor] Da clic en el icono \IUAgregar{} (Publicar vacante) en la interfaz \refElem{VCT-IU03}.
	\UCpaso [\UCsist] Muestra la interfaz \refElem{VCT-IU04a} en la interfaz \refElem{VCT-IU03}.
	\UCpaso [\UCactor] \label{VCT-CU04:vadata} Ingresa el título y el número de plazas de la vacante, ingresa el código postal, estado, municipio y colonia donde se va laboral.\refTray{A}
	\UCpaso [\UCactor] Ingresa el perfil, la experiencia y el tipo de contratación a la que la vacante va dirigida.
	\UCpaso [\UCactor] Ingresa el horario laboral y el rango salarial indicando si es neto o no el salario.
	\UCpaso [\UCsist] Valida que todos los campos marcados como obligatorios hayan sido ingresados de acuerdo a la regla de negocio \refIdElem{RN-N001}. \refTray{B}
	\UCpaso [\UCactor] Ingresa la descripción de la vacante.
	\UCpaso [\UCactor] \label{VCT-CU04:hab}Selecciona las habilidades y su correspondiente experiencia deseadas para la vacante.\refTray{D}
	\UCpaso [\UCactor] Selecciona la fecha de cierre de la vacante.
	\UCpaso [\UCactor] Da clic en el botón \IUbutton{Publicar} en la interfaz \refElem{VCT-IU04b}.\refTray{A}\refTray{C} \refTray{F}
	\UCpaso [\UCsist] \label{VCT-CU04:vadata2}Valida que todos los campos marcados como obligatorios hayan sido ingresados de acuerdo a la regla de negocio \refIdElem{RN-N001}.\refTray{E}
	\UCpaso [\UCsist] Muestra la interfaz \refElem{VCT-IU03} mostrando la nueva vacante registrada.
	\UCpaso [\UCsist] Notifica a los encargados y colaboradores de que hay una nueva vacante por revisar.
\end{UCtrayectoria}

%Trayectorias Alternativas
\begin{UCtrayectoriaA}[Fin de la trayectoria]{A}{El actor decide cancelar el registro.}
	\UCpaso [\UCactor] Da clic en el botón \IUbutton{Cancelar} en la interfaz \refElem{VCT-IU04a}.
	\UCpaso [\UCsist] Muestra el mensaje \refIdElem{MSG6} en la interfaz \refElem{VCT-IU04a}.
	\UCpaso [\UCactor] Da clic en el botón \IUbutton{Sí} en la interfaz \refElem{VCT-IU04a}.\refTray{G}
	\UCpaso [\UCsist] Muestra la interfaz \refElem{VCT-IU03}.
\end{UCtrayectoriaA} 

%Trayectorias Alternativas
\begin{UCtrayectoriaA}[Fin de la trayectoria]{B}{El actor no registro al menos un campo obligatorio.}
	\UCpaso [\UCsist] Muestra el mensaje \refIdElem{MSG4} en la interfaz \refElem{VCT-IU03a} en los campos que no
	fueron ingresados.
	\UCpaso [\UCsist] Continúa en el paso \ref{VCT-CU04:vadata} de la trayectoria principal.
\end{UCtrayectoriaA} 

%Trayectorias Alternativas
\begin{UCtrayectoriaA}[Fin de la trayectoria]{C}{El actor decide regresa a la pantalla anterior.}
	\UCpaso [\UCactor] Da clic en el botón \IUbutton{Regresar} en la interfaz \refElem{VCT-IU04b}.
	\UCpaso [\UCsist] Muestra la interfaz \refElem{VCT-IU04a}.
	\UCpaso [\UCsist] Continúa en el paso \ref{VCT-CU04:vadata} de la trayectoria principal.
\end{UCtrayectoriaA} 

%Trayectorias Alternativas
\begin{UCtrayectoriaA}[Fin de la trayectoria]{D}{El actor decide eliminar una habilidad.}
	\UCpaso [\UCactor] Da clic en el botón \IUbutton{x} de la habilidad seleccionada en la interfaz \refElem{VCT-IU04b}.
	\UCpaso [\UCsist] Muestra la interfaz \refElem{VCT-IU04b}.
	\UCpaso [\UCsist] Continúa en el paso \ref{VCT-CU04:hab} de la trayectoria principal.
\end{UCtrayectoriaA} 

%Trayectorias Alternativas
\begin{UCtrayectoriaA}[Fin de la trayectoria]{E}{El actor no registro al menos un campo obligatorio.}
	\UCpaso [\UCsist] Muestra el mensaje \refIdElem{MSG4} en la interfaz \refElem{VCT-IU03b} en los campos que no
	fueron ingresados.
	\UCpaso [\UCsist] Continúa en el paso \ref{VCT-CU04:vadata2} de la trayectoria principal.
\end{UCtrayectoriaA} 

%Trayectorias Alternativas
\begin{UCtrayectoriaA}[Fin de la trayectoria]{F}{El actor decide cancelar el registro.}
	\UCpaso [\UCactor] Da clic en el botón \IUbutton{Cancelar} en la interfaz \refElem{VCT-IU04b}.
	\UCpaso [\UCsist] Muestra el mensaje \refIdElem{MSG6} en la interfaz \refElem{VCT-IU04b}.
	\UCpaso [\UCactor] Da clic en el botón \IUbutton{Sí} en la interfaz \refElem{VCT-IU04b}.\refTray{H}
	\UCpaso [\UCsist] Muestra la interfaz \refElem{VCT-IU03}.
\end{UCtrayectoriaA} 

%Trayectorias Alternativas
\begin{UCtrayectoriaA}[Fin de la trayectoria]{G}{El actor decide cancelar la acción.}
	\UCpaso [\UCactor] Da clic en el botón \IUbutton{No} en la interfaz \refElem{VCT-IU04a}.
	\UCpaso [\UCsist] Muestra la interfaz \refElem{VCT-IU04a}.
	\UCpaso [\UCsist] Continúa en el paso \ref{VCT-CU04:vadata} de la trayectoria principal.
\end{UCtrayectoriaA} 

%Trayectorias Alternativas
\begin{UCtrayectoriaA}[Fin de la trayectoria]{H}{El actor decide cancelar la acción.}
	\UCpaso [\UCactor] Da clic en el botón \IUbutton{No} en la interfaz \refElem{VCT-IU04b}.
	\UCpaso [\UCsist] Muestra la interfaz \refElem{VCT-IU04b}.
	\UCpaso [\UCsist] Continúa en el paso \ref{VCT-CU04:vadata2} de la trayectoria principal.
\end{UCtrayectoriaA} 
	%\clearpage
\begin{UseCase}[]{VCT-CU04}{Registrar vacante}{
	Permite al reclutador de una empresa  publicar una vacante en el sistema durante cierto periodo de tiempo y así poder gestionas las postulaciones 
	que los usarios(canditos) hagan a dicha vacante.
	}
	%----------------------------------------------------------------
	% Datos generales del CU:
	\UCsection{Atributos}
	\UCitem{Actor(es)}{
		Reclutadores.

	}
	\UCitem[admin]{Prioridad}{
		Media
	}
	\UCitem[admin]{Complejidad}{
		Alta
	}
	\UCitem{Precondiciones}{
		El reclutador debe de estar registrado en el sistema.
	}
	\UCitem{Destino}{
		\Titem \refElem{VCT-IU03}
	}
	\UCitem{Reglas de Negocio}{
		\Titem \refIdElem{RN-N001}
		
	}
	\UCitem{Viene de}{
		\refElem{VCT-CU03}
	}	
\end{UseCase}

%Trayectoria Principal
\begin{UCtrayectoria}
	\UCpaso [\UCactor] Da clic en el icono \IUAgregar{} (Publicar vacante) en la interfaz \refElem{VCT-IU03}.
	\UCpaso [\UCsist] Muestra la interfaz \refElem{VCT-IU04a} en la interfaz \refElem{VCT-IU03}.
	\UCpaso [\UCactor] \label{VCT-CU04:vadata} Ingresa el título y el número de plazas de la vacante, ingresa el código postal, estado, municipio y colonia donde se va laboral.\refTray{A}
	\UCpaso [\UCactor] Ingresa el perfil, la experiencia y el tipo de contratación a la que la vacante va dirigida.
	\UCpaso [\UCactor] Ingresa el horario laboral y el rango salarial indicando si es neto o no el salario.
	\UCpaso [\UCsist] Valida que todos los campos marcados como obligatorios hayan sido ingresados de acuerdo a la regla de negocio \refIdElem{RN-N001}. \refTray{B}
	\UCpaso [\UCactor] Ingresa la descripción de la vacante.
	\UCpaso [\UCactor] \label{VCT-CU04:hab}Selecciona las habilidades y su correspondiente experiencia deseadas para la vacante.\refTray{D}
	\UCpaso [\UCactor] Selecciona la fecha de cierre de la vacante.
	\UCpaso [\UCactor] Da clic en el botón \IUbutton{Publicar} en la interfaz \refElem{VCT-IU04b}.\refTray{A}\refTray{C} \refTray{F}
	\UCpaso [\UCsist] \label{VCT-CU04:vadata2}Valida que todos los campos marcados como obligatorios hayan sido ingresados de acuerdo a la regla de negocio \refIdElem{RN-N001}.\refTray{E}
	\UCpaso [\UCsist] Muestra la interfaz \refElem{VCT-IU03} mostrando la nueva vacante registrada.
	\UCpaso [\UCsist] Notifica a los encargados y colaboradores de que hay una nueva vacante por revisar.
\end{UCtrayectoria}

%Trayectorias Alternativas
\begin{UCtrayectoriaA}[Fin de la trayectoria]{A}{El actor decide cancelar el registro.}
	\UCpaso [\UCactor] Da clic en el botón \IUbutton{Cancelar} en la interfaz \refElem{VCT-IU04a}.
	\UCpaso [\UCsist] Muestra el mensaje \refIdElem{MSG6} en la interfaz \refElem{VCT-IU04a}.
	\UCpaso [\UCactor] Da clic en el botón \IUbutton{Sí} en la interfaz \refElem{VCT-IU04a}.\refTray{G}
	\UCpaso [\UCsist] Muestra la interfaz \refElem{VCT-IU03}.
\end{UCtrayectoriaA} 

%Trayectorias Alternativas
\begin{UCtrayectoriaA}[Fin de la trayectoria]{B}{El actor no registro al menos un campo obligatorio.}
	\UCpaso [\UCsist] Muestra el mensaje \refIdElem{MSG4} en la interfaz \refElem{VCT-IU03a} en los campos que no
	fueron ingresados.
	\UCpaso [\UCsist] Continúa en el paso \ref{VCT-CU04:vadata} de la trayectoria principal.
\end{UCtrayectoriaA} 

%Trayectorias Alternativas
\begin{UCtrayectoriaA}[Fin de la trayectoria]{C}{El actor decide regresa a la pantalla anterior.}
	\UCpaso [\UCactor] Da clic en el botón \IUbutton{Regresar} en la interfaz \refElem{VCT-IU04b}.
	\UCpaso [\UCsist] Muestra la interfaz \refElem{VCT-IU04a}.
	\UCpaso [\UCsist] Continúa en el paso \ref{VCT-CU04:vadata} de la trayectoria principal.
\end{UCtrayectoriaA} 

%Trayectorias Alternativas
\begin{UCtrayectoriaA}[Fin de la trayectoria]{D}{El actor decide eliminar una habilidad.}
	\UCpaso [\UCactor] Da clic en el botón \IUbutton{x} de la habilidad seleccionada en la interfaz \refElem{VCT-IU04b}.
	\UCpaso [\UCsist] Muestra la interfaz \refElem{VCT-IU04b}.
	\UCpaso [\UCsist] Continúa en el paso \ref{VCT-CU04:hab} de la trayectoria principal.
\end{UCtrayectoriaA} 

%Trayectorias Alternativas
\begin{UCtrayectoriaA}[Fin de la trayectoria]{E}{El actor no registro al menos un campo obligatorio.}
	\UCpaso [\UCsist] Muestra el mensaje \refIdElem{MSG4} en la interfaz \refElem{VCT-IU03b} en los campos que no
	fueron ingresados.
	\UCpaso [\UCsist] Continúa en el paso \ref{VCT-CU04:vadata2} de la trayectoria principal.
\end{UCtrayectoriaA} 

%Trayectorias Alternativas
\begin{UCtrayectoriaA}[Fin de la trayectoria]{F}{El actor decide cancelar el registro.}
	\UCpaso [\UCactor] Da clic en el botón \IUbutton{Cancelar} en la interfaz \refElem{VCT-IU04b}.
	\UCpaso [\UCsist] Muestra el mensaje \refIdElem{MSG6} en la interfaz \refElem{VCT-IU04b}.
	\UCpaso [\UCactor] Da clic en el botón \IUbutton{Sí} en la interfaz \refElem{VCT-IU04b}.\refTray{H}
	\UCpaso [\UCsist] Muestra la interfaz \refElem{VCT-IU03}.
\end{UCtrayectoriaA} 

%Trayectorias Alternativas
\begin{UCtrayectoriaA}[Fin de la trayectoria]{G}{El actor decide cancelar la acción.}
	\UCpaso [\UCactor] Da clic en el botón \IUbutton{No} en la interfaz \refElem{VCT-IU04a}.
	\UCpaso [\UCsist] Muestra la interfaz \refElem{VCT-IU04a}.
	\UCpaso [\UCsist] Continúa en el paso \ref{VCT-CU04:vadata} de la trayectoria principal.
\end{UCtrayectoriaA} 

%Trayectorias Alternativas
\begin{UCtrayectoriaA}[Fin de la trayectoria]{H}{El actor decide cancelar la acción.}
	\UCpaso [\UCactor] Da clic en el botón \IUbutton{No} en la interfaz \refElem{VCT-IU04b}.
	\UCpaso [\UCsist] Muestra la interfaz \refElem{VCT-IU04b}.
	\UCpaso [\UCsist] Continúa en el paso \ref{VCT-CU04:vadata2} de la trayectoria principal.
\end{UCtrayectoriaA} 
	%\clearpage
\begin{UseCase}[]{VCT-CU04}{Registrar vacante}{
	Permite al reclutador de una empresa  publicar una vacante en el sistema durante cierto periodo de tiempo y así poder gestionas las postulaciones 
	que los usarios(canditos) hagan a dicha vacante.
	}
	%----------------------------------------------------------------
	% Datos generales del CU:
	\UCsection{Atributos}
	\UCitem{Actor(es)}{
		Reclutadores.

	}
	\UCitem[admin]{Prioridad}{
		Media
	}
	\UCitem[admin]{Complejidad}{
		Alta
	}
	\UCitem{Precondiciones}{
		El reclutador debe de estar registrado en el sistema.
	}
	\UCitem{Destino}{
		\Titem \refElem{VCT-IU03}
	}
	\UCitem{Reglas de Negocio}{
		\Titem \refIdElem{RN-N001}
		
	}
	\UCitem{Viene de}{
		\refElem{VCT-CU03}
	}	
\end{UseCase}

%Trayectoria Principal
\begin{UCtrayectoria}
	\UCpaso [\UCactor] Da clic en el icono \IUAgregar{} (Publicar vacante) en la interfaz \refElem{VCT-IU03}.
	\UCpaso [\UCsist] Muestra la interfaz \refElem{VCT-IU04a} en la interfaz \refElem{VCT-IU03}.
	\UCpaso [\UCactor] \label{VCT-CU04:vadata} Ingresa el título y el número de plazas de la vacante, ingresa el código postal, estado, municipio y colonia donde se va laboral.\refTray{A}
	\UCpaso [\UCactor] Ingresa el perfil, la experiencia y el tipo de contratación a la que la vacante va dirigida.
	\UCpaso [\UCactor] Ingresa el horario laboral y el rango salarial indicando si es neto o no el salario.
	\UCpaso [\UCsist] Valida que todos los campos marcados como obligatorios hayan sido ingresados de acuerdo a la regla de negocio \refIdElem{RN-N001}. \refTray{B}
	\UCpaso [\UCactor] Ingresa la descripción de la vacante.
	\UCpaso [\UCactor] \label{VCT-CU04:hab}Selecciona las habilidades y su correspondiente experiencia deseadas para la vacante.\refTray{D}
	\UCpaso [\UCactor] Selecciona la fecha de cierre de la vacante.
	\UCpaso [\UCactor] Da clic en el botón \IUbutton{Publicar} en la interfaz \refElem{VCT-IU04b}.\refTray{A}\refTray{C} \refTray{F}
	\UCpaso [\UCsist] \label{VCT-CU04:vadata2}Valida que todos los campos marcados como obligatorios hayan sido ingresados de acuerdo a la regla de negocio \refIdElem{RN-N001}.\refTray{E}
	\UCpaso [\UCsist] Muestra la interfaz \refElem{VCT-IU03} mostrando la nueva vacante registrada.
	\UCpaso [\UCsist] Notifica a los encargados y colaboradores de que hay una nueva vacante por revisar.
\end{UCtrayectoria}

%Trayectorias Alternativas
\begin{UCtrayectoriaA}[Fin de la trayectoria]{A}{El actor decide cancelar el registro.}
	\UCpaso [\UCactor] Da clic en el botón \IUbutton{Cancelar} en la interfaz \refElem{VCT-IU04a}.
	\UCpaso [\UCsist] Muestra el mensaje \refIdElem{MSG6} en la interfaz \refElem{VCT-IU04a}.
	\UCpaso [\UCactor] Da clic en el botón \IUbutton{Sí} en la interfaz \refElem{VCT-IU04a}.\refTray{G}
	\UCpaso [\UCsist] Muestra la interfaz \refElem{VCT-IU03}.
\end{UCtrayectoriaA} 

%Trayectorias Alternativas
\begin{UCtrayectoriaA}[Fin de la trayectoria]{B}{El actor no registro al menos un campo obligatorio.}
	\UCpaso [\UCsist] Muestra el mensaje \refIdElem{MSG4} en la interfaz \refElem{VCT-IU03a} en los campos que no
	fueron ingresados.
	\UCpaso [\UCsist] Continúa en el paso \ref{VCT-CU04:vadata} de la trayectoria principal.
\end{UCtrayectoriaA} 

%Trayectorias Alternativas
\begin{UCtrayectoriaA}[Fin de la trayectoria]{C}{El actor decide regresa a la pantalla anterior.}
	\UCpaso [\UCactor] Da clic en el botón \IUbutton{Regresar} en la interfaz \refElem{VCT-IU04b}.
	\UCpaso [\UCsist] Muestra la interfaz \refElem{VCT-IU04a}.
	\UCpaso [\UCsist] Continúa en el paso \ref{VCT-CU04:vadata} de la trayectoria principal.
\end{UCtrayectoriaA} 

%Trayectorias Alternativas
\begin{UCtrayectoriaA}[Fin de la trayectoria]{D}{El actor decide eliminar una habilidad.}
	\UCpaso [\UCactor] Da clic en el botón \IUbutton{x} de la habilidad seleccionada en la interfaz \refElem{VCT-IU04b}.
	\UCpaso [\UCsist] Muestra la interfaz \refElem{VCT-IU04b}.
	\UCpaso [\UCsist] Continúa en el paso \ref{VCT-CU04:hab} de la trayectoria principal.
\end{UCtrayectoriaA} 

%Trayectorias Alternativas
\begin{UCtrayectoriaA}[Fin de la trayectoria]{E}{El actor no registro al menos un campo obligatorio.}
	\UCpaso [\UCsist] Muestra el mensaje \refIdElem{MSG4} en la interfaz \refElem{VCT-IU03b} en los campos que no
	fueron ingresados.
	\UCpaso [\UCsist] Continúa en el paso \ref{VCT-CU04:vadata2} de la trayectoria principal.
\end{UCtrayectoriaA} 

%Trayectorias Alternativas
\begin{UCtrayectoriaA}[Fin de la trayectoria]{F}{El actor decide cancelar el registro.}
	\UCpaso [\UCactor] Da clic en el botón \IUbutton{Cancelar} en la interfaz \refElem{VCT-IU04b}.
	\UCpaso [\UCsist] Muestra el mensaje \refIdElem{MSG6} en la interfaz \refElem{VCT-IU04b}.
	\UCpaso [\UCactor] Da clic en el botón \IUbutton{Sí} en la interfaz \refElem{VCT-IU04b}.\refTray{H}
	\UCpaso [\UCsist] Muestra la interfaz \refElem{VCT-IU03}.
\end{UCtrayectoriaA} 

%Trayectorias Alternativas
\begin{UCtrayectoriaA}[Fin de la trayectoria]{G}{El actor decide cancelar la acción.}
	\UCpaso [\UCactor] Da clic en el botón \IUbutton{No} en la interfaz \refElem{VCT-IU04a}.
	\UCpaso [\UCsist] Muestra la interfaz \refElem{VCT-IU04a}.
	\UCpaso [\UCsist] Continúa en el paso \ref{VCT-CU04:vadata} de la trayectoria principal.
\end{UCtrayectoriaA} 

%Trayectorias Alternativas
\begin{UCtrayectoriaA}[Fin de la trayectoria]{H}{El actor decide cancelar la acción.}
	\UCpaso [\UCactor] Da clic en el botón \IUbutton{No} en la interfaz \refElem{VCT-IU04b}.
	\UCpaso [\UCsist] Muestra la interfaz \refElem{VCT-IU04b}.
	\UCpaso [\UCsist] Continúa en el paso \ref{VCT-CU04:vadata2} de la trayectoria principal.
\end{UCtrayectoriaA} 
	%\clearpage
\begin{UseCase}[]{VCT-CU04}{Registrar vacante}{
	Permite al reclutador de una empresa  publicar una vacante en el sistema durante cierto periodo de tiempo y así poder gestionas las postulaciones 
	que los usarios(canditos) hagan a dicha vacante.
	}
	%----------------------------------------------------------------
	% Datos generales del CU:
	\UCsection{Atributos}
	\UCitem{Actor(es)}{
		Reclutadores.

	}
	\UCitem[admin]{Prioridad}{
		Media
	}
	\UCitem[admin]{Complejidad}{
		Alta
	}
	\UCitem{Precondiciones}{
		El reclutador debe de estar registrado en el sistema.
	}
	\UCitem{Destino}{
		\Titem \refElem{VCT-IU03}
	}
	\UCitem{Reglas de Negocio}{
		\Titem \refIdElem{RN-N001}
		
	}
	\UCitem{Viene de}{
		\refElem{VCT-CU03}
	}	
\end{UseCase}

%Trayectoria Principal
\begin{UCtrayectoria}
	\UCpaso [\UCactor] Da clic en el icono \IUAgregar{} (Publicar vacante) en la interfaz \refElem{VCT-IU03}.
	\UCpaso [\UCsist] Muestra la interfaz \refElem{VCT-IU04a} en la interfaz \refElem{VCT-IU03}.
	\UCpaso [\UCactor] \label{VCT-CU04:vadata} Ingresa el título y el número de plazas de la vacante, ingresa el código postal, estado, municipio y colonia donde se va laboral.\refTray{A}
	\UCpaso [\UCactor] Ingresa el perfil, la experiencia y el tipo de contratación a la que la vacante va dirigida.
	\UCpaso [\UCactor] Ingresa el horario laboral y el rango salarial indicando si es neto o no el salario.
	\UCpaso [\UCsist] Valida que todos los campos marcados como obligatorios hayan sido ingresados de acuerdo a la regla de negocio \refIdElem{RN-N001}. \refTray{B}
	\UCpaso [\UCactor] Ingresa la descripción de la vacante.
	\UCpaso [\UCactor] \label{VCT-CU04:hab}Selecciona las habilidades y su correspondiente experiencia deseadas para la vacante.\refTray{D}
	\UCpaso [\UCactor] Selecciona la fecha de cierre de la vacante.
	\UCpaso [\UCactor] Da clic en el botón \IUbutton{Publicar} en la interfaz \refElem{VCT-IU04b}.\refTray{A}\refTray{C} \refTray{F}
	\UCpaso [\UCsist] \label{VCT-CU04:vadata2}Valida que todos los campos marcados como obligatorios hayan sido ingresados de acuerdo a la regla de negocio \refIdElem{RN-N001}.\refTray{E}
	\UCpaso [\UCsist] Muestra la interfaz \refElem{VCT-IU03} mostrando la nueva vacante registrada.
	\UCpaso [\UCsist] Notifica a los encargados y colaboradores de que hay una nueva vacante por revisar.
\end{UCtrayectoria}

%Trayectorias Alternativas
\begin{UCtrayectoriaA}[Fin de la trayectoria]{A}{El actor decide cancelar el registro.}
	\UCpaso [\UCactor] Da clic en el botón \IUbutton{Cancelar} en la interfaz \refElem{VCT-IU04a}.
	\UCpaso [\UCsist] Muestra el mensaje \refIdElem{MSG6} en la interfaz \refElem{VCT-IU04a}.
	\UCpaso [\UCactor] Da clic en el botón \IUbutton{Sí} en la interfaz \refElem{VCT-IU04a}.\refTray{G}
	\UCpaso [\UCsist] Muestra la interfaz \refElem{VCT-IU03}.
\end{UCtrayectoriaA} 

%Trayectorias Alternativas
\begin{UCtrayectoriaA}[Fin de la trayectoria]{B}{El actor no registro al menos un campo obligatorio.}
	\UCpaso [\UCsist] Muestra el mensaje \refIdElem{MSG4} en la interfaz \refElem{VCT-IU03a} en los campos que no
	fueron ingresados.
	\UCpaso [\UCsist] Continúa en el paso \ref{VCT-CU04:vadata} de la trayectoria principal.
\end{UCtrayectoriaA} 

%Trayectorias Alternativas
\begin{UCtrayectoriaA}[Fin de la trayectoria]{C}{El actor decide regresa a la pantalla anterior.}
	\UCpaso [\UCactor] Da clic en el botón \IUbutton{Regresar} en la interfaz \refElem{VCT-IU04b}.
	\UCpaso [\UCsist] Muestra la interfaz \refElem{VCT-IU04a}.
	\UCpaso [\UCsist] Continúa en el paso \ref{VCT-CU04:vadata} de la trayectoria principal.
\end{UCtrayectoriaA} 

%Trayectorias Alternativas
\begin{UCtrayectoriaA}[Fin de la trayectoria]{D}{El actor decide eliminar una habilidad.}
	\UCpaso [\UCactor] Da clic en el botón \IUbutton{x} de la habilidad seleccionada en la interfaz \refElem{VCT-IU04b}.
	\UCpaso [\UCsist] Muestra la interfaz \refElem{VCT-IU04b}.
	\UCpaso [\UCsist] Continúa en el paso \ref{VCT-CU04:hab} de la trayectoria principal.
\end{UCtrayectoriaA} 

%Trayectorias Alternativas
\begin{UCtrayectoriaA}[Fin de la trayectoria]{E}{El actor no registro al menos un campo obligatorio.}
	\UCpaso [\UCsist] Muestra el mensaje \refIdElem{MSG4} en la interfaz \refElem{VCT-IU03b} en los campos que no
	fueron ingresados.
	\UCpaso [\UCsist] Continúa en el paso \ref{VCT-CU04:vadata2} de la trayectoria principal.
\end{UCtrayectoriaA} 

%Trayectorias Alternativas
\begin{UCtrayectoriaA}[Fin de la trayectoria]{F}{El actor decide cancelar el registro.}
	\UCpaso [\UCactor] Da clic en el botón \IUbutton{Cancelar} en la interfaz \refElem{VCT-IU04b}.
	\UCpaso [\UCsist] Muestra el mensaje \refIdElem{MSG6} en la interfaz \refElem{VCT-IU04b}.
	\UCpaso [\UCactor] Da clic en el botón \IUbutton{Sí} en la interfaz \refElem{VCT-IU04b}.\refTray{H}
	\UCpaso [\UCsist] Muestra la interfaz \refElem{VCT-IU03}.
\end{UCtrayectoriaA} 

%Trayectorias Alternativas
\begin{UCtrayectoriaA}[Fin de la trayectoria]{G}{El actor decide cancelar la acción.}
	\UCpaso [\UCactor] Da clic en el botón \IUbutton{No} en la interfaz \refElem{VCT-IU04a}.
	\UCpaso [\UCsist] Muestra la interfaz \refElem{VCT-IU04a}.
	\UCpaso [\UCsist] Continúa en el paso \ref{VCT-CU04:vadata} de la trayectoria principal.
\end{UCtrayectoriaA} 

%Trayectorias Alternativas
\begin{UCtrayectoriaA}[Fin de la trayectoria]{H}{El actor decide cancelar la acción.}
	\UCpaso [\UCactor] Da clic en el botón \IUbutton{No} en la interfaz \refElem{VCT-IU04b}.
	\UCpaso [\UCsist] Muestra la interfaz \refElem{VCT-IU04b}.
	\UCpaso [\UCsist] Continúa en el paso \ref{VCT-CU04:vadata2} de la trayectoria principal.
\end{UCtrayectoriaA} 
	%\clearpage
\begin{UseCase}[]{VCT-CU04}{Registrar vacante}{
	Permite al reclutador de una empresa  publicar una vacante en el sistema durante cierto periodo de tiempo y así poder gestionas las postulaciones 
	que los usarios(canditos) hagan a dicha vacante.
	}
	%----------------------------------------------------------------
	% Datos generales del CU:
	\UCsection{Atributos}
	\UCitem{Actor(es)}{
		Reclutadores.

	}
	\UCitem[admin]{Prioridad}{
		Media
	}
	\UCitem[admin]{Complejidad}{
		Alta
	}
	\UCitem{Precondiciones}{
		El reclutador debe de estar registrado en el sistema.
	}
	\UCitem{Destino}{
		\Titem \refElem{VCT-IU03}
	}
	\UCitem{Reglas de Negocio}{
		\Titem \refIdElem{RN-N001}
		
	}
	\UCitem{Viene de}{
		\refElem{VCT-CU03}
	}	
\end{UseCase}

%Trayectoria Principal
\begin{UCtrayectoria}
	\UCpaso [\UCactor] Da clic en el icono \IUAgregar{} (Publicar vacante) en la interfaz \refElem{VCT-IU03}.
	\UCpaso [\UCsist] Muestra la interfaz \refElem{VCT-IU04a} en la interfaz \refElem{VCT-IU03}.
	\UCpaso [\UCactor] \label{VCT-CU04:vadata} Ingresa el título y el número de plazas de la vacante, ingresa el código postal, estado, municipio y colonia donde se va laboral.\refTray{A}
	\UCpaso [\UCactor] Ingresa el perfil, la experiencia y el tipo de contratación a la que la vacante va dirigida.
	\UCpaso [\UCactor] Ingresa el horario laboral y el rango salarial indicando si es neto o no el salario.
	\UCpaso [\UCsist] Valida que todos los campos marcados como obligatorios hayan sido ingresados de acuerdo a la regla de negocio \refIdElem{RN-N001}. \refTray{B}
	\UCpaso [\UCactor] Ingresa la descripción de la vacante.
	\UCpaso [\UCactor] \label{VCT-CU04:hab}Selecciona las habilidades y su correspondiente experiencia deseadas para la vacante.\refTray{D}
	\UCpaso [\UCactor] Selecciona la fecha de cierre de la vacante.
	\UCpaso [\UCactor] Da clic en el botón \IUbutton{Publicar} en la interfaz \refElem{VCT-IU04b}.\refTray{A}\refTray{C} \refTray{F}
	\UCpaso [\UCsist] \label{VCT-CU04:vadata2}Valida que todos los campos marcados como obligatorios hayan sido ingresados de acuerdo a la regla de negocio \refIdElem{RN-N001}.\refTray{E}
	\UCpaso [\UCsist] Muestra la interfaz \refElem{VCT-IU03} mostrando la nueva vacante registrada.
	\UCpaso [\UCsist] Notifica a los encargados y colaboradores de que hay una nueva vacante por revisar.
\end{UCtrayectoria}

%Trayectorias Alternativas
\begin{UCtrayectoriaA}[Fin de la trayectoria]{A}{El actor decide cancelar el registro.}
	\UCpaso [\UCactor] Da clic en el botón \IUbutton{Cancelar} en la interfaz \refElem{VCT-IU04a}.
	\UCpaso [\UCsist] Muestra el mensaje \refIdElem{MSG6} en la interfaz \refElem{VCT-IU04a}.
	\UCpaso [\UCactor] Da clic en el botón \IUbutton{Sí} en la interfaz \refElem{VCT-IU04a}.\refTray{G}
	\UCpaso [\UCsist] Muestra la interfaz \refElem{VCT-IU03}.
\end{UCtrayectoriaA} 

%Trayectorias Alternativas
\begin{UCtrayectoriaA}[Fin de la trayectoria]{B}{El actor no registro al menos un campo obligatorio.}
	\UCpaso [\UCsist] Muestra el mensaje \refIdElem{MSG4} en la interfaz \refElem{VCT-IU03a} en los campos que no
	fueron ingresados.
	\UCpaso [\UCsist] Continúa en el paso \ref{VCT-CU04:vadata} de la trayectoria principal.
\end{UCtrayectoriaA} 

%Trayectorias Alternativas
\begin{UCtrayectoriaA}[Fin de la trayectoria]{C}{El actor decide regresa a la pantalla anterior.}
	\UCpaso [\UCactor] Da clic en el botón \IUbutton{Regresar} en la interfaz \refElem{VCT-IU04b}.
	\UCpaso [\UCsist] Muestra la interfaz \refElem{VCT-IU04a}.
	\UCpaso [\UCsist] Continúa en el paso \ref{VCT-CU04:vadata} de la trayectoria principal.
\end{UCtrayectoriaA} 

%Trayectorias Alternativas
\begin{UCtrayectoriaA}[Fin de la trayectoria]{D}{El actor decide eliminar una habilidad.}
	\UCpaso [\UCactor] Da clic en el botón \IUbutton{x} de la habilidad seleccionada en la interfaz \refElem{VCT-IU04b}.
	\UCpaso [\UCsist] Muestra la interfaz \refElem{VCT-IU04b}.
	\UCpaso [\UCsist] Continúa en el paso \ref{VCT-CU04:hab} de la trayectoria principal.
\end{UCtrayectoriaA} 

%Trayectorias Alternativas
\begin{UCtrayectoriaA}[Fin de la trayectoria]{E}{El actor no registro al menos un campo obligatorio.}
	\UCpaso [\UCsist] Muestra el mensaje \refIdElem{MSG4} en la interfaz \refElem{VCT-IU03b} en los campos que no
	fueron ingresados.
	\UCpaso [\UCsist] Continúa en el paso \ref{VCT-CU04:vadata2} de la trayectoria principal.
\end{UCtrayectoriaA} 

%Trayectorias Alternativas
\begin{UCtrayectoriaA}[Fin de la trayectoria]{F}{El actor decide cancelar el registro.}
	\UCpaso [\UCactor] Da clic en el botón \IUbutton{Cancelar} en la interfaz \refElem{VCT-IU04b}.
	\UCpaso [\UCsist] Muestra el mensaje \refIdElem{MSG6} en la interfaz \refElem{VCT-IU04b}.
	\UCpaso [\UCactor] Da clic en el botón \IUbutton{Sí} en la interfaz \refElem{VCT-IU04b}.\refTray{H}
	\UCpaso [\UCsist] Muestra la interfaz \refElem{VCT-IU03}.
\end{UCtrayectoriaA} 

%Trayectorias Alternativas
\begin{UCtrayectoriaA}[Fin de la trayectoria]{G}{El actor decide cancelar la acción.}
	\UCpaso [\UCactor] Da clic en el botón \IUbutton{No} en la interfaz \refElem{VCT-IU04a}.
	\UCpaso [\UCsist] Muestra la interfaz \refElem{VCT-IU04a}.
	\UCpaso [\UCsist] Continúa en el paso \ref{VCT-CU04:vadata} de la trayectoria principal.
\end{UCtrayectoriaA} 

%Trayectorias Alternativas
\begin{UCtrayectoriaA}[Fin de la trayectoria]{H}{El actor decide cancelar la acción.}
	\UCpaso [\UCactor] Da clic en el botón \IUbutton{No} en la interfaz \refElem{VCT-IU04b}.
	\UCpaso [\UCsist] Muestra la interfaz \refElem{VCT-IU04b}.
	\UCpaso [\UCsist] Continúa en el paso \ref{VCT-CU04:vadata2} de la trayectoria principal.
\end{UCtrayectoriaA} 

	%\clearpage
\begin{UseCase}[]{VCT-CU04}{Registrar vacante}{
	Permite al reclutador de una empresa  publicar una vacante en el sistema durante cierto periodo de tiempo y así poder gestionas las postulaciones 
	que los usarios(canditos) hagan a dicha vacante.
	}
	%----------------------------------------------------------------
	% Datos generales del CU:
	\UCsection{Atributos}
	\UCitem{Actor(es)}{
		Reclutadores.

	}
	\UCitem[admin]{Prioridad}{
		Media
	}
	\UCitem[admin]{Complejidad}{
		Alta
	}
	\UCitem{Precondiciones}{
		El reclutador debe de estar registrado en el sistema.
	}
	\UCitem{Destino}{
		\Titem \refElem{VCT-IU03}
	}
	\UCitem{Reglas de Negocio}{
		\Titem \refIdElem{RN-N001}
		
	}
	\UCitem{Viene de}{
		\refElem{VCT-CU03}
	}	
\end{UseCase}

%Trayectoria Principal
\begin{UCtrayectoria}
	\UCpaso [\UCactor] Da clic en el icono \IUAgregar{} (Publicar vacante) en la interfaz \refElem{VCT-IU03}.
	\UCpaso [\UCsist] Muestra la interfaz \refElem{VCT-IU04a} en la interfaz \refElem{VCT-IU03}.
	\UCpaso [\UCactor] \label{VCT-CU04:vadata} Ingresa el título y el número de plazas de la vacante, ingresa el código postal, estado, municipio y colonia donde se va laboral.\refTray{A}
	\UCpaso [\UCactor] Ingresa el perfil, la experiencia y el tipo de contratación a la que la vacante va dirigida.
	\UCpaso [\UCactor] Ingresa el horario laboral y el rango salarial indicando si es neto o no el salario.
	\UCpaso [\UCsist] Valida que todos los campos marcados como obligatorios hayan sido ingresados de acuerdo a la regla de negocio \refIdElem{RN-N001}. \refTray{B}
	\UCpaso [\UCactor] Ingresa la descripción de la vacante.
	\UCpaso [\UCactor] \label{VCT-CU04:hab}Selecciona las habilidades y su correspondiente experiencia deseadas para la vacante.\refTray{D}
	\UCpaso [\UCactor] Selecciona la fecha de cierre de la vacante.
	\UCpaso [\UCactor] Da clic en el botón \IUbutton{Publicar} en la interfaz \refElem{VCT-IU04b}.\refTray{A}\refTray{C} \refTray{F}
	\UCpaso [\UCsist] \label{VCT-CU04:vadata2}Valida que todos los campos marcados como obligatorios hayan sido ingresados de acuerdo a la regla de negocio \refIdElem{RN-N001}.\refTray{E}
	\UCpaso [\UCsist] Muestra la interfaz \refElem{VCT-IU03} mostrando la nueva vacante registrada.
	\UCpaso [\UCsist] Notifica a los encargados y colaboradores de que hay una nueva vacante por revisar.
\end{UCtrayectoria}

%Trayectorias Alternativas
\begin{UCtrayectoriaA}[Fin de la trayectoria]{A}{El actor decide cancelar el registro.}
	\UCpaso [\UCactor] Da clic en el botón \IUbutton{Cancelar} en la interfaz \refElem{VCT-IU04a}.
	\UCpaso [\UCsist] Muestra el mensaje \refIdElem{MSG6} en la interfaz \refElem{VCT-IU04a}.
	\UCpaso [\UCactor] Da clic en el botón \IUbutton{Sí} en la interfaz \refElem{VCT-IU04a}.\refTray{G}
	\UCpaso [\UCsist] Muestra la interfaz \refElem{VCT-IU03}.
\end{UCtrayectoriaA} 

%Trayectorias Alternativas
\begin{UCtrayectoriaA}[Fin de la trayectoria]{B}{El actor no registro al menos un campo obligatorio.}
	\UCpaso [\UCsist] Muestra el mensaje \refIdElem{MSG4} en la interfaz \refElem{VCT-IU03a} en los campos que no
	fueron ingresados.
	\UCpaso [\UCsist] Continúa en el paso \ref{VCT-CU04:vadata} de la trayectoria principal.
\end{UCtrayectoriaA} 

%Trayectorias Alternativas
\begin{UCtrayectoriaA}[Fin de la trayectoria]{C}{El actor decide regresa a la pantalla anterior.}
	\UCpaso [\UCactor] Da clic en el botón \IUbutton{Regresar} en la interfaz \refElem{VCT-IU04b}.
	\UCpaso [\UCsist] Muestra la interfaz \refElem{VCT-IU04a}.
	\UCpaso [\UCsist] Continúa en el paso \ref{VCT-CU04:vadata} de la trayectoria principal.
\end{UCtrayectoriaA} 

%Trayectorias Alternativas
\begin{UCtrayectoriaA}[Fin de la trayectoria]{D}{El actor decide eliminar una habilidad.}
	\UCpaso [\UCactor] Da clic en el botón \IUbutton{x} de la habilidad seleccionada en la interfaz \refElem{VCT-IU04b}.
	\UCpaso [\UCsist] Muestra la interfaz \refElem{VCT-IU04b}.
	\UCpaso [\UCsist] Continúa en el paso \ref{VCT-CU04:hab} de la trayectoria principal.
\end{UCtrayectoriaA} 

%Trayectorias Alternativas
\begin{UCtrayectoriaA}[Fin de la trayectoria]{E}{El actor no registro al menos un campo obligatorio.}
	\UCpaso [\UCsist] Muestra el mensaje \refIdElem{MSG4} en la interfaz \refElem{VCT-IU03b} en los campos que no
	fueron ingresados.
	\UCpaso [\UCsist] Continúa en el paso \ref{VCT-CU04:vadata2} de la trayectoria principal.
\end{UCtrayectoriaA} 

%Trayectorias Alternativas
\begin{UCtrayectoriaA}[Fin de la trayectoria]{F}{El actor decide cancelar el registro.}
	\UCpaso [\UCactor] Da clic en el botón \IUbutton{Cancelar} en la interfaz \refElem{VCT-IU04b}.
	\UCpaso [\UCsist] Muestra el mensaje \refIdElem{MSG6} en la interfaz \refElem{VCT-IU04b}.
	\UCpaso [\UCactor] Da clic en el botón \IUbutton{Sí} en la interfaz \refElem{VCT-IU04b}.\refTray{H}
	\UCpaso [\UCsist] Muestra la interfaz \refElem{VCT-IU03}.
\end{UCtrayectoriaA} 

%Trayectorias Alternativas
\begin{UCtrayectoriaA}[Fin de la trayectoria]{G}{El actor decide cancelar la acción.}
	\UCpaso [\UCactor] Da clic en el botón \IUbutton{No} en la interfaz \refElem{VCT-IU04a}.
	\UCpaso [\UCsist] Muestra la interfaz \refElem{VCT-IU04a}.
	\UCpaso [\UCsist] Continúa en el paso \ref{VCT-CU04:vadata} de la trayectoria principal.
\end{UCtrayectoriaA} 

%Trayectorias Alternativas
\begin{UCtrayectoriaA}[Fin de la trayectoria]{H}{El actor decide cancelar la acción.}
	\UCpaso [\UCactor] Da clic en el botón \IUbutton{No} en la interfaz \refElem{VCT-IU04b}.
	\UCpaso [\UCsist] Muestra la interfaz \refElem{VCT-IU04b}.
	\UCpaso [\UCsist] Continúa en el paso \ref{VCT-CU04:vadata2} de la trayectoria principal.
\end{UCtrayectoriaA} 
	%\clearpage
\begin{UseCase}[]{VCT-CU04}{Registrar vacante}{
	Permite al reclutador de una empresa  publicar una vacante en el sistema durante cierto periodo de tiempo y así poder gestionas las postulaciones 
	que los usarios(canditos) hagan a dicha vacante.
	}
	%----------------------------------------------------------------
	% Datos generales del CU:
	\UCsection{Atributos}
	\UCitem{Actor(es)}{
		Reclutadores.

	}
	\UCitem[admin]{Prioridad}{
		Media
	}
	\UCitem[admin]{Complejidad}{
		Alta
	}
	\UCitem{Precondiciones}{
		El reclutador debe de estar registrado en el sistema.
	}
	\UCitem{Destino}{
		\Titem \refElem{VCT-IU03}
	}
	\UCitem{Reglas de Negocio}{
		\Titem \refIdElem{RN-N001}
		
	}
	\UCitem{Viene de}{
		\refElem{VCT-CU03}
	}	
\end{UseCase}

%Trayectoria Principal
\begin{UCtrayectoria}
	\UCpaso [\UCactor] Da clic en el icono \IUAgregar{} (Publicar vacante) en la interfaz \refElem{VCT-IU03}.
	\UCpaso [\UCsist] Muestra la interfaz \refElem{VCT-IU04a} en la interfaz \refElem{VCT-IU03}.
	\UCpaso [\UCactor] \label{VCT-CU04:vadata} Ingresa el título y el número de plazas de la vacante, ingresa el código postal, estado, municipio y colonia donde se va laboral.\refTray{A}
	\UCpaso [\UCactor] Ingresa el perfil, la experiencia y el tipo de contratación a la que la vacante va dirigida.
	\UCpaso [\UCactor] Ingresa el horario laboral y el rango salarial indicando si es neto o no el salario.
	\UCpaso [\UCsist] Valida que todos los campos marcados como obligatorios hayan sido ingresados de acuerdo a la regla de negocio \refIdElem{RN-N001}. \refTray{B}
	\UCpaso [\UCactor] Ingresa la descripción de la vacante.
	\UCpaso [\UCactor] \label{VCT-CU04:hab}Selecciona las habilidades y su correspondiente experiencia deseadas para la vacante.\refTray{D}
	\UCpaso [\UCactor] Selecciona la fecha de cierre de la vacante.
	\UCpaso [\UCactor] Da clic en el botón \IUbutton{Publicar} en la interfaz \refElem{VCT-IU04b}.\refTray{A}\refTray{C} \refTray{F}
	\UCpaso [\UCsist] \label{VCT-CU04:vadata2}Valida que todos los campos marcados como obligatorios hayan sido ingresados de acuerdo a la regla de negocio \refIdElem{RN-N001}.\refTray{E}
	\UCpaso [\UCsist] Muestra la interfaz \refElem{VCT-IU03} mostrando la nueva vacante registrada.
	\UCpaso [\UCsist] Notifica a los encargados y colaboradores de que hay una nueva vacante por revisar.
\end{UCtrayectoria}

%Trayectorias Alternativas
\begin{UCtrayectoriaA}[Fin de la trayectoria]{A}{El actor decide cancelar el registro.}
	\UCpaso [\UCactor] Da clic en el botón \IUbutton{Cancelar} en la interfaz \refElem{VCT-IU04a}.
	\UCpaso [\UCsist] Muestra el mensaje \refIdElem{MSG6} en la interfaz \refElem{VCT-IU04a}.
	\UCpaso [\UCactor] Da clic en el botón \IUbutton{Sí} en la interfaz \refElem{VCT-IU04a}.\refTray{G}
	\UCpaso [\UCsist] Muestra la interfaz \refElem{VCT-IU03}.
\end{UCtrayectoriaA} 

%Trayectorias Alternativas
\begin{UCtrayectoriaA}[Fin de la trayectoria]{B}{El actor no registro al menos un campo obligatorio.}
	\UCpaso [\UCsist] Muestra el mensaje \refIdElem{MSG4} en la interfaz \refElem{VCT-IU03a} en los campos que no
	fueron ingresados.
	\UCpaso [\UCsist] Continúa en el paso \ref{VCT-CU04:vadata} de la trayectoria principal.
\end{UCtrayectoriaA} 

%Trayectorias Alternativas
\begin{UCtrayectoriaA}[Fin de la trayectoria]{C}{El actor decide regresa a la pantalla anterior.}
	\UCpaso [\UCactor] Da clic en el botón \IUbutton{Regresar} en la interfaz \refElem{VCT-IU04b}.
	\UCpaso [\UCsist] Muestra la interfaz \refElem{VCT-IU04a}.
	\UCpaso [\UCsist] Continúa en el paso \ref{VCT-CU04:vadata} de la trayectoria principal.
\end{UCtrayectoriaA} 

%Trayectorias Alternativas
\begin{UCtrayectoriaA}[Fin de la trayectoria]{D}{El actor decide eliminar una habilidad.}
	\UCpaso [\UCactor] Da clic en el botón \IUbutton{x} de la habilidad seleccionada en la interfaz \refElem{VCT-IU04b}.
	\UCpaso [\UCsist] Muestra la interfaz \refElem{VCT-IU04b}.
	\UCpaso [\UCsist] Continúa en el paso \ref{VCT-CU04:hab} de la trayectoria principal.
\end{UCtrayectoriaA} 

%Trayectorias Alternativas
\begin{UCtrayectoriaA}[Fin de la trayectoria]{E}{El actor no registro al menos un campo obligatorio.}
	\UCpaso [\UCsist] Muestra el mensaje \refIdElem{MSG4} en la interfaz \refElem{VCT-IU03b} en los campos que no
	fueron ingresados.
	\UCpaso [\UCsist] Continúa en el paso \ref{VCT-CU04:vadata2} de la trayectoria principal.
\end{UCtrayectoriaA} 

%Trayectorias Alternativas
\begin{UCtrayectoriaA}[Fin de la trayectoria]{F}{El actor decide cancelar el registro.}
	\UCpaso [\UCactor] Da clic en el botón \IUbutton{Cancelar} en la interfaz \refElem{VCT-IU04b}.
	\UCpaso [\UCsist] Muestra el mensaje \refIdElem{MSG6} en la interfaz \refElem{VCT-IU04b}.
	\UCpaso [\UCactor] Da clic en el botón \IUbutton{Sí} en la interfaz \refElem{VCT-IU04b}.\refTray{H}
	\UCpaso [\UCsist] Muestra la interfaz \refElem{VCT-IU03}.
\end{UCtrayectoriaA} 

%Trayectorias Alternativas
\begin{UCtrayectoriaA}[Fin de la trayectoria]{G}{El actor decide cancelar la acción.}
	\UCpaso [\UCactor] Da clic en el botón \IUbutton{No} en la interfaz \refElem{VCT-IU04a}.
	\UCpaso [\UCsist] Muestra la interfaz \refElem{VCT-IU04a}.
	\UCpaso [\UCsist] Continúa en el paso \ref{VCT-CU04:vadata} de la trayectoria principal.
\end{UCtrayectoriaA} 

%Trayectorias Alternativas
\begin{UCtrayectoriaA}[Fin de la trayectoria]{H}{El actor decide cancelar la acción.}
	\UCpaso [\UCactor] Da clic en el botón \IUbutton{No} en la interfaz \refElem{VCT-IU04b}.
	\UCpaso [\UCsist] Muestra la interfaz \refElem{VCT-IU04b}.
	\UCpaso [\UCsist] Continúa en el paso \ref{VCT-CU04:vadata2} de la trayectoria principal.
\end{UCtrayectoriaA} 
	%\clearpage
\begin{UseCase}[]{VCT-CU04}{Registrar vacante}{
	Permite al reclutador de una empresa  publicar una vacante en el sistema durante cierto periodo de tiempo y así poder gestionas las postulaciones 
	que los usarios(canditos) hagan a dicha vacante.
	}
	%----------------------------------------------------------------
	% Datos generales del CU:
	\UCsection{Atributos}
	\UCitem{Actor(es)}{
		Reclutadores.

	}
	\UCitem[admin]{Prioridad}{
		Media
	}
	\UCitem[admin]{Complejidad}{
		Alta
	}
	\UCitem{Precondiciones}{
		El reclutador debe de estar registrado en el sistema.
	}
	\UCitem{Destino}{
		\Titem \refElem{VCT-IU03}
	}
	\UCitem{Reglas de Negocio}{
		\Titem \refIdElem{RN-N001}
		
	}
	\UCitem{Viene de}{
		\refElem{VCT-CU03}
	}	
\end{UseCase}

%Trayectoria Principal
\begin{UCtrayectoria}
	\UCpaso [\UCactor] Da clic en el icono \IUAgregar{} (Publicar vacante) en la interfaz \refElem{VCT-IU03}.
	\UCpaso [\UCsist] Muestra la interfaz \refElem{VCT-IU04a} en la interfaz \refElem{VCT-IU03}.
	\UCpaso [\UCactor] \label{VCT-CU04:vadata} Ingresa el título y el número de plazas de la vacante, ingresa el código postal, estado, municipio y colonia donde se va laboral.\refTray{A}
	\UCpaso [\UCactor] Ingresa el perfil, la experiencia y el tipo de contratación a la que la vacante va dirigida.
	\UCpaso [\UCactor] Ingresa el horario laboral y el rango salarial indicando si es neto o no el salario.
	\UCpaso [\UCsist] Valida que todos los campos marcados como obligatorios hayan sido ingresados de acuerdo a la regla de negocio \refIdElem{RN-N001}. \refTray{B}
	\UCpaso [\UCactor] Ingresa la descripción de la vacante.
	\UCpaso [\UCactor] \label{VCT-CU04:hab}Selecciona las habilidades y su correspondiente experiencia deseadas para la vacante.\refTray{D}
	\UCpaso [\UCactor] Selecciona la fecha de cierre de la vacante.
	\UCpaso [\UCactor] Da clic en el botón \IUbutton{Publicar} en la interfaz \refElem{VCT-IU04b}.\refTray{A}\refTray{C} \refTray{F}
	\UCpaso [\UCsist] \label{VCT-CU04:vadata2}Valida que todos los campos marcados como obligatorios hayan sido ingresados de acuerdo a la regla de negocio \refIdElem{RN-N001}.\refTray{E}
	\UCpaso [\UCsist] Muestra la interfaz \refElem{VCT-IU03} mostrando la nueva vacante registrada.
	\UCpaso [\UCsist] Notifica a los encargados y colaboradores de que hay una nueva vacante por revisar.
\end{UCtrayectoria}

%Trayectorias Alternativas
\begin{UCtrayectoriaA}[Fin de la trayectoria]{A}{El actor decide cancelar el registro.}
	\UCpaso [\UCactor] Da clic en el botón \IUbutton{Cancelar} en la interfaz \refElem{VCT-IU04a}.
	\UCpaso [\UCsist] Muestra el mensaje \refIdElem{MSG6} en la interfaz \refElem{VCT-IU04a}.
	\UCpaso [\UCactor] Da clic en el botón \IUbutton{Sí} en la interfaz \refElem{VCT-IU04a}.\refTray{G}
	\UCpaso [\UCsist] Muestra la interfaz \refElem{VCT-IU03}.
\end{UCtrayectoriaA} 

%Trayectorias Alternativas
\begin{UCtrayectoriaA}[Fin de la trayectoria]{B}{El actor no registro al menos un campo obligatorio.}
	\UCpaso [\UCsist] Muestra el mensaje \refIdElem{MSG4} en la interfaz \refElem{VCT-IU03a} en los campos que no
	fueron ingresados.
	\UCpaso [\UCsist] Continúa en el paso \ref{VCT-CU04:vadata} de la trayectoria principal.
\end{UCtrayectoriaA} 

%Trayectorias Alternativas
\begin{UCtrayectoriaA}[Fin de la trayectoria]{C}{El actor decide regresa a la pantalla anterior.}
	\UCpaso [\UCactor] Da clic en el botón \IUbutton{Regresar} en la interfaz \refElem{VCT-IU04b}.
	\UCpaso [\UCsist] Muestra la interfaz \refElem{VCT-IU04a}.
	\UCpaso [\UCsist] Continúa en el paso \ref{VCT-CU04:vadata} de la trayectoria principal.
\end{UCtrayectoriaA} 

%Trayectorias Alternativas
\begin{UCtrayectoriaA}[Fin de la trayectoria]{D}{El actor decide eliminar una habilidad.}
	\UCpaso [\UCactor] Da clic en el botón \IUbutton{x} de la habilidad seleccionada en la interfaz \refElem{VCT-IU04b}.
	\UCpaso [\UCsist] Muestra la interfaz \refElem{VCT-IU04b}.
	\UCpaso [\UCsist] Continúa en el paso \ref{VCT-CU04:hab} de la trayectoria principal.
\end{UCtrayectoriaA} 

%Trayectorias Alternativas
\begin{UCtrayectoriaA}[Fin de la trayectoria]{E}{El actor no registro al menos un campo obligatorio.}
	\UCpaso [\UCsist] Muestra el mensaje \refIdElem{MSG4} en la interfaz \refElem{VCT-IU03b} en los campos que no
	fueron ingresados.
	\UCpaso [\UCsist] Continúa en el paso \ref{VCT-CU04:vadata2} de la trayectoria principal.
\end{UCtrayectoriaA} 

%Trayectorias Alternativas
\begin{UCtrayectoriaA}[Fin de la trayectoria]{F}{El actor decide cancelar el registro.}
	\UCpaso [\UCactor] Da clic en el botón \IUbutton{Cancelar} en la interfaz \refElem{VCT-IU04b}.
	\UCpaso [\UCsist] Muestra el mensaje \refIdElem{MSG6} en la interfaz \refElem{VCT-IU04b}.
	\UCpaso [\UCactor] Da clic en el botón \IUbutton{Sí} en la interfaz \refElem{VCT-IU04b}.\refTray{H}
	\UCpaso [\UCsist] Muestra la interfaz \refElem{VCT-IU03}.
\end{UCtrayectoriaA} 

%Trayectorias Alternativas
\begin{UCtrayectoriaA}[Fin de la trayectoria]{G}{El actor decide cancelar la acción.}
	\UCpaso [\UCactor] Da clic en el botón \IUbutton{No} en la interfaz \refElem{VCT-IU04a}.
	\UCpaso [\UCsist] Muestra la interfaz \refElem{VCT-IU04a}.
	\UCpaso [\UCsist] Continúa en el paso \ref{VCT-CU04:vadata} de la trayectoria principal.
\end{UCtrayectoriaA} 

%Trayectorias Alternativas
\begin{UCtrayectoriaA}[Fin de la trayectoria]{H}{El actor decide cancelar la acción.}
	\UCpaso [\UCactor] Da clic en el botón \IUbutton{No} en la interfaz \refElem{VCT-IU04b}.
	\UCpaso [\UCsist] Muestra la interfaz \refElem{VCT-IU04b}.
	\UCpaso [\UCsist] Continúa en el paso \ref{VCT-CU04:vadata2} de la trayectoria principal.
\end{UCtrayectoriaA} 
	%\clearpage
\begin{UseCase}[]{VCT-CU04}{Registrar vacante}{
	Permite al reclutador de una empresa  publicar una vacante en el sistema durante cierto periodo de tiempo y así poder gestionas las postulaciones 
	que los usarios(canditos) hagan a dicha vacante.
	}
	%----------------------------------------------------------------
	% Datos generales del CU:
	\UCsection{Atributos}
	\UCitem{Actor(es)}{
		Reclutadores.

	}
	\UCitem[admin]{Prioridad}{
		Media
	}
	\UCitem[admin]{Complejidad}{
		Alta
	}
	\UCitem{Precondiciones}{
		El reclutador debe de estar registrado en el sistema.
	}
	\UCitem{Destino}{
		\Titem \refElem{VCT-IU03}
	}
	\UCitem{Reglas de Negocio}{
		\Titem \refIdElem{RN-N001}
		
	}
	\UCitem{Viene de}{
		\refElem{VCT-CU03}
	}	
\end{UseCase}

%Trayectoria Principal
\begin{UCtrayectoria}
	\UCpaso [\UCactor] Da clic en el icono \IUAgregar{} (Publicar vacante) en la interfaz \refElem{VCT-IU03}.
	\UCpaso [\UCsist] Muestra la interfaz \refElem{VCT-IU04a} en la interfaz \refElem{VCT-IU03}.
	\UCpaso [\UCactor] \label{VCT-CU04:vadata} Ingresa el título y el número de plazas de la vacante, ingresa el código postal, estado, municipio y colonia donde se va laboral.\refTray{A}
	\UCpaso [\UCactor] Ingresa el perfil, la experiencia y el tipo de contratación a la que la vacante va dirigida.
	\UCpaso [\UCactor] Ingresa el horario laboral y el rango salarial indicando si es neto o no el salario.
	\UCpaso [\UCsist] Valida que todos los campos marcados como obligatorios hayan sido ingresados de acuerdo a la regla de negocio \refIdElem{RN-N001}. \refTray{B}
	\UCpaso [\UCactor] Ingresa la descripción de la vacante.
	\UCpaso [\UCactor] \label{VCT-CU04:hab}Selecciona las habilidades y su correspondiente experiencia deseadas para la vacante.\refTray{D}
	\UCpaso [\UCactor] Selecciona la fecha de cierre de la vacante.
	\UCpaso [\UCactor] Da clic en el botón \IUbutton{Publicar} en la interfaz \refElem{VCT-IU04b}.\refTray{A}\refTray{C} \refTray{F}
	\UCpaso [\UCsist] \label{VCT-CU04:vadata2}Valida que todos los campos marcados como obligatorios hayan sido ingresados de acuerdo a la regla de negocio \refIdElem{RN-N001}.\refTray{E}
	\UCpaso [\UCsist] Muestra la interfaz \refElem{VCT-IU03} mostrando la nueva vacante registrada.
	\UCpaso [\UCsist] Notifica a los encargados y colaboradores de que hay una nueva vacante por revisar.
\end{UCtrayectoria}

%Trayectorias Alternativas
\begin{UCtrayectoriaA}[Fin de la trayectoria]{A}{El actor decide cancelar el registro.}
	\UCpaso [\UCactor] Da clic en el botón \IUbutton{Cancelar} en la interfaz \refElem{VCT-IU04a}.
	\UCpaso [\UCsist] Muestra el mensaje \refIdElem{MSG6} en la interfaz \refElem{VCT-IU04a}.
	\UCpaso [\UCactor] Da clic en el botón \IUbutton{Sí} en la interfaz \refElem{VCT-IU04a}.\refTray{G}
	\UCpaso [\UCsist] Muestra la interfaz \refElem{VCT-IU03}.
\end{UCtrayectoriaA} 

%Trayectorias Alternativas
\begin{UCtrayectoriaA}[Fin de la trayectoria]{B}{El actor no registro al menos un campo obligatorio.}
	\UCpaso [\UCsist] Muestra el mensaje \refIdElem{MSG4} en la interfaz \refElem{VCT-IU03a} en los campos que no
	fueron ingresados.
	\UCpaso [\UCsist] Continúa en el paso \ref{VCT-CU04:vadata} de la trayectoria principal.
\end{UCtrayectoriaA} 

%Trayectorias Alternativas
\begin{UCtrayectoriaA}[Fin de la trayectoria]{C}{El actor decide regresa a la pantalla anterior.}
	\UCpaso [\UCactor] Da clic en el botón \IUbutton{Regresar} en la interfaz \refElem{VCT-IU04b}.
	\UCpaso [\UCsist] Muestra la interfaz \refElem{VCT-IU04a}.
	\UCpaso [\UCsist] Continúa en el paso \ref{VCT-CU04:vadata} de la trayectoria principal.
\end{UCtrayectoriaA} 

%Trayectorias Alternativas
\begin{UCtrayectoriaA}[Fin de la trayectoria]{D}{El actor decide eliminar una habilidad.}
	\UCpaso [\UCactor] Da clic en el botón \IUbutton{x} de la habilidad seleccionada en la interfaz \refElem{VCT-IU04b}.
	\UCpaso [\UCsist] Muestra la interfaz \refElem{VCT-IU04b}.
	\UCpaso [\UCsist] Continúa en el paso \ref{VCT-CU04:hab} de la trayectoria principal.
\end{UCtrayectoriaA} 

%Trayectorias Alternativas
\begin{UCtrayectoriaA}[Fin de la trayectoria]{E}{El actor no registro al menos un campo obligatorio.}
	\UCpaso [\UCsist] Muestra el mensaje \refIdElem{MSG4} en la interfaz \refElem{VCT-IU03b} en los campos que no
	fueron ingresados.
	\UCpaso [\UCsist] Continúa en el paso \ref{VCT-CU04:vadata2} de la trayectoria principal.
\end{UCtrayectoriaA} 

%Trayectorias Alternativas
\begin{UCtrayectoriaA}[Fin de la trayectoria]{F}{El actor decide cancelar el registro.}
	\UCpaso [\UCactor] Da clic en el botón \IUbutton{Cancelar} en la interfaz \refElem{VCT-IU04b}.
	\UCpaso [\UCsist] Muestra el mensaje \refIdElem{MSG6} en la interfaz \refElem{VCT-IU04b}.
	\UCpaso [\UCactor] Da clic en el botón \IUbutton{Sí} en la interfaz \refElem{VCT-IU04b}.\refTray{H}
	\UCpaso [\UCsist] Muestra la interfaz \refElem{VCT-IU03}.
\end{UCtrayectoriaA} 

%Trayectorias Alternativas
\begin{UCtrayectoriaA}[Fin de la trayectoria]{G}{El actor decide cancelar la acción.}
	\UCpaso [\UCactor] Da clic en el botón \IUbutton{No} en la interfaz \refElem{VCT-IU04a}.
	\UCpaso [\UCsist] Muestra la interfaz \refElem{VCT-IU04a}.
	\UCpaso [\UCsist] Continúa en el paso \ref{VCT-CU04:vadata} de la trayectoria principal.
\end{UCtrayectoriaA} 

%Trayectorias Alternativas
\begin{UCtrayectoriaA}[Fin de la trayectoria]{H}{El actor decide cancelar la acción.}
	\UCpaso [\UCactor] Da clic en el botón \IUbutton{No} en la interfaz \refElem{VCT-IU04b}.
	\UCpaso [\UCsist] Muestra la interfaz \refElem{VCT-IU04b}.
	\UCpaso [\UCsist] Continúa en el paso \ref{VCT-CU04:vadata2} de la trayectoria principal.
\end{UCtrayectoriaA} 

	%\clearpage
\begin{UseCase}[]{VCT-CU04}{Registrar vacante}{
	Permite al reclutador de una empresa  publicar una vacante en el sistema durante cierto periodo de tiempo y así poder gestionas las postulaciones 
	que los usarios(canditos) hagan a dicha vacante.
	}
	%----------------------------------------------------------------
	% Datos generales del CU:
	\UCsection{Atributos}
	\UCitem{Actor(es)}{
		Reclutadores.

	}
	\UCitem[admin]{Prioridad}{
		Media
	}
	\UCitem[admin]{Complejidad}{
		Alta
	}
	\UCitem{Precondiciones}{
		El reclutador debe de estar registrado en el sistema.
	}
	\UCitem{Destino}{
		\Titem \refElem{VCT-IU03}
	}
	\UCitem{Reglas de Negocio}{
		\Titem \refIdElem{RN-N001}
		
	}
	\UCitem{Viene de}{
		\refElem{VCT-CU03}
	}	
\end{UseCase}

%Trayectoria Principal
\begin{UCtrayectoria}
	\UCpaso [\UCactor] Da clic en el icono \IUAgregar{} (Publicar vacante) en la interfaz \refElem{VCT-IU03}.
	\UCpaso [\UCsist] Muestra la interfaz \refElem{VCT-IU04a} en la interfaz \refElem{VCT-IU03}.
	\UCpaso [\UCactor] \label{VCT-CU04:vadata} Ingresa el título y el número de plazas de la vacante, ingresa el código postal, estado, municipio y colonia donde se va laboral.\refTray{A}
	\UCpaso [\UCactor] Ingresa el perfil, la experiencia y el tipo de contratación a la que la vacante va dirigida.
	\UCpaso [\UCactor] Ingresa el horario laboral y el rango salarial indicando si es neto o no el salario.
	\UCpaso [\UCsist] Valida que todos los campos marcados como obligatorios hayan sido ingresados de acuerdo a la regla de negocio \refIdElem{RN-N001}. \refTray{B}
	\UCpaso [\UCactor] Ingresa la descripción de la vacante.
	\UCpaso [\UCactor] \label{VCT-CU04:hab}Selecciona las habilidades y su correspondiente experiencia deseadas para la vacante.\refTray{D}
	\UCpaso [\UCactor] Selecciona la fecha de cierre de la vacante.
	\UCpaso [\UCactor] Da clic en el botón \IUbutton{Publicar} en la interfaz \refElem{VCT-IU04b}.\refTray{A}\refTray{C} \refTray{F}
	\UCpaso [\UCsist] \label{VCT-CU04:vadata2}Valida que todos los campos marcados como obligatorios hayan sido ingresados de acuerdo a la regla de negocio \refIdElem{RN-N001}.\refTray{E}
	\UCpaso [\UCsist] Muestra la interfaz \refElem{VCT-IU03} mostrando la nueva vacante registrada.
	\UCpaso [\UCsist] Notifica a los encargados y colaboradores de que hay una nueva vacante por revisar.
\end{UCtrayectoria}

%Trayectorias Alternativas
\begin{UCtrayectoriaA}[Fin de la trayectoria]{A}{El actor decide cancelar el registro.}
	\UCpaso [\UCactor] Da clic en el botón \IUbutton{Cancelar} en la interfaz \refElem{VCT-IU04a}.
	\UCpaso [\UCsist] Muestra el mensaje \refIdElem{MSG6} en la interfaz \refElem{VCT-IU04a}.
	\UCpaso [\UCactor] Da clic en el botón \IUbutton{Sí} en la interfaz \refElem{VCT-IU04a}.\refTray{G}
	\UCpaso [\UCsist] Muestra la interfaz \refElem{VCT-IU03}.
\end{UCtrayectoriaA} 

%Trayectorias Alternativas
\begin{UCtrayectoriaA}[Fin de la trayectoria]{B}{El actor no registro al menos un campo obligatorio.}
	\UCpaso [\UCsist] Muestra el mensaje \refIdElem{MSG4} en la interfaz \refElem{VCT-IU03a} en los campos que no
	fueron ingresados.
	\UCpaso [\UCsist] Continúa en el paso \ref{VCT-CU04:vadata} de la trayectoria principal.
\end{UCtrayectoriaA} 

%Trayectorias Alternativas
\begin{UCtrayectoriaA}[Fin de la trayectoria]{C}{El actor decide regresa a la pantalla anterior.}
	\UCpaso [\UCactor] Da clic en el botón \IUbutton{Regresar} en la interfaz \refElem{VCT-IU04b}.
	\UCpaso [\UCsist] Muestra la interfaz \refElem{VCT-IU04a}.
	\UCpaso [\UCsist] Continúa en el paso \ref{VCT-CU04:vadata} de la trayectoria principal.
\end{UCtrayectoriaA} 

%Trayectorias Alternativas
\begin{UCtrayectoriaA}[Fin de la trayectoria]{D}{El actor decide eliminar una habilidad.}
	\UCpaso [\UCactor] Da clic en el botón \IUbutton{x} de la habilidad seleccionada en la interfaz \refElem{VCT-IU04b}.
	\UCpaso [\UCsist] Muestra la interfaz \refElem{VCT-IU04b}.
	\UCpaso [\UCsist] Continúa en el paso \ref{VCT-CU04:hab} de la trayectoria principal.
\end{UCtrayectoriaA} 

%Trayectorias Alternativas
\begin{UCtrayectoriaA}[Fin de la trayectoria]{E}{El actor no registro al menos un campo obligatorio.}
	\UCpaso [\UCsist] Muestra el mensaje \refIdElem{MSG4} en la interfaz \refElem{VCT-IU03b} en los campos que no
	fueron ingresados.
	\UCpaso [\UCsist] Continúa en el paso \ref{VCT-CU04:vadata2} de la trayectoria principal.
\end{UCtrayectoriaA} 

%Trayectorias Alternativas
\begin{UCtrayectoriaA}[Fin de la trayectoria]{F}{El actor decide cancelar el registro.}
	\UCpaso [\UCactor] Da clic en el botón \IUbutton{Cancelar} en la interfaz \refElem{VCT-IU04b}.
	\UCpaso [\UCsist] Muestra el mensaje \refIdElem{MSG6} en la interfaz \refElem{VCT-IU04b}.
	\UCpaso [\UCactor] Da clic en el botón \IUbutton{Sí} en la interfaz \refElem{VCT-IU04b}.\refTray{H}
	\UCpaso [\UCsist] Muestra la interfaz \refElem{VCT-IU03}.
\end{UCtrayectoriaA} 

%Trayectorias Alternativas
\begin{UCtrayectoriaA}[Fin de la trayectoria]{G}{El actor decide cancelar la acción.}
	\UCpaso [\UCactor] Da clic en el botón \IUbutton{No} en la interfaz \refElem{VCT-IU04a}.
	\UCpaso [\UCsist] Muestra la interfaz \refElem{VCT-IU04a}.
	\UCpaso [\UCsist] Continúa en el paso \ref{VCT-CU04:vadata} de la trayectoria principal.
\end{UCtrayectoriaA} 

%Trayectorias Alternativas
\begin{UCtrayectoriaA}[Fin de la trayectoria]{H}{El actor decide cancelar la acción.}
	\UCpaso [\UCactor] Da clic en el botón \IUbutton{No} en la interfaz \refElem{VCT-IU04b}.
	\UCpaso [\UCsist] Muestra la interfaz \refElem{VCT-IU04b}.
	\UCpaso [\UCsist] Continúa en el paso \ref{VCT-CU04:vadata2} de la trayectoria principal.
\end{UCtrayectoriaA} 
	%\clearpage
\begin{UseCase}[]{VCT-CU04}{Registrar vacante}{
	Permite al reclutador de una empresa  publicar una vacante en el sistema durante cierto periodo de tiempo y así poder gestionas las postulaciones 
	que los usarios(canditos) hagan a dicha vacante.
	}
	%----------------------------------------------------------------
	% Datos generales del CU:
	\UCsection{Atributos}
	\UCitem{Actor(es)}{
		Reclutadores.

	}
	\UCitem[admin]{Prioridad}{
		Media
	}
	\UCitem[admin]{Complejidad}{
		Alta
	}
	\UCitem{Precondiciones}{
		El reclutador debe de estar registrado en el sistema.
	}
	\UCitem{Destino}{
		\Titem \refElem{VCT-IU03}
	}
	\UCitem{Reglas de Negocio}{
		\Titem \refIdElem{RN-N001}
		
	}
	\UCitem{Viene de}{
		\refElem{VCT-CU03}
	}	
\end{UseCase}

%Trayectoria Principal
\begin{UCtrayectoria}
	\UCpaso [\UCactor] Da clic en el icono \IUAgregar{} (Publicar vacante) en la interfaz \refElem{VCT-IU03}.
	\UCpaso [\UCsist] Muestra la interfaz \refElem{VCT-IU04a} en la interfaz \refElem{VCT-IU03}.
	\UCpaso [\UCactor] \label{VCT-CU04:vadata} Ingresa el título y el número de plazas de la vacante, ingresa el código postal, estado, municipio y colonia donde se va laboral.\refTray{A}
	\UCpaso [\UCactor] Ingresa el perfil, la experiencia y el tipo de contratación a la que la vacante va dirigida.
	\UCpaso [\UCactor] Ingresa el horario laboral y el rango salarial indicando si es neto o no el salario.
	\UCpaso [\UCsist] Valida que todos los campos marcados como obligatorios hayan sido ingresados de acuerdo a la regla de negocio \refIdElem{RN-N001}. \refTray{B}
	\UCpaso [\UCactor] Ingresa la descripción de la vacante.
	\UCpaso [\UCactor] \label{VCT-CU04:hab}Selecciona las habilidades y su correspondiente experiencia deseadas para la vacante.\refTray{D}
	\UCpaso [\UCactor] Selecciona la fecha de cierre de la vacante.
	\UCpaso [\UCactor] Da clic en el botón \IUbutton{Publicar} en la interfaz \refElem{VCT-IU04b}.\refTray{A}\refTray{C} \refTray{F}
	\UCpaso [\UCsist] \label{VCT-CU04:vadata2}Valida que todos los campos marcados como obligatorios hayan sido ingresados de acuerdo a la regla de negocio \refIdElem{RN-N001}.\refTray{E}
	\UCpaso [\UCsist] Muestra la interfaz \refElem{VCT-IU03} mostrando la nueva vacante registrada.
	\UCpaso [\UCsist] Notifica a los encargados y colaboradores de que hay una nueva vacante por revisar.
\end{UCtrayectoria}

%Trayectorias Alternativas
\begin{UCtrayectoriaA}[Fin de la trayectoria]{A}{El actor decide cancelar el registro.}
	\UCpaso [\UCactor] Da clic en el botón \IUbutton{Cancelar} en la interfaz \refElem{VCT-IU04a}.
	\UCpaso [\UCsist] Muestra el mensaje \refIdElem{MSG6} en la interfaz \refElem{VCT-IU04a}.
	\UCpaso [\UCactor] Da clic en el botón \IUbutton{Sí} en la interfaz \refElem{VCT-IU04a}.\refTray{G}
	\UCpaso [\UCsist] Muestra la interfaz \refElem{VCT-IU03}.
\end{UCtrayectoriaA} 

%Trayectorias Alternativas
\begin{UCtrayectoriaA}[Fin de la trayectoria]{B}{El actor no registro al menos un campo obligatorio.}
	\UCpaso [\UCsist] Muestra el mensaje \refIdElem{MSG4} en la interfaz \refElem{VCT-IU03a} en los campos que no
	fueron ingresados.
	\UCpaso [\UCsist] Continúa en el paso \ref{VCT-CU04:vadata} de la trayectoria principal.
\end{UCtrayectoriaA} 

%Trayectorias Alternativas
\begin{UCtrayectoriaA}[Fin de la trayectoria]{C}{El actor decide regresa a la pantalla anterior.}
	\UCpaso [\UCactor] Da clic en el botón \IUbutton{Regresar} en la interfaz \refElem{VCT-IU04b}.
	\UCpaso [\UCsist] Muestra la interfaz \refElem{VCT-IU04a}.
	\UCpaso [\UCsist] Continúa en el paso \ref{VCT-CU04:vadata} de la trayectoria principal.
\end{UCtrayectoriaA} 

%Trayectorias Alternativas
\begin{UCtrayectoriaA}[Fin de la trayectoria]{D}{El actor decide eliminar una habilidad.}
	\UCpaso [\UCactor] Da clic en el botón \IUbutton{x} de la habilidad seleccionada en la interfaz \refElem{VCT-IU04b}.
	\UCpaso [\UCsist] Muestra la interfaz \refElem{VCT-IU04b}.
	\UCpaso [\UCsist] Continúa en el paso \ref{VCT-CU04:hab} de la trayectoria principal.
\end{UCtrayectoriaA} 

%Trayectorias Alternativas
\begin{UCtrayectoriaA}[Fin de la trayectoria]{E}{El actor no registro al menos un campo obligatorio.}
	\UCpaso [\UCsist] Muestra el mensaje \refIdElem{MSG4} en la interfaz \refElem{VCT-IU03b} en los campos que no
	fueron ingresados.
	\UCpaso [\UCsist] Continúa en el paso \ref{VCT-CU04:vadata2} de la trayectoria principal.
\end{UCtrayectoriaA} 

%Trayectorias Alternativas
\begin{UCtrayectoriaA}[Fin de la trayectoria]{F}{El actor decide cancelar el registro.}
	\UCpaso [\UCactor] Da clic en el botón \IUbutton{Cancelar} en la interfaz \refElem{VCT-IU04b}.
	\UCpaso [\UCsist] Muestra el mensaje \refIdElem{MSG6} en la interfaz \refElem{VCT-IU04b}.
	\UCpaso [\UCactor] Da clic en el botón \IUbutton{Sí} en la interfaz \refElem{VCT-IU04b}.\refTray{H}
	\UCpaso [\UCsist] Muestra la interfaz \refElem{VCT-IU03}.
\end{UCtrayectoriaA} 

%Trayectorias Alternativas
\begin{UCtrayectoriaA}[Fin de la trayectoria]{G}{El actor decide cancelar la acción.}
	\UCpaso [\UCactor] Da clic en el botón \IUbutton{No} en la interfaz \refElem{VCT-IU04a}.
	\UCpaso [\UCsist] Muestra la interfaz \refElem{VCT-IU04a}.
	\UCpaso [\UCsist] Continúa en el paso \ref{VCT-CU04:vadata} de la trayectoria principal.
\end{UCtrayectoriaA} 

%Trayectorias Alternativas
\begin{UCtrayectoriaA}[Fin de la trayectoria]{H}{El actor decide cancelar la acción.}
	\UCpaso [\UCactor] Da clic en el botón \IUbutton{No} en la interfaz \refElem{VCT-IU04b}.
	\UCpaso [\UCsist] Muestra la interfaz \refElem{VCT-IU04b}.
	\UCpaso [\UCsist] Continúa en el paso \ref{VCT-CU04:vadata2} de la trayectoria principal.
\end{UCtrayectoriaA} 
	%\clearpage
\begin{UseCase}[]{VCT-CU04}{Registrar vacante}{
	Permite al reclutador de una empresa  publicar una vacante en el sistema durante cierto periodo de tiempo y así poder gestionas las postulaciones 
	que los usarios(canditos) hagan a dicha vacante.
	}
	%----------------------------------------------------------------
	% Datos generales del CU:
	\UCsection{Atributos}
	\UCitem{Actor(es)}{
		Reclutadores.

	}
	\UCitem[admin]{Prioridad}{
		Media
	}
	\UCitem[admin]{Complejidad}{
		Alta
	}
	\UCitem{Precondiciones}{
		El reclutador debe de estar registrado en el sistema.
	}
	\UCitem{Destino}{
		\Titem \refElem{VCT-IU03}
	}
	\UCitem{Reglas de Negocio}{
		\Titem \refIdElem{RN-N001}
		
	}
	\UCitem{Viene de}{
		\refElem{VCT-CU03}
	}	
\end{UseCase}

%Trayectoria Principal
\begin{UCtrayectoria}
	\UCpaso [\UCactor] Da clic en el icono \IUAgregar{} (Publicar vacante) en la interfaz \refElem{VCT-IU03}.
	\UCpaso [\UCsist] Muestra la interfaz \refElem{VCT-IU04a} en la interfaz \refElem{VCT-IU03}.
	\UCpaso [\UCactor] \label{VCT-CU04:vadata} Ingresa el título y el número de plazas de la vacante, ingresa el código postal, estado, municipio y colonia donde se va laboral.\refTray{A}
	\UCpaso [\UCactor] Ingresa el perfil, la experiencia y el tipo de contratación a la que la vacante va dirigida.
	\UCpaso [\UCactor] Ingresa el horario laboral y el rango salarial indicando si es neto o no el salario.
	\UCpaso [\UCsist] Valida que todos los campos marcados como obligatorios hayan sido ingresados de acuerdo a la regla de negocio \refIdElem{RN-N001}. \refTray{B}
	\UCpaso [\UCactor] Ingresa la descripción de la vacante.
	\UCpaso [\UCactor] \label{VCT-CU04:hab}Selecciona las habilidades y su correspondiente experiencia deseadas para la vacante.\refTray{D}
	\UCpaso [\UCactor] Selecciona la fecha de cierre de la vacante.
	\UCpaso [\UCactor] Da clic en el botón \IUbutton{Publicar} en la interfaz \refElem{VCT-IU04b}.\refTray{A}\refTray{C} \refTray{F}
	\UCpaso [\UCsist] \label{VCT-CU04:vadata2}Valida que todos los campos marcados como obligatorios hayan sido ingresados de acuerdo a la regla de negocio \refIdElem{RN-N001}.\refTray{E}
	\UCpaso [\UCsist] Muestra la interfaz \refElem{VCT-IU03} mostrando la nueva vacante registrada.
	\UCpaso [\UCsist] Notifica a los encargados y colaboradores de que hay una nueva vacante por revisar.
\end{UCtrayectoria}

%Trayectorias Alternativas
\begin{UCtrayectoriaA}[Fin de la trayectoria]{A}{El actor decide cancelar el registro.}
	\UCpaso [\UCactor] Da clic en el botón \IUbutton{Cancelar} en la interfaz \refElem{VCT-IU04a}.
	\UCpaso [\UCsist] Muestra el mensaje \refIdElem{MSG6} en la interfaz \refElem{VCT-IU04a}.
	\UCpaso [\UCactor] Da clic en el botón \IUbutton{Sí} en la interfaz \refElem{VCT-IU04a}.\refTray{G}
	\UCpaso [\UCsist] Muestra la interfaz \refElem{VCT-IU03}.
\end{UCtrayectoriaA} 

%Trayectorias Alternativas
\begin{UCtrayectoriaA}[Fin de la trayectoria]{B}{El actor no registro al menos un campo obligatorio.}
	\UCpaso [\UCsist] Muestra el mensaje \refIdElem{MSG4} en la interfaz \refElem{VCT-IU03a} en los campos que no
	fueron ingresados.
	\UCpaso [\UCsist] Continúa en el paso \ref{VCT-CU04:vadata} de la trayectoria principal.
\end{UCtrayectoriaA} 

%Trayectorias Alternativas
\begin{UCtrayectoriaA}[Fin de la trayectoria]{C}{El actor decide regresa a la pantalla anterior.}
	\UCpaso [\UCactor] Da clic en el botón \IUbutton{Regresar} en la interfaz \refElem{VCT-IU04b}.
	\UCpaso [\UCsist] Muestra la interfaz \refElem{VCT-IU04a}.
	\UCpaso [\UCsist] Continúa en el paso \ref{VCT-CU04:vadata} de la trayectoria principal.
\end{UCtrayectoriaA} 

%Trayectorias Alternativas
\begin{UCtrayectoriaA}[Fin de la trayectoria]{D}{El actor decide eliminar una habilidad.}
	\UCpaso [\UCactor] Da clic en el botón \IUbutton{x} de la habilidad seleccionada en la interfaz \refElem{VCT-IU04b}.
	\UCpaso [\UCsist] Muestra la interfaz \refElem{VCT-IU04b}.
	\UCpaso [\UCsist] Continúa en el paso \ref{VCT-CU04:hab} de la trayectoria principal.
\end{UCtrayectoriaA} 

%Trayectorias Alternativas
\begin{UCtrayectoriaA}[Fin de la trayectoria]{E}{El actor no registro al menos un campo obligatorio.}
	\UCpaso [\UCsist] Muestra el mensaje \refIdElem{MSG4} en la interfaz \refElem{VCT-IU03b} en los campos que no
	fueron ingresados.
	\UCpaso [\UCsist] Continúa en el paso \ref{VCT-CU04:vadata2} de la trayectoria principal.
\end{UCtrayectoriaA} 

%Trayectorias Alternativas
\begin{UCtrayectoriaA}[Fin de la trayectoria]{F}{El actor decide cancelar el registro.}
	\UCpaso [\UCactor] Da clic en el botón \IUbutton{Cancelar} en la interfaz \refElem{VCT-IU04b}.
	\UCpaso [\UCsist] Muestra el mensaje \refIdElem{MSG6} en la interfaz \refElem{VCT-IU04b}.
	\UCpaso [\UCactor] Da clic en el botón \IUbutton{Sí} en la interfaz \refElem{VCT-IU04b}.\refTray{H}
	\UCpaso [\UCsist] Muestra la interfaz \refElem{VCT-IU03}.
\end{UCtrayectoriaA} 

%Trayectorias Alternativas
\begin{UCtrayectoriaA}[Fin de la trayectoria]{G}{El actor decide cancelar la acción.}
	\UCpaso [\UCactor] Da clic en el botón \IUbutton{No} en la interfaz \refElem{VCT-IU04a}.
	\UCpaso [\UCsist] Muestra la interfaz \refElem{VCT-IU04a}.
	\UCpaso [\UCsist] Continúa en el paso \ref{VCT-CU04:vadata} de la trayectoria principal.
\end{UCtrayectoriaA} 

%Trayectorias Alternativas
\begin{UCtrayectoriaA}[Fin de la trayectoria]{H}{El actor decide cancelar la acción.}
	\UCpaso [\UCactor] Da clic en el botón \IUbutton{No} en la interfaz \refElem{VCT-IU04b}.
	\UCpaso [\UCsist] Muestra la interfaz \refElem{VCT-IU04b}.
	\UCpaso [\UCsist] Continúa en el paso \ref{VCT-CU04:vadata2} de la trayectoria principal.
\end{UCtrayectoriaA} 
	%\clearpage
\begin{UseCase}[]{VCT-CU04}{Registrar vacante}{
	Permite al reclutador de una empresa  publicar una vacante en el sistema durante cierto periodo de tiempo y así poder gestionas las postulaciones 
	que los usarios(canditos) hagan a dicha vacante.
	}
	%----------------------------------------------------------------
	% Datos generales del CU:
	\UCsection{Atributos}
	\UCitem{Actor(es)}{
		Reclutadores.

	}
	\UCitem[admin]{Prioridad}{
		Media
	}
	\UCitem[admin]{Complejidad}{
		Alta
	}
	\UCitem{Precondiciones}{
		El reclutador debe de estar registrado en el sistema.
	}
	\UCitem{Destino}{
		\Titem \refElem{VCT-IU03}
	}
	\UCitem{Reglas de Negocio}{
		\Titem \refIdElem{RN-N001}
		
	}
	\UCitem{Viene de}{
		\refElem{VCT-CU03}
	}	
\end{UseCase}

%Trayectoria Principal
\begin{UCtrayectoria}
	\UCpaso [\UCactor] Da clic en el icono \IUAgregar{} (Publicar vacante) en la interfaz \refElem{VCT-IU03}.
	\UCpaso [\UCsist] Muestra la interfaz \refElem{VCT-IU04a} en la interfaz \refElem{VCT-IU03}.
	\UCpaso [\UCactor] \label{VCT-CU04:vadata} Ingresa el título y el número de plazas de la vacante, ingresa el código postal, estado, municipio y colonia donde se va laboral.\refTray{A}
	\UCpaso [\UCactor] Ingresa el perfil, la experiencia y el tipo de contratación a la que la vacante va dirigida.
	\UCpaso [\UCactor] Ingresa el horario laboral y el rango salarial indicando si es neto o no el salario.
	\UCpaso [\UCsist] Valida que todos los campos marcados como obligatorios hayan sido ingresados de acuerdo a la regla de negocio \refIdElem{RN-N001}. \refTray{B}
	\UCpaso [\UCactor] Ingresa la descripción de la vacante.
	\UCpaso [\UCactor] \label{VCT-CU04:hab}Selecciona las habilidades y su correspondiente experiencia deseadas para la vacante.\refTray{D}
	\UCpaso [\UCactor] Selecciona la fecha de cierre de la vacante.
	\UCpaso [\UCactor] Da clic en el botón \IUbutton{Publicar} en la interfaz \refElem{VCT-IU04b}.\refTray{A}\refTray{C} \refTray{F}
	\UCpaso [\UCsist] \label{VCT-CU04:vadata2}Valida que todos los campos marcados como obligatorios hayan sido ingresados de acuerdo a la regla de negocio \refIdElem{RN-N001}.\refTray{E}
	\UCpaso [\UCsist] Muestra la interfaz \refElem{VCT-IU03} mostrando la nueva vacante registrada.
	\UCpaso [\UCsist] Notifica a los encargados y colaboradores de que hay una nueva vacante por revisar.
\end{UCtrayectoria}

%Trayectorias Alternativas
\begin{UCtrayectoriaA}[Fin de la trayectoria]{A}{El actor decide cancelar el registro.}
	\UCpaso [\UCactor] Da clic en el botón \IUbutton{Cancelar} en la interfaz \refElem{VCT-IU04a}.
	\UCpaso [\UCsist] Muestra el mensaje \refIdElem{MSG6} en la interfaz \refElem{VCT-IU04a}.
	\UCpaso [\UCactor] Da clic en el botón \IUbutton{Sí} en la interfaz \refElem{VCT-IU04a}.\refTray{G}
	\UCpaso [\UCsist] Muestra la interfaz \refElem{VCT-IU03}.
\end{UCtrayectoriaA} 

%Trayectorias Alternativas
\begin{UCtrayectoriaA}[Fin de la trayectoria]{B}{El actor no registro al menos un campo obligatorio.}
	\UCpaso [\UCsist] Muestra el mensaje \refIdElem{MSG4} en la interfaz \refElem{VCT-IU03a} en los campos que no
	fueron ingresados.
	\UCpaso [\UCsist] Continúa en el paso \ref{VCT-CU04:vadata} de la trayectoria principal.
\end{UCtrayectoriaA} 

%Trayectorias Alternativas
\begin{UCtrayectoriaA}[Fin de la trayectoria]{C}{El actor decide regresa a la pantalla anterior.}
	\UCpaso [\UCactor] Da clic en el botón \IUbutton{Regresar} en la interfaz \refElem{VCT-IU04b}.
	\UCpaso [\UCsist] Muestra la interfaz \refElem{VCT-IU04a}.
	\UCpaso [\UCsist] Continúa en el paso \ref{VCT-CU04:vadata} de la trayectoria principal.
\end{UCtrayectoriaA} 

%Trayectorias Alternativas
\begin{UCtrayectoriaA}[Fin de la trayectoria]{D}{El actor decide eliminar una habilidad.}
	\UCpaso [\UCactor] Da clic en el botón \IUbutton{x} de la habilidad seleccionada en la interfaz \refElem{VCT-IU04b}.
	\UCpaso [\UCsist] Muestra la interfaz \refElem{VCT-IU04b}.
	\UCpaso [\UCsist] Continúa en el paso \ref{VCT-CU04:hab} de la trayectoria principal.
\end{UCtrayectoriaA} 

%Trayectorias Alternativas
\begin{UCtrayectoriaA}[Fin de la trayectoria]{E}{El actor no registro al menos un campo obligatorio.}
	\UCpaso [\UCsist] Muestra el mensaje \refIdElem{MSG4} en la interfaz \refElem{VCT-IU03b} en los campos que no
	fueron ingresados.
	\UCpaso [\UCsist] Continúa en el paso \ref{VCT-CU04:vadata2} de la trayectoria principal.
\end{UCtrayectoriaA} 

%Trayectorias Alternativas
\begin{UCtrayectoriaA}[Fin de la trayectoria]{F}{El actor decide cancelar el registro.}
	\UCpaso [\UCactor] Da clic en el botón \IUbutton{Cancelar} en la interfaz \refElem{VCT-IU04b}.
	\UCpaso [\UCsist] Muestra el mensaje \refIdElem{MSG6} en la interfaz \refElem{VCT-IU04b}.
	\UCpaso [\UCactor] Da clic en el botón \IUbutton{Sí} en la interfaz \refElem{VCT-IU04b}.\refTray{H}
	\UCpaso [\UCsist] Muestra la interfaz \refElem{VCT-IU03}.
\end{UCtrayectoriaA} 

%Trayectorias Alternativas
\begin{UCtrayectoriaA}[Fin de la trayectoria]{G}{El actor decide cancelar la acción.}
	\UCpaso [\UCactor] Da clic en el botón \IUbutton{No} en la interfaz \refElem{VCT-IU04a}.
	\UCpaso [\UCsist] Muestra la interfaz \refElem{VCT-IU04a}.
	\UCpaso [\UCsist] Continúa en el paso \ref{VCT-CU04:vadata} de la trayectoria principal.
\end{UCtrayectoriaA} 

%Trayectorias Alternativas
\begin{UCtrayectoriaA}[Fin de la trayectoria]{H}{El actor decide cancelar la acción.}
	\UCpaso [\UCactor] Da clic en el botón \IUbutton{No} en la interfaz \refElem{VCT-IU04b}.
	\UCpaso [\UCsist] Muestra la interfaz \refElem{VCT-IU04b}.
	\UCpaso [\UCsist] Continúa en el paso \ref{VCT-CU04:vadata2} de la trayectoria principal.
\end{UCtrayectoriaA} 

	%\clearpage
\begin{UseCase}[]{VCT-CU04}{Registrar vacante}{
	Permite al reclutador de una empresa  publicar una vacante en el sistema durante cierto periodo de tiempo y así poder gestionas las postulaciones 
	que los usarios(canditos) hagan a dicha vacante.
	}
	%----------------------------------------------------------------
	% Datos generales del CU:
	\UCsection{Atributos}
	\UCitem{Actor(es)}{
		Reclutadores.

	}
	\UCitem[admin]{Prioridad}{
		Media
	}
	\UCitem[admin]{Complejidad}{
		Alta
	}
	\UCitem{Precondiciones}{
		El reclutador debe de estar registrado en el sistema.
	}
	\UCitem{Destino}{
		\Titem \refElem{VCT-IU03}
	}
	\UCitem{Reglas de Negocio}{
		\Titem \refIdElem{RN-N001}
		
	}
	\UCitem{Viene de}{
		\refElem{VCT-CU03}
	}	
\end{UseCase}

%Trayectoria Principal
\begin{UCtrayectoria}
	\UCpaso [\UCactor] Da clic en el icono \IUAgregar{} (Publicar vacante) en la interfaz \refElem{VCT-IU03}.
	\UCpaso [\UCsist] Muestra la interfaz \refElem{VCT-IU04a} en la interfaz \refElem{VCT-IU03}.
	\UCpaso [\UCactor] \label{VCT-CU04:vadata} Ingresa el título y el número de plazas de la vacante, ingresa el código postal, estado, municipio y colonia donde se va laboral.\refTray{A}
	\UCpaso [\UCactor] Ingresa el perfil, la experiencia y el tipo de contratación a la que la vacante va dirigida.
	\UCpaso [\UCactor] Ingresa el horario laboral y el rango salarial indicando si es neto o no el salario.
	\UCpaso [\UCsist] Valida que todos los campos marcados como obligatorios hayan sido ingresados de acuerdo a la regla de negocio \refIdElem{RN-N001}. \refTray{B}
	\UCpaso [\UCactor] Ingresa la descripción de la vacante.
	\UCpaso [\UCactor] \label{VCT-CU04:hab}Selecciona las habilidades y su correspondiente experiencia deseadas para la vacante.\refTray{D}
	\UCpaso [\UCactor] Selecciona la fecha de cierre de la vacante.
	\UCpaso [\UCactor] Da clic en el botón \IUbutton{Publicar} en la interfaz \refElem{VCT-IU04b}.\refTray{A}\refTray{C} \refTray{F}
	\UCpaso [\UCsist] \label{VCT-CU04:vadata2}Valida que todos los campos marcados como obligatorios hayan sido ingresados de acuerdo a la regla de negocio \refIdElem{RN-N001}.\refTray{E}
	\UCpaso [\UCsist] Muestra la interfaz \refElem{VCT-IU03} mostrando la nueva vacante registrada.
	\UCpaso [\UCsist] Notifica a los encargados y colaboradores de que hay una nueva vacante por revisar.
\end{UCtrayectoria}

%Trayectorias Alternativas
\begin{UCtrayectoriaA}[Fin de la trayectoria]{A}{El actor decide cancelar el registro.}
	\UCpaso [\UCactor] Da clic en el botón \IUbutton{Cancelar} en la interfaz \refElem{VCT-IU04a}.
	\UCpaso [\UCsist] Muestra el mensaje \refIdElem{MSG6} en la interfaz \refElem{VCT-IU04a}.
	\UCpaso [\UCactor] Da clic en el botón \IUbutton{Sí} en la interfaz \refElem{VCT-IU04a}.\refTray{G}
	\UCpaso [\UCsist] Muestra la interfaz \refElem{VCT-IU03}.
\end{UCtrayectoriaA} 

%Trayectorias Alternativas
\begin{UCtrayectoriaA}[Fin de la trayectoria]{B}{El actor no registro al menos un campo obligatorio.}
	\UCpaso [\UCsist] Muestra el mensaje \refIdElem{MSG4} en la interfaz \refElem{VCT-IU03a} en los campos que no
	fueron ingresados.
	\UCpaso [\UCsist] Continúa en el paso \ref{VCT-CU04:vadata} de la trayectoria principal.
\end{UCtrayectoriaA} 

%Trayectorias Alternativas
\begin{UCtrayectoriaA}[Fin de la trayectoria]{C}{El actor decide regresa a la pantalla anterior.}
	\UCpaso [\UCactor] Da clic en el botón \IUbutton{Regresar} en la interfaz \refElem{VCT-IU04b}.
	\UCpaso [\UCsist] Muestra la interfaz \refElem{VCT-IU04a}.
	\UCpaso [\UCsist] Continúa en el paso \ref{VCT-CU04:vadata} de la trayectoria principal.
\end{UCtrayectoriaA} 

%Trayectorias Alternativas
\begin{UCtrayectoriaA}[Fin de la trayectoria]{D}{El actor decide eliminar una habilidad.}
	\UCpaso [\UCactor] Da clic en el botón \IUbutton{x} de la habilidad seleccionada en la interfaz \refElem{VCT-IU04b}.
	\UCpaso [\UCsist] Muestra la interfaz \refElem{VCT-IU04b}.
	\UCpaso [\UCsist] Continúa en el paso \ref{VCT-CU04:hab} de la trayectoria principal.
\end{UCtrayectoriaA} 

%Trayectorias Alternativas
\begin{UCtrayectoriaA}[Fin de la trayectoria]{E}{El actor no registro al menos un campo obligatorio.}
	\UCpaso [\UCsist] Muestra el mensaje \refIdElem{MSG4} en la interfaz \refElem{VCT-IU03b} en los campos que no
	fueron ingresados.
	\UCpaso [\UCsist] Continúa en el paso \ref{VCT-CU04:vadata2} de la trayectoria principal.
\end{UCtrayectoriaA} 

%Trayectorias Alternativas
\begin{UCtrayectoriaA}[Fin de la trayectoria]{F}{El actor decide cancelar el registro.}
	\UCpaso [\UCactor] Da clic en el botón \IUbutton{Cancelar} en la interfaz \refElem{VCT-IU04b}.
	\UCpaso [\UCsist] Muestra el mensaje \refIdElem{MSG6} en la interfaz \refElem{VCT-IU04b}.
	\UCpaso [\UCactor] Da clic en el botón \IUbutton{Sí} en la interfaz \refElem{VCT-IU04b}.\refTray{H}
	\UCpaso [\UCsist] Muestra la interfaz \refElem{VCT-IU03}.
\end{UCtrayectoriaA} 

%Trayectorias Alternativas
\begin{UCtrayectoriaA}[Fin de la trayectoria]{G}{El actor decide cancelar la acción.}
	\UCpaso [\UCactor] Da clic en el botón \IUbutton{No} en la interfaz \refElem{VCT-IU04a}.
	\UCpaso [\UCsist] Muestra la interfaz \refElem{VCT-IU04a}.
	\UCpaso [\UCsist] Continúa en el paso \ref{VCT-CU04:vadata} de la trayectoria principal.
\end{UCtrayectoriaA} 

%Trayectorias Alternativas
\begin{UCtrayectoriaA}[Fin de la trayectoria]{H}{El actor decide cancelar la acción.}
	\UCpaso [\UCactor] Da clic en el botón \IUbutton{No} en la interfaz \refElem{VCT-IU04b}.
	\UCpaso [\UCsist] Muestra la interfaz \refElem{VCT-IU04b}.
	\UCpaso [\UCsist] Continúa en el paso \ref{VCT-CU04:vadata2} de la trayectoria principal.
\end{UCtrayectoriaA} 


\section{Módulo de Vacantes}
	En la figura \ref{adcu:usr} se muestra el diagrama de casos de uso del módulo vacantes del sistema.

	\begin{figure}[hbtp!]
		\begin{center}
			\includegraphics[width=.8\textwidth]{sprints/imagenes/MUVCT.png}
		\end{center}
		
		\caption{Diagrama de casos de uso del \textit{Módulo vacantes}.}
		\label{adcu:usr}
	\end{figure}

	\begin{itemize}
        \item Los casos de uso \IUazul{} , son aquellos que se pertenecen a esta primera entrega del proyecto.
        \item Los casos de uso \IUblanco{}, se tienen planeados para la segunda entrega del proyecto.
    \end{itemize} 

	\clearpage
\begin{UseCase}[]{VCT-CU04}{Registrar vacante}{
	Permite al reclutador de una empresa  publicar una vacante en el sistema durante cierto periodo de tiempo y así poder gestionas las postulaciones 
	que los usarios(canditos) hagan a dicha vacante.
	}
	%----------------------------------------------------------------
	% Datos generales del CU:
	\UCsection{Atributos}
	\UCitem{Actor(es)}{
		Reclutadores.

	}
	\UCitem[admin]{Prioridad}{
		Media
	}
	\UCitem[admin]{Complejidad}{
		Alta
	}
	\UCitem{Precondiciones}{
		El reclutador debe de estar registrado en el sistema.
	}
	\UCitem{Destino}{
		\Titem \refElem{VCT-IU03}
	}
	\UCitem{Reglas de Negocio}{
		\Titem \refIdElem{RN-N001}
		
	}
	\UCitem{Viene de}{
		\refElem{VCT-CU03}
	}	
\end{UseCase}

%Trayectoria Principal
\begin{UCtrayectoria}
	\UCpaso [\UCactor] Da clic en el icono \IUAgregar{} (Publicar vacante) en la interfaz \refElem{VCT-IU03}.
	\UCpaso [\UCsist] Muestra la interfaz \refElem{VCT-IU04a} en la interfaz \refElem{VCT-IU03}.
	\UCpaso [\UCactor] \label{VCT-CU04:vadata} Ingresa el título y el número de plazas de la vacante, ingresa el código postal, estado, municipio y colonia donde se va laboral.\refTray{A}
	\UCpaso [\UCactor] Ingresa el perfil, la experiencia y el tipo de contratación a la que la vacante va dirigida.
	\UCpaso [\UCactor] Ingresa el horario laboral y el rango salarial indicando si es neto o no el salario.
	\UCpaso [\UCsist] Valida que todos los campos marcados como obligatorios hayan sido ingresados de acuerdo a la regla de negocio \refIdElem{RN-N001}. \refTray{B}
	\UCpaso [\UCactor] Ingresa la descripción de la vacante.
	\UCpaso [\UCactor] \label{VCT-CU04:hab}Selecciona las habilidades y su correspondiente experiencia deseadas para la vacante.\refTray{D}
	\UCpaso [\UCactor] Selecciona la fecha de cierre de la vacante.
	\UCpaso [\UCactor] Da clic en el botón \IUbutton{Publicar} en la interfaz \refElem{VCT-IU04b}.\refTray{A}\refTray{C} \refTray{F}
	\UCpaso [\UCsist] \label{VCT-CU04:vadata2}Valida que todos los campos marcados como obligatorios hayan sido ingresados de acuerdo a la regla de negocio \refIdElem{RN-N001}.\refTray{E}
	\UCpaso [\UCsist] Muestra la interfaz \refElem{VCT-IU03} mostrando la nueva vacante registrada.
	\UCpaso [\UCsist] Notifica a los encargados y colaboradores de que hay una nueva vacante por revisar.
\end{UCtrayectoria}

%Trayectorias Alternativas
\begin{UCtrayectoriaA}[Fin de la trayectoria]{A}{El actor decide cancelar el registro.}
	\UCpaso [\UCactor] Da clic en el botón \IUbutton{Cancelar} en la interfaz \refElem{VCT-IU04a}.
	\UCpaso [\UCsist] Muestra el mensaje \refIdElem{MSG6} en la interfaz \refElem{VCT-IU04a}.
	\UCpaso [\UCactor] Da clic en el botón \IUbutton{Sí} en la interfaz \refElem{VCT-IU04a}.\refTray{G}
	\UCpaso [\UCsist] Muestra la interfaz \refElem{VCT-IU03}.
\end{UCtrayectoriaA} 

%Trayectorias Alternativas
\begin{UCtrayectoriaA}[Fin de la trayectoria]{B}{El actor no registro al menos un campo obligatorio.}
	\UCpaso [\UCsist] Muestra el mensaje \refIdElem{MSG4} en la interfaz \refElem{VCT-IU03a} en los campos que no
	fueron ingresados.
	\UCpaso [\UCsist] Continúa en el paso \ref{VCT-CU04:vadata} de la trayectoria principal.
\end{UCtrayectoriaA} 

%Trayectorias Alternativas
\begin{UCtrayectoriaA}[Fin de la trayectoria]{C}{El actor decide regresa a la pantalla anterior.}
	\UCpaso [\UCactor] Da clic en el botón \IUbutton{Regresar} en la interfaz \refElem{VCT-IU04b}.
	\UCpaso [\UCsist] Muestra la interfaz \refElem{VCT-IU04a}.
	\UCpaso [\UCsist] Continúa en el paso \ref{VCT-CU04:vadata} de la trayectoria principal.
\end{UCtrayectoriaA} 

%Trayectorias Alternativas
\begin{UCtrayectoriaA}[Fin de la trayectoria]{D}{El actor decide eliminar una habilidad.}
	\UCpaso [\UCactor] Da clic en el botón \IUbutton{x} de la habilidad seleccionada en la interfaz \refElem{VCT-IU04b}.
	\UCpaso [\UCsist] Muestra la interfaz \refElem{VCT-IU04b}.
	\UCpaso [\UCsist] Continúa en el paso \ref{VCT-CU04:hab} de la trayectoria principal.
\end{UCtrayectoriaA} 

%Trayectorias Alternativas
\begin{UCtrayectoriaA}[Fin de la trayectoria]{E}{El actor no registro al menos un campo obligatorio.}
	\UCpaso [\UCsist] Muestra el mensaje \refIdElem{MSG4} en la interfaz \refElem{VCT-IU03b} en los campos que no
	fueron ingresados.
	\UCpaso [\UCsist] Continúa en el paso \ref{VCT-CU04:vadata2} de la trayectoria principal.
\end{UCtrayectoriaA} 

%Trayectorias Alternativas
\begin{UCtrayectoriaA}[Fin de la trayectoria]{F}{El actor decide cancelar el registro.}
	\UCpaso [\UCactor] Da clic en el botón \IUbutton{Cancelar} en la interfaz \refElem{VCT-IU04b}.
	\UCpaso [\UCsist] Muestra el mensaje \refIdElem{MSG6} en la interfaz \refElem{VCT-IU04b}.
	\UCpaso [\UCactor] Da clic en el botón \IUbutton{Sí} en la interfaz \refElem{VCT-IU04b}.\refTray{H}
	\UCpaso [\UCsist] Muestra la interfaz \refElem{VCT-IU03}.
\end{UCtrayectoriaA} 

%Trayectorias Alternativas
\begin{UCtrayectoriaA}[Fin de la trayectoria]{G}{El actor decide cancelar la acción.}
	\UCpaso [\UCactor] Da clic en el botón \IUbutton{No} en la interfaz \refElem{VCT-IU04a}.
	\UCpaso [\UCsist] Muestra la interfaz \refElem{VCT-IU04a}.
	\UCpaso [\UCsist] Continúa en el paso \ref{VCT-CU04:vadata} de la trayectoria principal.
\end{UCtrayectoriaA} 

%Trayectorias Alternativas
\begin{UCtrayectoriaA}[Fin de la trayectoria]{H}{El actor decide cancelar la acción.}
	\UCpaso [\UCactor] Da clic en el botón \IUbutton{No} en la interfaz \refElem{VCT-IU04b}.
	\UCpaso [\UCsist] Muestra la interfaz \refElem{VCT-IU04b}.
	\UCpaso [\UCsist] Continúa en el paso \ref{VCT-CU04:vadata2} de la trayectoria principal.
\end{UCtrayectoriaA} 
	\clearpage
\subsection{USR-IU02 Consultar perfil}

\subsubsection{Objetivo}
En la figura \refElem{USR-IU02} se muestra la interfaz correspondiente con la funcionalidad descrita en las
trayectorias del caso de uso \refElem{USR-CU02} , la cual permite al actor la gestión su perfil y la consulta del mismo.

La interfaz esta compuesta ``secciones'' y cada sección corresponde a un formulario diferente, las secciones
son las siguientes:
\begin{itemize}
   \item \textbf{Datos personales}: esta sección tiene como objetivo que el actor actualice su información
   personal,sus datos de contacto y/o habilidades que posee (ver la figura \refElem{USR-IU02a}).
   \item \textbf{Objetivos y metas personales}: esta sección tiene como objetivo que el actor actualice sus objetivos, metas personales y laborales 
   (ver la figura \refElem{USR-IU02b}).
   \item \textbf{Historial académico}: esta sección tiene como objetivo que el actor actualice su información
   académica referente a todos los grados de estudios que tiene hasta la fecha (ver la figura \refElem{USR-IU02c}).
   \item \textbf{Idiomas}: esta sección tiene como objetivo preguntarle que el actor actualice la información de idiomas o en su caso, elimine
   o agregue nuevos idiomas a su perfil  (ver la figura \refElem{USR-IU02d}).
   \item \textbf{Experiencia laboral}: esta sección tiene como objetivo que el actor actualice su información de su experiencia
   laborar que tiene hasta la fecha (ver la figura \refElem{USR-IU02e}).
   \item \textbf{Cursos/Certificaciones}:  esta sección tiene como objetivo que el actor actualice su información de sus Certificaciones
   o cursos que ha tenido durante toda su trayectoria académica y laboral (ver la figura \refElem{USR-IU02f}).
\end{itemize}

\subsubsection{Comandos}
Los siguientes comandos aparecen durante toda la interfaz es decir, cada sección los tiene.

%\Titem \IUPass : Al da clic en el ícono, se muestra la contraseña de lo contrario aparecerá \IUOculto \thinspace sustituyendo cada caracter de la contraseña. \\

\Titem \IUEditar{} : Cuando presiona el ícono, habilita la sección para hacer los datos editables acorde a la sección o elemento seleccionado.
\Titem \IUEliminar{} : Cuando presiona el ícono, habilita la sección para eliminar el elemento seleccionado.
\Titem \IUAgregar{} : Cuando presiona el ícono, habilita la sección para agregar un nuevo el elemento.

\IUfig{.9}{CasosdeUso/USR-CU02/imagenes/USR-IU02.png}{USR-IU02}{Consultar perfil}  
\IUfig{.5}{CasosdeUso/USR-CU02/imagenes/USR-IU02a.png}{USR-IU02a}{Consultar perfil: Datos personales}
\IUfig{.9}{CasosdeUso/USR-CU02/imagenes/USR-IU02b.png}{USR-IU02b}{Consultar perfil: Objetivos y metas personales}  
\IUfig{.9}{CasosdeUso/USR-CU02/imagenes/USR-IU02c.png}{USR-IU02c}{Consultar perfil: Historial académico}  
\IUfig{.9}{CasosdeUso/USR-CU02/imagenes/USR-IU02d.png}{USR-IU02d}{Consultar perfil: Idiomas}
\IUfig{.9}{CasosdeUso/USR-CU02/imagenes/USR-IU02e.png}{USR-IU02e}{Consultar perfil: Experiencia laboral}  
\IUfig{.9}{CasosdeUso/USR-CU02/imagenes/USR-IU02f.png}{USR-IU02f}{Consultar perfil: Cursos/Certificaciones}  


\clearpage


	\clearpage
\begin{UseCase}[]{VCT-CU04}{Registrar vacante}{
	Permite al reclutador de una empresa  publicar una vacante en el sistema durante cierto periodo de tiempo y así poder gestionas las postulaciones 
	que los usarios(canditos) hagan a dicha vacante.
	}
	%----------------------------------------------------------------
	% Datos generales del CU:
	\UCsection{Atributos}
	\UCitem{Actor(es)}{
		Reclutadores.

	}
	\UCitem[admin]{Prioridad}{
		Media
	}
	\UCitem[admin]{Complejidad}{
		Alta
	}
	\UCitem{Precondiciones}{
		El reclutador debe de estar registrado en el sistema.
	}
	\UCitem{Destino}{
		\Titem \refElem{VCT-IU03}
	}
	\UCitem{Reglas de Negocio}{
		\Titem \refIdElem{RN-N001}
		
	}
	\UCitem{Viene de}{
		\refElem{VCT-CU03}
	}	
\end{UseCase}

%Trayectoria Principal
\begin{UCtrayectoria}
	\UCpaso [\UCactor] Da clic en el icono \IUAgregar{} (Publicar vacante) en la interfaz \refElem{VCT-IU03}.
	\UCpaso [\UCsist] Muestra la interfaz \refElem{VCT-IU04a} en la interfaz \refElem{VCT-IU03}.
	\UCpaso [\UCactor] \label{VCT-CU04:vadata} Ingresa el título y el número de plazas de la vacante, ingresa el código postal, estado, municipio y colonia donde se va laboral.\refTray{A}
	\UCpaso [\UCactor] Ingresa el perfil, la experiencia y el tipo de contratación a la que la vacante va dirigida.
	\UCpaso [\UCactor] Ingresa el horario laboral y el rango salarial indicando si es neto o no el salario.
	\UCpaso [\UCsist] Valida que todos los campos marcados como obligatorios hayan sido ingresados de acuerdo a la regla de negocio \refIdElem{RN-N001}. \refTray{B}
	\UCpaso [\UCactor] Ingresa la descripción de la vacante.
	\UCpaso [\UCactor] \label{VCT-CU04:hab}Selecciona las habilidades y su correspondiente experiencia deseadas para la vacante.\refTray{D}
	\UCpaso [\UCactor] Selecciona la fecha de cierre de la vacante.
	\UCpaso [\UCactor] Da clic en el botón \IUbutton{Publicar} en la interfaz \refElem{VCT-IU04b}.\refTray{A}\refTray{C} \refTray{F}
	\UCpaso [\UCsist] \label{VCT-CU04:vadata2}Valida que todos los campos marcados como obligatorios hayan sido ingresados de acuerdo a la regla de negocio \refIdElem{RN-N001}.\refTray{E}
	\UCpaso [\UCsist] Muestra la interfaz \refElem{VCT-IU03} mostrando la nueva vacante registrada.
	\UCpaso [\UCsist] Notifica a los encargados y colaboradores de que hay una nueva vacante por revisar.
\end{UCtrayectoria}

%Trayectorias Alternativas
\begin{UCtrayectoriaA}[Fin de la trayectoria]{A}{El actor decide cancelar el registro.}
	\UCpaso [\UCactor] Da clic en el botón \IUbutton{Cancelar} en la interfaz \refElem{VCT-IU04a}.
	\UCpaso [\UCsist] Muestra el mensaje \refIdElem{MSG6} en la interfaz \refElem{VCT-IU04a}.
	\UCpaso [\UCactor] Da clic en el botón \IUbutton{Sí} en la interfaz \refElem{VCT-IU04a}.\refTray{G}
	\UCpaso [\UCsist] Muestra la interfaz \refElem{VCT-IU03}.
\end{UCtrayectoriaA} 

%Trayectorias Alternativas
\begin{UCtrayectoriaA}[Fin de la trayectoria]{B}{El actor no registro al menos un campo obligatorio.}
	\UCpaso [\UCsist] Muestra el mensaje \refIdElem{MSG4} en la interfaz \refElem{VCT-IU03a} en los campos que no
	fueron ingresados.
	\UCpaso [\UCsist] Continúa en el paso \ref{VCT-CU04:vadata} de la trayectoria principal.
\end{UCtrayectoriaA} 

%Trayectorias Alternativas
\begin{UCtrayectoriaA}[Fin de la trayectoria]{C}{El actor decide regresa a la pantalla anterior.}
	\UCpaso [\UCactor] Da clic en el botón \IUbutton{Regresar} en la interfaz \refElem{VCT-IU04b}.
	\UCpaso [\UCsist] Muestra la interfaz \refElem{VCT-IU04a}.
	\UCpaso [\UCsist] Continúa en el paso \ref{VCT-CU04:vadata} de la trayectoria principal.
\end{UCtrayectoriaA} 

%Trayectorias Alternativas
\begin{UCtrayectoriaA}[Fin de la trayectoria]{D}{El actor decide eliminar una habilidad.}
	\UCpaso [\UCactor] Da clic en el botón \IUbutton{x} de la habilidad seleccionada en la interfaz \refElem{VCT-IU04b}.
	\UCpaso [\UCsist] Muestra la interfaz \refElem{VCT-IU04b}.
	\UCpaso [\UCsist] Continúa en el paso \ref{VCT-CU04:hab} de la trayectoria principal.
\end{UCtrayectoriaA} 

%Trayectorias Alternativas
\begin{UCtrayectoriaA}[Fin de la trayectoria]{E}{El actor no registro al menos un campo obligatorio.}
	\UCpaso [\UCsist] Muestra el mensaje \refIdElem{MSG4} en la interfaz \refElem{VCT-IU03b} en los campos que no
	fueron ingresados.
	\UCpaso [\UCsist] Continúa en el paso \ref{VCT-CU04:vadata2} de la trayectoria principal.
\end{UCtrayectoriaA} 

%Trayectorias Alternativas
\begin{UCtrayectoriaA}[Fin de la trayectoria]{F}{El actor decide cancelar el registro.}
	\UCpaso [\UCactor] Da clic en el botón \IUbutton{Cancelar} en la interfaz \refElem{VCT-IU04b}.
	\UCpaso [\UCsist] Muestra el mensaje \refIdElem{MSG6} en la interfaz \refElem{VCT-IU04b}.
	\UCpaso [\UCactor] Da clic en el botón \IUbutton{Sí} en la interfaz \refElem{VCT-IU04b}.\refTray{H}
	\UCpaso [\UCsist] Muestra la interfaz \refElem{VCT-IU03}.
\end{UCtrayectoriaA} 

%Trayectorias Alternativas
\begin{UCtrayectoriaA}[Fin de la trayectoria]{G}{El actor decide cancelar la acción.}
	\UCpaso [\UCactor] Da clic en el botón \IUbutton{No} en la interfaz \refElem{VCT-IU04a}.
	\UCpaso [\UCsist] Muestra la interfaz \refElem{VCT-IU04a}.
	\UCpaso [\UCsist] Continúa en el paso \ref{VCT-CU04:vadata} de la trayectoria principal.
\end{UCtrayectoriaA} 

%Trayectorias Alternativas
\begin{UCtrayectoriaA}[Fin de la trayectoria]{H}{El actor decide cancelar la acción.}
	\UCpaso [\UCactor] Da clic en el botón \IUbutton{No} en la interfaz \refElem{VCT-IU04b}.
	\UCpaso [\UCsist] Muestra la interfaz \refElem{VCT-IU04b}.
	\UCpaso [\UCsist] Continúa en el paso \ref{VCT-CU04:vadata2} de la trayectoria principal.
\end{UCtrayectoriaA} 
	\clearpage
\subsection{USR-IU02 Consultar perfil}

\subsubsection{Objetivo}
En la figura \refElem{USR-IU02} se muestra la interfaz correspondiente con la funcionalidad descrita en las
trayectorias del caso de uso \refElem{USR-CU02} , la cual permite al actor la gestión su perfil y la consulta del mismo.

La interfaz esta compuesta ``secciones'' y cada sección corresponde a un formulario diferente, las secciones
son las siguientes:
\begin{itemize}
   \item \textbf{Datos personales}: esta sección tiene como objetivo que el actor actualice su información
   personal,sus datos de contacto y/o habilidades que posee (ver la figura \refElem{USR-IU02a}).
   \item \textbf{Objetivos y metas personales}: esta sección tiene como objetivo que el actor actualice sus objetivos, metas personales y laborales 
   (ver la figura \refElem{USR-IU02b}).
   \item \textbf{Historial académico}: esta sección tiene como objetivo que el actor actualice su información
   académica referente a todos los grados de estudios que tiene hasta la fecha (ver la figura \refElem{USR-IU02c}).
   \item \textbf{Idiomas}: esta sección tiene como objetivo preguntarle que el actor actualice la información de idiomas o en su caso, elimine
   o agregue nuevos idiomas a su perfil  (ver la figura \refElem{USR-IU02d}).
   \item \textbf{Experiencia laboral}: esta sección tiene como objetivo que el actor actualice su información de su experiencia
   laborar que tiene hasta la fecha (ver la figura \refElem{USR-IU02e}).
   \item \textbf{Cursos/Certificaciones}:  esta sección tiene como objetivo que el actor actualice su información de sus Certificaciones
   o cursos que ha tenido durante toda su trayectoria académica y laboral (ver la figura \refElem{USR-IU02f}).
\end{itemize}

\subsubsection{Comandos}
Los siguientes comandos aparecen durante toda la interfaz es decir, cada sección los tiene.

%\Titem \IUPass : Al da clic en el ícono, se muestra la contraseña de lo contrario aparecerá \IUOculto \thinspace sustituyendo cada caracter de la contraseña. \\

\Titem \IUEditar{} : Cuando presiona el ícono, habilita la sección para hacer los datos editables acorde a la sección o elemento seleccionado.
\Titem \IUEliminar{} : Cuando presiona el ícono, habilita la sección para eliminar el elemento seleccionado.
\Titem \IUAgregar{} : Cuando presiona el ícono, habilita la sección para agregar un nuevo el elemento.

\IUfig{.9}{CasosdeUso/USR-CU02/imagenes/USR-IU02.png}{USR-IU02}{Consultar perfil}  
\IUfig{.5}{CasosdeUso/USR-CU02/imagenes/USR-IU02a.png}{USR-IU02a}{Consultar perfil: Datos personales}
\IUfig{.9}{CasosdeUso/USR-CU02/imagenes/USR-IU02b.png}{USR-IU02b}{Consultar perfil: Objetivos y metas personales}  
\IUfig{.9}{CasosdeUso/USR-CU02/imagenes/USR-IU02c.png}{USR-IU02c}{Consultar perfil: Historial académico}  
\IUfig{.9}{CasosdeUso/USR-CU02/imagenes/USR-IU02d.png}{USR-IU02d}{Consultar perfil: Idiomas}
\IUfig{.9}{CasosdeUso/USR-CU02/imagenes/USR-IU02e.png}{USR-IU02e}{Consultar perfil: Experiencia laboral}  
\IUfig{.9}{CasosdeUso/USR-CU02/imagenes/USR-IU02f.png}{USR-IU02f}{Consultar perfil: Cursos/Certificaciones}  


\clearpage


	\clearpage
\begin{UseCase}[]{VCT-CU04}{Registrar vacante}{
	Permite al reclutador de una empresa  publicar una vacante en el sistema durante cierto periodo de tiempo y así poder gestionas las postulaciones 
	que los usarios(canditos) hagan a dicha vacante.
	}
	%----------------------------------------------------------------
	% Datos generales del CU:
	\UCsection{Atributos}
	\UCitem{Actor(es)}{
		Reclutadores.

	}
	\UCitem[admin]{Prioridad}{
		Media
	}
	\UCitem[admin]{Complejidad}{
		Alta
	}
	\UCitem{Precondiciones}{
		El reclutador debe de estar registrado en el sistema.
	}
	\UCitem{Destino}{
		\Titem \refElem{VCT-IU03}
	}
	\UCitem{Reglas de Negocio}{
		\Titem \refIdElem{RN-N001}
		
	}
	\UCitem{Viene de}{
		\refElem{VCT-CU03}
	}	
\end{UseCase}

%Trayectoria Principal
\begin{UCtrayectoria}
	\UCpaso [\UCactor] Da clic en el icono \IUAgregar{} (Publicar vacante) en la interfaz \refElem{VCT-IU03}.
	\UCpaso [\UCsist] Muestra la interfaz \refElem{VCT-IU04a} en la interfaz \refElem{VCT-IU03}.
	\UCpaso [\UCactor] \label{VCT-CU04:vadata} Ingresa el título y el número de plazas de la vacante, ingresa el código postal, estado, municipio y colonia donde se va laboral.\refTray{A}
	\UCpaso [\UCactor] Ingresa el perfil, la experiencia y el tipo de contratación a la que la vacante va dirigida.
	\UCpaso [\UCactor] Ingresa el horario laboral y el rango salarial indicando si es neto o no el salario.
	\UCpaso [\UCsist] Valida que todos los campos marcados como obligatorios hayan sido ingresados de acuerdo a la regla de negocio \refIdElem{RN-N001}. \refTray{B}
	\UCpaso [\UCactor] Ingresa la descripción de la vacante.
	\UCpaso [\UCactor] \label{VCT-CU04:hab}Selecciona las habilidades y su correspondiente experiencia deseadas para la vacante.\refTray{D}
	\UCpaso [\UCactor] Selecciona la fecha de cierre de la vacante.
	\UCpaso [\UCactor] Da clic en el botón \IUbutton{Publicar} en la interfaz \refElem{VCT-IU04b}.\refTray{A}\refTray{C} \refTray{F}
	\UCpaso [\UCsist] \label{VCT-CU04:vadata2}Valida que todos los campos marcados como obligatorios hayan sido ingresados de acuerdo a la regla de negocio \refIdElem{RN-N001}.\refTray{E}
	\UCpaso [\UCsist] Muestra la interfaz \refElem{VCT-IU03} mostrando la nueva vacante registrada.
	\UCpaso [\UCsist] Notifica a los encargados y colaboradores de que hay una nueva vacante por revisar.
\end{UCtrayectoria}

%Trayectorias Alternativas
\begin{UCtrayectoriaA}[Fin de la trayectoria]{A}{El actor decide cancelar el registro.}
	\UCpaso [\UCactor] Da clic en el botón \IUbutton{Cancelar} en la interfaz \refElem{VCT-IU04a}.
	\UCpaso [\UCsist] Muestra el mensaje \refIdElem{MSG6} en la interfaz \refElem{VCT-IU04a}.
	\UCpaso [\UCactor] Da clic en el botón \IUbutton{Sí} en la interfaz \refElem{VCT-IU04a}.\refTray{G}
	\UCpaso [\UCsist] Muestra la interfaz \refElem{VCT-IU03}.
\end{UCtrayectoriaA} 

%Trayectorias Alternativas
\begin{UCtrayectoriaA}[Fin de la trayectoria]{B}{El actor no registro al menos un campo obligatorio.}
	\UCpaso [\UCsist] Muestra el mensaje \refIdElem{MSG4} en la interfaz \refElem{VCT-IU03a} en los campos que no
	fueron ingresados.
	\UCpaso [\UCsist] Continúa en el paso \ref{VCT-CU04:vadata} de la trayectoria principal.
\end{UCtrayectoriaA} 

%Trayectorias Alternativas
\begin{UCtrayectoriaA}[Fin de la trayectoria]{C}{El actor decide regresa a la pantalla anterior.}
	\UCpaso [\UCactor] Da clic en el botón \IUbutton{Regresar} en la interfaz \refElem{VCT-IU04b}.
	\UCpaso [\UCsist] Muestra la interfaz \refElem{VCT-IU04a}.
	\UCpaso [\UCsist] Continúa en el paso \ref{VCT-CU04:vadata} de la trayectoria principal.
\end{UCtrayectoriaA} 

%Trayectorias Alternativas
\begin{UCtrayectoriaA}[Fin de la trayectoria]{D}{El actor decide eliminar una habilidad.}
	\UCpaso [\UCactor] Da clic en el botón \IUbutton{x} de la habilidad seleccionada en la interfaz \refElem{VCT-IU04b}.
	\UCpaso [\UCsist] Muestra la interfaz \refElem{VCT-IU04b}.
	\UCpaso [\UCsist] Continúa en el paso \ref{VCT-CU04:hab} de la trayectoria principal.
\end{UCtrayectoriaA} 

%Trayectorias Alternativas
\begin{UCtrayectoriaA}[Fin de la trayectoria]{E}{El actor no registro al menos un campo obligatorio.}
	\UCpaso [\UCsist] Muestra el mensaje \refIdElem{MSG4} en la interfaz \refElem{VCT-IU03b} en los campos que no
	fueron ingresados.
	\UCpaso [\UCsist] Continúa en el paso \ref{VCT-CU04:vadata2} de la trayectoria principal.
\end{UCtrayectoriaA} 

%Trayectorias Alternativas
\begin{UCtrayectoriaA}[Fin de la trayectoria]{F}{El actor decide cancelar el registro.}
	\UCpaso [\UCactor] Da clic en el botón \IUbutton{Cancelar} en la interfaz \refElem{VCT-IU04b}.
	\UCpaso [\UCsist] Muestra el mensaje \refIdElem{MSG6} en la interfaz \refElem{VCT-IU04b}.
	\UCpaso [\UCactor] Da clic en el botón \IUbutton{Sí} en la interfaz \refElem{VCT-IU04b}.\refTray{H}
	\UCpaso [\UCsist] Muestra la interfaz \refElem{VCT-IU03}.
\end{UCtrayectoriaA} 

%Trayectorias Alternativas
\begin{UCtrayectoriaA}[Fin de la trayectoria]{G}{El actor decide cancelar la acción.}
	\UCpaso [\UCactor] Da clic en el botón \IUbutton{No} en la interfaz \refElem{VCT-IU04a}.
	\UCpaso [\UCsist] Muestra la interfaz \refElem{VCT-IU04a}.
	\UCpaso [\UCsist] Continúa en el paso \ref{VCT-CU04:vadata} de la trayectoria principal.
\end{UCtrayectoriaA} 

%Trayectorias Alternativas
\begin{UCtrayectoriaA}[Fin de la trayectoria]{H}{El actor decide cancelar la acción.}
	\UCpaso [\UCactor] Da clic en el botón \IUbutton{No} en la interfaz \refElem{VCT-IU04b}.
	\UCpaso [\UCsist] Muestra la interfaz \refElem{VCT-IU04b}.
	\UCpaso [\UCsist] Continúa en el paso \ref{VCT-CU04:vadata2} de la trayectoria principal.
\end{UCtrayectoriaA} 
	\clearpage
\subsection{USR-IU02 Consultar perfil}

\subsubsection{Objetivo}
En la figura \refElem{USR-IU02} se muestra la interfaz correspondiente con la funcionalidad descrita en las
trayectorias del caso de uso \refElem{USR-CU02} , la cual permite al actor la gestión su perfil y la consulta del mismo.

La interfaz esta compuesta ``secciones'' y cada sección corresponde a un formulario diferente, las secciones
son las siguientes:
\begin{itemize}
   \item \textbf{Datos personales}: esta sección tiene como objetivo que el actor actualice su información
   personal,sus datos de contacto y/o habilidades que posee (ver la figura \refElem{USR-IU02a}).
   \item \textbf{Objetivos y metas personales}: esta sección tiene como objetivo que el actor actualice sus objetivos, metas personales y laborales 
   (ver la figura \refElem{USR-IU02b}).
   \item \textbf{Historial académico}: esta sección tiene como objetivo que el actor actualice su información
   académica referente a todos los grados de estudios que tiene hasta la fecha (ver la figura \refElem{USR-IU02c}).
   \item \textbf{Idiomas}: esta sección tiene como objetivo preguntarle que el actor actualice la información de idiomas o en su caso, elimine
   o agregue nuevos idiomas a su perfil  (ver la figura \refElem{USR-IU02d}).
   \item \textbf{Experiencia laboral}: esta sección tiene como objetivo que el actor actualice su información de su experiencia
   laborar que tiene hasta la fecha (ver la figura \refElem{USR-IU02e}).
   \item \textbf{Cursos/Certificaciones}:  esta sección tiene como objetivo que el actor actualice su información de sus Certificaciones
   o cursos que ha tenido durante toda su trayectoria académica y laboral (ver la figura \refElem{USR-IU02f}).
\end{itemize}

\subsubsection{Comandos}
Los siguientes comandos aparecen durante toda la interfaz es decir, cada sección los tiene.

%\Titem \IUPass : Al da clic en el ícono, se muestra la contraseña de lo contrario aparecerá \IUOculto \thinspace sustituyendo cada caracter de la contraseña. \\

\Titem \IUEditar{} : Cuando presiona el ícono, habilita la sección para hacer los datos editables acorde a la sección o elemento seleccionado.
\Titem \IUEliminar{} : Cuando presiona el ícono, habilita la sección para eliminar el elemento seleccionado.
\Titem \IUAgregar{} : Cuando presiona el ícono, habilita la sección para agregar un nuevo el elemento.

\IUfig{.9}{CasosdeUso/USR-CU02/imagenes/USR-IU02.png}{USR-IU02}{Consultar perfil}  
\IUfig{.5}{CasosdeUso/USR-CU02/imagenes/USR-IU02a.png}{USR-IU02a}{Consultar perfil: Datos personales}
\IUfig{.9}{CasosdeUso/USR-CU02/imagenes/USR-IU02b.png}{USR-IU02b}{Consultar perfil: Objetivos y metas personales}  
\IUfig{.9}{CasosdeUso/USR-CU02/imagenes/USR-IU02c.png}{USR-IU02c}{Consultar perfil: Historial académico}  
\IUfig{.9}{CasosdeUso/USR-CU02/imagenes/USR-IU02d.png}{USR-IU02d}{Consultar perfil: Idiomas}
\IUfig{.9}{CasosdeUso/USR-CU02/imagenes/USR-IU02e.png}{USR-IU02e}{Consultar perfil: Experiencia laboral}  
\IUfig{.9}{CasosdeUso/USR-CU02/imagenes/USR-IU02f.png}{USR-IU02f}{Consultar perfil: Cursos/Certificaciones}  


\clearpage


	\clearpage
\begin{UseCase}[]{VCT-CU04}{Registrar vacante}{
	Permite al reclutador de una empresa  publicar una vacante en el sistema durante cierto periodo de tiempo y así poder gestionas las postulaciones 
	que los usarios(canditos) hagan a dicha vacante.
	}
	%----------------------------------------------------------------
	% Datos generales del CU:
	\UCsection{Atributos}
	\UCitem{Actor(es)}{
		Reclutadores.

	}
	\UCitem[admin]{Prioridad}{
		Media
	}
	\UCitem[admin]{Complejidad}{
		Alta
	}
	\UCitem{Precondiciones}{
		El reclutador debe de estar registrado en el sistema.
	}
	\UCitem{Destino}{
		\Titem \refElem{VCT-IU03}
	}
	\UCitem{Reglas de Negocio}{
		\Titem \refIdElem{RN-N001}
		
	}
	\UCitem{Viene de}{
		\refElem{VCT-CU03}
	}	
\end{UseCase}

%Trayectoria Principal
\begin{UCtrayectoria}
	\UCpaso [\UCactor] Da clic en el icono \IUAgregar{} (Publicar vacante) en la interfaz \refElem{VCT-IU03}.
	\UCpaso [\UCsist] Muestra la interfaz \refElem{VCT-IU04a} en la interfaz \refElem{VCT-IU03}.
	\UCpaso [\UCactor] \label{VCT-CU04:vadata} Ingresa el título y el número de plazas de la vacante, ingresa el código postal, estado, municipio y colonia donde se va laboral.\refTray{A}
	\UCpaso [\UCactor] Ingresa el perfil, la experiencia y el tipo de contratación a la que la vacante va dirigida.
	\UCpaso [\UCactor] Ingresa el horario laboral y el rango salarial indicando si es neto o no el salario.
	\UCpaso [\UCsist] Valida que todos los campos marcados como obligatorios hayan sido ingresados de acuerdo a la regla de negocio \refIdElem{RN-N001}. \refTray{B}
	\UCpaso [\UCactor] Ingresa la descripción de la vacante.
	\UCpaso [\UCactor] \label{VCT-CU04:hab}Selecciona las habilidades y su correspondiente experiencia deseadas para la vacante.\refTray{D}
	\UCpaso [\UCactor] Selecciona la fecha de cierre de la vacante.
	\UCpaso [\UCactor] Da clic en el botón \IUbutton{Publicar} en la interfaz \refElem{VCT-IU04b}.\refTray{A}\refTray{C} \refTray{F}
	\UCpaso [\UCsist] \label{VCT-CU04:vadata2}Valida que todos los campos marcados como obligatorios hayan sido ingresados de acuerdo a la regla de negocio \refIdElem{RN-N001}.\refTray{E}
	\UCpaso [\UCsist] Muestra la interfaz \refElem{VCT-IU03} mostrando la nueva vacante registrada.
	\UCpaso [\UCsist] Notifica a los encargados y colaboradores de que hay una nueva vacante por revisar.
\end{UCtrayectoria}

%Trayectorias Alternativas
\begin{UCtrayectoriaA}[Fin de la trayectoria]{A}{El actor decide cancelar el registro.}
	\UCpaso [\UCactor] Da clic en el botón \IUbutton{Cancelar} en la interfaz \refElem{VCT-IU04a}.
	\UCpaso [\UCsist] Muestra el mensaje \refIdElem{MSG6} en la interfaz \refElem{VCT-IU04a}.
	\UCpaso [\UCactor] Da clic en el botón \IUbutton{Sí} en la interfaz \refElem{VCT-IU04a}.\refTray{G}
	\UCpaso [\UCsist] Muestra la interfaz \refElem{VCT-IU03}.
\end{UCtrayectoriaA} 

%Trayectorias Alternativas
\begin{UCtrayectoriaA}[Fin de la trayectoria]{B}{El actor no registro al menos un campo obligatorio.}
	\UCpaso [\UCsist] Muestra el mensaje \refIdElem{MSG4} en la interfaz \refElem{VCT-IU03a} en los campos que no
	fueron ingresados.
	\UCpaso [\UCsist] Continúa en el paso \ref{VCT-CU04:vadata} de la trayectoria principal.
\end{UCtrayectoriaA} 

%Trayectorias Alternativas
\begin{UCtrayectoriaA}[Fin de la trayectoria]{C}{El actor decide regresa a la pantalla anterior.}
	\UCpaso [\UCactor] Da clic en el botón \IUbutton{Regresar} en la interfaz \refElem{VCT-IU04b}.
	\UCpaso [\UCsist] Muestra la interfaz \refElem{VCT-IU04a}.
	\UCpaso [\UCsist] Continúa en el paso \ref{VCT-CU04:vadata} de la trayectoria principal.
\end{UCtrayectoriaA} 

%Trayectorias Alternativas
\begin{UCtrayectoriaA}[Fin de la trayectoria]{D}{El actor decide eliminar una habilidad.}
	\UCpaso [\UCactor] Da clic en el botón \IUbutton{x} de la habilidad seleccionada en la interfaz \refElem{VCT-IU04b}.
	\UCpaso [\UCsist] Muestra la interfaz \refElem{VCT-IU04b}.
	\UCpaso [\UCsist] Continúa en el paso \ref{VCT-CU04:hab} de la trayectoria principal.
\end{UCtrayectoriaA} 

%Trayectorias Alternativas
\begin{UCtrayectoriaA}[Fin de la trayectoria]{E}{El actor no registro al menos un campo obligatorio.}
	\UCpaso [\UCsist] Muestra el mensaje \refIdElem{MSG4} en la interfaz \refElem{VCT-IU03b} en los campos que no
	fueron ingresados.
	\UCpaso [\UCsist] Continúa en el paso \ref{VCT-CU04:vadata2} de la trayectoria principal.
\end{UCtrayectoriaA} 

%Trayectorias Alternativas
\begin{UCtrayectoriaA}[Fin de la trayectoria]{F}{El actor decide cancelar el registro.}
	\UCpaso [\UCactor] Da clic en el botón \IUbutton{Cancelar} en la interfaz \refElem{VCT-IU04b}.
	\UCpaso [\UCsist] Muestra el mensaje \refIdElem{MSG6} en la interfaz \refElem{VCT-IU04b}.
	\UCpaso [\UCactor] Da clic en el botón \IUbutton{Sí} en la interfaz \refElem{VCT-IU04b}.\refTray{H}
	\UCpaso [\UCsist] Muestra la interfaz \refElem{VCT-IU03}.
\end{UCtrayectoriaA} 

%Trayectorias Alternativas
\begin{UCtrayectoriaA}[Fin de la trayectoria]{G}{El actor decide cancelar la acción.}
	\UCpaso [\UCactor] Da clic en el botón \IUbutton{No} en la interfaz \refElem{VCT-IU04a}.
	\UCpaso [\UCsist] Muestra la interfaz \refElem{VCT-IU04a}.
	\UCpaso [\UCsist] Continúa en el paso \ref{VCT-CU04:vadata} de la trayectoria principal.
\end{UCtrayectoriaA} 

%Trayectorias Alternativas
\begin{UCtrayectoriaA}[Fin de la trayectoria]{H}{El actor decide cancelar la acción.}
	\UCpaso [\UCactor] Da clic en el botón \IUbutton{No} en la interfaz \refElem{VCT-IU04b}.
	\UCpaso [\UCsist] Muestra la interfaz \refElem{VCT-IU04b}.
	\UCpaso [\UCsist] Continúa en el paso \ref{VCT-CU04:vadata2} de la trayectoria principal.
\end{UCtrayectoriaA} 
	\clearpage
\subsection{USR-IU02 Consultar perfil}

\subsubsection{Objetivo}
En la figura \refElem{USR-IU02} se muestra la interfaz correspondiente con la funcionalidad descrita en las
trayectorias del caso de uso \refElem{USR-CU02} , la cual permite al actor la gestión su perfil y la consulta del mismo.

La interfaz esta compuesta ``secciones'' y cada sección corresponde a un formulario diferente, las secciones
son las siguientes:
\begin{itemize}
   \item \textbf{Datos personales}: esta sección tiene como objetivo que el actor actualice su información
   personal,sus datos de contacto y/o habilidades que posee (ver la figura \refElem{USR-IU02a}).
   \item \textbf{Objetivos y metas personales}: esta sección tiene como objetivo que el actor actualice sus objetivos, metas personales y laborales 
   (ver la figura \refElem{USR-IU02b}).
   \item \textbf{Historial académico}: esta sección tiene como objetivo que el actor actualice su información
   académica referente a todos los grados de estudios que tiene hasta la fecha (ver la figura \refElem{USR-IU02c}).
   \item \textbf{Idiomas}: esta sección tiene como objetivo preguntarle que el actor actualice la información de idiomas o en su caso, elimine
   o agregue nuevos idiomas a su perfil  (ver la figura \refElem{USR-IU02d}).
   \item \textbf{Experiencia laboral}: esta sección tiene como objetivo que el actor actualice su información de su experiencia
   laborar que tiene hasta la fecha (ver la figura \refElem{USR-IU02e}).
   \item \textbf{Cursos/Certificaciones}:  esta sección tiene como objetivo que el actor actualice su información de sus Certificaciones
   o cursos que ha tenido durante toda su trayectoria académica y laboral (ver la figura \refElem{USR-IU02f}).
\end{itemize}

\subsubsection{Comandos}
Los siguientes comandos aparecen durante toda la interfaz es decir, cada sección los tiene.

%\Titem \IUPass : Al da clic en el ícono, se muestra la contraseña de lo contrario aparecerá \IUOculto \thinspace sustituyendo cada caracter de la contraseña. \\

\Titem \IUEditar{} : Cuando presiona el ícono, habilita la sección para hacer los datos editables acorde a la sección o elemento seleccionado.
\Titem \IUEliminar{} : Cuando presiona el ícono, habilita la sección para eliminar el elemento seleccionado.
\Titem \IUAgregar{} : Cuando presiona el ícono, habilita la sección para agregar un nuevo el elemento.

\IUfig{.9}{CasosdeUso/USR-CU02/imagenes/USR-IU02.png}{USR-IU02}{Consultar perfil}  
\IUfig{.5}{CasosdeUso/USR-CU02/imagenes/USR-IU02a.png}{USR-IU02a}{Consultar perfil: Datos personales}
\IUfig{.9}{CasosdeUso/USR-CU02/imagenes/USR-IU02b.png}{USR-IU02b}{Consultar perfil: Objetivos y metas personales}  
\IUfig{.9}{CasosdeUso/USR-CU02/imagenes/USR-IU02c.png}{USR-IU02c}{Consultar perfil: Historial académico}  
\IUfig{.9}{CasosdeUso/USR-CU02/imagenes/USR-IU02d.png}{USR-IU02d}{Consultar perfil: Idiomas}
\IUfig{.9}{CasosdeUso/USR-CU02/imagenes/USR-IU02e.png}{USR-IU02e}{Consultar perfil: Experiencia laboral}  
\IUfig{.9}{CasosdeUso/USR-CU02/imagenes/USR-IU02f.png}{USR-IU02f}{Consultar perfil: Cursos/Certificaciones}  


\clearpage


	\clearpage
\begin{UseCase}[]{VCT-CU04}{Registrar vacante}{
	Permite al reclutador de una empresa  publicar una vacante en el sistema durante cierto periodo de tiempo y así poder gestionas las postulaciones 
	que los usarios(canditos) hagan a dicha vacante.
	}
	%----------------------------------------------------------------
	% Datos generales del CU:
	\UCsection{Atributos}
	\UCitem{Actor(es)}{
		Reclutadores.

	}
	\UCitem[admin]{Prioridad}{
		Media
	}
	\UCitem[admin]{Complejidad}{
		Alta
	}
	\UCitem{Precondiciones}{
		El reclutador debe de estar registrado en el sistema.
	}
	\UCitem{Destino}{
		\Titem \refElem{VCT-IU03}
	}
	\UCitem{Reglas de Negocio}{
		\Titem \refIdElem{RN-N001}
		
	}
	\UCitem{Viene de}{
		\refElem{VCT-CU03}
	}	
\end{UseCase}

%Trayectoria Principal
\begin{UCtrayectoria}
	\UCpaso [\UCactor] Da clic en el icono \IUAgregar{} (Publicar vacante) en la interfaz \refElem{VCT-IU03}.
	\UCpaso [\UCsist] Muestra la interfaz \refElem{VCT-IU04a} en la interfaz \refElem{VCT-IU03}.
	\UCpaso [\UCactor] \label{VCT-CU04:vadata} Ingresa el título y el número de plazas de la vacante, ingresa el código postal, estado, municipio y colonia donde se va laboral.\refTray{A}
	\UCpaso [\UCactor] Ingresa el perfil, la experiencia y el tipo de contratación a la que la vacante va dirigida.
	\UCpaso [\UCactor] Ingresa el horario laboral y el rango salarial indicando si es neto o no el salario.
	\UCpaso [\UCsist] Valida que todos los campos marcados como obligatorios hayan sido ingresados de acuerdo a la regla de negocio \refIdElem{RN-N001}. \refTray{B}
	\UCpaso [\UCactor] Ingresa la descripción de la vacante.
	\UCpaso [\UCactor] \label{VCT-CU04:hab}Selecciona las habilidades y su correspondiente experiencia deseadas para la vacante.\refTray{D}
	\UCpaso [\UCactor] Selecciona la fecha de cierre de la vacante.
	\UCpaso [\UCactor] Da clic en el botón \IUbutton{Publicar} en la interfaz \refElem{VCT-IU04b}.\refTray{A}\refTray{C} \refTray{F}
	\UCpaso [\UCsist] \label{VCT-CU04:vadata2}Valida que todos los campos marcados como obligatorios hayan sido ingresados de acuerdo a la regla de negocio \refIdElem{RN-N001}.\refTray{E}
	\UCpaso [\UCsist] Muestra la interfaz \refElem{VCT-IU03} mostrando la nueva vacante registrada.
	\UCpaso [\UCsist] Notifica a los encargados y colaboradores de que hay una nueva vacante por revisar.
\end{UCtrayectoria}

%Trayectorias Alternativas
\begin{UCtrayectoriaA}[Fin de la trayectoria]{A}{El actor decide cancelar el registro.}
	\UCpaso [\UCactor] Da clic en el botón \IUbutton{Cancelar} en la interfaz \refElem{VCT-IU04a}.
	\UCpaso [\UCsist] Muestra el mensaje \refIdElem{MSG6} en la interfaz \refElem{VCT-IU04a}.
	\UCpaso [\UCactor] Da clic en el botón \IUbutton{Sí} en la interfaz \refElem{VCT-IU04a}.\refTray{G}
	\UCpaso [\UCsist] Muestra la interfaz \refElem{VCT-IU03}.
\end{UCtrayectoriaA} 

%Trayectorias Alternativas
\begin{UCtrayectoriaA}[Fin de la trayectoria]{B}{El actor no registro al menos un campo obligatorio.}
	\UCpaso [\UCsist] Muestra el mensaje \refIdElem{MSG4} en la interfaz \refElem{VCT-IU03a} en los campos que no
	fueron ingresados.
	\UCpaso [\UCsist] Continúa en el paso \ref{VCT-CU04:vadata} de la trayectoria principal.
\end{UCtrayectoriaA} 

%Trayectorias Alternativas
\begin{UCtrayectoriaA}[Fin de la trayectoria]{C}{El actor decide regresa a la pantalla anterior.}
	\UCpaso [\UCactor] Da clic en el botón \IUbutton{Regresar} en la interfaz \refElem{VCT-IU04b}.
	\UCpaso [\UCsist] Muestra la interfaz \refElem{VCT-IU04a}.
	\UCpaso [\UCsist] Continúa en el paso \ref{VCT-CU04:vadata} de la trayectoria principal.
\end{UCtrayectoriaA} 

%Trayectorias Alternativas
\begin{UCtrayectoriaA}[Fin de la trayectoria]{D}{El actor decide eliminar una habilidad.}
	\UCpaso [\UCactor] Da clic en el botón \IUbutton{x} de la habilidad seleccionada en la interfaz \refElem{VCT-IU04b}.
	\UCpaso [\UCsist] Muestra la interfaz \refElem{VCT-IU04b}.
	\UCpaso [\UCsist] Continúa en el paso \ref{VCT-CU04:hab} de la trayectoria principal.
\end{UCtrayectoriaA} 

%Trayectorias Alternativas
\begin{UCtrayectoriaA}[Fin de la trayectoria]{E}{El actor no registro al menos un campo obligatorio.}
	\UCpaso [\UCsist] Muestra el mensaje \refIdElem{MSG4} en la interfaz \refElem{VCT-IU03b} en los campos que no
	fueron ingresados.
	\UCpaso [\UCsist] Continúa en el paso \ref{VCT-CU04:vadata2} de la trayectoria principal.
\end{UCtrayectoriaA} 

%Trayectorias Alternativas
\begin{UCtrayectoriaA}[Fin de la trayectoria]{F}{El actor decide cancelar el registro.}
	\UCpaso [\UCactor] Da clic en el botón \IUbutton{Cancelar} en la interfaz \refElem{VCT-IU04b}.
	\UCpaso [\UCsist] Muestra el mensaje \refIdElem{MSG6} en la interfaz \refElem{VCT-IU04b}.
	\UCpaso [\UCactor] Da clic en el botón \IUbutton{Sí} en la interfaz \refElem{VCT-IU04b}.\refTray{H}
	\UCpaso [\UCsist] Muestra la interfaz \refElem{VCT-IU03}.
\end{UCtrayectoriaA} 

%Trayectorias Alternativas
\begin{UCtrayectoriaA}[Fin de la trayectoria]{G}{El actor decide cancelar la acción.}
	\UCpaso [\UCactor] Da clic en el botón \IUbutton{No} en la interfaz \refElem{VCT-IU04a}.
	\UCpaso [\UCsist] Muestra la interfaz \refElem{VCT-IU04a}.
	\UCpaso [\UCsist] Continúa en el paso \ref{VCT-CU04:vadata} de la trayectoria principal.
\end{UCtrayectoriaA} 

%Trayectorias Alternativas
\begin{UCtrayectoriaA}[Fin de la trayectoria]{H}{El actor decide cancelar la acción.}
	\UCpaso [\UCactor] Da clic en el botón \IUbutton{No} en la interfaz \refElem{VCT-IU04b}.
	\UCpaso [\UCsist] Muestra la interfaz \refElem{VCT-IU04b}.
	\UCpaso [\UCsist] Continúa en el paso \ref{VCT-CU04:vadata2} de la trayectoria principal.
\end{UCtrayectoriaA} 
	\clearpage
\subsection{USR-IU02 Consultar perfil}

\subsubsection{Objetivo}
En la figura \refElem{USR-IU02} se muestra la interfaz correspondiente con la funcionalidad descrita en las
trayectorias del caso de uso \refElem{USR-CU02} , la cual permite al actor la gestión su perfil y la consulta del mismo.

La interfaz esta compuesta ``secciones'' y cada sección corresponde a un formulario diferente, las secciones
son las siguientes:
\begin{itemize}
   \item \textbf{Datos personales}: esta sección tiene como objetivo que el actor actualice su información
   personal,sus datos de contacto y/o habilidades que posee (ver la figura \refElem{USR-IU02a}).
   \item \textbf{Objetivos y metas personales}: esta sección tiene como objetivo que el actor actualice sus objetivos, metas personales y laborales 
   (ver la figura \refElem{USR-IU02b}).
   \item \textbf{Historial académico}: esta sección tiene como objetivo que el actor actualice su información
   académica referente a todos los grados de estudios que tiene hasta la fecha (ver la figura \refElem{USR-IU02c}).
   \item \textbf{Idiomas}: esta sección tiene como objetivo preguntarle que el actor actualice la información de idiomas o en su caso, elimine
   o agregue nuevos idiomas a su perfil  (ver la figura \refElem{USR-IU02d}).
   \item \textbf{Experiencia laboral}: esta sección tiene como objetivo que el actor actualice su información de su experiencia
   laborar que tiene hasta la fecha (ver la figura \refElem{USR-IU02e}).
   \item \textbf{Cursos/Certificaciones}:  esta sección tiene como objetivo que el actor actualice su información de sus Certificaciones
   o cursos que ha tenido durante toda su trayectoria académica y laboral (ver la figura \refElem{USR-IU02f}).
\end{itemize}

\subsubsection{Comandos}
Los siguientes comandos aparecen durante toda la interfaz es decir, cada sección los tiene.

%\Titem \IUPass : Al da clic en el ícono, se muestra la contraseña de lo contrario aparecerá \IUOculto \thinspace sustituyendo cada caracter de la contraseña. \\

\Titem \IUEditar{} : Cuando presiona el ícono, habilita la sección para hacer los datos editables acorde a la sección o elemento seleccionado.
\Titem \IUEliminar{} : Cuando presiona el ícono, habilita la sección para eliminar el elemento seleccionado.
\Titem \IUAgregar{} : Cuando presiona el ícono, habilita la sección para agregar un nuevo el elemento.

\IUfig{.9}{CasosdeUso/USR-CU02/imagenes/USR-IU02.png}{USR-IU02}{Consultar perfil}  
\IUfig{.5}{CasosdeUso/USR-CU02/imagenes/USR-IU02a.png}{USR-IU02a}{Consultar perfil: Datos personales}
\IUfig{.9}{CasosdeUso/USR-CU02/imagenes/USR-IU02b.png}{USR-IU02b}{Consultar perfil: Objetivos y metas personales}  
\IUfig{.9}{CasosdeUso/USR-CU02/imagenes/USR-IU02c.png}{USR-IU02c}{Consultar perfil: Historial académico}  
\IUfig{.9}{CasosdeUso/USR-CU02/imagenes/USR-IU02d.png}{USR-IU02d}{Consultar perfil: Idiomas}
\IUfig{.9}{CasosdeUso/USR-CU02/imagenes/USR-IU02e.png}{USR-IU02e}{Consultar perfil: Experiencia laboral}  
\IUfig{.9}{CasosdeUso/USR-CU02/imagenes/USR-IU02f.png}{USR-IU02f}{Consultar perfil: Cursos/Certificaciones}  


\clearpage

	


	\section{Módulo de Postulaciones}
	En la figura \ref{adcu:usr} se muestra el diagrama de casos de uso del módulo postulaciones del sistema.

	\begin{figure}[hbtp!]
		\begin{center}
			\includegraphics[width=.8\textwidth]{sprints/imagenes/DCUPST.png}
		\end{center}
		
		\caption{Diagrama de casos de uso del \textit{Módulo postulaciones}.}
		\label{adcu:usr}
	\end{figure}

	\begin{itemize}
        \item Los casos de uso \IUazul{} , son aquellos que se pertenecen a esta primera entrega del proyecto.
        \item Los casos de uso \IUblanco{}, se tienen planeados para la segunda entrega del proyecto.
    \end{itemize} 

	\clearpage
\begin{UseCase}[]{VCT-CU04}{Registrar vacante}{
	Permite al reclutador de una empresa  publicar una vacante en el sistema durante cierto periodo de tiempo y así poder gestionas las postulaciones 
	que los usarios(canditos) hagan a dicha vacante.
	}
	%----------------------------------------------------------------
	% Datos generales del CU:
	\UCsection{Atributos}
	\UCitem{Actor(es)}{
		Reclutadores.

	}
	\UCitem[admin]{Prioridad}{
		Media
	}
	\UCitem[admin]{Complejidad}{
		Alta
	}
	\UCitem{Precondiciones}{
		El reclutador debe de estar registrado en el sistema.
	}
	\UCitem{Destino}{
		\Titem \refElem{VCT-IU03}
	}
	\UCitem{Reglas de Negocio}{
		\Titem \refIdElem{RN-N001}
		
	}
	\UCitem{Viene de}{
		\refElem{VCT-CU03}
	}	
\end{UseCase}

%Trayectoria Principal
\begin{UCtrayectoria}
	\UCpaso [\UCactor] Da clic en el icono \IUAgregar{} (Publicar vacante) en la interfaz \refElem{VCT-IU03}.
	\UCpaso [\UCsist] Muestra la interfaz \refElem{VCT-IU04a} en la interfaz \refElem{VCT-IU03}.
	\UCpaso [\UCactor] \label{VCT-CU04:vadata} Ingresa el título y el número de plazas de la vacante, ingresa el código postal, estado, municipio y colonia donde se va laboral.\refTray{A}
	\UCpaso [\UCactor] Ingresa el perfil, la experiencia y el tipo de contratación a la que la vacante va dirigida.
	\UCpaso [\UCactor] Ingresa el horario laboral y el rango salarial indicando si es neto o no el salario.
	\UCpaso [\UCsist] Valida que todos los campos marcados como obligatorios hayan sido ingresados de acuerdo a la regla de negocio \refIdElem{RN-N001}. \refTray{B}
	\UCpaso [\UCactor] Ingresa la descripción de la vacante.
	\UCpaso [\UCactor] \label{VCT-CU04:hab}Selecciona las habilidades y su correspondiente experiencia deseadas para la vacante.\refTray{D}
	\UCpaso [\UCactor] Selecciona la fecha de cierre de la vacante.
	\UCpaso [\UCactor] Da clic en el botón \IUbutton{Publicar} en la interfaz \refElem{VCT-IU04b}.\refTray{A}\refTray{C} \refTray{F}
	\UCpaso [\UCsist] \label{VCT-CU04:vadata2}Valida que todos los campos marcados como obligatorios hayan sido ingresados de acuerdo a la regla de negocio \refIdElem{RN-N001}.\refTray{E}
	\UCpaso [\UCsist] Muestra la interfaz \refElem{VCT-IU03} mostrando la nueva vacante registrada.
	\UCpaso [\UCsist] Notifica a los encargados y colaboradores de que hay una nueva vacante por revisar.
\end{UCtrayectoria}

%Trayectorias Alternativas
\begin{UCtrayectoriaA}[Fin de la trayectoria]{A}{El actor decide cancelar el registro.}
	\UCpaso [\UCactor] Da clic en el botón \IUbutton{Cancelar} en la interfaz \refElem{VCT-IU04a}.
	\UCpaso [\UCsist] Muestra el mensaje \refIdElem{MSG6} en la interfaz \refElem{VCT-IU04a}.
	\UCpaso [\UCactor] Da clic en el botón \IUbutton{Sí} en la interfaz \refElem{VCT-IU04a}.\refTray{G}
	\UCpaso [\UCsist] Muestra la interfaz \refElem{VCT-IU03}.
\end{UCtrayectoriaA} 

%Trayectorias Alternativas
\begin{UCtrayectoriaA}[Fin de la trayectoria]{B}{El actor no registro al menos un campo obligatorio.}
	\UCpaso [\UCsist] Muestra el mensaje \refIdElem{MSG4} en la interfaz \refElem{VCT-IU03a} en los campos que no
	fueron ingresados.
	\UCpaso [\UCsist] Continúa en el paso \ref{VCT-CU04:vadata} de la trayectoria principal.
\end{UCtrayectoriaA} 

%Trayectorias Alternativas
\begin{UCtrayectoriaA}[Fin de la trayectoria]{C}{El actor decide regresa a la pantalla anterior.}
	\UCpaso [\UCactor] Da clic en el botón \IUbutton{Regresar} en la interfaz \refElem{VCT-IU04b}.
	\UCpaso [\UCsist] Muestra la interfaz \refElem{VCT-IU04a}.
	\UCpaso [\UCsist] Continúa en el paso \ref{VCT-CU04:vadata} de la trayectoria principal.
\end{UCtrayectoriaA} 

%Trayectorias Alternativas
\begin{UCtrayectoriaA}[Fin de la trayectoria]{D}{El actor decide eliminar una habilidad.}
	\UCpaso [\UCactor] Da clic en el botón \IUbutton{x} de la habilidad seleccionada en la interfaz \refElem{VCT-IU04b}.
	\UCpaso [\UCsist] Muestra la interfaz \refElem{VCT-IU04b}.
	\UCpaso [\UCsist] Continúa en el paso \ref{VCT-CU04:hab} de la trayectoria principal.
\end{UCtrayectoriaA} 

%Trayectorias Alternativas
\begin{UCtrayectoriaA}[Fin de la trayectoria]{E}{El actor no registro al menos un campo obligatorio.}
	\UCpaso [\UCsist] Muestra el mensaje \refIdElem{MSG4} en la interfaz \refElem{VCT-IU03b} en los campos que no
	fueron ingresados.
	\UCpaso [\UCsist] Continúa en el paso \ref{VCT-CU04:vadata2} de la trayectoria principal.
\end{UCtrayectoriaA} 

%Trayectorias Alternativas
\begin{UCtrayectoriaA}[Fin de la trayectoria]{F}{El actor decide cancelar el registro.}
	\UCpaso [\UCactor] Da clic en el botón \IUbutton{Cancelar} en la interfaz \refElem{VCT-IU04b}.
	\UCpaso [\UCsist] Muestra el mensaje \refIdElem{MSG6} en la interfaz \refElem{VCT-IU04b}.
	\UCpaso [\UCactor] Da clic en el botón \IUbutton{Sí} en la interfaz \refElem{VCT-IU04b}.\refTray{H}
	\UCpaso [\UCsist] Muestra la interfaz \refElem{VCT-IU03}.
\end{UCtrayectoriaA} 

%Trayectorias Alternativas
\begin{UCtrayectoriaA}[Fin de la trayectoria]{G}{El actor decide cancelar la acción.}
	\UCpaso [\UCactor] Da clic en el botón \IUbutton{No} en la interfaz \refElem{VCT-IU04a}.
	\UCpaso [\UCsist] Muestra la interfaz \refElem{VCT-IU04a}.
	\UCpaso [\UCsist] Continúa en el paso \ref{VCT-CU04:vadata} de la trayectoria principal.
\end{UCtrayectoriaA} 

%Trayectorias Alternativas
\begin{UCtrayectoriaA}[Fin de la trayectoria]{H}{El actor decide cancelar la acción.}
	\UCpaso [\UCactor] Da clic en el botón \IUbutton{No} en la interfaz \refElem{VCT-IU04b}.
	\UCpaso [\UCsist] Muestra la interfaz \refElem{VCT-IU04b}.
	\UCpaso [\UCsist] Continúa en el paso \ref{VCT-CU04:vadata2} de la trayectoria principal.
\end{UCtrayectoriaA} 
	\clearpage
\subsection{USR-IU02 Consultar perfil}

\subsubsection{Objetivo}
En la figura \refElem{USR-IU02} se muestra la interfaz correspondiente con la funcionalidad descrita en las
trayectorias del caso de uso \refElem{USR-CU02} , la cual permite al actor la gestión su perfil y la consulta del mismo.

La interfaz esta compuesta ``secciones'' y cada sección corresponde a un formulario diferente, las secciones
son las siguientes:
\begin{itemize}
   \item \textbf{Datos personales}: esta sección tiene como objetivo que el actor actualice su información
   personal,sus datos de contacto y/o habilidades que posee (ver la figura \refElem{USR-IU02a}).
   \item \textbf{Objetivos y metas personales}: esta sección tiene como objetivo que el actor actualice sus objetivos, metas personales y laborales 
   (ver la figura \refElem{USR-IU02b}).
   \item \textbf{Historial académico}: esta sección tiene como objetivo que el actor actualice su información
   académica referente a todos los grados de estudios que tiene hasta la fecha (ver la figura \refElem{USR-IU02c}).
   \item \textbf{Idiomas}: esta sección tiene como objetivo preguntarle que el actor actualice la información de idiomas o en su caso, elimine
   o agregue nuevos idiomas a su perfil  (ver la figura \refElem{USR-IU02d}).
   \item \textbf{Experiencia laboral}: esta sección tiene como objetivo que el actor actualice su información de su experiencia
   laborar que tiene hasta la fecha (ver la figura \refElem{USR-IU02e}).
   \item \textbf{Cursos/Certificaciones}:  esta sección tiene como objetivo que el actor actualice su información de sus Certificaciones
   o cursos que ha tenido durante toda su trayectoria académica y laboral (ver la figura \refElem{USR-IU02f}).
\end{itemize}

\subsubsection{Comandos}
Los siguientes comandos aparecen durante toda la interfaz es decir, cada sección los tiene.

%\Titem \IUPass : Al da clic en el ícono, se muestra la contraseña de lo contrario aparecerá \IUOculto \thinspace sustituyendo cada caracter de la contraseña. \\

\Titem \IUEditar{} : Cuando presiona el ícono, habilita la sección para hacer los datos editables acorde a la sección o elemento seleccionado.
\Titem \IUEliminar{} : Cuando presiona el ícono, habilita la sección para eliminar el elemento seleccionado.
\Titem \IUAgregar{} : Cuando presiona el ícono, habilita la sección para agregar un nuevo el elemento.

\IUfig{.9}{CasosdeUso/USR-CU02/imagenes/USR-IU02.png}{USR-IU02}{Consultar perfil}  
\IUfig{.5}{CasosdeUso/USR-CU02/imagenes/USR-IU02a.png}{USR-IU02a}{Consultar perfil: Datos personales}
\IUfig{.9}{CasosdeUso/USR-CU02/imagenes/USR-IU02b.png}{USR-IU02b}{Consultar perfil: Objetivos y metas personales}  
\IUfig{.9}{CasosdeUso/USR-CU02/imagenes/USR-IU02c.png}{USR-IU02c}{Consultar perfil: Historial académico}  
\IUfig{.9}{CasosdeUso/USR-CU02/imagenes/USR-IU02d.png}{USR-IU02d}{Consultar perfil: Idiomas}
\IUfig{.9}{CasosdeUso/USR-CU02/imagenes/USR-IU02e.png}{USR-IU02e}{Consultar perfil: Experiencia laboral}  
\IUfig{.9}{CasosdeUso/USR-CU02/imagenes/USR-IU02f.png}{USR-IU02f}{Consultar perfil: Cursos/Certificaciones}  


\clearpage


	\clearpage
\begin{UseCase}[]{VCT-CU04}{Registrar vacante}{
	Permite al reclutador de una empresa  publicar una vacante en el sistema durante cierto periodo de tiempo y así poder gestionas las postulaciones 
	que los usarios(canditos) hagan a dicha vacante.
	}
	%----------------------------------------------------------------
	% Datos generales del CU:
	\UCsection{Atributos}
	\UCitem{Actor(es)}{
		Reclutadores.

	}
	\UCitem[admin]{Prioridad}{
		Media
	}
	\UCitem[admin]{Complejidad}{
		Alta
	}
	\UCitem{Precondiciones}{
		El reclutador debe de estar registrado en el sistema.
	}
	\UCitem{Destino}{
		\Titem \refElem{VCT-IU03}
	}
	\UCitem{Reglas de Negocio}{
		\Titem \refIdElem{RN-N001}
		
	}
	\UCitem{Viene de}{
		\refElem{VCT-CU03}
	}	
\end{UseCase}

%Trayectoria Principal
\begin{UCtrayectoria}
	\UCpaso [\UCactor] Da clic en el icono \IUAgregar{} (Publicar vacante) en la interfaz \refElem{VCT-IU03}.
	\UCpaso [\UCsist] Muestra la interfaz \refElem{VCT-IU04a} en la interfaz \refElem{VCT-IU03}.
	\UCpaso [\UCactor] \label{VCT-CU04:vadata} Ingresa el título y el número de plazas de la vacante, ingresa el código postal, estado, municipio y colonia donde se va laboral.\refTray{A}
	\UCpaso [\UCactor] Ingresa el perfil, la experiencia y el tipo de contratación a la que la vacante va dirigida.
	\UCpaso [\UCactor] Ingresa el horario laboral y el rango salarial indicando si es neto o no el salario.
	\UCpaso [\UCsist] Valida que todos los campos marcados como obligatorios hayan sido ingresados de acuerdo a la regla de negocio \refIdElem{RN-N001}. \refTray{B}
	\UCpaso [\UCactor] Ingresa la descripción de la vacante.
	\UCpaso [\UCactor] \label{VCT-CU04:hab}Selecciona las habilidades y su correspondiente experiencia deseadas para la vacante.\refTray{D}
	\UCpaso [\UCactor] Selecciona la fecha de cierre de la vacante.
	\UCpaso [\UCactor] Da clic en el botón \IUbutton{Publicar} en la interfaz \refElem{VCT-IU04b}.\refTray{A}\refTray{C} \refTray{F}
	\UCpaso [\UCsist] \label{VCT-CU04:vadata2}Valida que todos los campos marcados como obligatorios hayan sido ingresados de acuerdo a la regla de negocio \refIdElem{RN-N001}.\refTray{E}
	\UCpaso [\UCsist] Muestra la interfaz \refElem{VCT-IU03} mostrando la nueva vacante registrada.
	\UCpaso [\UCsist] Notifica a los encargados y colaboradores de que hay una nueva vacante por revisar.
\end{UCtrayectoria}

%Trayectorias Alternativas
\begin{UCtrayectoriaA}[Fin de la trayectoria]{A}{El actor decide cancelar el registro.}
	\UCpaso [\UCactor] Da clic en el botón \IUbutton{Cancelar} en la interfaz \refElem{VCT-IU04a}.
	\UCpaso [\UCsist] Muestra el mensaje \refIdElem{MSG6} en la interfaz \refElem{VCT-IU04a}.
	\UCpaso [\UCactor] Da clic en el botón \IUbutton{Sí} en la interfaz \refElem{VCT-IU04a}.\refTray{G}
	\UCpaso [\UCsist] Muestra la interfaz \refElem{VCT-IU03}.
\end{UCtrayectoriaA} 

%Trayectorias Alternativas
\begin{UCtrayectoriaA}[Fin de la trayectoria]{B}{El actor no registro al menos un campo obligatorio.}
	\UCpaso [\UCsist] Muestra el mensaje \refIdElem{MSG4} en la interfaz \refElem{VCT-IU03a} en los campos que no
	fueron ingresados.
	\UCpaso [\UCsist] Continúa en el paso \ref{VCT-CU04:vadata} de la trayectoria principal.
\end{UCtrayectoriaA} 

%Trayectorias Alternativas
\begin{UCtrayectoriaA}[Fin de la trayectoria]{C}{El actor decide regresa a la pantalla anterior.}
	\UCpaso [\UCactor] Da clic en el botón \IUbutton{Regresar} en la interfaz \refElem{VCT-IU04b}.
	\UCpaso [\UCsist] Muestra la interfaz \refElem{VCT-IU04a}.
	\UCpaso [\UCsist] Continúa en el paso \ref{VCT-CU04:vadata} de la trayectoria principal.
\end{UCtrayectoriaA} 

%Trayectorias Alternativas
\begin{UCtrayectoriaA}[Fin de la trayectoria]{D}{El actor decide eliminar una habilidad.}
	\UCpaso [\UCactor] Da clic en el botón \IUbutton{x} de la habilidad seleccionada en la interfaz \refElem{VCT-IU04b}.
	\UCpaso [\UCsist] Muestra la interfaz \refElem{VCT-IU04b}.
	\UCpaso [\UCsist] Continúa en el paso \ref{VCT-CU04:hab} de la trayectoria principal.
\end{UCtrayectoriaA} 

%Trayectorias Alternativas
\begin{UCtrayectoriaA}[Fin de la trayectoria]{E}{El actor no registro al menos un campo obligatorio.}
	\UCpaso [\UCsist] Muestra el mensaje \refIdElem{MSG4} en la interfaz \refElem{VCT-IU03b} en los campos que no
	fueron ingresados.
	\UCpaso [\UCsist] Continúa en el paso \ref{VCT-CU04:vadata2} de la trayectoria principal.
\end{UCtrayectoriaA} 

%Trayectorias Alternativas
\begin{UCtrayectoriaA}[Fin de la trayectoria]{F}{El actor decide cancelar el registro.}
	\UCpaso [\UCactor] Da clic en el botón \IUbutton{Cancelar} en la interfaz \refElem{VCT-IU04b}.
	\UCpaso [\UCsist] Muestra el mensaje \refIdElem{MSG6} en la interfaz \refElem{VCT-IU04b}.
	\UCpaso [\UCactor] Da clic en el botón \IUbutton{Sí} en la interfaz \refElem{VCT-IU04b}.\refTray{H}
	\UCpaso [\UCsist] Muestra la interfaz \refElem{VCT-IU03}.
\end{UCtrayectoriaA} 

%Trayectorias Alternativas
\begin{UCtrayectoriaA}[Fin de la trayectoria]{G}{El actor decide cancelar la acción.}
	\UCpaso [\UCactor] Da clic en el botón \IUbutton{No} en la interfaz \refElem{VCT-IU04a}.
	\UCpaso [\UCsist] Muestra la interfaz \refElem{VCT-IU04a}.
	\UCpaso [\UCsist] Continúa en el paso \ref{VCT-CU04:vadata} de la trayectoria principal.
\end{UCtrayectoriaA} 

%Trayectorias Alternativas
\begin{UCtrayectoriaA}[Fin de la trayectoria]{H}{El actor decide cancelar la acción.}
	\UCpaso [\UCactor] Da clic en el botón \IUbutton{No} en la interfaz \refElem{VCT-IU04b}.
	\UCpaso [\UCsist] Muestra la interfaz \refElem{VCT-IU04b}.
	\UCpaso [\UCsist] Continúa en el paso \ref{VCT-CU04:vadata2} de la trayectoria principal.
\end{UCtrayectoriaA} 
	\clearpage
\begin{UseCase}[]{VCT-CU04}{Registrar vacante}{
	Permite al reclutador de una empresa  publicar una vacante en el sistema durante cierto periodo de tiempo y así poder gestionas las postulaciones 
	que los usarios(canditos) hagan a dicha vacante.
	}
	%----------------------------------------------------------------
	% Datos generales del CU:
	\UCsection{Atributos}
	\UCitem{Actor(es)}{
		Reclutadores.

	}
	\UCitem[admin]{Prioridad}{
		Media
	}
	\UCitem[admin]{Complejidad}{
		Alta
	}
	\UCitem{Precondiciones}{
		El reclutador debe de estar registrado en el sistema.
	}
	\UCitem{Destino}{
		\Titem \refElem{VCT-IU03}
	}
	\UCitem{Reglas de Negocio}{
		\Titem \refIdElem{RN-N001}
		
	}
	\UCitem{Viene de}{
		\refElem{VCT-CU03}
	}	
\end{UseCase}

%Trayectoria Principal
\begin{UCtrayectoria}
	\UCpaso [\UCactor] Da clic en el icono \IUAgregar{} (Publicar vacante) en la interfaz \refElem{VCT-IU03}.
	\UCpaso [\UCsist] Muestra la interfaz \refElem{VCT-IU04a} en la interfaz \refElem{VCT-IU03}.
	\UCpaso [\UCactor] \label{VCT-CU04:vadata} Ingresa el título y el número de plazas de la vacante, ingresa el código postal, estado, municipio y colonia donde se va laboral.\refTray{A}
	\UCpaso [\UCactor] Ingresa el perfil, la experiencia y el tipo de contratación a la que la vacante va dirigida.
	\UCpaso [\UCactor] Ingresa el horario laboral y el rango salarial indicando si es neto o no el salario.
	\UCpaso [\UCsist] Valida que todos los campos marcados como obligatorios hayan sido ingresados de acuerdo a la regla de negocio \refIdElem{RN-N001}. \refTray{B}
	\UCpaso [\UCactor] Ingresa la descripción de la vacante.
	\UCpaso [\UCactor] \label{VCT-CU04:hab}Selecciona las habilidades y su correspondiente experiencia deseadas para la vacante.\refTray{D}
	\UCpaso [\UCactor] Selecciona la fecha de cierre de la vacante.
	\UCpaso [\UCactor] Da clic en el botón \IUbutton{Publicar} en la interfaz \refElem{VCT-IU04b}.\refTray{A}\refTray{C} \refTray{F}
	\UCpaso [\UCsist] \label{VCT-CU04:vadata2}Valida que todos los campos marcados como obligatorios hayan sido ingresados de acuerdo a la regla de negocio \refIdElem{RN-N001}.\refTray{E}
	\UCpaso [\UCsist] Muestra la interfaz \refElem{VCT-IU03} mostrando la nueva vacante registrada.
	\UCpaso [\UCsist] Notifica a los encargados y colaboradores de que hay una nueva vacante por revisar.
\end{UCtrayectoria}

%Trayectorias Alternativas
\begin{UCtrayectoriaA}[Fin de la trayectoria]{A}{El actor decide cancelar el registro.}
	\UCpaso [\UCactor] Da clic en el botón \IUbutton{Cancelar} en la interfaz \refElem{VCT-IU04a}.
	\UCpaso [\UCsist] Muestra el mensaje \refIdElem{MSG6} en la interfaz \refElem{VCT-IU04a}.
	\UCpaso [\UCactor] Da clic en el botón \IUbutton{Sí} en la interfaz \refElem{VCT-IU04a}.\refTray{G}
	\UCpaso [\UCsist] Muestra la interfaz \refElem{VCT-IU03}.
\end{UCtrayectoriaA} 

%Trayectorias Alternativas
\begin{UCtrayectoriaA}[Fin de la trayectoria]{B}{El actor no registro al menos un campo obligatorio.}
	\UCpaso [\UCsist] Muestra el mensaje \refIdElem{MSG4} en la interfaz \refElem{VCT-IU03a} en los campos que no
	fueron ingresados.
	\UCpaso [\UCsist] Continúa en el paso \ref{VCT-CU04:vadata} de la trayectoria principal.
\end{UCtrayectoriaA} 

%Trayectorias Alternativas
\begin{UCtrayectoriaA}[Fin de la trayectoria]{C}{El actor decide regresa a la pantalla anterior.}
	\UCpaso [\UCactor] Da clic en el botón \IUbutton{Regresar} en la interfaz \refElem{VCT-IU04b}.
	\UCpaso [\UCsist] Muestra la interfaz \refElem{VCT-IU04a}.
	\UCpaso [\UCsist] Continúa en el paso \ref{VCT-CU04:vadata} de la trayectoria principal.
\end{UCtrayectoriaA} 

%Trayectorias Alternativas
\begin{UCtrayectoriaA}[Fin de la trayectoria]{D}{El actor decide eliminar una habilidad.}
	\UCpaso [\UCactor] Da clic en el botón \IUbutton{x} de la habilidad seleccionada en la interfaz \refElem{VCT-IU04b}.
	\UCpaso [\UCsist] Muestra la interfaz \refElem{VCT-IU04b}.
	\UCpaso [\UCsist] Continúa en el paso \ref{VCT-CU04:hab} de la trayectoria principal.
\end{UCtrayectoriaA} 

%Trayectorias Alternativas
\begin{UCtrayectoriaA}[Fin de la trayectoria]{E}{El actor no registro al menos un campo obligatorio.}
	\UCpaso [\UCsist] Muestra el mensaje \refIdElem{MSG4} en la interfaz \refElem{VCT-IU03b} en los campos que no
	fueron ingresados.
	\UCpaso [\UCsist] Continúa en el paso \ref{VCT-CU04:vadata2} de la trayectoria principal.
\end{UCtrayectoriaA} 

%Trayectorias Alternativas
\begin{UCtrayectoriaA}[Fin de la trayectoria]{F}{El actor decide cancelar el registro.}
	\UCpaso [\UCactor] Da clic en el botón \IUbutton{Cancelar} en la interfaz \refElem{VCT-IU04b}.
	\UCpaso [\UCsist] Muestra el mensaje \refIdElem{MSG6} en la interfaz \refElem{VCT-IU04b}.
	\UCpaso [\UCactor] Da clic en el botón \IUbutton{Sí} en la interfaz \refElem{VCT-IU04b}.\refTray{H}
	\UCpaso [\UCsist] Muestra la interfaz \refElem{VCT-IU03}.
\end{UCtrayectoriaA} 

%Trayectorias Alternativas
\begin{UCtrayectoriaA}[Fin de la trayectoria]{G}{El actor decide cancelar la acción.}
	\UCpaso [\UCactor] Da clic en el botón \IUbutton{No} en la interfaz \refElem{VCT-IU04a}.
	\UCpaso [\UCsist] Muestra la interfaz \refElem{VCT-IU04a}.
	\UCpaso [\UCsist] Continúa en el paso \ref{VCT-CU04:vadata} de la trayectoria principal.
\end{UCtrayectoriaA} 

%Trayectorias Alternativas
\begin{UCtrayectoriaA}[Fin de la trayectoria]{H}{El actor decide cancelar la acción.}
	\UCpaso [\UCactor] Da clic en el botón \IUbutton{No} en la interfaz \refElem{VCT-IU04b}.
	\UCpaso [\UCsist] Muestra la interfaz \refElem{VCT-IU04b}.
	\UCpaso [\UCsist] Continúa en el paso \ref{VCT-CU04:vadata2} de la trayectoria principal.
\end{UCtrayectoriaA} 