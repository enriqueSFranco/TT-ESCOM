\chapter{Acotres del Sistema}
En esta sección detallamos se definen los actores identificados como participantes en los procesos del sistema 
\textit{Bolsa de Trabajo ESCOM}. Estos actores son los encargados de llevar a cabo determinadas tareas dentro de cada uno de los 
procesos y por ende les corresponde uno o más casos de uso de los descritos en este documento.


\begin{description}
	\item[Nombre:] Nombre con el que se le conoce al actor dentro del sistem.
	\item[Descripción:] Descripción breve del puesto o rol que juega dentro del sistema
	\item[Responsabilidades:] Se listan las responsabilidades oficiales o relacionadas con el sistema según aplique. En el caso de las descripciones tomadas de la normatividad se listan todas.
\end{description}

\begin{actor}{aReclutador}{Reclutador}{%
   Es el representante del area de recursos humanos de una empresa,es el primer punto de contacto entre su empresa y los candidatos registrados en el sistema.}
    \item[Responsabilidades:] \hfill
    \begin{itemize}
        \item  Revisar y analizar currículums de las postulaciones de sus vacantes.
        \item  Buscar candidatos potenciales y ponerse en contacto con ellos personalmente para dar seguimiento a la solucitud.
        \item  Redactar de la mejor manera posible las vacantes que el requiera publicar, para evitar malas interpretaciones 
        en su redacción y descripciones ambiguas.
        \item  Atender las correciones enviadas por los encargados y colaboradores del sistema a sus vacante que desean ser publicadas.
        \item Dar solución a los reportes de sus vacantes hechos por los otros usuarios del sistema.
    \end{itemize}
\end{actor}

\begin{actor}{aCandidato}{Candidato}{%
    Es el usuario que desea aplicar a la(s) vacante(s) de su elección registrada(s) por un reclutador dentro del sistema.
    No necesariamente tiene que ser un alumno del IPN, puede ser cualqueir persona que se registre dentro del sistema como candidato.}
     \item[Responsabilidades:] \hfill
     \begin{itemize}
         \item  Actualizar su perfil o almenos tener su currículum actualizado y cargado en el sistema.
         \item  Reportar alguna vacante siempre y cuando lo concidere necesario.
     \end{itemize}
 \end{actor}

 \begin{actor}{aColaborador}{Colaborador}{%
    Es el encargado de administrar el sistema junto con el Encargado.}
    \item[Responsabilidades:] \hfill
    \begin{itemize}
        \item Validar la información de las solicitudes de empresas y reclutadores que envian.
        \item Eliminar y crear cuentas de reclutadores o candidatos si es que lo requiere.
        \item Generar reportes de estado del sistema.
        \item Consultar y revisar las vacantes y sus reportes si es que asi lo requiere, además de cerrar vacantes si lo cree necesario.
        \item Si es necesario, podrá enviar observaciones a los reclutadores sobre anormalidades o errores que se presenten en la redacción de vacantes.
    \end{itemize}

\end{actor}

\begin{actor}{aEncargado}{Encargado}{%
    Dentro del sistema solo puede existir un único encargado, este usuario tendrá el control sobre todos los demás usuarios y cuentas.
   }   
   \item[Responsabilidades:] \hfill
    \begin{itemize}
        \item Eliminar y crear cuentas de colaboradores.
        \item Validar la información de las solicitudes de empresas y reclutadores que envian.
        \item Eliminar y crear cuentas de reclutadores o candidatos si es que lo requiere.
        \item Generar reportes de estado del sistema.
        \item Consultar y revisar las vacantes y sus reportes si es que asi lo requiere, además de cerrar vacantes si lo cree necesario.
        \item Si es necesario, podrá enviar observaciones a los reclutadores sobre anormalidades o errores que se presenten en la redacción de vacantes.
    \end{itemize}

\end{actor}