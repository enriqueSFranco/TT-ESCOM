\chapter{Modelo de estados}
A continucación describiremos de manera detallada el modelo de estados correspondiente a las entidades de vacantes, postulaciones, reclutadores
y empresas.
\begin{Maquina}{mq:vacante}{Máquina de estados de una vacante}{
	En todo momento una vacante tiene un ``estado'' en el sistema. Las acciones quelos actores pueden realizar sobre dicha vacante 
	dependen del estado y pueden tener como consecuencia la transición a otro estado.
	Los estados y transiciones posibles se muestran en la figura  y se describen a continuación.}{anexos/imagenes/mvacante.jpeg}

	La tabla \ref{figvacante} muestra el comportamiento que se tendrá en los íconos de la pantalla 
	\refElem{VCT-IU08}, los cuales son habilitados dependiendo del estado en el que se encuentre  la vacante.


	\begin{table}[htbp]
		\begin{center}
			\begin{tabular}{|c|c|c|c|}
				\hline
				Estado &\IUbutton{Cerrar vacante}& \IUEliminar & \IUEditar \\
				\hline \hline
				Abierta  & Habilitado & Habilitado & Habilitado \\ \hline
				Cerrada & No aparece & Habilitado & Habilitado \\ \hline
				Ocupada & Habilitado & Habilitado & No aparece \\ \hline
				Reportada & No aparece & Habilitado & Habilitado \\ \hline
				Cerrada por reporte & No aparece & Habilitado & Habilitado\\ \hline
			\end{tabular}
			\caption{Comportamiento de acciones para gestionar vacantes.} 
			\label{figvacante}

		\end{center}
	\end{table}
\end{Maquina}

\begin{Maquina}{mq:post}{Máquina de estados de una postulación}{
	En todo momento una postulación tiene un ``estado'' en el sistema. Las acciones que los actores pueden realizar sobre dicha 
	postulación dependen del estado y pueden tener como consecuencia la transición a otro estado.
	Los estados y transiciones posibles se muestran en la figura  y se describen a continuación.}{anexos/imagenes/mpost.png}

	La tabla \ref{figpost} muestra el comportamiento que se tendrá en los íconos de la pantalla 
	\refElem{VCT-IU08}, los cuales son habilitados dependiendo del estado en el que se encuentre  la vacante.


	\begin{table}[htbp]
		\begin{center}
			\begin{tabular}{|c|c|c|c|c|}
				\hline
				Estado &\IUbutton{Seguimiento}& \IUbutton{Descartar} & \IUbutton{Contratado} & \IUbutton{No Contratado}\\
				\hline \hline
				Por evaluar  & Habilitado & Habilitado & No aparece & No aparece\\ \hline
				Descartada & No aparece & No aparece & No aparece & No aparece\\ \hline
				En seguimiento & No aparece & No aparece & Habilitado & Habilitado\\ \hline
				Contratado & No aparece & No aparece & No aparece & No aparece\\ \hline
				No contratado & No aparece & No aparece & No aparece & No aparece\\ \hline
			\end{tabular}
			\caption{Comportamiento de acciones para gestionar postulaciones.}
			\label{figpost}
		\end{center}
	\end{table}
\end{Maquina}


\begin{Maquina}{mq:empresa}{Maquina de estados de una empresa}{
	Cuando un reclutador quiere publicar vacantes dentro del sistema, el encargado o colaborador que lo gestiona debe de validar
	que la empresa a la que el reclutador representa es una empresa constituida, por lo que la entidad empresa tiene un estado 
	desde que el reclutador envia un preregistro.}{anexos/imagenes/mqEMpresa.jpeg}
\end{Maquina}

\begin{Maquina}{mq:reclutador}{Maquina de estados de un reclutador}{
	Una vez que la empresa haya sido validada, el encargado debe de validar la información del reclutador, por lo que la cuenta de 
	tipo reclutador tiene siempre un estado dentro del sistema.}{anexos/imagenes/mqRrec.jpeg}
\end{Maquina}