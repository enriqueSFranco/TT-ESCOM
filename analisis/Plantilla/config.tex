

\setImgPortada[0.9]{LIT/LIT-banner}
\setImgHeader[40pt]{Plantilla/headerPar}{Plantilla/headerImp}
%\setDireccion{Flores Magon \#82 B-402, Edificion Lopez Mateos, Nonoalco Tlatelolco, Delegación Cuauhtemoc, cp: 06900, CDMX}
%\setTelefono{Correo del TT }
%\setCorreo{email del TT }

%\AtBeginDocument{%
%    \ThisURCornerWallPaper{0.3}{LIT/LIT}%
%}


%===== Comandos específicos del proyecto GRP =====

%% Los siguientes comandos se utilizan para indicar el tipo de mensaje que se muestra en la pantalla.
%% Esto crea una referencia al diseño de los mensajes.
%%
%% Ejemplo: \begin{mensaje}{MSG2}{Confirmar registro}{\MSGConfirmacion}...
%%
%% Salida: MSG2 Confirmar registro Tipo: Confirmación

\newcommand{\MSGConfirmacion}{%
	\hyperref[fig:interaccion:msg-confirmacion]{Mensaje emergente de confirmación}%
}

\newcommand{\MSGExito}{%
	\hyperref[fig:interaccion:msg-toast-exito]{Mensaje emergente de éxito}%
}

\newcommand{\MSGInfo}{%
	\hyperref[fig:interaccion:msg-toast-info]{Mensaje emergente de información}%
}

\newcommand{\MSGAlerta}{%
	\hyperref[fig:interaccion:msg-toast-alerta]{Mensaje emergente de alerta}%
}

\newcommand{\MSGError}{%
	\hyperref[fig:interaccion:msg-toast-error]{Mensaje emergente de error}%
}

\newcommand{\MSGCampoError}{%
	\hyperref[fig:interaccion:msg-campo-error]{Mensaje de error en campo}%
}

\newcommand{\MSGTooltip}{%
	\hyperref[fig:interaccion:msg-tooltip]{Mensaje emergente de ayuda}%
}

\newcommand{\MSGTexto}{%
	\hyperref[fig:interaccion:msg-texto]{Mensaje de texto en pantalla}%
}

\newcommand{\MSGNotif}{%
	\hyperref[fig:interaccion:msg-notif]{Mensaje de notificación}%
}

%% El comando \UCerrCamposReq se utiliza para mostrar un error común en la tabla del CU
%% indicando el motivo del error, la reacción del sistema, la pantalla en la que se muestra y el paso en el que contúa
%% de la trayectoria principal.
%%
%% Recibe 4 parámetros:
%%		1. Número de error: Uno, Dos, Tres, Cuatro...
%%		2. Mensaje que se muestra dependiendo del error: MSG1, MSG2...
%%		3. Pantalla en la que se muestra el mensaje: CH-EM-IU1...
%%		4. Referencia al paso de la trayectoria principal en el que continúa después de mostrar el mensaje: \ref{paso}...
%%
%% Ejemplo: \UCerrCamposReq{Uno}{MSG1}{CH-EM-IU1}{\ref{paso}}
%%
%% Salida: Uno: Cuando el actor no ingresó todos los los campos requeridos. Muestra el mensaje MSG1 debajo de los campos correcpondientes en la pantalla CH-EM-IU1 
%%				y continúa en el paso 3 de la trayectoria principal.

%Error de campos requeridos omitidos
\newcommand{\UCerrCamposReq}[4]{%
	\UCerr{#1}%
	{Cuando el actor no ingresó todos los campos requeridos.}%
	{el sistema muestra el mensaje \refElem{#2} debajo de los campos correspondientes en la pantalla \refElem{#3} y continúa en el paso #4 de la trayectoria principal.}%
}

%Error de tipo de dato inválido
\newcommand{\UCerrTipoDato}[4]{%
	\UCerr{#1}%
	{Cuando el actor ingresó algún dato que no coincide con el tipo de dato especificado en el modelo de información.}%
	{el sistema muestra el mensaje \refElem{#2} debajo de los campos correspondientes en la pantalla \refElem{#3} y continúa en el paso #4 de la trayectoria principal.}%
}

%Error de longitud
\newcommand{\UCerrLongitud}[4]{%
	\UCerr{#1}%
	{Cuando el actor ingresó algún dato que no coincide con la longitud especificada en el modelo de información.}%
	{el sistema muestra el mensaje \refElem{#2} debajo de los campos correspondientes en la pantalla \refElem{#3} y continúa en el paso #4 de la trayectoria principal.}%
}

%Error de formato
\newcommand{\UCerrFormato}[4]{%
	\UCerr{#1}%
	{Cuando el actor ingresó algún dato que no coincide con el formato especificado en el modelo de información.}%
	{el sistema muestra el mensaje \refElem{#2} debajo de los campos correspondientes en la pantalla \refElem{#3} y continúa en el paso #4 de la trayectoria principal.}%
}

%------------------------------------------------

