% Actualizar para indicar cual es el archivo raíz a compilar.
%!TEX root = ../../main2.tex

\begin{UseCase}[]{GRAL-CU1}{Iniciar sesión.}{
	Permite al actor ingresar al sistema por medio de una cuenta y así poder acceder a las
	acciones que tiene permitidas dentro del sistema según los permisos del actor.
}
	%----------------------------------------------------------------
	% Datos generales del CU:
	\UCsection{Atributos}
	\UCitem{Actor(es)}{
		Todos. 
	}
	\UCitem[admin]{Prioridad}{
		Alta
	}
	\UCitem[admin]{Complejidad}{
		Media.
	}
	\UCitem{Precondiciones}{
		El actor debe de ingresar al sitio web de la Bolsa de Trabajo ESCOM.
	}
	\UCitem{Postcondiciones}{
		El actor podrá acceder al sistema y dependeiendo el tipo de usuario se habilitarán las acciones correspondientes.
	}
	\UCitems{Entradas}{
		\begin{Titemize}
			\Titem Nombre de usuario
			\Titem Contraseña
		\end{Titemize}

	}
	\UCitems{Salidas}{	
		\begin{Titemize}
			\Titem Nombre de usuario
			\Titem Contraseña
		\end{Titemize}	
	}   
	\UCitem{Destino}{
		GRAL-USR-IU1.
	}
	\UCitem{Viene de}{
		Caso de uso primario.
	}	
\end{UseCase}

%Trayectoria Principal
\begin{UCtrayectoria}
	\UCpaso [\UCactor] Ingresa al sitio web de la \textbf{Bolsa de trabajo de ESCOM} desde el navegador de su preferencia.
    \UCpaso [\UCsist] Muestra la interfaz \refIdElem{GRAL-IU1}.
	\UCpaso [\UCactor] Da clic en el botón \IUbutton{Alumno}. \refTray{A}
	\UCpaso [\UCsist] Muestra la interfaz \refIdElem{GRAL-USR-IU1}.
	\UCpaso [\UCactor] Introduce los datos para ingresar al sistema: usuario y contraseña.\refTray{B}\label{gral-01-1}
	\UCpaso [\UCactor] Da clic en el botón \IUbutton{Iniciar Sesión}. \refTray{C}\refTray{D}
    \UCpaso [\UCsist] Valida que el nombre de usuario y la contraseña ingresados esten registrados en el sistema.\refTray{E}
    \UCpaso [\UCsist] Muestra la interfaz \refIdElem{AA-IU1}.
\end{UCtrayectoria}

%Trayectorias Alternativas
\begin{UCtrayectoriaA}[Fin del caso de uso]{A}{El actor da clic en el botón \IUbutton{Empresa}.}
	\UCpaso [\UCsist] Muestra la interfaz \refIdElem{GRAL-USR-IU1}.
	\UCpaso [\UCactor] Introduce los datos para ingresar al sistema: usuario y contraseña.\refTray{B}\label{gral-01-2}
	\UCpaso [\UCactor] Da clic en el botón \IUbutton{Iniciar Sesión}. \refTray{F}\refTray{D}
    \UCpaso [\UCsist] Valida que el nombre de usuario y la contraseña ingresados esten registrados en el sistema. \refTray{E}
    \UCpaso [\UCsist] Muestra la interfaz \refIdElem{EMP-IU1}.
\end{UCtrayectoriaA} 

\begin{UCtrayectoriaA}[Fin de la Trayectoria]{B}{El actor no ingresó uno o más campos obligatorios.}
	\UCpaso [\UCsist] Muestra el mensaje \textbf{MSG} en la interfaz \refIdElem{GRAL-USR-IU1} .
	\UCpaso [\UCsist] Continua en el paso \ref{gral-01-1} de la Trayectoria Principal o en el paso \ref{gral-01-2} de la Trayectoria Alternativa A.
\end{UCtrayectoriaA} 

\begin{UCtrayectoriaA}[Fin de la Trayectoria]{C}{El actor desea registrarse en el sistema.}
	\UCpaso [\UCsist] Presiona el botón \IUbutton{¿No tienes una cuenta?}.
	%\extendUC{GRAL-CU2}
\end{UCtrayectoriaA} 

\begin{UCtrayectoriaA}[Fin de la Trayectoria]{D}{El actor ha olvidado su contraseña.}
	\UCpaso [\UCsist] Presiona el botón \IUbutton{¿Olvidaste tu contraseña?}.
	%\extendUC{GRAL-CU3}
\end{UCtrayectoriaA} 

\begin{UCtrayectoriaA}[Fin de la Trayectoria]{E}{El actor no ingresó un correo o contraseña válidos.}
	\UCpaso [\UCsist] Muestra el mensaje \textbf{MSG} en la interfaz \refIdElem{GRAL-USR-IU1} .
	\UCpaso [\UCsist] Continua en el paso \ref{gral-01-1} de la Trayectoria Principal o en el paso \ref{gral-01-2} de la Trayectoria Alternativa A.
\end{UCtrayectoriaA} 

\begin{UCtrayectoriaA}[Fin de la Trayectoria]{F}{El actor desea hacer el pre-registro en el sistema.}
	\UCpaso [\UCsist] Presiona el botón \IUbutton{¿No tienes una cuenta?}.
	%\extendUC{GRAL-CU4}
\end{UCtrayectoriaA} 




\subsubsection{Puntos de extensión}

%\UCExtensionPoint{Registrar usuario}{El usuario requiere registrarse en el sistema}{En la trayectoria alternativa B}{\refIdElem{AUCU2}}

%\UCExtensionPoint{Recuperar contraseña}{El usuario requiere recuperar su contraseña para ingresar al sistema}{En la trayectoria alternativa C}{\refIdElem{AUCU3}}