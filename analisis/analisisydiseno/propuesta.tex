\section{Objetivo}
\subsection{Objetivo general}
\begin{itemize}
    \item Implementar un sistema web para la Bolsa de trabajo ESCOM, mejorando la búsqueda de vacantes y pueda 
    ofrecer apoyo al proceso de reclutamiento para las empresas con el fin de resolver los problemas de gestión 
    identificados.
\end{itemize}

\subsection{Objetivos específicos}
        \begin{itemize}
            \item Desarrollar una plataforma web para la bolsa de trabajo ESCOM, independiente de la página de Facebook.
            \item Diseñar un procedimiento para registro de empresas al sistema con base al sistema manejado a la actual
            implementación de la bolsa de trabajo ESCOM.
            \item Facilitar la publicación de vacantes para los reclutadores, implementando la capacidad de realizar la
            publicación directamente por ellos.
            \item Implementar un algoritmo de clasificación que automatice la búsqueda de los mejores postulantes, analizando
            su curriculum acorde a sus intereses y los intereses de la empresa.
        \end{itemize}

%\section{Flujo del sistema}

%        En esta sección vamos a describir de forma general el flujo del sistema y los usuarios que tendran interacción con el.
 %       Para dar una mejor claridar de comprensión al proyecto se listas los usuarios que van a interacturar con el 
 %       sistema:
 %       \begin{itemize}
  %          \item Encargado de la bolsa de trabajo ESCOM: solo podra existir un solo encargado en el sistema,
  %          el trandrá todos los permisos necesarios para gestionar el sistema adecuadamente, sera el unico
  %          que puede dar de alta a otros colaboradores.
        
   %         \item Colaborador: son aquellas personas que ayudan al encargado de la bolsa de trabajo, lo unico que no 
   %         podran hacer es dar de alta a otros colaboradores.
   %     
    %        \item Reclutador: son todaslas personas de recursos humanos que trabajan para empresas constituidas y 
   %         que quieren publicar sus vacantes.
    %    
    %        \item Alumno(probalemente cambie el nombre): son aquellos que buscan un empleo dentro del sistema y puden ser 
    %        o no alumnos de ESCOM.
     %   \end{itemize}
        
        
        

