\clearpage
\chapter{Mensajes del sistema}

En esta sección se describen los mensajes utilizados en el prototipo actual del sistema.
Los mensajes se refieren a todos aquellos avisos que el sistema muestra al actor para comunicar la ocurrencia de algún evento 
tal como un error o una operación exitosa. Estos mensajes se pueden mostrar a través de diversos canales, por ejemplo, pantalla 
o correo electrónico.

Cuando un mensaje es recurrente se parametrizan sus elementos, por ejemplo los mensajes: ``Aún no existen registros de tus
\textit{vacantes} en el sistema.'', ``Aún no existen registros de \textit{candidatos} en el sistema.'', 
tienen una estructura similar por lo que, con el objetivo de que el mensaje sea genérico y pueda utilizarse en todos los 
casos de uso que se considere necesario, se utilizan parámetros para definir el mensaje.\\

Los parámetros también se utilizan cuando la redacción del mensaje tiene datos que son ingresados por el actor o que dependen del 
resultado de la operación, por ejemplo: 
``La \textit{Unidad Académica ESCOM} ha sido \textit{modificada} exitosamente.''. 
En este caso la redacción se presenta parametrizada de la forma: \newline
``$<DETERMINADO> <ENTIDAD> <VALOR>$ ha sido $<OPERACION>$ exitosamente.'' \newline
Y los parámetros se describen de la siguiente forma:

\begin{itemize}
	\item DETERMINADO ENTIDAD: Es un artículo determinado más el nombre de la entidad sobre la cual se realizó la acción.
	\item VALOR: Es el valor asignado al atributo de la entidad, generalmente es el nombre o la clave.
	\item OPERACIÓN: Es la acción que el actor solicitó realizar.
\end{itemize}

En el ejemplo anterior se hace referencia a $<VALOR>$, es decir: \textit{ESCOM} es el \textbf{valor} de la \textbf{entidad} 
\textit{escuela}. Cada mensaje enlista los parámetros  que utiliza, sin embargo aquí se definen los más comunes a fin de 
simplificar la descripción de los mensajes:

\begin{description}
	\item [$<ARTICULO>$:] Se refiere a un {\em artículo} el cual puede ser DETERMINADO (El $\mid$ La $\mid$ Lo $\mid$ Los $\mid$ Las) o INDETERMINADO (Un $\mid$ Una $\mid$ 
	Uno $\mid$ Unos $\mid$Unas) se aplica generalmente sobre una ENTIDAD, ATRIBUTO o VALOR.
	
	\item [$<CAMPO>$:] Se refiere a un campo del formulario. Por lo regular es el nombre de un atributo en una entidad.

	\item [$<CONDICION>$:] Define una expresión booleana cuyo resultado deriva en {\em falso} o {\em verdadero} y suele ser la causa del mensaje.
	
	\item [$<DATO>$:] Es un sustantivo y generalmente se refiere a un atributo de una entidad descrito en el modelo estructural del negocio, por ejemplo: nombre de la unidad de aprendizaje, nombre del alumno, RFC del profesor, etc. %ATRIBUTO
	
	\item [$<ENTIDAD>$:] Es un sustantivo y generalmente se refiere a una entidad del modelo estructural del negocio, por ejemplo: programa académico, alumno, profesor, etc.
	\item [$<OPERACION>$:] Se refiere a una acción que se debe realizar sobre los datos de una o varias entidades. Por ejemplo: registrar, eliminar, actualizar, por mencionar algunos. Comúnmente la OPERACIÓN va concatenada con el sustantivo, por ejemplo: Registro de un nuevo beneficio, registro de una actividad, eliminar una tarea y demás.
	
	\item [$<VALOR>$:] Es un sustantivo concreto y generalmente se refiere a un valor en específico. Por ejemplo: \textit{Histología I} es un \textbf{valor} concreto de la \textbf{entidad} \textit{Unidad de Aprendizaje}.
	
\end{description}
%___________________________Plantilla_______________________________
%===========================  MSGX ==================================
%\begin{mensaje}{MSGX}{}{}
%	\item[Canal:] 
%	\item[Propósito:] 
%	\item[Redacción:]
%	\item[Parámetros:] 
%	\begin{itemize}
%		\item 
%	\end{itemize}
%	\item[Ejemplo:]  
%	\item[Referenciado por: ]
%\end{mensaje}


%===========================  MSG6 ==================================
\begin{mensaje}{MSG1}{Operación exitosa}{Mensaje de éxito en pantalla}
  
    %   \item[Redacción:] $<ARTICULO>$ $ENTIDAD$ se ha(n) $OPERACION$ exitosamente.
    %
    %    \item[Parámetros:] 
    %    \begin{itemize}
    %        \item $<ARTICULO>$: Se refiere a un {\em artículo} el cual puede ser DETERMINADO (El $\mid$
    %            La $\mid$ Lo $\mid$ Los $\mid$ Las) o INDETERMINADO (Un $\mid$ Una $\mid$ Uno $\mid$
    %            Unos $\mid$Unas).
    %        \item $<ENTIDAD>$: Se refiere a la entidad o atributo en el cual estamos realizando la operación. Por ejemplo: evento, supervisor, sustentante, etc.
    %        \item $<OPERACION>$: Se refiere a una acción que se debe realizar sobre los datos de una o varias entidades, el verbo puede tener una conjugación en participio. Por ejemplo: finalizado, registrado, eliminado, actualizado, por mencionar algunos.
    %    \end{itemize}   
    %
    %    \item[Ejemplo:] \textit{El} \textit{evento} se ha \textit{registrado} exitosamente.  
      
        \item[Redacción:] $ENTIDAD$ $OPERACION$.
    
        \item[Parámetros:] 
        \begin{itemize}
    %        \item $<ARTICULO>$: Se refiere a un {\em artículo} el cual puede ser DETERMINADO (El $\mid$
    %            La $\mid$ Lo $\mid$ Los $\mid$ Las) o INDETERMINADO (Un $\mid$ Una $\mid$ Uno $\mid$
    %            Unos $\mid$Unas).
            \item $<ENTIDAD>$: Se refiere a la entidad o atributo en el cual estamos realizando la operación. Por ejemplo: evento, supervisor, sustentante, etc.
            \item $<OPERACION>$: Se refiere a una acción que se debe realizar sobre los datos de una o varias entidades, el verbo puede tener una conjugación en participio. Por ejemplo: finalizado, registrado, eliminado, actualizado, por mencionar algunos.
        \end{itemize}   
    
        \item[Ejemplo:] \textit{Evento} \textit{registrado}.
    \end{mensaje}

 
%%====================  MSG2 =====================
\begin{mensaje}{MSG2}{Formato de campo inválido}{Mensaje de error}
	\item[Redacción:] Favor de ingresar un $<CAMPO>$  válido.
    \item[Parámetros:] 
    \begin{itemize}
        
        \item $<CAMPO>$: Se refiere al nombre del campo que se ha ingresado y no es válido. Por ejemplo: número de teléfono, correo electrónico, etc.
    \end{itemize}   

    \item[Ejemplo:]  \begin{itemize}
        \item  Favor de ingresar un número de teléfono  válido.
        \item  Favor de ingresar un correo electrónico válido.
    \end{itemize} 
\end{mensaje}


%===========================  MSG3 ==================================
\begin{mensaje}{MSG3}{Dato ingresado ya registrado}{Mensaje de error}
	\item[Redacción:] Ya existe un $<elemento>$ registrado con la información ingresada.
	
	%el $<DATO>$ $<ACCION>$	
	
	\item[Parámetros:] 
    \begin{itemize}
      %  \item $<DATO>$: Se refiere a la información que esta siendo ingresada nuevamente. Por ejemplo: CURP, correo electrónico, etc.
     %   \item $<ACCION>$: Se refiere a la operación que esta siendo ejecutada de nueva cuenta, entre las cuales se encuentran: registrado, ingresado, confirmado, etc. 
        \item $<ACTOR>$: Se refiere a la persona que está siendo afectado por la acción. Por ejemplo: sustentante, supervisor potencial, supervisor, enlace institucional, etc.
    \end{itemize}   

    \item[Ejemplo:]  
    \begin{itemize}
        \item  Ya existe un $correo electronico$ registrado con la información ingresada.
	
        \item  Ya existe un $correo electronico$ registrado con la información ingresada.
	
    \end{itemize} 
\end{mensaje}

   
%===========================  MSG1 ==================================
\begin{mensaje}{MSG4}{Dato obligatorio faltante}{Mensaje de error en campo}
	\item[Redacción:] Campo obligatorio, favor de $<OPERACION>$.
    \item[Parámetros:] $<OPERACION>$: Se refiere a una acción que se debe realizar sobre los campos marcados como obligatorios. Por ejemplo: ingresarlo, seleccionarlo.
    

    \item[Ejemplo:]\begin{itemize}
        
        \item Campo obligatorio, favor de ingresarlo.  
        \item Campo obligatorio, favor de seleccionarlo.
    \end{itemize} 
\end{mensaje}


%%====================  MSG5 =====================
\begin{mensaje}{MSG5}{Credenciales incorrectas}{Mensaje de error en pantalla}
	\item[Redacción:] Usuario y/o contraseña incorrectos.
\end{mensaje}

