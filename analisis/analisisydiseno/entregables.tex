En este caítulo se presentan los avances que se tienen del sistema en análisis y diseño por cada sprint ...

\section{Sprits del proyecto}
En casi todos sprints se cubren requerimientos funcionales del sistema, estos requerimientos utilizan una 
clave para ser identificados:

\begin{itemize}
	\item Para los requerimientos \textbf{fucionales} se utliza el formato \textbf{RF-X}, donde:
    \begin{itemize}
        \item \textbf{RF} Es la clave para todos los requerimientos funcionales.
        \item \textbf{X} Es un número consecutivo: 01, 02, 03, ...
    \end{itemize}

    \item Para los requerimientos \textbf{no fucionales} se utliza el formato \textbf{RNF-X}, donde:
    \begin{itemize}
        \item \textbf{RNF} Es la clave para todos los requerimientos funcionales.
        \item \textbf{X} Es un número consecutivo: 01, 02, 03, ...
    \end{itemize}
\end{itemize}

\subsection{Sprint 1: Pruebas de concepto}
Esta iteración tuvo como propósito capacitarnos en las tecnologías que se van ha estar utilizando 
a lo largo del desarrollo del sistema.La actividades que se realizaron las siguientes:

\begin{enumerate}
    \item Crear el product backlog así como elementos de apoyo para dar prioridad y seguimiento.
    \item Especificar el lugar físico en el que el código del proyecto se va almacenar y que permita que
    todos los miembros del equipo puedan darle mantenimiento.
    \item Configurar las diferentes plataformas y tecnologías necesarias para el proyecto.
\end{enumerate}

\subsection{Sprint 2: Análisis y planificación general del sistema}
Esta iteración tuvo como establecer los fundamentos teóricos y prácticos para su desarrollo.Las actividades que se realizaron 
de manera general fueron las siguientes:
\begin{enumerate}
    \item Se hizo el levantamiento de requerimientos a travez de juntas con el personal de la bolsa de trabajo de ESCOM.
    \item Se adquirieron las licencias de Balsamiq Cloud y Visual Paradigm  para todo el análisis de
    requerimientos y de diseño.
    \item Se creo un repositorio usando el Sistema de Control de Versiones Git para almacenar el código y la documentación del proyecto
    utilizando la prataforma de GitHub.
    \item Se configuró el proyecto en front end utilizando React Js con la versión 17.0.2.
\end{enumerate} 


\subsection{Sprint 3: Gestión de Usuarios}
Esta iteración tuvo como propósito implementar un mecanismo que permita a los usuarios del sistema autenticarse y/o crear cuentas
para acceder al mismo, las actividades realizadas fueron:
\begin{enumerate}
    \item Se diseñó e implementó las interfaces de usuario para la gestión de cuentas y accesos.
    \item Se especificaron los requerimientos mediante casos de uso con sus respectivas interfaces de usuario.
    \item Se elaboró el a de casos de uso para este sprint.
    \item Se elaboró la maquina de estados de un reclutador.
\end{enumerate} 

Los requerimientos funcionales de este sprint se muestran en la siguiente tabla.
\begin{requerimientos}{funcionales}
    \RFitem{RF-01}{Iniciar sesión }{El sistema debe proporcionar un mecanismo que permita al actor acceder al sistema mediante un correo y una contraseña.}
    \RFitem{RF-02}{Crear cuenta}{El sistema debe proporcionar un mecanismo que permita al actor registrarse como nuevo usuario en el sistema.}
    \RFitem{RF-04}{Consultar perfil}{El sistema debe permitir al actor acceder a su perfil para consultar la información registrada.}
    \RFitem{RF-05}{Editar perfil}{El sistema debe permitir al actor modificar la información registrada en su perfil, como nombre, apellidos, información de contacto y si el usuario es un candidato, debe de permitir modificar su información académica y datos de habilidades y conocimientos del mismo.}
    \RFitem{RF-24}{Enviar solicitud para acceder al sistema}{El sistema debe proporcionar un mecanismo para que al representante de una empresa pueda enviar una solicitud o pre-registro de poder acceder y publicar vacantes.}
\end{requerimientos}

Los casos de uso que se implementaron en este sprint fueron:
\begin{enumerate}
    \item \refElem{GRL-CU01}
    \item \refElem{GRL-CU02}
    \item \refElem{GRL-CU03}
    %\item \refElem{USR-CU02-2}
\end{enumerate} 


\subsection{Sprint 4: Gestión de Vacantes}

Esta iteración tuvo como propósito implementar un mecanismo para la gestión de vacantes dentro del sistema, las actividades 
realizadas fueron:
\begin{enumerate}
    \item Se diseñó e implementó las interfaces de usuario para la gestión de vacantes ú el tipo de cuenta.
    \item Se especificaron los requerimientos mediante casos de uso con sus respectivas interfaces de usuario.
    \item Se elaboró el a de casos de uso para este sprint.
    \item Se elaboró la maquina de estados de una vacante.
\end{enumerate} 

Los requerimientos funcionales de este sprint se muestran en la siguiente tabla.
\begin{requerimientos}{funcionales}
    \RFitem{RF-09}{Consultar vacantes}{El sistema debe permitir a cualquier persona consultar las vacantes que se tengan registradas siempre y cuando aun estén abiertas.}
    \RFitem{RF-10}{Publicar vacantes }{El sistema debe permitir  a los usuarios registrados publicar vacantes.}
    \RFitem{RF-11}{Editar vacantes}{El sistema debe permitir a los usuarios editar las vacantes que ellos hayan publicado.}
    \RFitem{RF-12}{Reportar vacantes}{El sistema debe permitir a los usuarios reportar vacantes publicadas si es que el usuario lo crea necesario.}
    \RFitem{RF-13}{Eliminar vacantes}{El sistema debe permitir a los usuarios eliminar las vacantes que ellos hayan publicado.}
\end{requerimientos}

Los casos de uso que se implementaron en este sprint fueron:
\begin{enumerate}
    \item VCT-CU01 Buscar vacantes
    \item VCT-CU03-1 Cerrar vacante
    \item VCT-CU06 Listar vacantes
    \item VCT-CU06-1 Consultar vacante
    \item VCT-CU03-2 Editar vacante
    \item VCT-CU03-3 Elimnar vacante
    \item VCT-CU03 Publicar vacante
\end{enumerate} 



\subsection{Sprint 5: Gestión de Postulaciones}

Esta iteración tuvo como propósito implementar un mecanismo para gestionar las postulaciones de los usuarios registrdos dentro del 
sistema, las actividades realizadas fueron:
\begin{enumerate}
    \item Se diseñó e implementó las interfaces de usuario para la gestión de posutlaciones según el tipo de cuenta.
    \item Se especificaron los requerimientos mediante casos de uso con sus respectivas interfaces de usuario.
    \item Se elaboró el diagrama de casos de uso para este sprint.
    \item Se elaboró la maquina de estados de una postulaciones.
\end{enumerate} 

Los requerimientos funcionales de este sprint se muestran en la siguiente tabla.
\begin{requerimientos}{funcionales}
    \RFitem{RF-14}{Postularse a vacantes}{El sistema debe permitir a los usuarios registrados enviar postularse a  las vacantes que esté abiertas.}
    \RFitem{RF-15}{Consultar el estado de postulaciones}{El sistema debe permitir al actor consultar el estado de su o sus postulaciones que haya hecho.}
    \RFitem{RF-16}{Dar seguimiento a postulaciones}{El sistema debe permitir al actor indicar  el estado de las postulaciones que tiene según sus vacantes publicadas.}
    \RFitem{RF-17}{Consultar postulaciones}{El sistema debe permitir a los actores registrados consultar las postulaciones que se hayan hecho de acuerdo a cada vacante.}
        
\end{requerimientos}


Los casos de uso que se implementaron en este sprint fueron:
\begin{enumerate}
    \item PST-CU1 Consultar postulaciones
    \item PST-CU2 Postularse a una vacante
    \item PST-CU1-2 Rechazar postulación
\end{enumerate} 


\section{Casos de uso}


%\clearpage
\begin{UseCase}[]{VCT-CU04}{Registrar vacante}{
	Permite al reclutador de una empresa  publicar una vacante en el sistema durante cierto periodo de tiempo y así poder gestionas las postulaciones 
	que los usarios(canditos) hagan a dicha vacante.
	}
	%----------------------------------------------------------------
	% Datos generales del CU:
	\UCsection{Atributos}
	\UCitem{Actor(es)}{
		Reclutadores.

	}
	\UCitem[admin]{Prioridad}{
		Media
	}
	\UCitem[admin]{Complejidad}{
		Alta
	}
	\UCitem{Precondiciones}{
		El reclutador debe de estar registrado en el sistema.
	}
	\UCitem{Destino}{
		\Titem \refElem{VCT-IU03}
	}
	\UCitem{Reglas de Negocio}{
		\Titem \refIdElem{RN-N001}
		
	}
	\UCitem{Viene de}{
		\refElem{VCT-CU03}
	}	
\end{UseCase}

%Trayectoria Principal
\begin{UCtrayectoria}
	\UCpaso [\UCactor] Da clic en el icono \IUAgregar{} (Publicar vacante) en la interfaz \refElem{VCT-IU03}.
	\UCpaso [\UCsist] Muestra la interfaz \refElem{VCT-IU04a} en la interfaz \refElem{VCT-IU03}.
	\UCpaso [\UCactor] \label{VCT-CU04:vadata} Ingresa el título y el número de plazas de la vacante, ingresa el código postal, estado, municipio y colonia donde se va laboral.\refTray{A}
	\UCpaso [\UCactor] Ingresa el perfil, la experiencia y el tipo de contratación a la que la vacante va dirigida.
	\UCpaso [\UCactor] Ingresa el horario laboral y el rango salarial indicando si es neto o no el salario.
	\UCpaso [\UCsist] Valida que todos los campos marcados como obligatorios hayan sido ingresados de acuerdo a la regla de negocio \refIdElem{RN-N001}. \refTray{B}
	\UCpaso [\UCactor] Ingresa la descripción de la vacante.
	\UCpaso [\UCactor] \label{VCT-CU04:hab}Selecciona las habilidades y su correspondiente experiencia deseadas para la vacante.\refTray{D}
	\UCpaso [\UCactor] Selecciona la fecha de cierre de la vacante.
	\UCpaso [\UCactor] Da clic en el botón \IUbutton{Publicar} en la interfaz \refElem{VCT-IU04b}.\refTray{A}\refTray{C} \refTray{F}
	\UCpaso [\UCsist] \label{VCT-CU04:vadata2}Valida que todos los campos marcados como obligatorios hayan sido ingresados de acuerdo a la regla de negocio \refIdElem{RN-N001}.\refTray{E}
	\UCpaso [\UCsist] Muestra la interfaz \refElem{VCT-IU03} mostrando la nueva vacante registrada.
	\UCpaso [\UCsist] Notifica a los encargados y colaboradores de que hay una nueva vacante por revisar.
\end{UCtrayectoria}

%Trayectorias Alternativas
\begin{UCtrayectoriaA}[Fin de la trayectoria]{A}{El actor decide cancelar el registro.}
	\UCpaso [\UCactor] Da clic en el botón \IUbutton{Cancelar} en la interfaz \refElem{VCT-IU04a}.
	\UCpaso [\UCsist] Muestra el mensaje \refIdElem{MSG6} en la interfaz \refElem{VCT-IU04a}.
	\UCpaso [\UCactor] Da clic en el botón \IUbutton{Sí} en la interfaz \refElem{VCT-IU04a}.\refTray{G}
	\UCpaso [\UCsist] Muestra la interfaz \refElem{VCT-IU03}.
\end{UCtrayectoriaA} 

%Trayectorias Alternativas
\begin{UCtrayectoriaA}[Fin de la trayectoria]{B}{El actor no registro al menos un campo obligatorio.}
	\UCpaso [\UCsist] Muestra el mensaje \refIdElem{MSG4} en la interfaz \refElem{VCT-IU03a} en los campos que no
	fueron ingresados.
	\UCpaso [\UCsist] Continúa en el paso \ref{VCT-CU04:vadata} de la trayectoria principal.
\end{UCtrayectoriaA} 

%Trayectorias Alternativas
\begin{UCtrayectoriaA}[Fin de la trayectoria]{C}{El actor decide regresa a la pantalla anterior.}
	\UCpaso [\UCactor] Da clic en el botón \IUbutton{Regresar} en la interfaz \refElem{VCT-IU04b}.
	\UCpaso [\UCsist] Muestra la interfaz \refElem{VCT-IU04a}.
	\UCpaso [\UCsist] Continúa en el paso \ref{VCT-CU04:vadata} de la trayectoria principal.
\end{UCtrayectoriaA} 

%Trayectorias Alternativas
\begin{UCtrayectoriaA}[Fin de la trayectoria]{D}{El actor decide eliminar una habilidad.}
	\UCpaso [\UCactor] Da clic en el botón \IUbutton{x} de la habilidad seleccionada en la interfaz \refElem{VCT-IU04b}.
	\UCpaso [\UCsist] Muestra la interfaz \refElem{VCT-IU04b}.
	\UCpaso [\UCsist] Continúa en el paso \ref{VCT-CU04:hab} de la trayectoria principal.
\end{UCtrayectoriaA} 

%Trayectorias Alternativas
\begin{UCtrayectoriaA}[Fin de la trayectoria]{E}{El actor no registro al menos un campo obligatorio.}
	\UCpaso [\UCsist] Muestra el mensaje \refIdElem{MSG4} en la interfaz \refElem{VCT-IU03b} en los campos que no
	fueron ingresados.
	\UCpaso [\UCsist] Continúa en el paso \ref{VCT-CU04:vadata2} de la trayectoria principal.
\end{UCtrayectoriaA} 

%Trayectorias Alternativas
\begin{UCtrayectoriaA}[Fin de la trayectoria]{F}{El actor decide cancelar el registro.}
	\UCpaso [\UCactor] Da clic en el botón \IUbutton{Cancelar} en la interfaz \refElem{VCT-IU04b}.
	\UCpaso [\UCsist] Muestra el mensaje \refIdElem{MSG6} en la interfaz \refElem{VCT-IU04b}.
	\UCpaso [\UCactor] Da clic en el botón \IUbutton{Sí} en la interfaz \refElem{VCT-IU04b}.\refTray{H}
	\UCpaso [\UCsist] Muestra la interfaz \refElem{VCT-IU03}.
\end{UCtrayectoriaA} 

%Trayectorias Alternativas
\begin{UCtrayectoriaA}[Fin de la trayectoria]{G}{El actor decide cancelar la acción.}
	\UCpaso [\UCactor] Da clic en el botón \IUbutton{No} en la interfaz \refElem{VCT-IU04a}.
	\UCpaso [\UCsist] Muestra la interfaz \refElem{VCT-IU04a}.
	\UCpaso [\UCsist] Continúa en el paso \ref{VCT-CU04:vadata} de la trayectoria principal.
\end{UCtrayectoriaA} 

%Trayectorias Alternativas
\begin{UCtrayectoriaA}[Fin de la trayectoria]{H}{El actor decide cancelar la acción.}
	\UCpaso [\UCactor] Da clic en el botón \IUbutton{No} en la interfaz \refElem{VCT-IU04b}.
	\UCpaso [\UCsist] Muestra la interfaz \refElem{VCT-IU04b}.
	\UCpaso [\UCsist] Continúa en el paso \ref{VCT-CU04:vadata2} de la trayectoria principal.
\end{UCtrayectoriaA} 
%\clearpage
\begin{UseCase}[]{VCT-CU04}{Registrar vacante}{
	Permite al reclutador de una empresa  publicar una vacante en el sistema durante cierto periodo de tiempo y así poder gestionas las postulaciones 
	que los usarios(canditos) hagan a dicha vacante.
	}
	%----------------------------------------------------------------
	% Datos generales del CU:
	\UCsection{Atributos}
	\UCitem{Actor(es)}{
		Reclutadores.

	}
	\UCitem[admin]{Prioridad}{
		Media
	}
	\UCitem[admin]{Complejidad}{
		Alta
	}
	\UCitem{Precondiciones}{
		El reclutador debe de estar registrado en el sistema.
	}
	\UCitem{Destino}{
		\Titem \refElem{VCT-IU03}
	}
	\UCitem{Reglas de Negocio}{
		\Titem \refIdElem{RN-N001}
		
	}
	\UCitem{Viene de}{
		\refElem{VCT-CU03}
	}	
\end{UseCase}

%Trayectoria Principal
\begin{UCtrayectoria}
	\UCpaso [\UCactor] Da clic en el icono \IUAgregar{} (Publicar vacante) en la interfaz \refElem{VCT-IU03}.
	\UCpaso [\UCsist] Muestra la interfaz \refElem{VCT-IU04a} en la interfaz \refElem{VCT-IU03}.
	\UCpaso [\UCactor] \label{VCT-CU04:vadata} Ingresa el título y el número de plazas de la vacante, ingresa el código postal, estado, municipio y colonia donde se va laboral.\refTray{A}
	\UCpaso [\UCactor] Ingresa el perfil, la experiencia y el tipo de contratación a la que la vacante va dirigida.
	\UCpaso [\UCactor] Ingresa el horario laboral y el rango salarial indicando si es neto o no el salario.
	\UCpaso [\UCsist] Valida que todos los campos marcados como obligatorios hayan sido ingresados de acuerdo a la regla de negocio \refIdElem{RN-N001}. \refTray{B}
	\UCpaso [\UCactor] Ingresa la descripción de la vacante.
	\UCpaso [\UCactor] \label{VCT-CU04:hab}Selecciona las habilidades y su correspondiente experiencia deseadas para la vacante.\refTray{D}
	\UCpaso [\UCactor] Selecciona la fecha de cierre de la vacante.
	\UCpaso [\UCactor] Da clic en el botón \IUbutton{Publicar} en la interfaz \refElem{VCT-IU04b}.\refTray{A}\refTray{C} \refTray{F}
	\UCpaso [\UCsist] \label{VCT-CU04:vadata2}Valida que todos los campos marcados como obligatorios hayan sido ingresados de acuerdo a la regla de negocio \refIdElem{RN-N001}.\refTray{E}
	\UCpaso [\UCsist] Muestra la interfaz \refElem{VCT-IU03} mostrando la nueva vacante registrada.
	\UCpaso [\UCsist] Notifica a los encargados y colaboradores de que hay una nueva vacante por revisar.
\end{UCtrayectoria}

%Trayectorias Alternativas
\begin{UCtrayectoriaA}[Fin de la trayectoria]{A}{El actor decide cancelar el registro.}
	\UCpaso [\UCactor] Da clic en el botón \IUbutton{Cancelar} en la interfaz \refElem{VCT-IU04a}.
	\UCpaso [\UCsist] Muestra el mensaje \refIdElem{MSG6} en la interfaz \refElem{VCT-IU04a}.
	\UCpaso [\UCactor] Da clic en el botón \IUbutton{Sí} en la interfaz \refElem{VCT-IU04a}.\refTray{G}
	\UCpaso [\UCsist] Muestra la interfaz \refElem{VCT-IU03}.
\end{UCtrayectoriaA} 

%Trayectorias Alternativas
\begin{UCtrayectoriaA}[Fin de la trayectoria]{B}{El actor no registro al menos un campo obligatorio.}
	\UCpaso [\UCsist] Muestra el mensaje \refIdElem{MSG4} en la interfaz \refElem{VCT-IU03a} en los campos que no
	fueron ingresados.
	\UCpaso [\UCsist] Continúa en el paso \ref{VCT-CU04:vadata} de la trayectoria principal.
\end{UCtrayectoriaA} 

%Trayectorias Alternativas
\begin{UCtrayectoriaA}[Fin de la trayectoria]{C}{El actor decide regresa a la pantalla anterior.}
	\UCpaso [\UCactor] Da clic en el botón \IUbutton{Regresar} en la interfaz \refElem{VCT-IU04b}.
	\UCpaso [\UCsist] Muestra la interfaz \refElem{VCT-IU04a}.
	\UCpaso [\UCsist] Continúa en el paso \ref{VCT-CU04:vadata} de la trayectoria principal.
\end{UCtrayectoriaA} 

%Trayectorias Alternativas
\begin{UCtrayectoriaA}[Fin de la trayectoria]{D}{El actor decide eliminar una habilidad.}
	\UCpaso [\UCactor] Da clic en el botón \IUbutton{x} de la habilidad seleccionada en la interfaz \refElem{VCT-IU04b}.
	\UCpaso [\UCsist] Muestra la interfaz \refElem{VCT-IU04b}.
	\UCpaso [\UCsist] Continúa en el paso \ref{VCT-CU04:hab} de la trayectoria principal.
\end{UCtrayectoriaA} 

%Trayectorias Alternativas
\begin{UCtrayectoriaA}[Fin de la trayectoria]{E}{El actor no registro al menos un campo obligatorio.}
	\UCpaso [\UCsist] Muestra el mensaje \refIdElem{MSG4} en la interfaz \refElem{VCT-IU03b} en los campos que no
	fueron ingresados.
	\UCpaso [\UCsist] Continúa en el paso \ref{VCT-CU04:vadata2} de la trayectoria principal.
\end{UCtrayectoriaA} 

%Trayectorias Alternativas
\begin{UCtrayectoriaA}[Fin de la trayectoria]{F}{El actor decide cancelar el registro.}
	\UCpaso [\UCactor] Da clic en el botón \IUbutton{Cancelar} en la interfaz \refElem{VCT-IU04b}.
	\UCpaso [\UCsist] Muestra el mensaje \refIdElem{MSG6} en la interfaz \refElem{VCT-IU04b}.
	\UCpaso [\UCactor] Da clic en el botón \IUbutton{Sí} en la interfaz \refElem{VCT-IU04b}.\refTray{H}
	\UCpaso [\UCsist] Muestra la interfaz \refElem{VCT-IU03}.
\end{UCtrayectoriaA} 

%Trayectorias Alternativas
\begin{UCtrayectoriaA}[Fin de la trayectoria]{G}{El actor decide cancelar la acción.}
	\UCpaso [\UCactor] Da clic en el botón \IUbutton{No} en la interfaz \refElem{VCT-IU04a}.
	\UCpaso [\UCsist] Muestra la interfaz \refElem{VCT-IU04a}.
	\UCpaso [\UCsist] Continúa en el paso \ref{VCT-CU04:vadata} de la trayectoria principal.
\end{UCtrayectoriaA} 

%Trayectorias Alternativas
\begin{UCtrayectoriaA}[Fin de la trayectoria]{H}{El actor decide cancelar la acción.}
	\UCpaso [\UCactor] Da clic en el botón \IUbutton{No} en la interfaz \refElem{VCT-IU04b}.
	\UCpaso [\UCsist] Muestra la interfaz \refElem{VCT-IU04b}.
	\UCpaso [\UCsist] Continúa en el paso \ref{VCT-CU04:vadata2} de la trayectoria principal.
\end{UCtrayectoriaA} 
%\clearpage
\subsection{USR-IU02 Consultar perfil}

\subsubsection{Objetivo}
En la figura \refElem{USR-IU02} se muestra la interfaz correspondiente con la funcionalidad descrita en las
trayectorias del caso de uso \refElem{USR-CU02} , la cual permite al actor la gestión su perfil y la consulta del mismo.

La interfaz esta compuesta ``secciones'' y cada sección corresponde a un formulario diferente, las secciones
son las siguientes:
\begin{itemize}
   \item \textbf{Datos personales}: esta sección tiene como objetivo que el actor actualice su información
   personal,sus datos de contacto y/o habilidades que posee (ver la figura \refElem{USR-IU02a}).
   \item \textbf{Objetivos y metas personales}: esta sección tiene como objetivo que el actor actualice sus objetivos, metas personales y laborales 
   (ver la figura \refElem{USR-IU02b}).
   \item \textbf{Historial académico}: esta sección tiene como objetivo que el actor actualice su información
   académica referente a todos los grados de estudios que tiene hasta la fecha (ver la figura \refElem{USR-IU02c}).
   \item \textbf{Idiomas}: esta sección tiene como objetivo preguntarle que el actor actualice la información de idiomas o en su caso, elimine
   o agregue nuevos idiomas a su perfil  (ver la figura \refElem{USR-IU02d}).
   \item \textbf{Experiencia laboral}: esta sección tiene como objetivo que el actor actualice su información de su experiencia
   laborar que tiene hasta la fecha (ver la figura \refElem{USR-IU02e}).
   \item \textbf{Cursos/Certificaciones}:  esta sección tiene como objetivo que el actor actualice su información de sus Certificaciones
   o cursos que ha tenido durante toda su trayectoria académica y laboral (ver la figura \refElem{USR-IU02f}).
\end{itemize}

\subsubsection{Comandos}
Los siguientes comandos aparecen durante toda la interfaz es decir, cada sección los tiene.

%\Titem \IUPass : Al da clic en el ícono, se muestra la contraseña de lo contrario aparecerá \IUOculto \thinspace sustituyendo cada caracter de la contraseña. \\

\Titem \IUEditar{} : Cuando presiona el ícono, habilita la sección para hacer los datos editables acorde a la sección o elemento seleccionado.
\Titem \IUEliminar{} : Cuando presiona el ícono, habilita la sección para eliminar el elemento seleccionado.
\Titem \IUAgregar{} : Cuando presiona el ícono, habilita la sección para agregar un nuevo el elemento.

\IUfig{.9}{CasosdeUso/USR-CU02/imagenes/USR-IU02.png}{USR-IU02}{Consultar perfil}  
\IUfig{.5}{CasosdeUso/USR-CU02/imagenes/USR-IU02a.png}{USR-IU02a}{Consultar perfil: Datos personales}
\IUfig{.9}{CasosdeUso/USR-CU02/imagenes/USR-IU02b.png}{USR-IU02b}{Consultar perfil: Objetivos y metas personales}  
\IUfig{.9}{CasosdeUso/USR-CU02/imagenes/USR-IU02c.png}{USR-IU02c}{Consultar perfil: Historial académico}  
\IUfig{.9}{CasosdeUso/USR-CU02/imagenes/USR-IU02d.png}{USR-IU02d}{Consultar perfil: Idiomas}
\IUfig{.9}{CasosdeUso/USR-CU02/imagenes/USR-IU02e.png}{USR-IU02e}{Consultar perfil: Experiencia laboral}  
\IUfig{.9}{CasosdeUso/USR-CU02/imagenes/USR-IU02f.png}{USR-IU02f}{Consultar perfil: Cursos/Certificaciones}  


\clearpage


Los casos de uso son una herramienta usada para representar las transacciones entre un actor y el sistema, las cuales siempre tendrán un valor agregado o un propósito para que el actor las realice.
En estos se representan a traves de aas los cuales se conforan por los siguientes elementos:

\begin{itemize}
	 \UCpaso Representa al sistema mediante un óvalo.
	\item \UCactor Representa al actor que va a interactuar con el sistema.
	\item Relación $<$- - -$<<extends>>$- - -. Indica que un caso de uso \textbf{puede} ejecutarse a partir de otro.
	\item Relación - - -$<<include>>$- - -$>$. Indica que un caso de uso \textbf{debe} ejecutarse a partir de otro.
\end{itemize}

La conexión entre un actor y un caso de uso es por medio de una línea como se muestra en la figura \ref{fig:acUC}.

\begin{figure}[hbtp!]
	\begin{center}
		\includegraphics[width=.4\textwidth]{LIT/ActorUC}
	\end{center}
	\label{fig:acUC}
	\caption{Interacción del actor con el caso de uso}
\end{figure}

Los casos de uso se encontrarán dentro de paquetes (representados por carpetas) indicando así que pertenecen a un mismo módulo como se muestra en la figura \ref{fig:pack}.

\begin{figure}[hbtp!]
	\begin{center}
		\includegraphics[width=.7\textwidth]{LIT/Paquete}
	\end{center}
	\label{fig:pack}
	\caption{Un Actor con varios caso de uso dentro de un módulo}
\end{figure}

\pagebreak
\subsection{Diagrama de casos de uso Sprint 3}
\begin{figure}[hbtp!]
	\begin{center}
		\includegraphics[width=.5\textwidth]{propuesta/imagenes/sprint3.jpeg}
	\end{center}
	\label{fig:acUC}
	\caption{Interacción del actor con el caso de uso}
\end{figure}

\subsection{Diagrama de casos de uso Sprint 4}
\begin{figure}[hbtp!]
	\begin{center}
		\includegraphics[width=.5\textwidth]{propuesta/imagenes/sprin4R.png}
	\end{center}
	\label{fig:acUC}
	\caption{Interacción del actor con el caso de uso}
\end{figure}

\subsection{Diagrama de casos de uso Sprint 5}
\begin{figure}[hbtp!]
	\begin{center}
		\includegraphics[width=.5\textwidth]{propuesta/imagenes/Sprint5.png}
	\end{center}
	\label{fig:acUC}
	\caption{Interacción del actor con el caso de uso}
\end{figure}


%---------------------------------------------------------
\subsection{Modelado de casos de uso}
A continuación se listas los componentes que conformam un caso de uso completo las cuales se pueden consultar en
el apendice de este documento.

\subsubsection{Modelo de estados}

Diversas entidades en el sistema  que tienen un comportamiento dinámico en el sistema, 
los que se consideraron en el desarrollo del sistema y ameritaban ser modelados a través de una 
maquina de estados.\\

\subsubsection{Actores del sistema}
 Se definen los actores identificados como participantes en los procesos del sistema. 
 Estos actores son los encargados de llevar a cabo determinadas tareas dentro de cada proceso.\\

 \subsubsection{Reglas de negocio}

 Se van describir las reglas de negocio identificadas en el sistema las cuales son una condición que 
 se debe satisfacer cuando se realiza una actividad de negocio.\\
 \subsubsection{Reglas del sistema ???}
 
\subsubsection{Casos de uso}

Las funcionalidades del sistema son representadas mediante casos de uso. Un caso de uso es una transacción entre un actor y el sistema que tiene un valor agregado o un propósito para que el actor las realice. Los casos de uso se modelan a partir de dos elementos: un aa de casos de uso y la
descripción del caso de uso.\\



\subsubsection{Mensajes del sistema}

Los mensajes se refieren a todos aquellos avisos que el sistema muestra al actor para comunicar la ocurrencia de algún evento tal como un error o una operación exitosa. 

