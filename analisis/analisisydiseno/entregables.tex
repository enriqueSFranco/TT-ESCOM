\section{Entregables y Sprits}
Como resultado final del desarrollo del proyecto se espera que éste concluya con un sistema web de la siguiente forma:
\begin{enumerate}
    \item Módulo de Usuarios: Este módulo se encargará de la gestión de los candidatos, administradores y
    reclutadores
    \begin{enumerate}
        \item Candidatos: Estos podrán crear un perfil, realizar búsquedas y aplicar a vacantes.
        \item Reclutador: Estos podrán crear un perfil, publicar, modificar, eliminar vacantes y administrar el
proceso de selección.
        \item Administrador: Este podrá consultar, dar de alta, modificar y eliminar a un reclutador y a colaboradores para 
        la gestión de la bolsa de trabajo.

    \end{enumerate}
    \item Módulo de Administración de Solicitudes: En este módulo se implementará un mecanismo para la administración
de las solicitudes.
\item Módulo Currículum: En este módulo se implementará el generado automático de currículums y las
sugerencias para el si es que las requiere.
\item Módulo de Administración y Filtro de Vacantes: Este módulo permite a los reclutadores gestionar las
vacantes de la empresa a la que pertenece y a los candidatos consultarlas según sus aptitudes y habilidades.
\end{enumerate}
Los Sprits que se trabajarón durante TT1 son:
\subsection{Sprint 1: Pruebas de concepto}
Esta iteración tuvo como propósito capacitarnos en las tecnologías que se van ha estar utilizando 
a lo largo del desarrollo del sistema.La actividades que se realizaron las siguientes:

\begin{enumerate}
    \item Crear el product backlog así como elementos de apoyo para dar prioridad y seguimiento.
    \item Especificar el lugar físico en el que el código del proyecto se va almacenar y que permita que
    todos los miembros del equipo puedan darle mantenimiento.
    \item Configurar las diferentes plataformas y tecnologías necesarias para el proyectot.
\end{enumerate}

\subsection{Sprint 2: Análisis y Planificación general del sistema Product Backlog}
Esta iteración tuvo como establecer los fundamentos teóricos y prácticos para su desarrollo.Las actividades que se realizaron 
de manera general fueron las siguientes:
\begin{enumerate}
    \item Implementar los requerimientos de prioridad alta del módulo de usuarios.
    \item Diseñar e implementar las interfaces de usuario para la gestión de autenticación y gestión de
    participantes.
    \item Especificar los requerimientos mediante casos de uso y sus respectivas interfaces de usuario.
    \item Para este sprint se definieron los requerimientos funcionales para llevar a cabo los módulos
    correspondientes al desarrollo de esta iteración.
\end{enumerate} 

\subsection{Sprint 3: Gestión de Usuarios}
Esta iteración tuvo como propósito implementar un mecanismo que permita a
los usuarios del sistema autenticarse o crear cuentas para acceder al sistema, así como definir los
roles y los participantes del proyecto.


los requerimientos funcionales de este sprint son:

\begin{itemize}
    \item RF-001: Iniciar sesión
    \item RF-002: Crear cuenta
    \item RF-003: Consultar perfil
    %\item RF-004: Editar perfil
    \item RF-005: Enviar pre-registro
\end{itemize}



\subsection{Sprint 4: Gestión de Vacantes}

Esta iteración tuvo como propósito implementar un mecanismo para gestionar las vacantes dentro del sistema.


Los requerimientos funcionales de este sprint son:

\begin{itemize}
    \item RF-006: Publicar vacante
    %\item RF-008: Editar vacante
    %\item RF-009: Eliminar vacante
    %\item RF-010: Reportar vacante
    \item RF-011: Consultar vacante
    \item RF-012: Listar vacante
\end{itemize}


\subsection{Sprint 5: Gestión de Postulaciones}

Esta iteración tuvo como propósito implementar un mecanismo para gestionar las postulaciones dentro del sistema.


Los requerimientos funcionales de este sprint son:

\begin{itemize}
    \item RF-013: Enviar Postulación
    %\item RF-014: Consultar Postulación
    %\item RF-015: Descartar Postulación
    \item RF-016: Listar Postulaciones
\end{itemize}

