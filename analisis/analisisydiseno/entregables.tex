\section{Entregables y Sprits}
Como resultado final del desarrollo del proyecto se espera que éste concluya con un sistema web de la siguiente forma:
\begin{enumerate}
    \item Usuarios: Este módulo se encargará de la gestión de los candidatos, administradores y
    reclutadores
    \begin{enumerate}
        \item Candidatos: Estos podrán crear un perfil, realizar búsquedas y aplicar a vacantes.
        \item Reclutador: Estos podrán crear un perfil, publicar, modificar, eliminar vacantes y administrar el
        proceso de selección.
        \item Administrador: Este podrá consultar, dar de alta, modificar y eliminar a un reclutador y a colaboradores para 
        la gestión de la bolsa de trabajo.
    \end{enumerate}

    \item Administración de Solicitudes: En este módulo se implementará un mecanismo para la administración
    de las solicitudes.
    \item Módulo Currículum: En este módulo se implementará el generado automático de currículums y las
    sugerencias para el si es que las requiere.
    \item Administración y Filtro de Vacantes: Este módulo permite a los reclutadores gestionar las
    vacantes de la empresa a la que pertenece y a los candidatos consultarlas según sus aptitudes y habilidades.
\end{enumerate}

Los Sprits que se trabajarón durante TT-1 son:

\subsection{Sprint 1: Pruebas de concepto}
Esta iteración tuvo como propósito capacitarnos en las tecnologías que se van ha estar utilizando 
a lo largo del desarrollo del sistema.La actividades que se realizaron las siguientes:

\begin{enumerate}
    \item Crear el product backlog así como elementos de apoyo para dar prioridad y seguimiento.
    \item Especificar el lugar físico en el que el código del proyecto se va almacenar y que permita que
    todos los miembros del equipo puedan darle mantenimiento.
    \item Configurar las diferentes plataformas y tecnologías necesarias para el proyectot.
\end{enumerate}

\subsection{Sprint 2: Análisis y Planificación general del sistema Product Backlog}
Esta iteración tuvo como establecer los fundamentos teóricos y prácticos para su desarrollo.Las actividades que se realizaron 
de manera general fueron las siguientes:
\begin{enumerate}
    \item Implementar los requerimientos de prioridad alta.
    \item Diseñar e implementar las interfaces de usuario para la gestión de autenticación y gestión de
    participantes.
    \item Especificar los requerimientos mediante casos de uso y sus respectivas interfaces de usuario.
    \item Para este sprint se definieron los requerimientos funcionales para llevar a cabo los módulos
    correspondientes al desarrollo de esta iteración.
\end{enumerate} 

\subsection{Sprint 3: Gestión de Usuarios}
Esta iteración tuvo como propósito implementar un mecanismo que permita a
los usuarios del sistema autenticarse o crear cuentas para acceder al sistema, así como definir los
roles y los participantes del proyecto.


Los requerimientos funcionales de este sprint son:

\begin{longtable}{| p{0.15\textwidth}  | p{0.30\textwidth} | p{0.45\textwidth}  |}

    \label{table:herramientasSimilares}
        \rowcolor{black}
        %\multicolumn{7}{ |c| }{\bf\cellcolor{black}\color{white}{Herramientas para la especificación de requerimientos}} \\ \hline
        \bf\color{white} ID & \bf \color{white}Nombre	& \bf \color{white}Descripción \\ \hline
    \endhead
    RF-001 &Iniciar sesión & \\ \hline
    RF-002 &Crear cuenta & \\ \hline
    RF-003 &Enviar pre-registro & \\ \hline
    RF-004 &Configurar perfil & \\ \hline
    %\caption{Comparación Plataformas para buscar empleo}
\end{longtable}



\subsection{Sprint 4: Gestión de Vacantes}

Esta iteración tuvo como propósito implementar un mecanismo para gestionar las vacantes dentro del sistema.


Los requerimientos funcionales de este sprint son:
\begin{longtable}{| p{0.15\textwidth}  | p{0.30\textwidth} | p{0.45\textwidth}  |}

    \label{table:herramientasSimilares}
        \rowcolor{black}
        %\multicolumn{7}{ |c| }{\bf\cellcolor{black}\color{white}{Herramientas para la especificación de requerimientos}} \\ \hline
        \bf\color{white} ID & \bf \color{white}Nombre	& \bf \color{white}Descripción \\ \hline
    \endhead
    RF-006 &Publicar vacante& \\ \hline
    RF-011 &Consultar vacante & \\ \hline
    RF-012 &Listar vacante & \\ \hline
    %\caption{Comparación Plataformas para buscar empleo}
\end{longtable}

\subsection{Sprint 5: Gestión de Postulaciones}

Esta iteración tuvo como propósito implementar un mecanismo para gestionar las postulaciones dentro del sistema.


Los requerimientos funcionales de este sprint son:

\begin{longtable}{| p{0.15\textwidth}  | p{0.30\textwidth} | p{0.45\textwidth}  |}

    \label{table:herramientasSimilares}
        \rowcolor{black}
        %\multicolumn{7}{ |c| }{\bf\cellcolor{black}\color{white}{Herramientas para la especificación de requerimientos}} \\ \hline
        \bf\color{white} ID & \bf \color{white}Nombre	& \bf \color{white}Descripción
    \endhead
    RF-013 &Enviar Postulación& \\ \hline
    RF-016 &Listar Postulaciones & \\ \hline
    %\caption{Comparación Plataformas para buscar empleo}
\end{longtable}

\section{Casos de uso}


%\clearpage
\begin{UseCase}[]{VCT-CU04}{Registrar vacante}{
	Permite al reclutador de una empresa  publicar una vacante en el sistema durante cierto periodo de tiempo y así poder gestionas las postulaciones 
	que los usarios(canditos) hagan a dicha vacante.
	}
	%----------------------------------------------------------------
	% Datos generales del CU:
	\UCsection{Atributos}
	\UCitem{Actor(es)}{
		Reclutadores.

	}
	\UCitem[admin]{Prioridad}{
		Media
	}
	\UCitem[admin]{Complejidad}{
		Alta
	}
	\UCitem{Precondiciones}{
		El reclutador debe de estar registrado en el sistema.
	}
	\UCitem{Destino}{
		\Titem \refElem{VCT-IU03}
	}
	\UCitem{Reglas de Negocio}{
		\Titem \refIdElem{RN-N001}
		
	}
	\UCitem{Viene de}{
		\refElem{VCT-CU03}
	}	
\end{UseCase}

%Trayectoria Principal
\begin{UCtrayectoria}
	\UCpaso [\UCactor] Da clic en el icono \IUAgregar{} (Publicar vacante) en la interfaz \refElem{VCT-IU03}.
	\UCpaso [\UCsist] Muestra la interfaz \refElem{VCT-IU04a} en la interfaz \refElem{VCT-IU03}.
	\UCpaso [\UCactor] \label{VCT-CU04:vadata} Ingresa el título y el número de plazas de la vacante, ingresa el código postal, estado, municipio y colonia donde se va laboral.\refTray{A}
	\UCpaso [\UCactor] Ingresa el perfil, la experiencia y el tipo de contratación a la que la vacante va dirigida.
	\UCpaso [\UCactor] Ingresa el horario laboral y el rango salarial indicando si es neto o no el salario.
	\UCpaso [\UCsist] Valida que todos los campos marcados como obligatorios hayan sido ingresados de acuerdo a la regla de negocio \refIdElem{RN-N001}. \refTray{B}
	\UCpaso [\UCactor] Ingresa la descripción de la vacante.
	\UCpaso [\UCactor] \label{VCT-CU04:hab}Selecciona las habilidades y su correspondiente experiencia deseadas para la vacante.\refTray{D}
	\UCpaso [\UCactor] Selecciona la fecha de cierre de la vacante.
	\UCpaso [\UCactor] Da clic en el botón \IUbutton{Publicar} en la interfaz \refElem{VCT-IU04b}.\refTray{A}\refTray{C} \refTray{F}
	\UCpaso [\UCsist] \label{VCT-CU04:vadata2}Valida que todos los campos marcados como obligatorios hayan sido ingresados de acuerdo a la regla de negocio \refIdElem{RN-N001}.\refTray{E}
	\UCpaso [\UCsist] Muestra la interfaz \refElem{VCT-IU03} mostrando la nueva vacante registrada.
	\UCpaso [\UCsist] Notifica a los encargados y colaboradores de que hay una nueva vacante por revisar.
\end{UCtrayectoria}

%Trayectorias Alternativas
\begin{UCtrayectoriaA}[Fin de la trayectoria]{A}{El actor decide cancelar el registro.}
	\UCpaso [\UCactor] Da clic en el botón \IUbutton{Cancelar} en la interfaz \refElem{VCT-IU04a}.
	\UCpaso [\UCsist] Muestra el mensaje \refIdElem{MSG6} en la interfaz \refElem{VCT-IU04a}.
	\UCpaso [\UCactor] Da clic en el botón \IUbutton{Sí} en la interfaz \refElem{VCT-IU04a}.\refTray{G}
	\UCpaso [\UCsist] Muestra la interfaz \refElem{VCT-IU03}.
\end{UCtrayectoriaA} 

%Trayectorias Alternativas
\begin{UCtrayectoriaA}[Fin de la trayectoria]{B}{El actor no registro al menos un campo obligatorio.}
	\UCpaso [\UCsist] Muestra el mensaje \refIdElem{MSG4} en la interfaz \refElem{VCT-IU03a} en los campos que no
	fueron ingresados.
	\UCpaso [\UCsist] Continúa en el paso \ref{VCT-CU04:vadata} de la trayectoria principal.
\end{UCtrayectoriaA} 

%Trayectorias Alternativas
\begin{UCtrayectoriaA}[Fin de la trayectoria]{C}{El actor decide regresa a la pantalla anterior.}
	\UCpaso [\UCactor] Da clic en el botón \IUbutton{Regresar} en la interfaz \refElem{VCT-IU04b}.
	\UCpaso [\UCsist] Muestra la interfaz \refElem{VCT-IU04a}.
	\UCpaso [\UCsist] Continúa en el paso \ref{VCT-CU04:vadata} de la trayectoria principal.
\end{UCtrayectoriaA} 

%Trayectorias Alternativas
\begin{UCtrayectoriaA}[Fin de la trayectoria]{D}{El actor decide eliminar una habilidad.}
	\UCpaso [\UCactor] Da clic en el botón \IUbutton{x} de la habilidad seleccionada en la interfaz \refElem{VCT-IU04b}.
	\UCpaso [\UCsist] Muestra la interfaz \refElem{VCT-IU04b}.
	\UCpaso [\UCsist] Continúa en el paso \ref{VCT-CU04:hab} de la trayectoria principal.
\end{UCtrayectoriaA} 

%Trayectorias Alternativas
\begin{UCtrayectoriaA}[Fin de la trayectoria]{E}{El actor no registro al menos un campo obligatorio.}
	\UCpaso [\UCsist] Muestra el mensaje \refIdElem{MSG4} en la interfaz \refElem{VCT-IU03b} en los campos que no
	fueron ingresados.
	\UCpaso [\UCsist] Continúa en el paso \ref{VCT-CU04:vadata2} de la trayectoria principal.
\end{UCtrayectoriaA} 

%Trayectorias Alternativas
\begin{UCtrayectoriaA}[Fin de la trayectoria]{F}{El actor decide cancelar el registro.}
	\UCpaso [\UCactor] Da clic en el botón \IUbutton{Cancelar} en la interfaz \refElem{VCT-IU04b}.
	\UCpaso [\UCsist] Muestra el mensaje \refIdElem{MSG6} en la interfaz \refElem{VCT-IU04b}.
	\UCpaso [\UCactor] Da clic en el botón \IUbutton{Sí} en la interfaz \refElem{VCT-IU04b}.\refTray{H}
	\UCpaso [\UCsist] Muestra la interfaz \refElem{VCT-IU03}.
\end{UCtrayectoriaA} 

%Trayectorias Alternativas
\begin{UCtrayectoriaA}[Fin de la trayectoria]{G}{El actor decide cancelar la acción.}
	\UCpaso [\UCactor] Da clic en el botón \IUbutton{No} en la interfaz \refElem{VCT-IU04a}.
	\UCpaso [\UCsist] Muestra la interfaz \refElem{VCT-IU04a}.
	\UCpaso [\UCsist] Continúa en el paso \ref{VCT-CU04:vadata} de la trayectoria principal.
\end{UCtrayectoriaA} 

%Trayectorias Alternativas
\begin{UCtrayectoriaA}[Fin de la trayectoria]{H}{El actor decide cancelar la acción.}
	\UCpaso [\UCactor] Da clic en el botón \IUbutton{No} en la interfaz \refElem{VCT-IU04b}.
	\UCpaso [\UCsist] Muestra la interfaz \refElem{VCT-IU04b}.
	\UCpaso [\UCsist] Continúa en el paso \ref{VCT-CU04:vadata2} de la trayectoria principal.
\end{UCtrayectoriaA} 
%\clearpage
\begin{UseCase}[]{VCT-CU04}{Registrar vacante}{
	Permite al reclutador de una empresa  publicar una vacante en el sistema durante cierto periodo de tiempo y así poder gestionas las postulaciones 
	que los usarios(canditos) hagan a dicha vacante.
	}
	%----------------------------------------------------------------
	% Datos generales del CU:
	\UCsection{Atributos}
	\UCitem{Actor(es)}{
		Reclutadores.

	}
	\UCitem[admin]{Prioridad}{
		Media
	}
	\UCitem[admin]{Complejidad}{
		Alta
	}
	\UCitem{Precondiciones}{
		El reclutador debe de estar registrado en el sistema.
	}
	\UCitem{Destino}{
		\Titem \refElem{VCT-IU03}
	}
	\UCitem{Reglas de Negocio}{
		\Titem \refIdElem{RN-N001}
		
	}
	\UCitem{Viene de}{
		\refElem{VCT-CU03}
	}	
\end{UseCase}

%Trayectoria Principal
\begin{UCtrayectoria}
	\UCpaso [\UCactor] Da clic en el icono \IUAgregar{} (Publicar vacante) en la interfaz \refElem{VCT-IU03}.
	\UCpaso [\UCsist] Muestra la interfaz \refElem{VCT-IU04a} en la interfaz \refElem{VCT-IU03}.
	\UCpaso [\UCactor] \label{VCT-CU04:vadata} Ingresa el título y el número de plazas de la vacante, ingresa el código postal, estado, municipio y colonia donde se va laboral.\refTray{A}
	\UCpaso [\UCactor] Ingresa el perfil, la experiencia y el tipo de contratación a la que la vacante va dirigida.
	\UCpaso [\UCactor] Ingresa el horario laboral y el rango salarial indicando si es neto o no el salario.
	\UCpaso [\UCsist] Valida que todos los campos marcados como obligatorios hayan sido ingresados de acuerdo a la regla de negocio \refIdElem{RN-N001}. \refTray{B}
	\UCpaso [\UCactor] Ingresa la descripción de la vacante.
	\UCpaso [\UCactor] \label{VCT-CU04:hab}Selecciona las habilidades y su correspondiente experiencia deseadas para la vacante.\refTray{D}
	\UCpaso [\UCactor] Selecciona la fecha de cierre de la vacante.
	\UCpaso [\UCactor] Da clic en el botón \IUbutton{Publicar} en la interfaz \refElem{VCT-IU04b}.\refTray{A}\refTray{C} \refTray{F}
	\UCpaso [\UCsist] \label{VCT-CU04:vadata2}Valida que todos los campos marcados como obligatorios hayan sido ingresados de acuerdo a la regla de negocio \refIdElem{RN-N001}.\refTray{E}
	\UCpaso [\UCsist] Muestra la interfaz \refElem{VCT-IU03} mostrando la nueva vacante registrada.
	\UCpaso [\UCsist] Notifica a los encargados y colaboradores de que hay una nueva vacante por revisar.
\end{UCtrayectoria}

%Trayectorias Alternativas
\begin{UCtrayectoriaA}[Fin de la trayectoria]{A}{El actor decide cancelar el registro.}
	\UCpaso [\UCactor] Da clic en el botón \IUbutton{Cancelar} en la interfaz \refElem{VCT-IU04a}.
	\UCpaso [\UCsist] Muestra el mensaje \refIdElem{MSG6} en la interfaz \refElem{VCT-IU04a}.
	\UCpaso [\UCactor] Da clic en el botón \IUbutton{Sí} en la interfaz \refElem{VCT-IU04a}.\refTray{G}
	\UCpaso [\UCsist] Muestra la interfaz \refElem{VCT-IU03}.
\end{UCtrayectoriaA} 

%Trayectorias Alternativas
\begin{UCtrayectoriaA}[Fin de la trayectoria]{B}{El actor no registro al menos un campo obligatorio.}
	\UCpaso [\UCsist] Muestra el mensaje \refIdElem{MSG4} en la interfaz \refElem{VCT-IU03a} en los campos que no
	fueron ingresados.
	\UCpaso [\UCsist] Continúa en el paso \ref{VCT-CU04:vadata} de la trayectoria principal.
\end{UCtrayectoriaA} 

%Trayectorias Alternativas
\begin{UCtrayectoriaA}[Fin de la trayectoria]{C}{El actor decide regresa a la pantalla anterior.}
	\UCpaso [\UCactor] Da clic en el botón \IUbutton{Regresar} en la interfaz \refElem{VCT-IU04b}.
	\UCpaso [\UCsist] Muestra la interfaz \refElem{VCT-IU04a}.
	\UCpaso [\UCsist] Continúa en el paso \ref{VCT-CU04:vadata} de la trayectoria principal.
\end{UCtrayectoriaA} 

%Trayectorias Alternativas
\begin{UCtrayectoriaA}[Fin de la trayectoria]{D}{El actor decide eliminar una habilidad.}
	\UCpaso [\UCactor] Da clic en el botón \IUbutton{x} de la habilidad seleccionada en la interfaz \refElem{VCT-IU04b}.
	\UCpaso [\UCsist] Muestra la interfaz \refElem{VCT-IU04b}.
	\UCpaso [\UCsist] Continúa en el paso \ref{VCT-CU04:hab} de la trayectoria principal.
\end{UCtrayectoriaA} 

%Trayectorias Alternativas
\begin{UCtrayectoriaA}[Fin de la trayectoria]{E}{El actor no registro al menos un campo obligatorio.}
	\UCpaso [\UCsist] Muestra el mensaje \refIdElem{MSG4} en la interfaz \refElem{VCT-IU03b} en los campos que no
	fueron ingresados.
	\UCpaso [\UCsist] Continúa en el paso \ref{VCT-CU04:vadata2} de la trayectoria principal.
\end{UCtrayectoriaA} 

%Trayectorias Alternativas
\begin{UCtrayectoriaA}[Fin de la trayectoria]{F}{El actor decide cancelar el registro.}
	\UCpaso [\UCactor] Da clic en el botón \IUbutton{Cancelar} en la interfaz \refElem{VCT-IU04b}.
	\UCpaso [\UCsist] Muestra el mensaje \refIdElem{MSG6} en la interfaz \refElem{VCT-IU04b}.
	\UCpaso [\UCactor] Da clic en el botón \IUbutton{Sí} en la interfaz \refElem{VCT-IU04b}.\refTray{H}
	\UCpaso [\UCsist] Muestra la interfaz \refElem{VCT-IU03}.
\end{UCtrayectoriaA} 

%Trayectorias Alternativas
\begin{UCtrayectoriaA}[Fin de la trayectoria]{G}{El actor decide cancelar la acción.}
	\UCpaso [\UCactor] Da clic en el botón \IUbutton{No} en la interfaz \refElem{VCT-IU04a}.
	\UCpaso [\UCsist] Muestra la interfaz \refElem{VCT-IU04a}.
	\UCpaso [\UCsist] Continúa en el paso \ref{VCT-CU04:vadata} de la trayectoria principal.
\end{UCtrayectoriaA} 

%Trayectorias Alternativas
\begin{UCtrayectoriaA}[Fin de la trayectoria]{H}{El actor decide cancelar la acción.}
	\UCpaso [\UCactor] Da clic en el botón \IUbutton{No} en la interfaz \refElem{VCT-IU04b}.
	\UCpaso [\UCsist] Muestra la interfaz \refElem{VCT-IU04b}.
	\UCpaso [\UCsist] Continúa en el paso \ref{VCT-CU04:vadata2} de la trayectoria principal.
\end{UCtrayectoriaA} 
%\clearpage
\subsection{USR-IU02 Consultar perfil}

\subsubsection{Objetivo}
En la figura \refElem{USR-IU02} se muestra la interfaz correspondiente con la funcionalidad descrita en las
trayectorias del caso de uso \refElem{USR-CU02} , la cual permite al actor la gestión su perfil y la consulta del mismo.

La interfaz esta compuesta ``secciones'' y cada sección corresponde a un formulario diferente, las secciones
son las siguientes:
\begin{itemize}
   \item \textbf{Datos personales}: esta sección tiene como objetivo que el actor actualice su información
   personal,sus datos de contacto y/o habilidades que posee (ver la figura \refElem{USR-IU02a}).
   \item \textbf{Objetivos y metas personales}: esta sección tiene como objetivo que el actor actualice sus objetivos, metas personales y laborales 
   (ver la figura \refElem{USR-IU02b}).
   \item \textbf{Historial académico}: esta sección tiene como objetivo que el actor actualice su información
   académica referente a todos los grados de estudios que tiene hasta la fecha (ver la figura \refElem{USR-IU02c}).
   \item \textbf{Idiomas}: esta sección tiene como objetivo preguntarle que el actor actualice la información de idiomas o en su caso, elimine
   o agregue nuevos idiomas a su perfil  (ver la figura \refElem{USR-IU02d}).
   \item \textbf{Experiencia laboral}: esta sección tiene como objetivo que el actor actualice su información de su experiencia
   laborar que tiene hasta la fecha (ver la figura \refElem{USR-IU02e}).
   \item \textbf{Cursos/Certificaciones}:  esta sección tiene como objetivo que el actor actualice su información de sus Certificaciones
   o cursos que ha tenido durante toda su trayectoria académica y laboral (ver la figura \refElem{USR-IU02f}).
\end{itemize}

\subsubsection{Comandos}
Los siguientes comandos aparecen durante toda la interfaz es decir, cada sección los tiene.

%\Titem \IUPass : Al da clic en el ícono, se muestra la contraseña de lo contrario aparecerá \IUOculto \thinspace sustituyendo cada caracter de la contraseña. \\

\Titem \IUEditar{} : Cuando presiona el ícono, habilita la sección para hacer los datos editables acorde a la sección o elemento seleccionado.
\Titem \IUEliminar{} : Cuando presiona el ícono, habilita la sección para eliminar el elemento seleccionado.
\Titem \IUAgregar{} : Cuando presiona el ícono, habilita la sección para agregar un nuevo el elemento.

\IUfig{.9}{CasosdeUso/USR-CU02/imagenes/USR-IU02.png}{USR-IU02}{Consultar perfil}  
\IUfig{.5}{CasosdeUso/USR-CU02/imagenes/USR-IU02a.png}{USR-IU02a}{Consultar perfil: Datos personales}
\IUfig{.9}{CasosdeUso/USR-CU02/imagenes/USR-IU02b.png}{USR-IU02b}{Consultar perfil: Objetivos y metas personales}  
\IUfig{.9}{CasosdeUso/USR-CU02/imagenes/USR-IU02c.png}{USR-IU02c}{Consultar perfil: Historial académico}  
\IUfig{.9}{CasosdeUso/USR-CU02/imagenes/USR-IU02d.png}{USR-IU02d}{Consultar perfil: Idiomas}
\IUfig{.9}{CasosdeUso/USR-CU02/imagenes/USR-IU02e.png}{USR-IU02e}{Consultar perfil: Experiencia laboral}  
\IUfig{.9}{CasosdeUso/USR-CU02/imagenes/USR-IU02f.png}{USR-IU02f}{Consultar perfil: Cursos/Certificaciones}  


\clearpage


Los casos de uso son una herramienta usada para representar las transacciones entre un actor y el sistema, las cuales siempre tendrán un valor agregado o un propósito para que el actor las realice.
En estos se representan a traves de diagramas los cuales se conforan por los siguientes elementos:

\begin{itemize}
	 \UCpaso Representa al sistema mediante un óvalo.
	\item \UCactor Representa al actor que va a interactuar con el sistema.
	\item Relación $<$- - -$<<extends>>$- - -. Indica que un caso de uso \textbf{puede} ejecutarse a partir de otro.
	\item Relación - - -$<<include>>$- - -$>$. Indica que un caso de uso \textbf{debe} ejecutarse a partir de otro.
\end{itemize}

La conexión entre un actor y un caso de uso es por medio de una línea como se muestra en la figura \ref{fig:acUC}.

\begin{figure}[hbtp!]
	\begin{center}
		\includegraphics[width=.4\textwidth]{LIT/ActorUC}
	\end{center}
	\label{fig:acUC}
	\caption{Interacción del actor con el caso de uso}
\end{figure}

Los casos de uso se encontrarán dentro de paquetes (representados por carpetas) indicando así que pertenecen a un mismo módulo como se muestra en la figura \ref{fig:pack}.

\begin{figure}[hbtp!]
	\begin{center}
		\includegraphics[width=.7\textwidth]{LIT/Paquete}
	\end{center}
	\label{fig:pack}
	\caption{Un Actor con varios caso de uso dentro de un módulo}
\end{figure}

\pagebreak

\subsection{Identificadores de casos de uso}

Los casos de uso se identificarán de acuerdo a la siguiente nomenclatura:

\begin{figure}[hbtp!]
	\begin{center}
		\includegraphics[width=.7\textwidth]{LIT/UCnombre}
	\end{center}
	\label{fig:nomenclatura}
\end{figure}

%\subsection{Modelado de Máquinas de Estados}
  %      \label{sec:ModMaquinas}

%\subsection{Modelado de Actores}
 %       \label{sec:ModActores}
%- - - - - - - - - - - - - - - - - - - - - - - - - - - - -
\subsection{Especificación de casos de uso}

Para poder entender un caso de uso más allá de un diagrama, se lleva a cabo la tarea de especificar cada uno de los casos de uso identificados en el sistema con el fin de describir las secuencias de acciones que realiza el sistema y que lleva a un resultado de valor a un actor específico. Los casos de uso tienen atributos los cuales se describen a continuación:

\begin{description}
	\item[Id] Identificador del caso de uso, el cual debe ser único.
	\item[Nombre] Nombre del caso de uso el cual es descriptivo basándose en la transacción que se realiza.
	\item[Resumen] Es una descripción resumida en la que se especifica la transacción realizada por el caso de uso.
	\item[Actores] Lista de los actores que interactúan con el caso de uso.
	\item[Entradas] Lista los datos de entrada que el caso de uso recibe, los cuales harán referencia al modelo de información.
	\item[Salidas] Lista los datos de salida que el caso de uso genera, por ejemplo: 
	\item[Destino] Indica a dónde se dirigen los datos de salida, por ejemplo: pantalla, impresora, repositorio, hacia un servidor o un archivo.
	\item[Precondiciones] Enlista las cosas que deben haber sucedido para que el caso  de uso se lleve a cabo.
	\item[Postcondiciones] Enlista las cosas que suceden en el sistema o negocio de forma inmediata o a corto plazo una vez que se ejecute el caso de uso.
	\item[Reglas de Negocio] Lista las reglas de negocio que se van a ejecutar en el caso de uso.
	\item[Viene de] Indica cuando el caso de uso se extiende de otro o se incluye en otro.
	\item[Trayectoria principal] Secuencia de pasos que llevan al caso de uso al éxito.
	\item[Trayectoria(s) alternativa(s)] Secuencias de pasos que llevan al caso de uso al éxito o al fracaso.
	\item[Puntos de extensión] Cuando existen casos de uso que pueden ejecutarse a partir del caso de uso en proceso.
\end{description}


%---------------------------------------------------------
\subsection{Modelado de casos de uso}

\subsubsection{Modelo de estados}

Diversas entidades en el sistema  que tienen un comportamiento dinámico en el sistema, 
los que se consideraron en el desarrollo del sistema y ameritaban ser modelados a través de una 
maquina de estados.\\

\subsubsection{Actores del sistema}
 Se definen los actores identificados como participantes en los procesos del sistema. 
 Estos actores son los encargados de llevar a cabo determinadas tareas dentro de cada proceso.\\
 
 \subsubsection{Reglas de negocio}

 Se van describir las reglas de negocio identificadas en el sistema las cuales son una condición que 
 se debe satisfacer cuando se realiza una actividad de negocio.\\
 
\subsubsection{Casos de uso}

Las funcionalidades del sistema son representadas mediante casos de uso. Un caso de uso es una transacción entre un actor y el sistema que tiene un valor agregado o un propósito para que el actor las realice. Los casos de uso se modelan a partir de dos elementos: un diagrama de casos de uso y la
descripción del caso de uso.\\

\subsubsection{Mensajes del sistema}

Los mensajes se refieren a todos aquellos avisos que el sistema muestra al actor para comunicar la ocurrencia de algún evento tal como un error o una operación exitosa. 


