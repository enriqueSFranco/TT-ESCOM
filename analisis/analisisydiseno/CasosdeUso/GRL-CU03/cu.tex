\begin{UseCase}[]{GRL-CU03}{Iniciar Sesión}{
	Permite al actor ingresar al sistema por medio de su cuenta de usuario para poder acceder a las
acciones que tiene permitidas dentro del sistema de acuerdo a su tipo de usuario.
}
	%----------------------------------------------------------------
	% Datos generales del CU:
	\UCsection{Atributos}
	\UCitem{Actor(es)}{
		\Titem Reglutador.
		\Titem Alumno.
		\Titem Encargado del sistema.
		\Titem Colaborador del sistema. 
	}
	\UCitem[admin]{Prioridad}{
		Alta
	}
	\UCitem[admin]{Complejidad}{
		Alta
	}
	\UCitem{Precondiciones}{
		El usuario debe de tener una cuenta dentro del sistema.
	}
	\UCitem{Postcondiciones}{
		
		
	}
	\UCitems{Entradas}{
		\imprimeUC{entradas}
	}
	\UCitem{Destino}{
		\Titem \refElem{GRL-IU03}
	}
	\UCitem{Reglas de Negocio}{
		\Titem \refIdElem{RN-N001}
		\Titem \refIdElem{RN-N002}
		\Titem \refIdElem{RN-N006}
		
	}
	\UCitem{Viene de}{
		Caso de uso primario.
	}	
\end{UseCase}

%Trayectoria Principal
\begin{UCtrayectoria}
	\UCpaso [\UCactor] Presiona el botón \IUbutton{Eres alumno?} desde la interfaz \refElem{GRL-IU06}.\refTray{A}
    \UCpaso [\UCsist] Muestra la interfaz \refElem{GRL-IU03}.
	\UCpaso [\UCactor] Ingresa el \entrada[correo electrónico]{users.username} y \entrada[contraseña]{users.password}.\label{cu03-grl1}\refTray{A}
	\UCpaso [\UCactor] Da clic en el botón \IUbutton{Iniciar Sesión}. \refTray{B} \refTray{C} 
	\UCpaso [\UCsist] Verifica que los campos marcados como obligatorios hayan sido ingresados de acuerdo a la
	regla de negocio \refIdElem{RN-N001}.\refTray{D}
	\UCpaso [\UCsist] Verifica que el correo electrónico ingresado cumpla con el formato correcto de acuerdo a la regla de negocio  \refIdElem{RN-N002}.  \refTray{E}
	\UCpaso [\UCsist] Verifica que la contraseña ingresada cumpla con el formato correcto de acuerdo a la regla de negocio  \refIdElem{RN-N006}.\refTray{E}
	\UCpaso [\UCsist] Verifica que los datos ingresados cumplan con la regla de negocio \refIdElem{RN-N007}. \refTray{F}
    \UCpaso [\UCsist] Muestra la interfaz \refElem{GRL-IU03-1}. %\refTray{G} \refTray{H}
\end{UCtrayectoria}

%Trayectorias Alternativas
\begin{UCtrayectoriaA}[Fin de la trayectoria]{A}{El actor presiona el botón \IUbutton{Eres reclutador?}.}
	\UCpaso [\UCsist] Muestra la interfaz \refElem{GRL-IU03}.
	\UCpaso [\UCsist] Continúa en el paso \ref{cu03-grl1} de la trayectoria principal.
\end{UCtrayectoriaA}

\begin{UCtrayectoriaA}[Fin de la trayectoria]{B}{El actor desea crear una cuenta y es alumno.}
	\UCpaso [\UCactor] Da clic en el botón \IUbutton{¿No tienes una cuenta?}
	\extendUC{GRL-CU01}.
\end{UCtrayectoriaA} 

\begin{UCtrayectoriaA}[Fin de la trayectoria]{C}{El actor desea crear una cuenta y es reclutador.}
	\UCpaso [\UCactor] Da clic en el botón \IUbutton{¿No tienes una cuenta?}
	\extendUC{GRL-CU02}.
\end{UCtrayectoriaA} 

\begin{UCtrayectoriaA}[Fin de la trayectoria]{D}{El actor no ingresó uno o más datos obligatorios.}
	\UCpaso [\UCsist] Muestra el mensaje \refIdElem{MSG4} en los campos que no
	fueron ingresados.
	\UCpaso [\UCsist] Continúa en el paso \ref{cu03-grl1} de la trayectoria principal.
\end{UCtrayectoriaA} 

\begin{UCtrayectoriaA}[Fin de la trayectoria]{E}{El actor ingresó el dato con formato incorrecto.}
	\UCpaso [\UCsist] Muestra el mensaje \refIdElem{MSG2}.
	\UCpaso [\UCsist] Continúa en el paso \ref{cu03-grl1} de la trayectoria principal.
\end{UCtrayectoriaA} 

\begin{UCtrayectoriaA}[Fin de la trayectoria]{F}{El actor ingresó un correo y contrasela validos.}
	\UCpaso [\UCsist] Muestra el mensaje \refIdElem{MSG5}.
	\UCpaso [\UCsist] Continúa en el paso \ref{cu03-grl1} de la trayectoria principal.
\end{UCtrayectoriaA} 




\subsubsection{Puntos de extensión}

\UCExtensionPoint{Crear cuenta}{El actor requiere registrarse en el sistema}{En el paso  de la trayectoria alternativa B}{\refIdElem{GRL-CU01}}
\UCExtensionPoint{Enviar Pre-Registro}{El actor requiere registrarse en el sistema}{En el paso  de la trayectoria alternativa C}{\refIdElem{GRL-CU02}}
