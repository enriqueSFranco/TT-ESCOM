\begin{UseCase}{GRL-CU01}{Crear Cuenta.}{
	Permite al actor registrarse como en el sistema y así gestionar su perfil y aplicar a las vacantes
	de su elección.
}
	%----------------------------------------------------------------
	% Datos generales del CU:
	\UCsection{Atributos}
	\UCitem{Actor(es)}{
		Alumno. 
	}
	\UCitem{Prioridad}{
		Alta
	}
	\UCitem{Complejidad}{
		Alta
	}
	\UCitem{Precondiciones}{
		El correo electrónico del usuario no debe estar registrado en el sistema.
	}
	\UCitem{Postcondiciones}{
		EL actor pordra gestionar su perfil y aplicar a las vacantes publicadas en el sistema.
	}
	\UCitem{Entradas}{
		\imprimeUC{entrada}
	}
	  
	\UCitem{Destino}{
		\refIdElem{GRL-IU01}
		
	}
	\UCitem{Reglas de Negocio}{
		\Titem \refElem{RN-N001}
		\Titem \refElem{RN-N002}
		\Titem \refElem{RN-N003}
		\Titem \refElem{RN-N004}
	}

	\UCitem{Viene de}{
		%Caso de uso primario.
	}	
\end{UseCase}

%Trayectoria Principal
\begin{UCtrayectoria}
	\UCpaso [\UCactor] Presiona el botón \textbf{Crea tu cuenta} desde la pantalla \refElem{GRL-IU00}.
    \UCpaso [\UCsist] Muestra la pantalla \refElem{GRL-IU01}.
	\UCpaso [\UCactor] Ingresa el \entrada[correo electrónico]{users.username} y \entrada[contraseña]{users.password} del usuario en los compos correspondientes.\label{cu01-grl}
	\UCpaso [\UCactor] Acepta los terminos de privacidad y condiciones del sistema.
	\UCpaso [\UCactor] Da clic en el botón \IUbutton{Crear Cuenta}. \refTray{A}
    \UCpaso [\UCsist] Verifica que los campos marcados como obligatorios hayan sido ingresados de acuerdo a la
	regla de negocio \refIdElem{RN-N001}.\refTray{B}
	\UCpaso [\UCsist] Verifica que el correo electrónico ingresado cumpla con el formato correcto de acuerdo a las reglas de negocio  \refIdElem{RN-N002} y \refIdElem{RN-N003} .\refTray{C}
	\UCpaso [\UCsist] Verifica que el correo electrónico ingresado no coincida con otro registrado en el sistema deacuedo a la regla de negocio  \refIdElem{RN-N004} Unicidad de datos.\refTray{D}
	\UCpaso [\UCsist] Verifica que la contraseña cumpla con el formato correcto de acuerdo a la reglas de negocio \refIdElem{RN-N002} y \refIdElem{RN-N003} .\refTray{C}
	\UCpaso [\UCsist] Asigna los permisos correspondientes de un usuario tipo Alumno a la cuenta creada.
    \UCpaso [\UCsist] Muestra la pantalla del Home del Alumno.
\end{UCtrayectoria}

%Trayectorias Alternativas
\begin{UCtrayectoriaA}[Fin del caso de uso]{A}{El actor decide cancelar el registro de un usuario.}
	\UCpaso [\UCactor] Da clic en el botón \IUbutton{Cancelar}.
	\UCpaso [\UCsist] Muestra la pantalla \refElem{GRL-IU00}.
\end{UCtrayectoriaA} 

%Trayectorias Alternativas
\begin{UCtrayectoriaA}[Fin del caso de uso]{B}{El actor no ingresó uno o más datos obligatorios.}
	\UCpaso [\UCsist] Muestra el mensaje \refIdElem{MSG1} en los campos que no
	fueron ingresados.
	\UCpaso [\UCsist] Continúa en el paso \ref{cu01-grl} de la trayectoria principal.
\end{UCtrayectoriaA} 

%Trayectorias Alternativas
\begin{UCtrayectoriaA}[Fin del caso de uso]{C}{El actor ingresó el dato con formato incorrecto.}
	\UCpaso [\UCsist] Muestra el mensaje \refIdElem{MSG2}.
	\UCpaso [\UCsist] Continúa en el paso \ref{cu01-grl} de la trayectoria principal.
\end{UCtrayectoriaA} 

%Trayectorias Alternativas
\begin{UCtrayectoriaA}[Fin del caso de uso]{D}{El actor ingresó uun correo electrónico ya registrado en el sistema.}
	\UCpaso [\UCsist] Muestra el mensaje \refIdElem{MSG3}.
	\UCpaso [\UCsist] Continúa en el paso \ref{cu01-grl} de la trayectoria principal.
\end{UCtrayectoriaA} 
