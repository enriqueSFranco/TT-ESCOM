\section{Requerimientos del sistema}
Esta sección describe todos los requerimientos necesarios para el funcionamiento del sistema con base en las responsabilidades de cada
usuarios, descritas en la sección anterior.
Los requerimientos utilizan una clave para ser identificados:

\begin{itemize}
	\item Para los requerimientos \textbf{fucionales} se utliza el formato \textbf{RF-X}, donde:
    \begin{itemize}
        \item \textbf{RF} Es la clave para todos los requerimientos funcionales.
        \item \textbf{X} Es un número consecutivo: 01, 02, 03, ...
    \end{itemize}

    \item Para los requerimientos \textbf{no fucionales} se utliza el formato \textbf{RNF-X}, donde:
    \begin{itemize}
        \item \textbf{RNF} Es la clave para todos los requerimientos funcionales.
        \item \textbf{X} Es un número consecutivo: 01, 02, 03, ...
    \end{itemize}

\end{itemize}

	\begin{requerimientos}{funcionales}
		\RFitem{RF-01}{Iniciar sesión }{El sistema debe proporcionar un mecanismo que permita al actor acceder al sistema mediante un correo y una contraseña.}
        \RFitem{RF-02}{Crear cuenta}{El sistema debe proporcionar un mecanismo que permita al actor registrarse como nuevo usuario en el sistema.}
        \RFitem{RF-03}{Recuperar contraseña}{El sistema debe proporcionar un mecanismo que permita al actor recuperar su contraseña en caso de haberla olvidado.}
        \RFitem{RF-04}{Consultar perfil}{El sistema debe permitir al actor acceder a su perfil para consultar la información registrada.}
        \RFitem{RF-05}{Editar perfil}{El sistema debe permitir al actor modificar la información registrada en su perfil, como nombre, apellidos, información de contacto y si el usuario es un candidato, debe de permitir modificar su información académica y datos de habilidades y conocimientos del mismo.}
        \RFitem{RF-06}{Cambiar contraseña}{El sistema debe permitir al actor modificar su contraseña actual por una nueva en el momento que él lo requiera.}
        \RFitem{RF-07}{Eliminar cuenta}{El sistema debe proporcionar un mecanismo que permita eliminar la cuenta de un actor, ya sea que el propio actor la elimine o si en dado caso es el encargado o colaborador la pueda eliminar.}
        \RFitem{RF-08}{Recuperar cuenta}{El sistema debe proporcionar  un mecanismo que le permita al actor recuperar su cuenta que anteriormente haya eliminado.}
        \RFitem{RF-09}{Consultar vacantes}{El sistema debe permitir a cualquier persona consultar las vacantes que se tengan registradas siempre y cuando aun estén abiertas.}
        \RFitem{RF-10}{Publicar vacantes }{El sistema debe permitir  a los usuarios registrados publicar vacantes.}
        \RFitem{RF-11}{Editar vacantes}{El sistema debe permitir a los usuarios editar las vacantes que ellos hayan publicado.}
        \RFitem{RF-12}{Reportar vacantes}{El sistema debe permitir a los usuarios reportar vacantes publicadas si es que el usuario lo crea necesario.}
        \RFitem{RF-13}{Eliminar vacantes}{El sistema debe permitir a los usuarios eliminar las vacantes que ellos hayan publicado.}
        \RFitem{RF-14}{Postularse a vacantes}{El sistema debe permitir a los usuarios registrados enviar postularse a  las vacantes que esté abiertas.}
        \RFitem{RF-15}{Consultar el estado de postulaciones}{El sistema debe permitir al actor consultar el estado de su o sus postulaciones que haya hecho.}
        \RFitem{RF-16}{Dar seguimiento a postulaciones}{El sistema debe permitir al actor indicar  el estado de las postulaciones que tiene según sus vacantes publicadas.}
        \RFitem{RF-17}{Consultar postulaciones}{El sistema debe permitir a los actores registrados consultar las postulaciones que se hayan hecho de acuerdo a cada vacante.}
        \RFitem{RF-18}{Consultar candidatos}{El sistema debe permitir a los actores registrados consultar los candidatos registrados hayan o no postulado a una vacante.}
        \RFitem{RF-19}{Publicar comunicados}{El sistema debe permitir publicar comunicados a los actores que tengan cuenta. }
        \RFitem{RF-20}{Consultar comunicados}{El sistema debe permitir consultar todos los comunicados que se tengan registrados y que aún están vigentes}
        \RFitem{RF-21}{Modificar comunicados}{El sistema debe permitir editar todos los comunicados que se tengan registrados y que aún están vigentes}
        \RFitem{RF-22}{Eliminar comunicados}{El sistema debe permitir eliminar todos los comunicados que se tengan registrados.}
        \RFitem{RF-23}{Editar información de la empresa}{El sistema debe proporcionar un mecanismo para editar la información general de una empres: misión, visión, objetivos, nombre, RFC  y razón social}
        \RFitem{RF-24}{Enviar solicitud para acceder al sistema}{El sistema debe proporcionar un mecanismo para que al representante de una empresa pueda enviar una solicitud o pre-registro de poder acceder y publicar vacantes.}
        \RFitem{RF-25}{Consultar empresas }{El sistema debe proporcionar un mecanismo para consultar la información general de una empres: misión, visión, objetivos, nombre, RFC  y razón social}
        \RFitem{RF-26}{Validar empresas}{El sistema debe proporcionar un mecanismo para validar la información general de una empresa y así corroborar que sean empresas constituidas}
        \RFitem{RF-27}{Validar reclutadores}{El sistema debe proporcionar un mecanismo para validar la información de un reclutador y así corroborar que trabajen para la empresa que representan.}
        \RFitem{RF-28}{Generar reportes de estado}{El sistema debe proporcionar un mecanismo generar reportes sobre vacantes, cuentas, empresas, candidatos, reclutadores, postulaciones y uso de la plataforma.}
        \RFitem{RF-29}{Gestionar vacantes reportadas}{El sistema debe proporcionar un mecanismo para dar solución a los casos de vacantes reportadas por los propios usuarios}
        \RFitem{RF-30}{Gestionar cuentas de usuarios}{El sistema debe proporcionar un mecanismo para que los colaboradores y el encargado puedan gestionar las cuentas de otros usuario es decir, eliminarlas, crearlas y recuperarlas si así lo requiere.}
        \RFitem{RF-31}{Agregar colaboradores}{El sistema debe proporcionar un mecanismo para validar la información de un reclutador y así corroborar que trabajen para la empresa que representan.}
        \RFitem{RF-32}{Recomendar candidatos}{El sistema debe proporcionar un mecanismo para recomendar y filtrar candidatos a los reclutadores, con base en las características y requerimientos de las vacantes publicadas por el propio reclutador  }
        \RFitem{RF-33}{Recomendar vacantes}{El sistema debe proporcionar un mecanismo para recomendar y filtrar vacantes a los candidatos, con base en las características, habilidades, conocimientos y preferencias del candidato.}
        \RFitem{RF-34}{Generar el pdf de perfil }{El sistema debe proporcionar un mecanismo para generar un curriculum en pdf con la información registrada del perfil de un candidato}
        \RFitem{RF-35}{Enviar notificaciones de estado}{El sistema debe proporcionar un mecanismo para enviar notificaciones a los usuarios sobre los estados de sus vacantes, publicaciones, reportes y actividad en general de su cuenta.}
        \RFitem{RF-36}{Enviar correos de confirmación de cuenta}{El sistema debe proporcionar un mecanismo para confirmar por medio de correos el envío de credenciales para las cuentas de reclutadores, recuperar contraseñas y recuperar cuentas}
        \RFitem{RF-37}{Guardar historial de cuentas}{El sistema debe proporcionar un mecanismo para guardar todos los registros de las cuentas creadas, eliminadas y recuperadas de los últimos 5 años}
        \RFitem{RF-38}{Guardar historial de uso }{El sistema debe proporcionar un mecanismo para guardar todos los registros de las vacantes, empresas, postulaciones y reportes de los últimos 5 años}
        \RFitem{RF-39}{Descargar archivos }{El sistema debe proporcionar un mecanismo para descargar y subir archivos según lo requieran los actores}
        \RFitem{RF-40}{Crear credenciales de acceso}{El sistema debe proporcionar un mecanismo para crear usuarios y contraseñas de nuevas cuentas}
	\end{requerimientos}

