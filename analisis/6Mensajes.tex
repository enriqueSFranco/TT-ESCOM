\chapter{Mensajes del sistema}
\label{ch:mensajes}
\cdtLabel{apendice:Mensajes}{}

%En esta sección se describen los mensajes utilizados en el prototipo actual del sistema.
Los mensajes se refieren a todos aquellos avisos que el sistema muestra al actor para comunicar la ocurrencia de algún evento tal como un error o una operación exitosa. Estos mensajes se pueden mostrar a través de diversos canales, por ejemplo, pantalla o correo electrónico.

Cuando un mensaje es recurrente se parametrizan sus elementos, por ejemplo los mensajes: ``Aún no existen registros de \textit{Unidades Académicas} en el sistema.'', ``Aún no existen registros de \textit{Programas Académicos} en el sistema.'', 
``Aún no existen registros de \textit{alumnos} en el sistema.'', tienen una estructura similar 
por lo que, con el objetivo de que el mensaje sea genérico y pueda utilizarse en todos los casos de uso que se considere necesario, se utilizan parámetros para definir el mensaje.\\

Los parámetros también se utilizan cuando la redacción del mensaje tiene datos que son ingresados por el actor o que dependen del resultado de la operación, por ejemplo: 
``La \textit{Unidad Académica ESCOM} ha sido \textit{modificada} exitosamente.''. En este caso la redacción se presenta parametrizada de la forma: ``$<DETERMINADO> <ENTIDAD> <VALOR>$ ha sido $<OPERACION>$ exitosamente.'' y los parámetros se describen de la siguiente forma:

\begin{itemize}
	\item DETERMINADO ENTIDAD: Es un artículo determinado más el nombre de la entidad sobre la cual se realizó la acción.
	\item VALOR: Es el valor asignado al atributo de la entidad, generalmente es el nombre o la clave.
	\item OPERACIÓN: Es la acción que el actor solicitó realizar.
\end{itemize}

En el ejemplo anterior se hace referencia a $<VALOR>$, es decir: \textit{ESCOM} es el \textbf{valor} de la \textbf{entidad} \textit{escuela}. Cada mensaje enlista los parámetros  que utiliza, sin embargo aquí se definen los más comunes a fin de simplificar la descripción de los mensajes:

\begin{description}
	\item [$<ARTICULO>$:] Se refiere a un {\em artículo} el cual puede ser DETERMINADO (El $\mid$ La $\mid$ Lo $\mid$ Los $\mid$ Las) o INDETERMINADO (Un $\mid$ Una $\mid$ 
	Uno $\mid$ Unos $\mid$Unas) se aplica generalmente sobre una ENTIDAD, ATRIBUTO o VALOR.
	
	\item [$<CAMPO>$:] Se refiere a un campo del formulario. Por lo regular es el nombre de un atributo en una entidad.
	
	\item [$<CONDICION>$:] Define una expresión booleana cuyo resultado deriva en {\em falso} o {\em verdadero} y suele ser la causa del mensaje.
	
	\item [$<DATO>$:] Es un sustantivo y generalmente se refiere a un atributo de una entidad descrito en el modelo estructural del negocio, por ejemplo: nombre de la unidad de aprendizaje, nombre del alumno, RFC del profesor, etc. %ATRIBUTO
	
	\item [$<ENTIDAD>$:] Es un sustantivo y generalmente se refiere a una entidad del modelo estructural del negocio, por ejemplo: programa académico, alumno, profesor, etc.
	\item [$<OPERACION>$:] Se refiere a una acción que se debe realizar sobre los datos de una o varias entidades. Por ejemplo: registrar, eliminar, actualizar, por mencionar algunos. Comúnmente la OPERACIÓN va concatenada con el sustantivo, por ejemplo: Registro de un nuevo beneficio, registro de una actividad, eliminar una tarea y demás.
	
	\item [$<VALOR>$:] Es un sustantivo concreto y generalmente se refiere a un valor en específico. Por ejemplo: \textit{Histología I} es un \textbf{valor} concreto de la \textbf{entidad} \textit{Unidad de Aprendizaje}.
	
\end{description}
%___________________________Plantilla_______________________________
%===========================  MSGX ==================================
%\begin{mensaje}{MSGX}{}{}
%	\item[Canal:] 
%	\item[Propósito:] 
%	\item[Redacción:]
%	\item[Parámetros:] 
%	\begin{itemize}
%		\item 
%	\end{itemize}
%	\item[Ejemplo:]  
%	\item[Referenciado por: ]
%\end{mensaje}

%===========================  MSG1 ==================================
\begin{mensaje}{MSG1}{Operación Exitosa}{Informativo }
	\item[Canal:] Sistema
	\item[Propósito:] Notificar al actor que la operación solicitada al sistema se llevó a cabo exitosamente.
	\item[Redacción:] La $<OPERACION>$ se llevo a cabo correctamente.
	\item[Parámetros:] OPERACION: Actividad que el actor debe realizar.
	\item[Ejemplo:] La reincripción se llevo a cabo correctamente.
\end{mensaje}

%===========================  MSG12 ==================================
\begin{mensaje}{MSG12}{Confirmar operación sin cambios posteriores}{Confirmación}
	\item[Canal:] Sistema
	\item[Propósito:] Solicitar al actor la confirmación de una operación de impacto que no podrá ser revertida.
	\item[Redacción:] ¿Desea confirmar la finalización $<OPERACION>$? La operación no podrá ser revertida una vez confirmada.
	Calmécac te notifica que es necesaria tú aprobación para concluir la $<OPERACION>$
	\item[Parámetros:] OPERACION: Aquella operación que el actor debe confirmar.
	\item[Ejemplo:] ¿Desea confirmar la finalización del registro de aspirantes? La operación no podrá ser revertida una vez confirmada.
	Calmécac te notifica que es necesario tú aprobación para concluir el registro de aspirantes
	
\end{mensaje}

%===========================  MSG15 ==================================
\begin{mensaje}{MSG15}{No	 existe información en el sistema}{Error}
	\item[Canal:] Sistema
	\item[Propósito:] Notificar al actor que no se puede llevar a cabo la operación solicitada pues no existe la información suficiente en el sistema.
	\item[Redacción:] !Error! Falta información necesaria sobre $<INFORMACION>$ en el sistema para poder realizar la/el $<OPERACION>$ \\
	\item[Parámetros:] 
	\begin{Titemize}
		\Titem INFORMACIÓN: El tipo de información que hace falta en el sistema.
		\Titem OPERACION: Tarea a realizar por el actor
	\end{Titemize}
	\item[Ejemplo:] ¡Error! Falta información necesaria sobre Modalidad Educativa en el sistema para poder realizar el registro de Unidades de Aprendizaje
	\refIdElem{HR-UA-CU4}, \refIdElem{HR-UA-GG-CU1}
\end{mensaje}

%===========================  MSG17 ==================================
\begin{mensaje}{MSG17}{Eliminar elemento}{Confirmación}
	\item[Canal:] Sistema
	\item[Propósito:] Solicitar la confirmación del actor para la eliminación de un elemento.
	\item[Redacción:] ¿Desea eliminar $<ARTICULO>$ + $<ELEMENTO>$ + $<SELECCION>$?
	\item[Parámetros:] 
	\begin{itemize}
		\item ARTICULO: Es la parte de la oración que se ocupa de expresar el género (masculino/femenino).
		\item ELEMENTO: Es el elemento que se requiere eliminar.
		\item SELECCION:Dependiendo del género en la oración se ocupa, seleccionado(masculino)  ó seleccionada(femenino).
	\end{itemize}
	\item[Ejemplo:] ¿Desea eliminar el ciclo escolar seleccionado?
	
\end{mensaje}

%===========================  MSG58 ==================================
\begin{mensaje}{MSG58}{Operación fuera de periodo}{Error}
	\item[Canal:] Sistema
	\item[Propósito:] Notificar al actor que la operación que desea realizar está fuera del periodo permitido
	\item[Redacción:] No es posible realizar $<ARTICULO\_OPERACION>$ debido al que la fecha actual está fuera del periodo permitido para realizarla.
	\item[Parámetros:] 
	\begin{itemize}
		\item ARTICULO\_OPERACION: La operación que el actor desea llevar a cabo.
	\end{itemize}
	\item[Ejemplo:] No es posible realizar el envío de la solicitud debido a que la fecha actual está fuera del periodo permitido para realizarla.
	\refIdElem{HR-PR-CU1}
\end{mensaje}


