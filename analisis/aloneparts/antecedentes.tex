\selectfont

En este capítulo se explican las diferentes formas de buscar empleo en diferentes escenarios, se 
exploran las diferentes plataformas que se ayudan a difundir la información de las vacantes de las empresas.

\section{Marco teorico}
    El ``7to. Estudio de Búsqueda de Empleo por Internet en México 2020'' por la Asociación de Internet MX (AIMX)
    organismo líder en temas de comunicación y tecnologías digitales, en esta edición se enfocaron en cómo la
    contingencia sanitaria afectó los hábitos tanto de los internautas en su búsqueda de empleo, así como en la manera en
    que las empresas realizaron sus procesos de reclutamiento en internet.\\
    \newline
    La transformación digital y la necesidad de inmediatez ha impulsado de manera significativa el uso de los smartphones
    en la búsqueda de empleo en línea con un 45\% de preferencia, las computadoras portátiles son el segundo dispositivo
    más utilizado para este fin (34\%) y con 49\% el primero utilizado para el reclutamiento de personal. Las bolsas de
    trabajo en línea siguen siendo el medio más utilizado tanto para buscar trabajo con el 83\% de preferencia; en segundo
    lugar, para los candidatos destaca el uso de las redes sociales, con el 54\% de preferencia y las bolsas de trabajo
    universitarias con un 42\%. 
    El Top of mind es aquella marca que ocupa un lugar privilegiado en la mente del público, OCCMundial se mantiene
    en el Top of Mind con una preferencia del 72\% por parte de candidatos y 62\% por parte de las empresas, mientras que
    Facebook cobra alta relevancia para las empresas en sus procesos de reclutamiento con el 63\% de preferencia.
    Aunque los internautas y las empresas también recurren a las redes sociales para buscar trabajo, las bolsas de empleo
    siguen siendo la punta de lanza en este tipo de servicios. Las 5 bolsas de trabajo más utilizadas para buscar empleo en
    2020 fueron:
    \begin{enumerate}
        \item OCCMundial 92\%
        \item CompuTrabajo 56\%
        \item Indeed 50\%
        \item LinkedIn 44\%
        \item Bumeran 15\%
    \end{enumerate}
    \newline

    El estudio indica que la mayoría de los internautas que buscan empleo son de licenciatura con una edad entre 18 y 37
    años, 5 de cada 10 personas se encuentran sin empleo, entre ellos están aquellos egresados o que están en sus últimos
    semestres que buscan incursionar en mundo laboral opciones.
    \newline

    Muchas universidades públicas y privadas en México cuentan con bolsa de trabajo para su comunidad académica y
    hoy en día universidades como la UAM, ANAHUAC y el IPN tienen su bolsa de trabajo virtual, aunque no todas se
    explotan al máximo o no tienen la difusión correcta. La más clara desventaja para los que buscan empleo y son
    1estudiantes es que hay poca oferta de vacantes con poca o nula experiencia. Mientras que para los reclutadores es que
    el servicio es de paga y conforme a lo pagado es el alcance de la vacante y el tiempo disponible para usar las
    herramientas de la plataforma. Dado que los estudiantes de ESCOM en su mayoría no cuentan con experticia laboral
    es más complicado encontrar una vacante donde sus habilidades se desarrollen. 

    ESCOM cuenta con una página en Facebook donde se publican los boletines de vacantes disponibles que le llegan al
    plantel, se reciben vacantes que pertenecen tanto del sector público como del privado, el departamento de Extensión
    y Apoyos Educativos es el encargado de filtrar la información, concentrarla y publicarla en la página de Facebook
    oficial.
    La información que se publica en ocasiones no está completa y se limita a informar sobre los conocimientos necesarios
    para la vacante y un contacto con la empresa dejando ciertas inquietudes como: saber el formato con el que deben ser
    enviados los datos del aspirante, si se debe dirigir con alguien en específico o si la vacante aún está disponible, por
    mencionar algunas.
    Por otro lado, el proceso de selección en ocasiones es problemático para el reclutador, el número de solicitudes que
    recibe excede de 50 para una sola vacante lo cual implica revisar y descargar cada solicitud una por una y si no se
    tiene un buen control puede que se traspapelen solicitudes y aunque el candidato envíe la información necesaria y en
    el formato necesario para ser considerado, no se realice el seguimiento a su solicitud.

\section{Estado del Arte}
Todas las plataformas de bolsa de trabajo tienen como objetivo publicar y difundir las vacantes de sus empresas participantes
y almismo tiempo dan a conocer el perfil de los candidatos a las mismas empresas.
\newline
A continuación se listas las bolsas de trabajo más populares entre las personas, asi como  la bolsa de trabajo
oficial del Instituto:
\begin{itemize}
    \item OCC Mundial:
    La OCC Mundial es una de las plataformas de busqueda de empleo más conocidas y usadas en México. La variedad de empleos y la variedad de empresas y empleadores es un gran atractivo tanto para los reclutadores y los candidatos para la gestión de contrataciones. Aunque es una buena opción para realizar contrataciones su uso para reclutdores es de paga mediante contratación de planes de pago de acuerdo al impacto y puestos requeridos. Razón por la que en algunas situaciones es una solución muy sobrada para empresas de menor tamaño.
    
    \item CompuTrabajo:
    CompuTranbajo es una plataforma de busqueda de empleo bastante utilizada en Latinoamerica (Como lo menciona en su página de incio) donde se pueden publicar vacantes para buscar y publicar ofertas de empleo. Al igual que indeed se pueden publicar vacantes de manera gratuita y potenciar las visitas de la vacantes. El precio base de la entrada premium para los reclutadores es accessible pero la plataforma insta a los reclutadores a invertir más dinero para mejorar el alcance de sus publicaciones.
    
    \item Indeed:
    Indeed a la par de OCC Mundial Indeed es una plataforma de busqueda de empleo muy conocida en México, la variedad de empresas y sectores de trabajo disponible además que pse pueden encontrar empleos en el extranjero a traves de esta plataforma, ya que el idioma predominante dentro de ella es el inglés. Al igual que otras plataformas de busqueda de empleo esta es de paga pero tambien permite realizar publicaciones gratuitas con el inconveniente de que el alcance se vera afectado porque las vacantes ofrecidad por usuarios de paga seran presentadas primero a los candidatos y además solo se permitira abrir un puesto para la vacante por cada oferta publicada gratis. El sistema de gestión de la vacantes se basa en la compra de tokens para potenciar el alcance, remarcar la urgencia de la contratación y ofertar más de un puesto por vacante.
    
    \item SIBOLTRA:
    Es la opción ofrecida por el Instituto Politecnico Nacional para el uso exclusivo de su comunidad, dado que para poder registrarse en ella es necesario contar con un número de boleta valido. Al hacer de uso de la comunidad del IPN la plataforma cuenta con la función de recuperar la trayectoria academica del alumno en cuestión asi como la información personal que se tenga registrada en las bases de datos de IPN, con ello los reclutadores pueden hacer una búsqueda de candidatos con información actualizada que se adapten a sus necesidades.
    
    Aunque estas funciones son atractivas las pruebas de uso que se han hecho en la plataforma nos muestran que no siempre funciona la sincronización de datos con la plataforma, la navegación dentro de la plataforma es poco intuitiva y además la difusión de la existencia de la plataforma no es mucha por lo que la comunidad del IPN casi no sabe de su exitencia.
    
    \item ESCOMobile
    ESCOMobile fue un proyecto de TT realizado por estudiantes de ESCOM con el proposito de crear una aplicación movíl para los alumnos de ESCOM donde se concentrara información útil de la Institución, avisos publicados por la unidad académica, el estado académico del usuario y la implementación de la bolsa de trabajo de ESCOM. El proyecto concluyo  el módulo de mostrar el estado academico del usuario y avisos generales publicados por los encargados pero el módulo de bolsa de trabajo no se completo a la entrega del proyecto, el avance del módulo fue la actualización del boletín de ofertas dentro de la pltaforma, resultando en una impelementación casi igual a el de la página de Facebook "Bolsa de trabajo ESCOM" que ya se maneja.
    
\end{itemize}

\newpage
\begin{longtable}{| p{0.13\textwidth}  | p{0.13\textwidth} | p{0.15\textwidth}  | p{0.14\textwidth}  | p{0.13\textwidth}  | p{0.13\textwidth}  | }

\label{table:herramientasSimilares}
    \rowcolor{black}
%    \multicolumn{7}{ |c| }{\bf\cellcolor{black}\color{white}{Herramientas para la especificación de requerimientos}} \\ \hline
    \bf\color{white} Herramienta & \bf \color{white}OCCMundial	& \bf \color{white}CompuTrabajo &  \bf \color{white}Indeed & 
    \bf \color{white}SIBOLTRA(IPN) & \bf \color{white}ESCOMobile \\ \hline
\endhead
Rango de precio por publicación de una vacante (MXN) &\$1,448.84 \$2,535.76 & Gratis \$775.00  &  Gratis/Patrocinada &  Gratis  & No aplica  \\ \hline
Duración de la vacante publicada &30 días &60 días, depende de la inversión.&Depende de la inversión. &No aplica.&No aplica. \\ \hline
Contacto directo reclutador-candidato dentro de la plataforma &  Si & No  & Si  & No  & No \\ \hline
Filtrado automático de currículos recibidos & Si &  Si 
&  No&  No & No \\ \hline
Autogenerado de CV & Si & Si &Si  &  No &Si \\ \hline
Seguimiento del estado de solicitud &Si  & Si & No & No  & Si\\ \hline
\caption{Comparación Plataformas para buscar empleo}
\end{longtable}