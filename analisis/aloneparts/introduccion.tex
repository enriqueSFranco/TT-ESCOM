%!TEX root = ../example.tex
\selectfont
	El presente documento contiene la especificación del reporte técnico del sistema ``Nombre del TT''
	correspondiente a la unidad de Aprendizaje: ``Trabajo Terminal I''.
	En este capítulo se resume su contenido asi como la estructura, con el fin de introducir al lector en la tema con el fin de
	tener una mejor comprensión.

	\section{Descripción del contexto}
		Internet ha alterado de forma radical los aspectos que configuran la búsqueda de empleo y de selección de personal, consolidándose 
		como uno de los medios más utilizados por quienes buscan nuevas oportunidades profesionales. Un estudio realizado por la Asociación 
		de Internet MX indica que la mayoría de los internautas que buscan empleo son de licenciatura con una edad entre 18 y 37 años [1], 
		es decir, egresados o aquellos que están en sus últimos semestres. 
		Algunas universidades públicas y privadas en México ofrecen apoyo a las empresas implementando bolsas de trabajo exclusivas para sus 
		comunidades estudiantiles. En ESCOM se implementó un boletín de ofertas de trabajo en la red social Facebook, crearón una pagina 
		llamada  ``Bolsa de Trabajo ESCOM''. 

		El problema que pretende atacar este proyecto es la falta de una plataforma que facilite la gestión de la Bolsa de Trabajo rapida
		y eficazmente. 
		La Bolsa de Trabajo ESCOM no es una plataforma en sí, el enfoque que tiene no es el apropiado para facilitar las búsquedas 
		de vacantes a los interesados, ni la gestión de ofertas para encargados de la Bolsa de Trabajo ni la selección que tienen que 
		hacer las empresas con las solicitudes que llegan a sus correos electrónicos. 



    \section{Estructura del documento}

       
	El presente documento se encuentra organizado en ``AGREGAR NUMERO DE PARTES'' partes, cada una dividida a su vez en capítulos, 
	como se muestra a continuación. \\

	%El capítulo \ref{ch:nomenclatura} presenta la nomenclatura utilizada a lo largo del documento.

	\begin{itemize}
		
		\item Capítulo \ref{aloneparts:antecedentes} Antecedentes:
		se definen todos aquellos conceptos  relevantes e importantes para una mejor compresión del contenido. Se presenta una 
		investigación en la que se describen las nuevas formas de bucar emplo y como han ido adaptando a las nuevas situaciones de hoy
		en día. Para finalizar se define la problemática que se espera resolver con el desarrollo de este proyecto así como su 
		posible solución y el estado del arte documental que permite el estudio de otras plataformas similares a proyecto.
		
		
		\item Parte I Analisis y Diseño: 
		se explican los avances realizados en el análisis y diseño proyecto hasta la fecha, se detalla todo el análisis y los 
		requerimientos de proyecto en los siguientes capítulos.

			\begin{itemize}
				\item Capítulo \ref{aloneparts:propuesta} Solución Propuesta: 
				se definen los objetivos que se han estimado alcanzar al concluir el ``Trabajo Termina I'',se va a presenta un diagrama
				del proceso que sigue el proyecto y la arquitectura propuesta del sistema.

				\item Capítulo  Modelo de negocio: 

				\item Capítulo Modelo de casos de uso: 

				\item Capítulo  Modelo de interacción: 

				\item Capítulo Avances en Trabajo Termina I: 
			\end{itemize}

		

		\item Parte  Anexos: 
			\begin{itemize}
				\item Capítulo Modelo de datos: 
			\end{itemize}
	\end{itemize}


\section{Notación, símbolos y convenciones utilizadas}

Los requerimientos funcionales utilizan una clave RFX, donde:
\begin{itemize}
	\item[X] Es un número consecutivo: 1, 2, 3, ...
	\item[RF] Es la clave para todos los {\bf R}equerimientos {\bf F}uncionales.\\
\end{itemize}

    Los requerimientos {\bf no} funcionales utilizan una clave RNFX, donde:
\begin{itemize}
	\item[X] Es un número consecutivo: 1, 2, 3, ...
	\item[RNF] Es la clave para todos los {\bf R}equerimientos {\bf N}o {\bf F}uncionales.\\
\end{itemize}

	Los requerimientos del usuario utilizan una clave RU-ZZ-X, donde:
	
\begin{itemize}
	\item[X] Es un número consecutivo: 1, 2, 3, ...
	\item[ZZ] Es la clave para los nombres de los encargados o responsables de las plantas que componen el sistema.
	\item[RU] Es la clave para todos los {\bf } Requerimientos del {\bf U}suario.\\
\end{itemize}