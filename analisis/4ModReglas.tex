%!TEX root = ../main-ejemplo.tex
\chapter{Modelo de Reglas de Negocio}
\label{ch:modReglas}


%{BR-100}{Estudiante Activo}
%======================================================================
\begin{BusinessRule}[%
	Autor/Ulises Vélez Saldaña,%
	Version/2.1,%
	Estado/En revisión%
]{BR-100}{Estudiante Activo}%
	\BRitem[control]{Revisada por}{Pendiente por asignar.}
    \BRitem[control]{Aprobada por}{Pendiente por asignar.}
    \BRsection[control]{Atributos}
    \BRitem[admin]{Clase}{\bcDerivation}% Clase: \bcCondition, \bcIntegridad, \bcAutorization, \bcDerivation.
    \BRitem[admin]{Nivel}{\blControlling}%  Nivel: \blControlling, \blInfluencing.
	\BRitem{Descripción}{
		Un estudiante se encuentra activo mientras no hayan transcurrido más de dos ciclos escolares desde su última inscripción, considerado a partir del inicio del ciclo escolar o cuenta con un dictamen de inscripción vigente para el ciclo escolar actual.
	}\BRitem[admin]{Sentencia}{%
		Sean 2019-2020/1, 2018-2019/2, 2018-2019/1 y 2017-2018/2 los últimos cuatro ciclos escolares. Un estudiante que no se haya inscrito en los ciclos escolares 2018-2019/2 y 2018-2019/1 pero si en 2017-2018/2, dejará de ser activo a partir del día de inicio del ciclo escolar 2019-2020/1 si no se inscribe en dicho ciclo escolar.
	}\BRitem{Motivación}{
		Evitar que los estudiantes que no atienden propiamente su carrera ocupen indefinidamente lugares que pueden ser aprovechados por otros estudiantes.
	}\BRitem{Ejemplo positivo}{Son estudiantes activos:
	    \begin{itemize}
		    \item Al 27 de agosto del 2018 el estudiante cuyo último ciclo escolar inscrito es en 2018-2019/1
		    \item Al 27 de agosto del 2018 el estudiante cuyo último ciclo escolar inscrito es en  2017-2018/2
		    \item Al 27 de agosto del 2018 el estudiante cuyo último ciclo escolar inscrito es en  2017-2018/1
		    \item Al 2 de agosto del 2018 el estudiante cuyo último ciclo escolar inscrito es en  2016-2018/2 considerando que el ciclo escolar 2018-2019/1 comienza en 5 de agosto del 2018.
	    \end{itemize}}
	\BRitem{Ejemplo negativo}{No es un estudiante activo si:
	\begin{itemize}
		\item Al 27 de agosto del 2018 el estudiante cuyo último ciclo escolar inscrito es en 2016-2017/1 y el ciclo escolar comienzo el 5 de agosto.
	\end{itemize}}
\end{BusinessRule}


%{BR-129}{Estudiante en proceso de titulación}
%======================================================================
\begin{BusinessRule}[%
	Autor/Ulises Vélez Saldaña,%
	Version/2.1,%
	Estado/En revisión%
]{BR-129}{Estudiante en proceso de titulación}%
	\BRitem[control]{Revisada por}{Pendiente por asignar.}
    \BRitem[control]{Aprobada por}{Pendiente por asignar.}
    \BRsection[control]{Atributos}
    \BRitem[admin]{Clase}{\bcDerivation}% Clase: \bcCondition, \bcIntegridad, \bcAutorization, \bcDerivation.
    \BRitem[admin]{Nivel}{\blControlling}%  Nivel: \blControlling, \blInfluencing.
	\BRitem{Descripción}{
		Un estudiante está en proceso de titulación si cuenta con más del 92\% de créditos aprobados.
	}\BRitem[admin]{Sentencia}{%
		Sean $M_{i}(e)$ las materias aprobadas del estudiante $e$ y $e.carrera$ entonces el estudiante $e$ está en proceso de titulación si y solo si: 
		\[{\sum M_{i}(e) \over e.carrera.TotalCreditos} >= 0.92\]
	}\BRitem{Motivación}{
		Garantizar que todos los alumnos que presenten un seminario de tesis cuenten con la formación adecuada.
	}\BRitem{Ejemplo positivo}{Cumplen la regla:
	    \begin{itemize}
		    \item Estudiante con las materias aprobadas: M1 (20 créditos), M2 (18 créditos), M3 (22 créditos) y la carrera con créditos totales 32. El porcentaje de creditos es ${30 \over 32}=0.9375>0.92$.
	    \end{itemize}}
	\BRitem{Ejemplo negativo}{No cumplen con la regla: 
	\begin{itemize}
		\item 
	\end{itemize}}
\end{BusinessRule}

%{BR-130}{Seminarios elegibles por carrera}
%%======================================================================
\begin{BusinessRule}[%
	Autor/Ulises Vélez Saldaña,%
	Version/2.1,%
	Estado/En revisión%
]{BR-130}{Seminarios elegibles por carrera}%
	\BRitem[control]{Revisada por}{Pendiente por asignar.}
    \BRitem[control]{Aprobada por}{Pendiente por asignar.}
    \BRsection[control]{Atributos}
    \BRitem[admin]{Clase}{\bcIntegridad}% Clase: \bcCondition, \bcIntegridad, \bcAutorization, \bcDerivation.
    \BRitem[admin]{Tipo}{\btEnabler}% Tipo:  \btEnabler,     \btTimer,      \btExecutive.
    \BRitem[admin]{Nivel}{\blControlling}%  Nivel: \blControlling, \blInfluencing.
	\BRitem{Descripción}{
		.
	}\BRitem[admin]{Sentencia}{%
		.
	}\BRitem{Motivación}{
		.
	}\BRitem{Ejemplo positivo}{Cumplen la regla:
	    \begin{itemize}
		    \item 
	    \end{itemize}}
	\BRitem{Ejemplo negativo}{No cumplen con la regla: 
	\begin{itemize}
		\item 
	\end{itemize}}
\end{BusinessRule}


%{BR-131}{Seminarios disponibles}
%%======================================================================
\begin{BusinessRule}[%
	Autor/Ulises Vélez Saldaña,%
	Version/2.1,%
	Estado/En revisión%
]{BR-131}{Seminarios disponibles}%
	\BRitem[control]{Revisada por}{Pendiente por asignar.}
    \BRitem[control]{Aprobada por}{Pendiente por asignar.}
    \BRsection[control]{Atributos}
    \BRitem[admin]{Clase}{\bcCondition}% Clase: \bcCondition, \bcIntegridad, \bcAutorization, \bcDerivation.
    \BRitem[admin]{Nivel}{\blControlling}%  Nivel: \blControlling, \blInfluencing.
	\BRitem{Descripción}{
		.
	}\BRitem[admin]{Sentencia}{%
		.
	}\BRitem{Motivación}{
		.
	}\BRitem{Ejemplo positivo}{Cumplen la regla:
	    \begin{itemize}
		    \item 
	    \end{itemize}}
	\BRitem{Ejemplo negativo}{No cumplen con la regla: 
	\begin{itemize}
		\item 
	\end{itemize}}
\end{BusinessRule}



%{BR-132}{Disponibilidad de un seminario}
%======================================================================
\begin{BusinessRule}[%
	Autor/Ulises Vélez Saldaña,%
	Version/2.1,%
	Estado/En revisión%
]{BR-132}{Disponibilidad de un seminario}%
	\BRitem[control]{Revisada por}{Pendiente por asignar.}
    \BRitem[control]{Aprobada por}{Pendiente por asignar.}
    \BRsection[control]{Atributos}
    \BRitem[admin]{Clase}{\bcDerivation}% \bcCondition, \bcIntegridad, \bcAutorization, \bcDerivation.
    %\BRitem[admin]{Tipo}{\btEnabler}% \btEnabler,     \btTimer,      \btExecutive.
    \BRitem[admin]{Nivel}{\blControlling}% \blControlling, \blInfluencing.
	\BRitem{Descripción}{
		.
	}\BRitem[admin]{Sentencia}{%
		.
	}\BRitem{Motivación}{
		.
	}\BRitem{Ejemplo positivo}{Cumplen la regla:
	    \begin{itemize}
		    \item 
	    \end{itemize}}
	\BRitem{Ejemplo negativo}{No cumplen con la regla: 
	\begin{itemize}
		\item 
	\end{itemize}}
\end{BusinessRule}

%{BR-143}{Cálculo del costo de seminario por tipo de estudiante}
%======================================================================
\begin{BusinessRule}[%
	Autor/Ulises Vélez Saldaña,%
	Version/2.1,%
	Estado/En revisión%
]{BR-143}{Cálculo del costo de seminario por tipo de estudiante}%
	\BRitem[control]{Revisada por}{Pendiente por asignar.}
    \BRitem[control]{Aprobada por}{Pendiente por asignar.}
    \BRsection[control]{Atributos}
    \BRitem[admin]{Clase}{\bcDerivation}% \bcCondition, \bcIntegridad, \bcAutorization, \bcDerivation.
    %\BRitem[admin]{Tipo}{\btEnabler}% \btEnabler,     \btTimer,      \btExecutive.
    \BRitem[admin]{Nivel}{\blControlling}% \blControlling, \blInfluencing.
	\BRitem{Descripción}{
		.
	}\BRitem[admin]{Sentencia}{%
		.
	}\BRitem{Motivación}{
		.
	}\BRitem{Ejemplo positivo}{Cumplen la regla:
	    \begin{itemize}
		    \item 
	    \end{itemize}}
	\BRitem{Ejemplo negativo}{No cumplen con la regla: 
	\begin{itemize}
		\item 
	\end{itemize}}
\end{BusinessRule}


%{BR-152}{Aplicación de impuestos a servicios generales}
%======================================================================
\begin{BusinessRule}[%
	Autor/Ulises Vélez Saldaña,%
	Version/2.1,%
	Estado/En revisión%
]{BR-152}{Aplicación de impuestos a servicios generales}%
	\BRitem[control]{Revisada por}{Pendiente por asignar.}
    \BRitem[control]{Aprobada por}{Pendiente por asignar.}
    \BRsection[control]{Atributos}
    \BRitem[admin]{Clase}{\bcCondition}% \bcCondition, \bcIntegridad, \bcAutorization, \bcDerivation.
    \BRitem[admin]{Tipo}{\btEnabler}%    \btEnabler,   \btTimer,      \btExecutive.
    \BRitem[admin]{Nivel}{\blControlling}% \blControlling, \blInfluencing.
	\BRitem{Descripción}{
		.
	}\BRitem[admin]{Sentencia}{%
		.
	}\BRitem{Motivación}{
		.
	}\BRitem{Ejemplo positivo}{Cumplen la regla:
	    \begin{itemize}
		    \item 
	    \end{itemize}}
	\BRitem{Ejemplo negativo}{No cumplen con la regla: 
	\begin{itemize}
		\item 
	\end{itemize}}
\end{BusinessRule}

