\clearpage
\section{Marco conceptual}
    La IA es una rama de las ciencias computacionales encargada de estudiar modelos de cómputo capaces de realizar actividades propias de los seres humanos con base a dos de sus características primordiales: el razonamiento y la conducta. 1: Sin embargo, a diferencia de las personas, los dispositivos basados en IA no necesitan descansar y pueden analizar grandes volúmenes de información a la vez. Asimismo, la proporción de errores es significativamente menor en las máquinas que realizan las mismas tareas que sus contrapartes humanas.\\
    \newline
    La idea de que los ordenadores o los programas informáticos puedan tanto aprender como tomar decisiones es particularmente importante y algo sobre lo que deberíamos ser conscientes, ya que sus procesos están creciendo exponencialmente con el tiempo. Debido a estas dos capacidades, los sistemas de inteligencia artificial pueden realizar ahora muchas de las tareas que antes estaban reservadas sólo a los humanos. \\
    \newline
    Las tecnologías basadas en la IA ya están siendo utilizadas para ayudar a los humanos a beneficiarse de mejoras significativas y disfrutar de una mayor eficiencia en casi todos los ámbitos de la vida. Pero el gran crecimiento de la IA también nos obliga a estar atentos para prevenir y analizar las posibles desventajas directas o indirectas que pueda generar la proliferación de la IA. La IA se puede aplicar en casi todas las situaciones. Éstas son sólo algunas de las aplicaciones técnicas de la IA que están creciendo rápidamente en la actualidad:
    \begin{itemize}
        \item Reconocimiento de imágenes estáticas, clasificación y etiquetado:  estas herramientas son útiles para una amplia gama de industrias.
        \item Mejoras del desempeño de la estrategia algorítmica comercial: ya ha sido implementada de diversas maneras en el sector financiero.
        \item Procesamiento eficiente y escalable de datos de pacientes: esto ayudará a que la atención médica sea más efectiva y eficiente.
        \item Mantenimiento predictivo: otra herramienta ampliamente aplicable en diferentes sectores industriales.
        \item Detección y clasificación de objetos: puede verse en la industria de vehículos autónomos, aunque también tiene potencial para muchos otros campos.
        \item Distribución de contenido en las redes sociales: se trata principalmente de una herramienta de marketing utilizada en las redes sociales, pero también puede usarse para crear conciencia entre las organizaciones sin ánimo de lucro o para difundir información rápidamente como servicio público.
        \item Protección contra amenazas de seguridad cibernética: es una herramienta importante para los bancos y los sistemas que envían y reciben pagos en línea.
    \end{itemize}

    \subsection{Aprendizaje automático}

    El aprendizaje automático (en inglés, machine learning) es uno de los enfoques principales de la inteligencia artificial. En pocas palabras, se trata de un aspecto de la informática en el que los ordenadores o las máquinas tienen la capacidad de aprender sin estar programados para ello. Un resultado típico serían las sugerencias o predicciones en una situación particular. Gracias al aprendizaje automático, muchos de los dispositivos que verás en el futuro obtendrán experiencia y conocimientos a partir de la forma en que son utilizados para poder ofrecer una experiencia al usuario personalizada.\\
    \newline
    El aprendizaje automático usa algoritmos para aprender patrones de datos. Por ejemplo, los filtros de spam de correo electrónico utilizan este tipo de aprendizaje con el fin de detectar qué mensajes son correo basura y separarlos de aquellos que no lo son. Éste es un sencillo ejemplo de cómo los algoritmos pueden usarse para aprender patrones y utilizar el conocimiento adquirido para tomar decisiones.\\
    \newline
    La figura  muestra a continuación tres subconjuntos del aprendizaje automático que pueden utilizarse: aprendizaje supervisado, no supervisado y de refuerzo.
    
    
    En el \textit{aprendizaje supervisado}, los algoritmos usan datos que ya han sido etiquetados u organizados previamente para indicar cómo tendría que ser categorizada la nueva información. Con este método, se requiere la intervención humana para proporcionar retroalimentación.
    
    En el \textit{aprendizaje no supervisado}, los algoritmos no usan ningún dato etiquetado u organizado previamente para indicar cómo tendría que ser categorizada la nueva información, sino que tienen que encontrar la manera de clasificarlas ellos mismos. Por tanto, este método no requiere la intervención humana.\\
    \newline
    Por último, con el \textit{aprendizaje por refuerzo}, los algoritmos aprenden de la experiencia. En otras palabras, tenemos que darles «un refuerzo positivo» cada vez que aciertan. La forma en que estos algoritmos aprenden se puede comparar con la de los perros cuando les damos «recompensas» al aprender a sentarse.
    
    \subsection{Aprendizaje profundo}
    Una de las aplicaciones más poderosas y de mayor crecimiento de la inteligencia artificial es el aprendizaje profundo (en inglés, deep learning). Se trata de un subcampo del aprendizaje automático que se utiliza para resolver problemas muy complejos y que normalmente implican grandes cantidades de datos. 
    
    El aprendizaje profundo se produce mediante el uso de redes neuronales, que se organizan en capas para reconocer relaciones y patrones complejos en los datos. Su aplicación requiere un enorme conjunto de información y una potente capacidad de procesamiento. Actualmente, se utiliza en el reconocimiento de voz, el procesamiento del lenguaje natural, la visión artificial y la identificación de vehículos en los sistemas de asistencia al conductor.
    
    \subsection{Redes neuronales}
    
    Las redes neuronales son más que otra forma de emular ciertas características propias de los humanos, como la capacidad de memorizar y de asociar hechos. Si se examinan con atención aquellos problemas que no pueden expresarse a través de un algoritmo, se observará que todos ellos tienen una característica en común: la experiencia. El hombre es capaz de resolver estas situaciones acudiendo a la experiencia acumulada. Así, parece claro que una forma de aproximarse al problema consista en la construcción de sistemas que sean capaces de reproducir esta característica humana. En definitiva, las redes neuronales no son más que un modelo artificial y simplificado del cerebro humano, que es el ejemplo más perfecto del que disponemos para un sistema que es capaz de adquirir conocimiento a través de la experiencia. Una red neuronal es “un nuevo sistema para el tratamiento de la información, cuya unidad básica de procesamiento está inspirada en la célula fundamental del sistema nervioso humano: la neurona”.\\
    \newline
    
    Todos los procesos del cuerpo humano se relacionan en alguna u otra forma con la (in)actividad de estas neuronas. Las mismas son un componente relativamente simple del ser humano, pero cuando millares de ellas se conectan en forma conjunta se hacen muy poderosas. Lo que básicamente ocurre en una neurona biológica es lo siguiente: la neurona es estimulada o excitada a través de sus entradas (inputs) y cuando se alcanza un cierto umbral, la neurona se dispara o activa, pasando una señal hacia el axon.\\
    \newline
    
    Como ya se sabe, el pensamiento tiene lugar en el cerebro, que consta de billones de neuronas interconectadas. Así, el secreto de la “inteligencia” -sin importar como se defina- se sitúa dentro de estas neuronas interconectadas y de su interacción. También, es bien conocido que los humanos son capaces de aprender. Aprendizaje significa que aquellos problemas que inicialmente no pueden resolverse, pueden ser resueltos después de obtener más información acerca del problema. Por lo tanto, las Redes Neuronales…
    \begin{itemize}
        \item Consisten de unidades de procesamiento que intercambian datos o información. 
        \item Se utilizan para reconocer patrones, incluyendo imágenes, manuscritos y secuencias de tiempo (por ejemplo: tendencias financieras).
        \item Tienen capacidad de aprender y mejorar su funcionamiento.
    \end{itemize}
    Una primera clasificación de los modelos de redes neuronales podría ser,
    atendiendo a su similitud con la realidad biológica:
    
    \begin{itemize}
        \item El modelo de tipo biológico. Este comprende las redes que tratan de simular los sistemas neuronales biológicos, así como las funciones auditivas o algunas funciones básicas de la visión.
        \item El modelo dirigido a aplicación. Este modelo no tiene por qué guardar similitud con los sistemas biológicos. Su arquitectura está fuertemente ligada a las necesidades de las aplicaciones para la que está diseñada.
    
    \end{itemize}
    
    Debido a su constitución y a sus fundamentos, las redes neuronales artificiales presentan un gran número de características semejantes a las del cerebro. Por ejemplo, son capaces de aprender de la experiencia, de generalizar de casos anteriores a nuevos casos, de abstraer características esenciales a partir de entradas que representan información irrelevante, etc. Esto hace que ofrezcan numerosas ventajas y que este tipo de tecnología se esté aplicando en múltiples áreas. Entre las ventajas se incluyen: 
    \begin{itemize}
        \item Aprendizaje Adaptativo: Capacidad de aprender a realizar tareas basadas en un entrenamiento o en una experiencia inicial. 
        \item Auto-organización: Una red neuronal puede crear su propia organización o representación de la información que recibe mediante una etapa de aprendizaje.
        \item Tolerancia a fallos: La destrucción parcial de una red conduce a una degradación de su estructura; sin embargo, algunas capacidades de la red se pueden retener, incluso sufriendo un gran daño.
        \item Operación en tiempo real: Los cómputos neuronales pueden ser realizados en paralelo; para esto se diseñan y fabrican máquinas con hardware especial para obtener esta capacidad.
        \item Fácil inserción dentro de la tecnología existente. Se pueden obtener chips especializados para redes neuronales que mejoran su capacidad en ciertas tareas. Ello facilitará la integración modular en los sistemas existentes
    \end{itemize}
    
    %\subsection{Introducción a la inteligencia artificial}
    \subsection{Sistemas de recomendación}
    Los sistemas de recomendación están en todas partes. Impulsan muchos de los servicios que nos encantan y que usamos todos los días. 
    Desde hacer compras hasta el streaming y el uso de los motores de búsqueda, los sistemas de recomendación están diseñados para 
    ayudar a las personas a tener una experiencia más personalizada.\\
    \newline
    Un sistema de recomendación es aquél que produce una lista de sugerencias para un usuario. Cada objeto de la lista se llama “ítem” 
    de manera genérica, pues dichas sugerencias pueden ser artículos, películas, música, etc.\\
    \newline
    Aunque todos los sistemas de recomendación tienen el mismo objetivo final, difieren en aspectos como la información utilizada para 
    hacer recomendaciones y en el funcionamiento del algoritmo. A continuación, se presentan las clasificaciones más comunes.
    
    \begin{description}
        \item[Filtrado colaborativo] En esta categoría se intenta predecir la “clasificación” de un ítem para un usuario con base en las 
        clasificaciones de otros usuarios; se encuentran personas con gustos (clasificaciones) similares al usuario y se le recomiendan 
        ítems que les gusten a esas personas.\\
        \newline
    Uno de los problemas más grandes en este tipo de recomendación es la falta de información, pues no todos los usuarios han 
    clasificado todos los ítems. En la mayoría de los casos, los usuarios acceden (y clasifican) los ítems que les gustan. Otro problema
    es el cold start (inicio frío); es difícil recomendar ítems nuevos debido a falta de clasificaciones y también es difícil hacer 
     recomendaciones a un usuario nuevo por la falta de historial de clasificaciones. \cite{mc1}
        \item[Basado en contenido] Los sistemas de recomendación basados en contenido hacen sus recomendaciones con base en descripciones de los 
    ítems y los intereses del usuario. Cada ítem tiene atributos, y se determina su similitud con otros ítems dependiendo de los 
    atributos que compartan. Por ejemplo, si a un usuario le gustan películas del género “Acción”, el sistema recomendaría más películas de ese mismo género.\\
    \newline
    Un problema con los sistemas basados en contenido es que los ítems no tengan suficientes atributos para ser diferenciados por el 
    sistema de acuerdo con un usuario. Por ejemplo, aunque al usuario le gusten las películas de acción, no quiere decir que le gusten 
    todas las películas de este género; quizá no le gustan las películas con finales tristes, pero el sistema no tenía ese atributo 
    contemplado. \cite{mc4}
    
    \item[Demográficos] Este tipo de sistema toma en cuenta datos del perfil demográfico de un usuario. Por ejemplo, se pueden hacer recomendaciones basadas en la región o país del usuario, o recomendar con base en la edad del usuario.\cite{mc5}
    \item[Híbridos] Como el nombre indica, son aquellos que combinan aspectos de diferentes tipos de sistemas de recomendación. Hoy en día son el tipo más común, pues se intenta minimizar las desventajas de algún tipo con la inclusión de aspectos de otro tipo.
    
    
    \end{description}
    
    \subsection{Beneficios de implementar sistemas de recomendación}
    Los sistemas de recomendación han logrado cambiar la forma en la que consumimos nuevos contenidos y descubrimos productos nuevos. Uno de los ejemplos más claros los podemos disfrutar en las páginas de compra de productos como Amazon o Mercado Libre. Con un nivel de precisión alto, estos sistemas web con algunos pocos datos pueden proporcionarnos de sugerencias de productos adaptadas a nuestras necesidades. De igual forma sucede con las plataformas de contenidos como YouTube, Spotify o Netflix. Sus recomendaciones precisas nos ayudan a descubrir nuevas series, videos o artistas al analizar nuestros gustos y preferencias.\\
    \newline
    Esto se traduce en una mejor satisfacción de las necesidades del cliente. La experiencia del usuario se convierte en una actividad más agradable, ya que estos sistemas actúan como un asistente personal que estimula a la persona a seguir descubriendo elementos. Adicionalmente estos sistemas aportan una eficiencia excepcional a las conversiones de los sitios web. Las recomendaciones de productos personalizados acerca al cliente a lo que desea, mejorando las posibilidades de que este efectivamente compre o consuma el contenido sugerido. \cite{mc6}
    


\chapter{Resultados obtenidos}

\section{Resultados}


De acuerdo a las fechas de trabajo programadas, al momento de presentar este documento para revisión estamos en el sprint 8, recolección de datos y prueba de algoritmo. Hasta este momento los objetivos que se ha completado son:

\begin{itemize}
   \item Creación de un proyecto en Django en el backend y con React js para el frontend con funciones básicas para probar el uso de las tecnologías.
   \item Configuración del entorno de programación del proyecto con Django.
   \item CRUD de usuarios en backend con Django.
   \item CRUD de vacantes en backend con Django.
   \item CRUD de solicitudes en backend con Django.
   \item CRUD de compañías en backend con Django.
   \item CRUD de comunicados en backend con Django.
   \item Implementación de Django API Rest Framework en el backend
   \item Serializadores de Django API Rest para los CRUDs creados en el backend.
   \item Viewset correspondientes a cada acción de los CRUDs creados en backend y las serializaciones creadas con DJango API Rest.
   \item Endpoints para los viewsets creados con Django API Rest.
   \item Creación e implementación de los catálogos propuestos para el sistema en la base de datos.
   \item Configuración del entorno de programación para el frontend con React js
   \item Hooks en JavaScript para gestionar la comunicación entre el frontend y backend.
   \item Servicios para establecer la comunicación entre el frontend y backend, usando los endpoints de la API creada.
   \item Plantillas para las vistas mostradas al usuario en el frontend con React js.
   \item Generación y gestión de tokens en backend para inicios de sesión en el backend con Django API Rest y JSON Web Token.
   \item Hooks en JavaScript para solicitar y usar tokens en frontend para inicio de sesión y uso del sistema en frontend.
   \item Uso de rutas privadas en el frontend para su uso dependiendo del inicio de sesión.
   \item CRUD del usuario Alumno en el frontend.
   \item Gestionar vacantes desde el frontend.
   \item Postularse a vacantes desde el frontend
   \item Gestionar postulaciones desde el frontend.
\end{itemize}
    
    
    