%\section{}
\section{Marco teórico}
    El ``7to. Estudio de Búsqueda de Empleo por Internet en México 2020'' por la Asociación de Internet MX (AIMX)
    organismo líder en temas de comunicación y tecnologías digitales, en esta edición se enfocaron en cómo la
    contingencia sanitaria afectó los hábitos tanto de los internautas en su búsqueda de empleo, así como en la manera en
    que las empresas realizaron sus procesos de reclutamiento en internet.\\
    \newline
    La transformación digital y la necesidad de inmediatez ha impulsado de manera significativa el uso de los smartphones
    en la búsqueda de empleo en línea con un 45\% de preferencia, las computadoras portátiles son el segundo dispositivo
    más utilizado para este fin (34\%) y con 49\% el primero utilizado para el reclutamiento de personal. Las bolsas de
    trabajo en línea siguen siendo el medio más utilizado tanto para buscar trabajo con el 83\% de preferencia; en segundo
    lugar, para los candidatos destaca el uso de las redes sociales, con el 54\% de preferencia y las bolsas de trabajo
    universitarias con un 42\%. 
    El Top of mind es aquella marca que ocupa un lugar privilegiado en la mente del público, OCCMundial se mantiene
    en el Top of Mind con una preferencia del 72\% por parte de candidatos y 62\% por parte de las empresas, mientras que
    Facebook cobra alta relevancia para las empresas en sus procesos de reclutamiento con el 63\% de preferencia.
    Aunque los internautas y las empresas también recurren a las redes sociales para buscar trabajo, las bolsas de empleo
    siguen siendo la punta de lanza en este tipo de servicios. Las 5 bolsas de trabajo más utilizadas para buscar empleo en
    2020 fueron:
    \begin{enumerate}
        \item OCCMundial 92\%
        \item CompuTrabajo 56\%
        \item Indeed 50\%
        \item LinkedIn 44\%
        \item Bumeran 15\%
    \end{enumerate}

    El estudio indica que la mayoría de los internautas que buscan empleo son de licenciatura con una edad entre 18 y 37
    años, 5 de cada 10 personas se encuentran sin empleo, entre ellos están aquellos egresados o que están en sus últimos
    semestres que buscan incursionar en mundo laboral opciones.
    \newline

    Muchas universidades públicas y privadas en México cuentan con bolsa de trabajo para su comunidad académica y
    hoy en día universidades como la UAM, ANAHUAC y el IPN tienen su bolsa de trabajo virtual, aunque no todas se
    explotan al máximo o no tienen la difusión correcta. La más clara desventaja para los que buscan empleo y son
    1estudiantes es que hay poca oferta de vacantes con poca o nula experiencia. Mientras que para los reclutadores es que
    el servicio es de paga y conforme a lo pagado es el alcance de la vacante y el tiempo disponible para usar las
    herramientas de la plataforma. Dado que los estudiantes de ESCOM en su mayoría no cuentan con experticia laboral
    es más complicado encontrar una vacante donde sus habilidades se desarrollen. 

    ESCOM cuenta con una página en Facebook donde se publican los boletines de vacantes disponibles que le llegan al instituto, 
    se reciben vacantes que pertenecen tanto del sector público como del privado, el departamento de Extensión y Apoyos Educativos 
    es el encargado de filtrar la información, concentrarla y publicarla en la página de Facebook
    oficial.
    La información que se publica en ocasiones no está completa y se limita a informar sobre los conocimientos necesarios
    para la vacante y un contacto con la empresa dejando ciertas inquietudes como: saber el formato con el que deben ser
    enviados los datos del aspirante, si se debe dirigir con alguien en específico o si la vacante aún está disponible, por
    mencionar algunas.
    Por otro lado, el proceso de selección en ocasiones es problemático para el reclutador, el número de solicitudes que
    recibe excede de 50 para una sola vacante lo cual implica revisar y descargar cada solicitud una por una y si no se
    tiene un buen control puede que se traspapelen solicitudes y aunque el candidato envíe la información necesaria y en
    el formato necesario para ser considerado, no se realice el seguimiento a su solicitud.
