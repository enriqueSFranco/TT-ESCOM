\section{Descripción del contexto}
El Internet es una constante de cambio en la actualidad, su impacto ha puesto una significativa evolución en los métodos tradicionales de búsqueda de empleo en pocos años. Quedó atrás el envío currículums en formato de papel junto con cartas de presentación o de recomendación, la era digital supone que uno mismo es su currículo.\cite{Evo}
El primer empleo puede parecer una montaña elevadísima, difícil de escalar y con muchos impedimentos, al ser una persona recién egresada y en la mayoría de las veces sin experiencia es difícil ser contratado ya que para la vacante que te postulaste no eres el único que se postuló y mucho menos el que tiene mayor experiencia, pero siempre existen vías para entrar en el mercado laboral sin tener experiencia. \\
\newline
Las bolsas de trabajo son el enlace entre las empresas y los candidatos, ya que su servicio consiste en servir como medio para que las empresas den a conocer las ofertas de trabajo que tienen; por lo que estos espacios sirven para reunir tanto a empresas como a candidatos.\cite{Occ3}
En la actualidad, las personas buscan trabajo a través de diferentes portales web, por lo que las empresas deben mantener y crear una imagen digital. Una de las bolsas de trabajo en línea más populares de nuestro país es OCCMundial.\\
Las bolsas de trabajo universitarias son la primera oportunidad de adquirir un empleo como profesionista, publican ofertas de empleo orientadas, en principio, a los recién licenciados y egresados de su comunidad, dando la oportunidad a empresas de buscar candidatos cumplan los requerimientos de sus áreas de trabajo y que los propios candidatos se postulen a dichas vacantes de una forma sencilla.\cite{Universia}  
Casi todas las universidades públicas y privadas en México ofrecen apoyo tanto a sus alumnos como a las empresas para facilitar la búsqueda de trabajo exclusivas para sus comunidades estudiantiles, es un buen primer paso para encontrar un hueco en el mercado laboral.\\
\newline
El Instituto Politécnico Nacional (IPN) cuenta con su propia bolsa de trabajo, la cual es una plataforma web llamada ``Siboltra'', donde únicamente las empresas \textit{constituidas} pueden registrarse para publicar sus vacantes y los estudiantes ya registrados pueden postularse a ellas. El desuso, la falta de difusión y de mantenimiento a la plataforma ha provocado que las unidades profesionales del IPN acudan a otros medios para publicar y difundir sus vacantes.\cite{Siboltra} \\
La Escuela Superior de Cómputo (ESCOM) implementa el uso de una bolsa de trabajo en la red social ``Facebook'', los encargados del departamento
de apoyos estudiantiles crearon una página llamada ``Bolsa de Trabajo ESCOM'', la cual se encarga de difundir sus vacantes por medio 
de boletines, su gestión se basa en el trabajo manual de los involucrados. 

