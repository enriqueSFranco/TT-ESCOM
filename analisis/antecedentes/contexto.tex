\section{Descripción del contexto}

	El Internet es una constante de cambio en la actualidad, su impacto ha puesto una significativa evolución en los métodos 
	tradicionales de búsqueda de empleo en pocos años. Quedó atrás el envío currículums en formato de papel junto con cartas de 
	presentación o de recomendación, la era digital supone que uno mismo es su currículo.\\
	\newline
	El primer empleo puede parecer una montaña elevadísima, difícil de escalar y con muchos impedimentos, pero siempre existen vías 
	para entrar en el mercado laboral sin tener experiencia. Las bolsas de trabajo universitarias son el primer puente que existe 
	entre un estudiante y las empresas; son la primera oportunidad de adquirir un empleo como profesionista.\\
	\newline
	Casi todas las universidades públicas y privadas en México ofrecen apoyo tanto a sus alumnos como a las empresas para facilitar 
	la busqueda de trabajo a sus alumnos recien egresados  implementando bolsas de trabajo exclusivas para sus comunidades 
	estudiantiles, es un buen primer paso para encontrar un hueco en el mercado laboral.La razón de ser de las bolsas de trabajo 
	universitarias es ofrecer un espacio en el que tanto empresas como candidatos puedan tener un lugar de encuentro, y todos 
	obtengan un beneficio y para  algunos un nuevo comienzo en su vida profesional.\\
	\newline
	El Instito Politécnico Nacional (IPN) cuenta con su propia bolsa de trabajo, la cual es una plataforma web llamada ``Siboltra'', 
	donde empresas \textit{constituidas} pueden registrarse para publicar sus vacantes y los estudiantes ya registrados pueden 
	postularse a ellas. 
	El desuso, la falta de difusión y de mantenimiento a la plataforma ha probocado que las unidades profesionales del IPN acudan a otros 
	medios para publicar y difundir sus vacantes.
	
	