\section{Descripción del contexto}
	Internet ha alterado de forma radical los aspectos que configuran la búsqueda de empleo y de selección de personal, 
	consolidándose como uno de los medios más utilizados por quienes buscan nuevas oportunidades profesionales. Un estudio 
	realizado por la Asociación de Internet MX indica que la mayoría de los internautas que buscan empleo son de licenciatura con 
	una edad entre 18 y 37 años, es decir, egresados o aquellos que están en sus últimos semestres. Algunas universidades 
	públicas y privadas en México ofrecen apoyo a las empresas implementando bolsas de trabajo exclusivas para sus comunidades 
	estudiantiles. En ESCOM se implementó un boletín de ofertas de trabajo en la red social Facebook, crearon una página llamada 
	``Bolsa de Trabajo ESCOM''. \newline
	El problema que pretende atacar este proyecto es la falta de una plataforma que facilite la gestión de la Bolsa de Trabajo 
	rápida y eficazmente. La Bolsa de Trabajo ESCOM no es una plataforma en sí, el enfoque que tiene no es el apropiado para 
	facilitar las búsquedas de vacantes a los interesados, ni la gestión de ofertas para encargados de la Bolsa de Trabajo ni 
	la selección que tienen que hacer las empresas con las solicitudes que llegan a sus correos electrónicos.


