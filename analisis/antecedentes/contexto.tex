\section{Descripción del contexto}
	Internet ha alterado de forma radical los aspectos que configuran la búsqueda de empleo y de selección de personal, 
	consolidándose como uno de los medios más utilizados por quienes buscan nuevas oportunidades profesionales. 
	La mayoría de los internautas que buscan empleo son de licenciatura con una edad entre 18 y 37 años, los cuales en su
	mayoria son egresados o aquellos que están en sus últimos semestres.
	Casi todas las universidades públicas y privadas en México ofrecen apoyo tanto a sus alumnos como a las empresas 
	para facilitar la busqueda de trabajo a sus alumnos recien egresados  implementando bolsas de trabajo 
	exclusivas para sus comunidades estudiantiles.\\
	\newline
	El Instito Politécnico Nacional (IPN) cuenta con su propia bolsa de trabajo, la cual es una plataforma web llamada ``SIBOLTRA'', 
	donde empresas \textit{constituidas} pueden registrarse para publicar sus vacantes y los estudiantes ya registrados pueden postularse a ellas. 
	El desuso, la falta de difusión y de mantenimiento a la plataforma ha probocado que las unidades profesionales del IPN acudan a otros 
	medios para publicar y difundir sus vacantes.\\
	
	