\section{Marco conceptual}
\subsection{Introducción a la inteligencia artificial}

\subsubsection{Redes neuronales}

Las redes neuronales son más que otra forma de emular ciertas características propias de los humanos, como la capacidad de memorizar y de asociar hechos. Si se examinan con atención aquellos problemas que no pueden expresarse a través de un algoritmo, se observará que todos ellos tienen una característica en común: la experiencia. El hombre es capaz de resolver estas situaciones acudiendo a la experiencia acumulada. Así, parece claro que una forma de aproximarse al problema consista en la construcción de sistemas que sean capaces de reproducir esta característica humana. En definitiva, las redes neuronales no son más que un modelo artificial y simplificado del cerebro humano, que es el ejemplo más perfecto del que disponemos para un sistema que es capaz de adquirir conocimiento a través de la experiencia. Una red neuronal es “un nuevo sistema para el tratamiento de la información, cuya unidad básica de procesamiento está inspirada en la célula fundamental del sistema nervioso humano: la neurona”.

Todos los procesos del cuerpo humano se relacionan en alguna u otra forma con la (in)actividad de estas neuronas. Las mismas son un componente relativamente simple del ser humano, pero cuando millares de ellas se conectan en forma conjunta se hacen muy poderosas. Lo que básicamente ocurre en una neurona biológica es lo siguiente: la neurona es estimulada o excitada a través de sus entradas (inputs) y cuando se alcanza un cierto umbral, la neurona se dispara o activa, pasando una señal hacia el axon.

Como ya se sabe, el pensamiento tiene lugar en el cerebro, que consta de billones de neuronas interconectadas. Así, el secreto de la “inteligencia” -sin importar como se defina- se sitúa dentro de estas neuronas interconectadas y de su interacción. También, es bien conocido que los humanos son capaces de aprender. Aprendizaje significa que aquellos problemas que inicialmente no pueden resolverse, pueden ser resueltos después de obtener más información acerca del problema. Por lo tanto, las Redes Neuronales…

Consisten de unidades de procesamiento que intercambian datos o información. 
Se utilizan para reconocer patrones, incluyendo imágenes, manuscritos y secuencias de tiempo (por ejemplo: tendencias financieras).
Tienen capacidad de aprender y mejorar su funcionamiento.

Una primera clasificación de los modelos de redes neuronales podría ser,
atendiendo a su similitud con la realidad biológica:

El modelo de tipo biológico. Este comprende las redes que tratan de simular los sistemas neuronales biológicos, así como las funciones auditivas o algunas funciones básicas de la visión.
El modelo dirigido a aplicación. Este modelo no tiene por qué guardar similitud con los sistemas biológicos. Su arquitectura está fuertemente ligada a las necesidades de las aplicaciones para la que está diseñada.


\subsubsection{Definiciones de una red neuronal}


Debido a su constitución y a sus fundamentos, las redes neuronales artificiales presentan un gran número de características semejantes a las del cerebro. Por ejemplo, son capaces de aprender de la experiencia, de generalizar de casos anteriores a nuevos casos, de abstraer características esenciales a partir de entradas que representan información irrelevante, etc. Esto hace que ofrezcan numerosas ventajas y que este tipo de tecnología se esté aplicando en múltiples áreas. Entre las ventajas se incluyen: 
Aprendizaje Adaptativo. Capacidad de aprender a realizar tareas basadas en un entrenamiento o en una experiencia inicial. 
Auto-organización. Una red neuronal puede crear su propia organización o representación de la información que recibe mediante una etapa de aprendizaje.
Tolerancia a fallos. La destrucción parcial de una red conduce a una degradación de su estructura; sin embargo, algunas capacidades de la red se pueden retener, incluso sufriendo un gran daño.
Operación en tiempo real. Los cómputos neuronales pueden ser realizados en paralelo; para esto se diseñan y fabrican máquinas con hardware especial para obtener esta capacidad.
Fácil inserción dentro de la tecnología existente. Se pueden obtener chips especializados para redes neuronales que mejoran su capacidad en ciertas tareas. Ello facilitará la integración modular en los sistemas existentes

\subsubsection{Redes bayesianas}

Las redes bayesianas modelan un fenómeno mediante un conjunto de variables y las relaciones de dependencia entre ellas. Dado este modelo, se puede hacer inferencia bayesiana; es decir, estimar la probabilidad posterior de las variables no conocidas, con base a las variables conocidas. Estos modelos pueden tener diversas aplicaciones, para clasificación, predicción, diagnóstico, etc. Además, pueden dar información interesante en cuanto a cómo se relacionan las variables del dominio, las cuales pueden ser interpretadas en ocasiones como relaciones de causa-efecto. 

Inicialmente, estos modelos eran construidos ``a mano'' basados en un conocimiento experto, pero en los últimos años se han desarrollado diversas técnicas para aprender a partir de datos, tanto la estructura como los parámetros asociados al modelo. También es posible el combinar conocimiento experto con los datos para aprender el modelo.



