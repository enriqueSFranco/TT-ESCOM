\section{Marco conceptual}
\subsection{Introducción a la inteligencia artificial}
\subsection{Sistemas de recomendación}
Los sistemas de recomendación están en todas partes. Impulsan muchos de los servicios que nos encantan y que usamos todos los días. 
Desde hacer compras hasta el streaming y el uso de los motores de búsqueda, los sistemas de recomendación están diseñados para 
ayudar a las personas a tener una experiencia más personalizada.\\
\newline
Un sistema de recomendación es aquél que produce una lista de sugerencias para un usuario. Cada objeto de la lista se llama “ítem” 
de manera genérica, pues dichas sugerencias pueden ser artículos, películas, música, etc.\\
\newline
Aunque todos los sistemas de recomendación tienen el mismo objetivo final, difieren en aspectos como la información utilizada para 
hacer recomendaciones y en el funcionamiento del algoritmo. A continuación, se presentan las clasificaciones más comunes.

\begin{description}
    \item[Filtrado colaborativo] En esta categoría se intenta predecir la “clasificación” de un ítem para un usuario con base en las 
    clasificaciones de otros usuarios; se encuentran personas con gustos (clasificaciones) similares al usuario y se le recomiendan 
    ítems que les gusten a esas personas.\\
    \newline
Uno de los problemas más grandes en este tipo de recomendación es la falta de información, pues no todos los usuarios han 
clasificado todos los ítems. En la mayoría de los casos, los usuarios acceden (y clasifican) los ítems que les gustan. Otro problema
es el cold start (inicio frío); es difícil recomendar ítems nuevos debido a falta de clasificaciones y también es difícil hacer 
 recomendaciones a un usuario nuevo por la falta de historial de clasificaciones. \cite{mc1}
    \item[Basado en contenido] Los sistemas de recomendación basados en contenido hacen sus recomendaciones con base en descripciones de los 
ítems y los intereses del usuario. Cada ítem tiene atributos, y se determina su similitud con otros ítems dependiendo de los 
atributos que compartan. Por ejemplo, si a un usuario le gustan películas del género “Acción”, el sistema recomendaría más películas de ese mismo género.\\
\newline
Un problema con los sistemas basados en contenido es que los ítems no tengan suficientes atributos para ser diferenciados por el 
sistema de acuerdo con un usuario. Por ejemplo, aunque al usuario le gusten las películas de acción, no quiere decir que le gusten 
todas las películas de este género; quizá no le gustan las películas con finales tristes, pero el sistema no tenía ese atributo 
contemplado. \cite{mc4}

\item[Demográficos] Este tipo de sistema toma en cuenta datos del perfil demográfico de un usuario. Por ejemplo, se pueden hacer recomendaciones basadas en la región o país del usuario, o recomendar con base en la edad del usuario. \cite[mc5]

\item[Híbridos] Como el nombre indica, son aquellos que combinan aspectos de diferentes tipos de sistemas de recomendación. Hoy en día son el tipo más común, pues se intenta minimizar las desventajas de algún tipo con la inclusión de aspectos de otro tipo.


\end{description}

\subsection{Beneficios de implementar sistemas de recomendación}
Los sistemas de recomendación han logrado cambiar la forma en la que consumimos nuevos contenidos y descubrimos productos nuevos. Uno de los ejemplos más claros los podemos disfrutar en las páginas de compra de productos como Amazon o Mercado Libre. Con un nivel de precisión alto, estos sistemas web con algunos pocos datos pueden proporcionarnos de sugerencias de productos adaptadas a nuestras necesidades. De igual forma sucede con las plataformas de contenidos como YouTube, Spotify o Netflix. Sus recomendaciones precisas nos ayudan a descubrir nuevas series, videos o artistas al analizar nuestros gustos y preferencias.\\
\newline
Esto se traduce en una mejor satisfacción de las necesidades del cliente. La experiencia del usuario se convierte en una actividad más agradable, ya que estos sistemas actúan como un asistente personal que estimula a la persona a seguir descubriendo elementos. Adicionalmente estos sistemas aportan una eficiencia excepcional a las conversiones de los sitios web. Las recomendaciones de productos personalizados acerca al cliente a lo que desea, mejorando las posibilidades de que este efectivamente compre o consuma el contenido sugerido. \cite{mc6}



