\section{Estado del Arte}
Todas las plataformas de bolsa de trabajo tienen como objetivo publicar y difundir las vacantes de sus empresas participantes
y almismo tiempo dan a conocer el perfil de los candidatos a las mismas empresas.
\newline
A continuación se listas las bolsas de trabajo más populares entre las personas, aSí como  la bolsa de trabajo
oficial del Instituto:
\begin{itemize}
    \item OCCMundial: La OCC Mundial es una de las plataformas de búsqueda de empleo más conocidas y usadas en México. La variedad 
    de empleos y la variedad de empresas y empleadores es un gran atractivo tanto para los reclutadores y los candidatos para la 
    gestión de contrataciones. Aunque es una buena opción para realizar contrataciones su uso para reclutadores es de paga mediante 
    contratación de planes de pago de acuerdo al impacto y puestos requeridos. Razón por la que en algunas situaciones es una 
    solución muy sobrada para empresas de menor tamaño.

    Muchas universidades públicas y privadas en México cuentan con bolsa de trabajo para su comunidad académica y hoy en día 
    universidades como la UAM, ANAHUAC y el IPN tienen su bolsa de trabajo virtual, aunque no todas se explotan al máximo o no 
    tienen la difusión correcta.
    
    \item CompuTrabajo: CompuTranbajo es una plataforma de búsqueda de empleo bastante utilizada en Latinoamérica (Como lo menciona 
    en su página de inicio) donde se pueden publicar vacantes para buscar y publicar ofertas de empleo. Al igual que indeed se 
    pueden publicar vacantes de manera gratuita y potenciar las visitas de las vacantes. El precio base de la entrada premium para 
    los reclutadores es accesible pero la plataforma insta a los reclutadores a invertir más dinero para mejorar el alcance de sus 
    publicaciones.
    
    \item Indeed: Indeed a la par de OCCMundial Indeed es una plataforma de búsqueda de empleo muy conocida en México, la variedad 
    de empresas y sectores de trabajo disponible además que se pueden encontrar empleos en el extranjero a través de esta 
    plataforma, ya que el idioma predominante dentro de ella es el inglés. Al igual que otras plataformas de búsqueda de empleo 
    esta es de paga pero también permite realizar publicaciones gratuitas con el inconveniente de que el alcance se verá afectado 
    porque las vacantes ofrecidas por usuarios de paga serán presentadas primero a los candidatos y además solo se permitirá abrir 
    un puesto para la vacante por cada oferta publicada gratis. El sistema de gestión de las vacantes se basa en la compra de 
    tokens para potenciar el alcance, remarcar la urgencia de la contratación y ofertar más de un puesto por vacante.
    
    \item SIBOLTRA: Es la opción ofrecida por el Instituto Politécnico Nacional para el uso exclusivo de su comunidad, dado que 
    para poder registrarse en ella es necesario contar con un número de boleta válido. Al hacer de uso de la comunidad del IPN la 
    plataforma cuenta con la función de recuperar la trayectoria académica del alumno en cuestión aSí como la información personal 
    que se tenga registrada en las bases de datos de IPN, con ello los reclutadores pueden hacer una búsqueda de candidatos con 
    información actualizada que se adapten a sus necesidades. Aunque estas funciones son atractivas las pruebas de uso que se han 
    hecho en la plataforma nos muestran que no siempre funciona la sincronización de datos con la plataforma, la navegación dentro 
    de la plataforma es poco intuitiva y además la difusión de la existencia de la plataforma no es mucha por lo que la comunidad 
    del IPN caSí no sabe de su existencia. 
    
    \item ESCOMobile: ESCOMobile fue un proyecto de TT realizado por estudiantes de ESCOM con el propósito de crear una aplicación 
    móvil para los alumnos de ESCOM donde se concentrará información útil de la Institución, avisos publicados por la unidad 
    académica, el estado académico del usuario y la implementación de la bolsa de trabajo de ESCOM. El proyecto concluyó el módulo 
    de mostrar el estado académico del usuario y avisos generales publicados por los encargados pero el módulo de bolsa de trabajo 
    no se completo a la entrega del proyecto, el avance del módulo fue la actualización del boletín de ofertas dentro de la 
    plataforma, resultando en una implementación caSí igual a el de la página de Facebook "Bolsa de trabajo ESCOM"que ya se maneja.

\end{itemize}

\newpage
\begin{longtable}{| p{0.15\textwidth}  | p{0.15\textwidth} | p{0.15\textwidth}  | p{0.14\textwidth}  | p{0.13\textwidth}  | p{0.13\textwidth}  | }

\label{table:herramientasSimilares}
    \rowcolor{black}
    %\multicolumn{7}{ |c| }{\bf\cellcolor{black}\color{white}{Herramientas para la especificación de requerimientos}} \\ \hline
    \bf\color{white} Herramienta & \bf \color{white}OCCMundial	& \bf \color{white}CompuTrabajo &  \bf \color{white}Indeed & 
    \bf \color{white}SIBOLTRA (IPN) & \bf \color{white}ESCOMobile \\ \hline
\endhead
Rango de precio por publicación de una vacante (MXN) &\$1,448.84 \$2,535.76 & Gratis \$775.00  &  Gratis Patrocinada &  Gratis  & No aplica  \\ \hline
Duración de la vacante publicada &30 días &60 días, depende de la inversión.&Depende de la inversión. &No aplica.&No aplica. \\ \hline
Contacto directo reclutador-candidato dentro de la plataforma &  Sí & No  & Sí  & No  & No \\ \hline
Filtrado automático de currículos recibidos & Sí &  Sí 
&  No&  No & No \\ \hline
Autogenerado de CV & Sí & Sí & Sí  &  No &Sí \\ \hline
Seguimiento del estado de solicitud &Sí  & Sí & No & No  & Si\\ \hline
\caption{Comparación Plataformas para buscar empleo}
\end{longtable}