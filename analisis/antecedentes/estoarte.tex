\clearpage
\section{Estado del Arte}
Todas las plataformas de bolsa de trabajo tienen como objetivo publicar y difundir las vacantes de sus empresas participantes y al mismo tiempo dan a conocer el perfil de los candidatos a las mismas empresas.
Las plataformas de bolsa de trabajo universitarias al ser diseñadas únicamente para su comunidad estudiantil no se puede acceder 
a su información tan fácilmente y están diseñadas para satisfacer sus objetivos y necesidades de la propia universidad y comunidad por lo que los objetivos y las necesidades que debe de cubrir una bolsa de trabajo para la ESCOM o  el IPN en general son diferentes en comparación con el Tecnológico de Monterrey o de la Universidad Autónoma de México.\\
\newline
A continuación se listan las bolsas de trabajo más populares entre las personas que más se apegan al objetivo del Trabajo Terminal,
 así como  la bolsa de trabajo oficial del Instituto:
\begin{itemize}
    \item OCCMundial
    \item CompuTrabajo
    \item Indeed
    \item Siboltra 
    \item ESCOMobile: a pesar de que esta aplicación no es una bolsa de trabajo, en ella se pueden consultar los boletines de trabajo
    publicados por la ESCOM.
\end{itemize}
En la tabla \ref{table:herramientasSimilares} se muestran las características de cada una de ellas y las características que pretende tener el Trabajo Terminal.
%\newpage


\begin{longtable}{| p{0.15\textwidth}  | p{0.10\textwidth} | p{0.10\textwidth}  | p{0.10\textwidth}  | p{0.10\textwidth}  | p{0.10\textwidth}  |  p{0.10\textwidth}  |}

    \hline

    \textbf{Herramienta} & \IUocc{}	& \IUcompuT{}&  \IUIneed{} & \IUsisae{} & \scriptsize \textbf{ESCOMobile} & \scriptsize Trabajo Terminal\\ 
    \hline

    Contacto directo reclutador-candidato dentro de la plataforma & Sí & No  & Sí  & No  & No & Si.\\ \hline
    Filtrado automático de currículos recibidos & Sí & Sí & No & No & No & Si.\\ \hline
    Autogenerado de CV & Sí & Sí & Sí  & No & Sí & No. \\ \hline
    Seguimiento del estado de solicitud & Sí  & Sí & No & No  & Si& Sí. \\ \hline
    Búsqueda de candidatos & Sí & Sí & Sí & Sí & No & Sí\\ \hline

    \caption{Comparación Plataformas para buscar empleo}
    \label{table:herramientasSimilares}
\end{longtable}

En general todas las plataformas cumplen con publicar vacantes y consultar perfiles de candidatos, pero el publicar una vacante y dejarla visible en la plataforma para que se postulen no siempre es gratuito, la OCCMundial, CompuTrabajo e Indeed cobran por vacante publicada y el tiempo que dure dicha vacante abierta, así como los servicios que te gustaria tener. Si quieres que se te recomienden candidatos cada cierto tiempo tiene un precio extra.\\

En Ineed si quieres que tu vacante publicada tenga cierto patrocinio es decir, que sean de las primeras en ser vistas se te cobra una comisión extra.\\
\newline
Nuestro sistema no se pedirá ninguna comisión por su uso (por el simple hecho que es que para la comunidad de ESCOM), tampoco tendrá límite de vacantes publicadas, el reclutador tendrá total libertad en consultar candidatos y configurar la fecha de publicación y cierre sus vacantes. \\

Un punto importante es que resolverá los problemas de gestión de vacantes  y correos para validar la información de las empresas que actualmente tiene el departamento de extensión y apoyos educativos de la ESCOM
