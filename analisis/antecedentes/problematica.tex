\clearpage
\section{Problemática}

Las bolsas de trabajo universitarias publican ofertas de empleo orientadas, en principio, a 
los recién licenciados y egresados dando la oportunidad a empresas de buscar candidatos cumplan los requerimientos de sus áreas
de trabajo.\cite{Universia} \\
\newline
Siboltra es la bolsa de trabajo oficial del Instituto Politécnico Nacional (IPN), la cual poco a poco se ha dejado de usar
por falta de mantenimiento a la propia plataforma y por falta de difución en la misma, esto ha olbigado a las unidades profesionales
del instituto a difundir sus vacantes a traves de redes sociales.

En ESCOM se implementa el uso de una bolsa de trabajo en la red social ``Facebook'', los encargados del departamento
de apoyos estudiantiles crearon una página llamada ``Bolsa de Trabajo ESCOM'', la cual se encarga de difundir sus vacantes por medio 
de boletines, su gestión se basa en el trabajo manual de los involucrados. \\
\newline
El primer involucrado es el departamento que gestiona la bolsa, lo cual implica recibir correos electronicos de empresas y filtrarlos, verificar si la empresa es constituida y que esté registrada
en Siboltra, concentrar toda la información de cada vacante y publicarla en un boletín por semana o cada dos semanas dentro de la página de facebook.
El segundo involucrado es la empresa, la cual debe de filtrar y descargar cada pdf que los alumnos enviaron al correo electrónico o 
en su defecto contestar mensajes en WhatsApp o hacer llamadas a los alumnos.\newline
El tercer involucrado y el que tiene mayores problemas es el alumno, el cual tiene que consultar la pagina y buscar los los boletines 
publicados y posteriormente leer cada uno de ellos para poder encontrar una vacante. El algoritmo de Facebook ordena cada 
publicación dentro del perfil de una persona según lo que cree que es más relevante el, y en una página las va ordenando 
por la fecha más reciente de publicación es y en la   página ``Bolsa de Trabajo ESCOM'' no solo se publican boletines ypor consecuencia
las vacantes se van perdiendo poco a poco entre las publicaciones de la página.\\
\newline

Tanto los reclutadores como los encargados deben de coordinarse para que se puedan realizar ofertas por este medio, este  es uno 
de los motivos qUe retrasa la publicación de nuevas ofertas realizadas para alumnos de ESCOM, haciendo que se pierdan oportunidades
 de trabajo a los candidatos objetivo.\\
\newline
Con base en lo anterior, el enfoque que tiene no es el apropiado para facilitar las búsquedas y consultas de vacantes a los interesados, la gestión
de ofertas para encargados de la Bolsa de Trabajo y muchos menos la selección que tienen que hacer las empresas con las solicitudes
que llegan a sus correos electrónicos.