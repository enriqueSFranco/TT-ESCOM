\clearpage
\section{Problemática}

Buscar empleo es una de las actividades que los recién egresados realizan para aplicar los conocimientos prácticos aprendidos 
durante su trayectoria escolar. Aunque las opciones de búsqueda son variadas siempre se enfrentan a la competencia de otros 
interesados en encontrar un empleo y a su vez buscan diversidad en las prestaciones. \\
Parte de la problemática que resuelven las bolsas de trabajo universitarias es que ofrecen a empresas la oportunidad de buscar 
candidatos directamente con egresados que cumplan los requerimientos de sus áreas de trabajo.\\

\newline
Siboltra es la bolsa de trabajo oficial para el Instituto Politécnico Nacional (IPN), la cual poco a poco se ha dejado de usar
por falta de mantenimiento a la propia plataforma y por falta de difución en la misma, esto ha olbigado a las unidades profesionales
del instituto a difundir sus vacantes a traves de redes sociales.\\
\newline

En ESCOM se implementa el uso de una bolsa de trabajo en la red social ``Facebook'', 
crearon una página llamada ``Bolsa de Trabajo ESCOM'', la cual se encarga de difundir sus vacantes por medio de boletines, su gestión se basa en el trabajo manual de los involucrados, tanto los reclutadores como los encargados deben de coordinarse 
para que se puedan realizar ofertas por este medio, este  es uno de los motivos qee retrasa la publicación de nuevas ofertas 
realizadas para alumnos de ESCOM, haciendo que se pierdan oportunidades de trabajo a los candidatos objetivo.\\
\newline
En otras palabras el enfoque que tiene no es el apropiado para facilitar las búsquedas de  vacantes a los interesados, ni la gestión
de ofertas para encargados de la Bolsa de Trabajo y muchos menos la selección que tienen que hacer las empresas con las solicitudes
que llegan a sus correos electrónicos.