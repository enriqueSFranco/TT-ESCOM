\clearpage
\section{Problemática}
Como ya se mencionó anteriormente, las bolsas de trabajo universitarias son ideales y convenientes para recién egresados y universitarios.
En la actualidad ``Siboltra'' (la bolsa de trabajo del IPN) poco a poco a quedado en desuso por falta de mantenimiento en la plataforma, a consecuencia muchas universidades del instituto han creado sus bolsas de trabajo digitales en plataformas como Facebook, ESCOM es un claro ejemplo de ello.\\
\newline
El hecho de que ya no se utilice ``Siboltra'' para difundir vacantes a la comunidad no implica que la dejen de utilizar al 100\%, toda persona o empresa que quiera difundir sus vacantes en el IPN debe de registrarse por puro protocolo.\\
\newline
Por objetivo del proyecto nosotros solo nos enfocaremos en la situación en la que se encuentra la EScuela Superior de Computo.
En ESCOM, la página ``Bolsa de Trabajo ESCOM'' es gestionada manualmente por el departamento de extensión y apoyos educativos de la ESCOM, a continuación se describe el proceso que actualmente hacen para publicar una vacante:

\begin{itemize}
    \item Las empresas que quieren publicar sus vacantes a la comunidad de ESCOM, mandan un correo con la información de la vacante e indicando que les interesa difundirla a la comunidad de ESCOM.

    \item Departamento de extensión y apoyos educativos de la ESCOM recibe los correos electrónicos de empresas, le solicitan a la empresa esté registrada en Siboltra y que envíe un documento oficial que indique su empresa es constituida, les anexan un formato con la información que debe de tener su vacante. 

    \item Una vez que la empresa esté registrada y el departamento corrobore que se trate de una empresa constituida, el departamento concentra la información en un documento WORD de la vacante, junto con las demás vacantes enviadas de otras empresas, a este documento se le llama boletín.

    \item Cuando el departamento termina dicho boletín se publican en la página en formato PDF .

    \item Las personas que consulten la página de Facebook pueden ver los boletines publicados entre las demás publicaciones de la página. 
\end{itemize}
Con base en el proceso, para publicar vacantes en la página de Facebook identificamos los siguiente problemas que se presentan:
\begin{itemize}
    \item El algoritmo Facebook ordena cada publicación dentro del perfil de una persona según lo que cree que es más relevante para él, las va ordenando por la fecha más reciente de publicación y en la página ``Bolsa de Trabajo ESCOM'' no solo se publican boletines únicamente por consecuencia, los boletines se van perdiendo poco a poco entre las publicaciones de la página.

    \item Las vacantes no son publicadas de manera inmediata: la vacante es publicada en el boletín hasta que haya pasado por el punto 1 y 2 del proceso y para esto pueden pasar días hasta que la empresa cumpla con los requisitos lo cual retrasa la publicación de nuevas ofertas realizadas para alumnos de ESCOM.

    \item No facilita  las búsquedas de las vacantes a los interesados, al ser una publicación con un documento pdf adjunto, el componente de búsqueda no detecta la información que hay en cada boletín, por  que si se quiere consultar una vacante en específico es imposible buscándola por alguna palabra clave.

    \item El departamento no puede generar estadísticas o reportes de las vacantes que se han publicado, y en qué fechas ni mucho menos cuantos se postularon a que vacante o en su caso, si hubo una contratación exitosa o no. 

    \item No facilita  a los reclutadores la selección de los candidatos, ya que tienen que buscar en su bandeja de entrada o correo no deseado los CV’s de los alumnos, descargarlos y posteriormente consultarlos, provocando que se lleguen a extraviar los curriculums de algunos candidatos.

\end{itemize}