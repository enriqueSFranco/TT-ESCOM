%!TEX root = ../example.tex
\chapter{Modelo del Alcance}
\label{ch:modAlcance}

\section{Modelado de Usuarios}

\clearpage
\begin{actor}{aGerenteVentas}{Gerente de Ventas}{Es el encargado de todas las operaciones de ventas al mayoreo y menudeo, coordina y supervisa el trabajo de los agentes de Ventas y encargados de Tienda. Reporta directamente al Gerente de operaciones.}
    \item[Área:] Gerencia
    \item[Responsabilidades:] \hfill
        \begin{itemize}
            \item Supervisar la operacion d eventas.
            \item Plantear y Supervisar el logro de las metas d eventas de la empresa y su crecimiento economico.
            \item Cuidar a los \refElem{aGerenteVentasJr}
        \end{itemize}
    \item[Perfil:] \hfill
        \begin{itemize}
            \item Amplia experiencia en el ramo
            \item Licenciatura como mínimo
        \end{itemize}
    \item[Cantidad:] 1 unidad.
\end{actor}

\begin{actor}{aGerenteVentasJr}{Gerente de Ventas Jr}{Hereda del \refElem{aGerenteVentas} Es el chambitas del señor gerente de ventar, aveces va por los garrafones, antojos y cosas de papeleria.}
    \item[Área:] Gerencia Jr
    \item[Responsabilidades:] \hfill
        \begin{itemize}
            \item Supervisar la operacion d eventas.
            \item Plantear y Supervisar el logro de las metas d eventas de la empresa y su crecimiento economico.
            \item ... Hacer algo en \refIdElem{HR-PR-CU1}
        \end{itemize}
    \item[Perfil:] \hfill
        \begin{itemize}
            \item Amplia experiencia en el ramo
            \item Licenciatura como mínimo
        \end{itemize}
    \item[Cantidad:] 1 unidad.
\end{actor}
